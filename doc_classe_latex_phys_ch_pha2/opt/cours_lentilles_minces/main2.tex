\cours{Les lentilles minces}
 

\section{Qu'est-ce qu'une lentille ?}

\subsection{D�finition}

\subsection{Les diff�rents types de lentilles}

\subsection{Caract�ristiques de lentilles monces}

\section{Les lentilles convergentes}



\section{Les lentilles divergentes}


\newpage

\null

\newpage

\section*{Relations de grandissement}


\begin{center}
\begin{pspicture}(-7,-2)(7,2)
%\psgrid[subgriddiv=1,griddots=10]

\psline{->}(-7,1.5)(-6.5,1.5)
\rput(-6.75,1.25){$+$}
\psline{->}(-7,1.5)(-7,2)
\rput(-7.25,1.75){$+$}

\pnode(-5,0){A}
\pnode(-5,1){B}

\psline{->}(A)(B)
\uput{0.1}[225](A){$A$}
\uput{0.1}[225](B){$B$}

\psline{<->}(0,-2)(0,2)
\rput(-0.25,-0.25){$O$}

\psline(2,-0.1)(2,+0.1)
\rput(2,-0.25){$F'$}

\psline(-2,-0.1)(-2,+0.1)
\rput(-2,-0.25){$F$}

\psline{->}(-7,0)(7,0)
\end{pspicture}
\end{center}

\subsection{D�finition}

\formule{\gamma = \rapportma{A'B'}{AB}}

\subsection{Relations}

\[\ma{AB} = \ma{OI}\]


\[\ma{A'B'} = \ma{OJ}\]



\subsubsection{Relation de grandissement avec origine au centre $O$}

\begin{multicols}{2}

\begin{pspicture}(-3,-1.5)(3,1.5)
\psgrid[subgriddiv=1,griddots=10]
\end{pspicture}

\medskip

\formule{\gamma = \rapportma{A'B'}{AB} = +\rapportma{OA'}{OA}}

\end{multicols}



\subsubsection{Relation de grandissement avec origine  au foyer objet $F$}

\begin{multicols}{2}

\begin{pspicture}(-3,-1.5)(3,1.5)
\psgrid[subgriddiv=1,griddots=10]
\end{pspicture}

\medskip

\formule{\gamma = \rapportma{A'B'}{AB} = -\rapportma{OF}{FA}}
\end{multicols}


\subsubsection{Relation de grandissement avec origine  au foyer image $F'$}

\begin{multicols}{2}

\begin{pspicture}(-3,-1.5)(3,1.5)
\psgrid[subgriddiv=1,griddots=10]
\end{pspicture}

\medskip

\formule{\gamma = \rapportma{A'B'}{AB} = -\rapportma{F'A'}{OF'}}
\end{multicols}

\newpage
\section*{Relations de conjugaison}

\setcounter{subsection}{0}

\subsection{Formule de Newton (origine aux foyers)}
On utilise les deux relations de grandissement avec origine aux foyers
(objet et image) : $-\rapportma{OF}{FA} = -\rapportma{F'A'}{OF'}$. On
multiplie par $-1$, on a alors $\displaystyle \rapportma{OF}{FA} = \rapportma{F'A'}{OF'}$

% \begin{center}
% \fbox{$\displaystyle \ma{FA} \cdot \ma{F'A'} = \ma{OF} \cdot \ma{OF'}$}
% \end{center}

\formule{\ma{FA} \cdot \ma{F'A'} = \ma{OF} \cdot \ma{OF'}}


Cette formule, nomm�e aussi \emph{formule de Newton} se note �galement
: $x \cdot x' = f f'$.

Or comme pour une lentille mince on a $\ma{OF} = - \ma{OF'}$ soit $f = -f'$ on peut alors �crire :


% \begin{center}
% \fbox{$\displaystyle x \cdot x' = - {f'}^2$}
% \end{center}


\formule{x \cdot x' = - {f'}^2}


\subsection{Formule de Descartes (origine au centre)}
On utilise la relation de grandissement avec origine au centre et celle avec origine au foyer image.

\[+\rapportma{OA'}{OA} = -\rapportma{F'A'}{OF'}\]

Soit en mettant en ligne : $\ma{OA'} \cdot \ma{OF'} = -\ma{F'A'} \cdot \ma{OA}$

Or $\ma{F'A'} = \ma{F'O} + \ma{OA'} = \ma{OA'} - \ma{OF'}$.


Il vient alors $\ma{OA'} \cdot \ma{OF'} = -(\ma{OA'} - \ma{OF'}) \cdot \ma{OA}$.


En d�veloppant, on obtient : $\ma{OA'} \cdot \ma{OF'} = -\ma{OA'} \cdot \ma{OA} + \ma{OF'} \cdot \ma{OA}$

%Comme $\ma{OF} = -\ma{OF'}$

Soit aussi : $\ma{OF'} \cdot \ma{OA} -\ma{OA'} \cdot \ma{OF'} = \ma{OA'} \cdot \ma{OA}$



On divise alors par $\ma{OA} \cdot \ma{OA'} \cdot \ma{OF'}$ et on obtient :

% \begin{center}
% \fbox{$\displaystyle \frac{1}{\ma{OA'}} - \frac{1}{\ma{OA}} = \frac{1}{\ma{OF'}}$}
% \end{center}

\formule{\frac{1}{\ma{OA'}} - \frac{1}{\ma{OA}} = \frac{1}{\ma{OF'}}}


Cette formule, nomm�e �galement \emph{Relation de Descartes}, se note aussi :

\formule{\frac{1}{p'} - \frac{1}{p} = \frac{1}{f'}}

\subsection*{R�sum�}

\begin{tabular}{cc}

$\ma{OF} = f$    &  distance focale objet\\
$\ma{OF'} = f'$  &  distance focale image\\
$\ma{OA} = p$    &  distance objet-lentille\\
$\ma{OA'} = p'$  &  distance image-lentille \\
$\ma{FA} = x$    &  distance foyer objet - objet\\
$\ma{F'A'} = x'$ &  distance foyer image - image \\

\end{tabular}

\subsubsection*{Relation de Descartes}
\formule{\frac{1}{p'} - \frac{1}{p} = \frac{1}{f'}}

\subsubsection*{Formule de Newton}
\formule{x \cdot x' = - {f'}^2}

\subsubsection*{Grandissement}
\formule{\gamma = \rapportma{A'B'}{AB} = \frac{p'}{p} = -\frac{f}{x} =
  -\frac{x'}{f'}}
