 % Comment obtenir ce document ?

\thispagestyle{empty}

%\vressort{1}
\vressort{3}

Ce document est r�alis� avec \LaTeX2e\footnote{\url{http://www.latex-project.org}}.

\vressort{1}

\section*{Obtenir une copie de ce document}
Vous pouvez obtenir une copie de ce document au format \verb+.pdf+
Acrobat Reader sur :
\begin{itemize}
\item \url{http://s.cls.free.fr/wikini/wakka.php?wiki=Enseignement}
\item
  \url{http://svn.berlios.de/wsvn/openphysic/doc_classe_latex_phys_ch_pha/trunk/pdf/main_dvips_suite_2.pdf} pour le document r�duit en 2 pages en 1
\item
  \url{http://svn.berlios.de/wsvn/openphysic/doc_classe_latex_phys_ch_pha/trunk/pdf/main.pdf} pour le document non r�duit
\end{itemize}


\vressort{1}


\section*{Obtenir les fichiers \LaTeX2e par Subversion pour les modifier}

Il est possible d'obtenir les fichiers \LaTeX2e  � l'aide du gestionnaire
de version Subversion\footnote{\url{http://subversion.tigris.org}}.

\subsection*{Acc�s web}
%\begin{verbatim}
\noindent
\url{http://svn.berlios.de/viewcvs/openphysic/doc_classe_latex_phys_ch_pha/trunk}
%\end{verbatim}

\subsection*{Acc�s anonyme}
\begin{verbatim}
svn checkout svn://svn.berlios.de/openphysic/doc_classe_latex_phys_ch_pha/trunk
\end{verbatim}

\subsection*{Acc�s d�veloppeur}
\begin{verbatim}
svn checkout svn+ssh://scls19fr@svn.berlios.de/svnroot/repos/openphysic/
 .../doc_classe_latex_phys_ch_pha/trunk
\end{verbatim}


\vressort{1}

\section*{Compiler le document}
Pour compiler ce document, il est n�cessaire d'installer une distribution de
\LaTeX2e comme teTex\footnote{\url{http://www.tug.org/teTeX}} ou
Miktex\footnote{\url{http://www.miktex.org}}.

Pour �diter les fichiers \verb+.tex+, on peut utiliser un �diteur
sp�cialis� pour \LaTeX  comme
TeXnicCenter\footnote{\url{http://www.toolscenter.org}} ou
Kile\footnote{\url{http://kile.sourceforge.net}}. On peut aussi
utiliser un �diteur de texte g�n�raliste comme GNU Emacs\footnote{\url{http://www.gnu.org/software/emacs/emacs.html}}.

Dans un environnement GNU/Linux ou Cygwin\footnote{\url{http://www.cygwin.com}} comportant GNU Make\footnote{\url{http://www.gnu.org/software/make}} il suffit de taper :

\begin{itemize}
\item \verb+make view-ps+ pour obtenir un fichier PostScript \verb+.ps+
\item \verb+make view-ps-2+ pour obtenir un fichier PostScript
  \verb+.ps+ avec une r�duction 2 pages en 1
\item \verb+make view-pdf+ pour obtenir un fichier Acrobat Reader \verb+.pdf+
\item \verb+make view-pdf-2+ pour obtenir un fichier Acrobat Reader
  \verb+.pdf+ avec une r�duction 2 pages en 1
\end{itemize}


\vressort{3}

