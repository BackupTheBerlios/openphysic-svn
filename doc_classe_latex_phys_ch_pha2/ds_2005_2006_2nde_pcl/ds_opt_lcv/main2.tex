\ds{Devoir Surveill� }{
 \item Optique - lentilles minces
}

\nomprenomclasse
%\notationfinale{20}{5cm}


%\begin{exercice}{Titre de l'exercice}
%
%\end{exercice}



\renewcommand{\reproduire}{
%\emph{Questions de cours : Optique - lentilles minces}\\
\begin{enumerate}
\item Rappelez le symbole d'une lentille mince
  \begin{enumerate}
  \item convergente,
  \item divergente.
  \end{enumerate}
\item Placez les deux foyers $F$ et $F'$ et le centre optique $O$ dans le cas d'une lentille mince 
  \begin{enumerate}
  \item convergente,
  \item divergente.
  \end{enumerate}
\item Repr�sentez les trois rayons particuliers traversant une
  lentille mince
  \begin{enumerate}
  \item convergente,
  \item divergente.
  \end{enumerate} 
\item Rappelez la formule de conjugaison des lentilles minces et
  pr�cisez les unit�s.
\item Rappelez les deux formules de grandissement et pr�cisez les
  unit�s.
\item D�finissez la vergence d'une lentille et pr�cisez \emph{les} unit�s.
\end{enumerate}
}


\vressort{1}

\begin{large}

% \begin{exercice}{Questions de cours}
% \reproduire
% \end{exercice}


% \vressort{1}


\begin{exercice}{\\}
On place un objet lumineux plan $AB$ de $1~cm$ de hauteur, � $6~cm$ en
avant d'une lentille $L$ convergente de centre optique $O$ et de
distance focale $f'=4~cm$.

\begin{enumerate}
\item Calculez la vergence de la lentille.
\item Notez sur un sch�ma � l'�chelle $1$ ses foyers $F$ et $F'$.
\item Trouvez graphiquement
  \begin{enumerate}
  \item la position $\ma{OA'}$ de l'image $A'B'$,
  \item la taille $\ma{A'B'}$ de l'image $A'B'$ .
  \end{enumerate}
\item Donnez les caract�ristiques de l'image $A'B'$
  \begin{enumerate}
  \item sa nature (r�elle/virtuelle)
  \item son sens (m�me sens que l'objet/renvers�e)
  \item sa taille (agrandie/r�tr�cie)
  \end{enumerate}
\item D�terminez par le calcul
  \begin{enumerate}
  \item la position $\ma{OA'}$ de l'image $A'B'$,
  \item la grandeur $\ma{A'B'}$ de l'image $A'B'$.
  \end{enumerate}
\end{enumerate}

\end{exercice}


\vressort{1}

\begin{exercice}{\\}
Une lentille convergente a une distance focale $f' = 10~cm$. Un objet de $1~cm$ de haut se trouve � $5~cm$ en avant de la lentille.


\begin{enumerate}
\item Calculez la vergence de la lentille.
\item Notez sur un sch�ma � l'�chelle $1$ ses foyers $F$ et $F'$.
\item Trouvez graphiquement
  \begin{enumerate}
  \item la position $\ma{OA'}$ de l'image $A'B'$,
  \item la taille $\ma{A'B'}$ de l'image $A'B'$ .
  \end{enumerate}
\item Donnez les caract�ristiques de l'image $A'B'$
  \begin{enumerate}
  \item sa nature (r�elle/virtuelle)
  \item son sens (m�me sens que l'objet/renvers�e)
  \item sa taille (agrandie/r�tr�cie)
  \end{enumerate}
\item D�terminez par le calcul
  \begin{enumerate}
  \item la position $\ma{OA'}$ de l'image $A'B'$,
  \item la grandeur $\ma{A'B'}$ de l'image $A'B'$.
  \end{enumerate}
\end{enumerate}

% \begin{enumerate}
% \item Calculez la vergence de la lentille.
% \item Notez sur un sch�ma � l'�chelle $1$ ses foyers $F$ et $F'$.
% \item D�terminez par construction la position, la nature et la taille
%   de l'image.
% \item D�terminez par calcul la position et la taille de l'image.
% \end{enumerate}

\end{exercice}

\end{large}

\vressort{1}

\newpage

\null

