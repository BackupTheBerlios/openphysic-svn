\cours{Visualisation � l'aide d'un oscilloscope\\d'une tension d�livr�e par un G�n�rateur Basse Fr�quence}



Un oscilloscope permet de visualiser des tensions alternatives.


C'est un voltm�tre qui permet d'afficher la tension sur un �cran.


\vspace*{\stretch{3}}

\section{Tension continue ou tension alternative ?}

\subsection{Tension continue}
C'est une tension dont la valeur est constant dans le temps.


On d�livre g�n�ralement cette tension � l'aide d'un g�n�rateur de tension continue.


\emph{Exemple :} $U = 3~V$

Sa valeur �~maximale~� est constante et vaut $3~V$.

% chronogramme
\begin{center}
\begin{pspicture}(-1,-1)(4,2)
%\psgrid[subgriddiv=1,griddots=10]
\psset{xunit=1}
\psset{yunit=1}
\psline{->}(0,-0.25)(0,1.75) % Axe des y
\rput{0}(4,-0.25){$t (s)$}
\psline{->}(-1.25,0)(4,0) % Axe des x
\rput{0}(-0.25,1.75){$u (V)$}
%\psplot{0}{2}{x 1 sub} % y=x-1 (add(1+2) div(1/2) mul(1*2) sub(1-2) exp(1^2) neg abs sin cos ln log
%\psplot{0}{4}{2.71828 x neg exp } % exp(-x)
\psline[linecolor=black,linewidth=2pt](-1,1)(3,1)
%\psplot[linecolor=red,linewidth=2pt]{0}{4}{2.71828 x neg exp neg 1 add} % 1-exp(-x)
\rput{0}(-0.25,1.1){$3$}
%\rput{0}(-0.25,0.76){$2$}
%\rput{0}(-0.25,0.43){$1$}
\end{pspicture}
\end{center}

% TO DO utiliser \psaxe


Un voltm�tre �~classique~� est suffisant pour pouvoir mesurer une tension continue.

\vspace*{\stretch{1}}



% oscillogramme

%\newpsstyle {BlueDots}{plotstyle=dots, 
%linecolor=blue,linewidth=0.02,plotpoints=50}
%\Oscillo[amplitude1=3, plotstyle2 =BlueDots,amplitude2=2]V

\newpsstyle{BlackStyle}{plotstyle=line,linecolor=black,linewidth=0.075}
\newpsstyle{InvisibleStyle}{linecolor=black,linewidth=0}

\begin{center}
\Oscillo[offset1=3,amplitude1=0,plotstyle1=BlackStyle,plotstyle2=InvisibleStyle]
\end{center}


\pagebreak

\subsection{Tension alternative}

C'est une tension qui varie dans le temps. On d�livre g�n�ralement une tension alternative � l'aide d'un G�n�rateur Basse Fr�quence (G.B.F.). Pour pouvoir caract�riser une tension alternative il est n�cessaire de donner ses caract�ristiques.

\subsubsection{Le type de signal}

\begin{itemize}
\item sinuso�dal

\begin{center}
\scalebox{0.7}{\Oscillo[amplitude1=1,plotstyle1=BlackStyle,plotstyle2=InvisibleStyle]}
\end{center}


\item triangulaire

\begin{center}
\scalebox{0.7}{\Oscillo[Wave1=\TriangleA,plotstyle1=BlackStyle,plotstyle2=InvisibleStyle]}
\end{center}


\item rectangulaire ou en cr�neaux

\begin{center}
\scalebox{0.7}{\Oscillo[Wave1=\RectangleA,plotstyle1=BlackStyle,plotstyle2=InvisibleStyle]}
\end{center}


\end{itemize}


\subsubsection{La p�riode $T$ du signal}

La p�riode $T$ est la plus petite dur�e au bout de laquelle la tension reprend la m�me valeur en variant dans le m�me sens.

$T$ s'exprime en seconde $s$.


\subsubsection{La fr�quence}
Le nombre de p�riodes par seconde est appel� fr�quence $f$.

\[f = \frac{1}{T}\]

\begin{itemize}
\item $f$ : fr�quence (en Hertz $Hz$)
\item $T$ : p�riode (en seconde $s$)
\end{itemize}

\subsubsection{Tension cr�te � cr�te}
C'est la diff�rence de tension entre les valeurs maximales et minimales de la tension ; on la note $U_{CC}$.

L'amplitude de la tension est $U_M = \frac{U_{CC}}{2}$ car elle varie entre $-U_M$ et $+U_M$.


\section{Utiliser un oscilloscope}


\section{Principe de fonctionnement de l'oscilloscope}


\section*{Exercices sur l'oscilloscope}



\begin{center}
Redressement monoalternance
\Oscillo[amplitude1=1,amplitude2=0.5,offset2=0.5,Wave2=\RectangleB,operation=mul,combine=true,plotstyle1=InvisibleStyle,plotstyle2=InvisibleStyle]
\end{center}


\begin{center}
Redressement bialternance
\Oscillo[amplitude1=1,amplitude2=1,Wave2=\RectangleB,operation=mul,combine=true,plotstyle1=InvisibleStyle,plotstyle2=InvisibleStyle]
\end{center}
