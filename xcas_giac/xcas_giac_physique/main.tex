\documentclass[a4paper,10pt]{book}

\usepackage[frenchb]{babel}     % specification francaise
\usepackage[latin1]{inputenc}   % entree clavier latin1
\usepackage[T1]{fontenc}        % sortie


\usepackage[french]{minitoc}


\title{Utilisation de Xcas - Giac\\ en Sciences Physiques}
\author{S�bastien Celles}
\date{\today} 


\begin{document}

% Titre
\maketitle



% Sommaire
\dominitoc % table des mati�res locales 
\tableofcontents


% r�f�rences biblio
% La physique avec Maple 1�re et 2�me ann�e
%   MP PC PSI PT
%   Vincent BOURGES
%   Ed. Ellipse


% Maple cours et applications (2�me �dition)
%   MPSI PCSI PTSI
%   MP PSI PC PT
%   Lionel PORCHERON
%   Ed. DUNOD

% MATLAB pour l'ing�nieur
%   Adrian BIRAN et Moshe BREINER
%   Ed. PEARSON Education


%\part{Part1}






\newcommand{\chapitre}[1]{%
\chapter{#1}
\vspace*{\stretch{1}}
\minitoc
\vspace*{\stretch{3}}
\newpage
}




\chapitre{M�canique}

\section{Chute libre}
\section{Mouvement d'un satellite}
% portrait de phase
% points singuliers points fixes points critique diagramme de
% bifurcation

\section{Oscillateurs m�caniques}

\subsection{Oscillateur harmonique libre non amorti}

\subsection{Oscillateur harmonique libre amorti}

\subsection{Portrait de phase d'un oscillateur m�canique}

\subsection{Oscillateur harmonique amorti en r�gime forc�}

\subsection{Oscillateurs coupl�s}




%\part{Part2}


\chapitre{\'Electronique}

\section{Diagramme de Bode}

\section{Comparateur}

\section{Oscillateurs}





\chapitre{\'Electrostatique}
\section{Spectre du champ �lectrique}

\section{Mouvement d'une particule dans un champ �lectrique}





\chapitre{Magn�tostatique}
\section{Spectre du champ magn�tique}
\section{Mouvement d'une particule dans un champ magn�tique}





\chapitre{Optique}
\section{Optique g�om�trique}
\section{Interf�rences}
\section{Diffraction}
\section{R�seau}





\chapitre{Physique ondulatoire}
\section{\'Etude d'une onde � une dimension}
\section{Ondes stationnaires}
%\section{\'Etude d'une onde � deux dimensions}
%\subsection{d




\chapitre{Analyse vectorielle}
\section{Gradient}
\section{Divergence}
\section{Rotationnel}
\section{Laplacien}



\chapitre{M�canique des fluides}
\section{\'Etude d'un �coulement laminaire autour d'une sp�h�re}
\section{\'Ecoulement}
% voir la simu Scilab de Biansan (Optg�o)



\chapitre{Traitement du signal}
\section{Analyse de Fourier}
\subsection{Transform�e de Fourier � l'aide du calcul formel}
\subsection{Transform�e de Fourier Rapide (FFT)}


\section{Transformation de Laplace}




\chapitre{Thermodynamique}
\section{Diffusion}





\chapitre{Physique non lin�aire}
\section{Espace de phase}
\section{Trajectoire de phase}



\chapitre{Chimie}
\section{Orbitales de l'atome d'hydog�ne}
\section{R�action acido-basiques}
\section{Cin�tique chimique}
\section{Diagramme potentiel-pH}





\chapitre{Fractales}
\section{Ensemble de Cantor}
\section{Courbe de Koch}
\section{Courbe de P�ano}
\section{Tamis de Siepinski}
% sapin/arbre de noel
% foug�re

\section{Arbres de Pythagore}
% arbre � broccoli

\section{Conway : le jeu de la vie}
% -> automate

\section{Courbe du crabe}

\section{Courbe de Hilbert}

\section{Ensemble de Julia}

\section{Ensemble de Mendelbrot}

\section{Gaussienne}

\section{Profil brownien d'une montagne}







\chapitre{Chaos}

\section{Sc�nario de Feigenbaum}

\section{Attracteur de H�non}

\section{Attracteur de R�ssler}

\section{Attracteur de Lorentz}




%\section{Pendule}

% accrochage d'une machine synchrone sur le r�seau ?






\end{document}


