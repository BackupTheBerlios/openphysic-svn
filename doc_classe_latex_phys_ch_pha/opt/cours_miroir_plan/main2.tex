\cours{Miroir plan}

\section{Exp�rience des deux bougies}



\emph{Conclusion :} $A'$ est le sym�trique de $A$ par rapport au plan
du miroir.




$AB$ est l'objet (r�el).


$A'B'$ est un image virtuelle : on ne peut pas l'observer directement
sur un �cran.




\section{Trac�s}

\subsection{Comment tracer le rayon issu d'un point et se
  r�fl�chissant  sur un miroir plan ?}


\subsubsection{M�thode 1 : report d'angle}



Le rayon incident, le rayon r�fl�chi et la normale au miroir plan sont
dan le m�me plan appel� plan d'incidence.


L'angle de r�flexion est �gal � l'angle d'incidence ($i = r$)


\subsubsection{M�thode 2 : en utilisant le point image}


On trace par sym�trie par rapport au plan du miroir l'image $A'$ de
$A$.


Le rayon r�fl�chi est issu de $A'$


\emph{Remarque : } aucun rayon r�el ne se coupe en $A'$ ; $A'$ est
donc une image virtuelle.



\newpage
\subsection{Faisceau r�fl�chi par un miroir}


\newpage


\subsection{Champ d'un miroir plan}

Soit un observateur $O$ regardant un miroir plan $M$ de dimension
finie. On cherche l'ensemble des points $A$ visible par l'obsevateur
au travers le miroir.



\section{Applications}
\subsection{P�riscope}

Un p�riscope permet � un sous-marin d'observer � la surface de l'eau
tout en �tant immerg�. 


\subsection{Appareil photo �~r�flex~�}




\subsection{Miroir orthogonaux}


\emph{Remarque : } apr�s 2 r�flexions le rayon r�fl�chi repart avec la
m�me direction mais dans le sens oppos�.




\section{Relation de conjugaison et grandissement}

\subsection{Conventions de signe}



\subsection{Relation de conjugaison}

Une \emph{relation de conjugaison} est une relation qui donne la
position de l'image en fonction de la position de l'objet et des
caract�ristiques du syst�me optique.


Une relation de conjugaison est toujours �crite de mani�re
\emph{alg�brique}, c'est � dire avec des longueurs qui peuvent �tre
positives ou n�gatives.

\emph{Exemple : }

$OA = 10~cm$ mais $\ma{OA} = -10~cm$

ici
\begin{itemize}
\item $\ma{OA} < 0$ et $\ma{OA'} > 0$
\item $\ma{AB} > 0$ et $\ma{A'B}' > 0$
\end{itemize}


$AO = OA'$ devient, avec la convention de signe $\ma{AO} = \ma{OA'}$
soit $-\ma{OA} = \ma{OA'}$



On obtient alors la relation de conjugaison pour un miroir plan

\[\ma{OA'} + \ma{OA} = 0\]


\subsection{Grandissement}

Le grandissement $\gamma$ permet de d�terminer la taille et le sens de
l'image.


\[\gamma = \frac{\ma{A'B'}}{\ma{AB}}\]


Le grandissement $\gamma$ peut �tre positif ou n�gatif,



% sch�ma





Pour un miroir plan, le grandissement $\gamma$ vaut : 


\[\gamma = 1\]


car $\ma{AB} = \ma{A'B'} = \ma{OO_2}$

