\docsanstitre{Quelques constructions g�om�triques pour les lentilles minces
  convergentes}

\renewcommand{\reproduire}{%

\thispagestyle{empty} %plain

\section*{\large Quelques constructions g�om�triques pour les lentilles  minces convergentes}


\[f' = \overline{OF'} > 0\]


\vressort{3}

\noindent
\optbiglcvdv{Cas 1}{<->}{objet r�el}{$\overline{OA} \in ]-\infty ; 2f[$}

\vressort{1}

\noindent
\optbiglcvdv{Cas 2}{<->}{objet r�el}{$\overline{OA} \in ]2f ; f[$}

\vressort{1}

\noindent
\optbiglcvdv{Cas 3}{<->}{objet r�el dans le plan focal
  objet}{$\overline{OA} = f$}

\vressort{1}

\noindent
\optbiglcvdv{Cas 4}{<->}{objet r�el entre le plan focal objet et la lentille}{$\overline{OA} \in ]f ; 0[$}


\vressort{1}
}

\reproduire

\newpage

\reproduire


