

\begin{landscape}



\tp{La loupe}

\thispagestyle{empty}


%\vspace*{2cm}

\vressort{3}

% \objectifs{
% %\normalsize
% \item savoir placer convenablement un objet afin de visualiser
%   convenablement son image au travers une loupe.
% \item tracer des rayons lumineux
% \item utiliser la formule de conjugaison de Descartes
% \item utiliser les formules de grandissement
% }

% \vressort{1}


% \materiel{
% %\normalsize
% \item lentille convergente de vergence $V = +8~\delta$
% }

% \vressort{3}


%\Large



\begin{enumerate}


%\item Que se passe-t'il si l'objet est situ� entre le foyer objet $F$ et le
%centre optique $O$ d'une lentille ?

%\vressort{1}

\item Prenez la lentille de vergence $V = +8~\delta$ et placez l'objet tel
que $\ma{OA} = -8~cm$.

%\vressort{1}

\item Sch�matisez sur la feuille de papier millim�tr� � l'�chelle $1/1$, la
loupe et les rayons lumineux issus de l'objet.
On placera la lentille � $2~cm$ du bord droit du papier millim�tr�.

%\vressort{1}

\begin{enumerate}
\item O� l'objet est-il plac� par rapport aux foyers ($F$ ou $F'$) de
  la lentille et � son centre optique $O$ ?

\item Pouvez-vous recueillir l'image de l'objet sur un �cran ?

\item Regardez alors � travers la lentille, qu'observez-vous ?

\item D�terminez graphiquement la position de l'image ainsi que sa
  grandeur $A'B'$ pour un objet $AB$ de $1~cm$ de hauteur.

\item Retrouvez ces valeurs par le calcul.
\end{enumerate}


%\vressort{3}

% papier milli

%\hspace*{-1.5cm}
\begin{center}
\begin{pspicture}(0,0)(26,6)
\psmilli
%\psmilliblack
%\psmillicolor
%\psmilligray{0.5}
\end{pspicture}
\end{center}


\item \emph{Conclusion :} Lorsque l'objet est situ� entre le foyer objet $F$
et le centre optique $O$ d'une lentille, l'image \trou{ne peut pas}
�tre recueillie sur un �cran. On dit que l'on a une image
\trou{virtuelle}. De plus, cette derni�re est \trou{plus grande} et
\trou{dans le m�me sens}�que l'objet.


Ainsi, une loupe est simplement une \trou{lentille mince convergente}
plac�e dans une position particuli�re par rapport � l'objet.


\end{enumerate}

%\pagepapiermilli

\end{landscape}


%\normalsize

% \newpage

% \begin{enumerate}
% \item 
% \end{enumerate}