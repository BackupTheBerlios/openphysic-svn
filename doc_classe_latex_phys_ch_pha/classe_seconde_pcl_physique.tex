\classe{Seconde\ \\
option Physique et Chimie\ \\
de Laboratoire\ \\
Partie Physique}{Seconde option PCL - Partie Physique}

\chapitre{Courant et tension �lectrique}
\inclure{elec/cours_elec_intro} % cours elec circuit �lectrique

\inclure{elec/tp_i_u} % tp mesure de tension et d'intensit�

\inclure{elec/fiche_methode_montage} % document fiche
    % m�thode montage
    % utilisation d'un multim�tre, ...

\inclure{elec/tp_loi_ohm} % tp loi d'ohm

\inclure{elec/tp_asso_r} % tp associations de R

% association r s�rie (trac� U en fonction de I et montrer qu'on ajoute U)
% association r parall�le (trac� U en fonction de I et montrer qu'on ajoute I)

%\chapitre{G�n�rateurs et r�cepteurs}
%\inclure{elec/cours_elec_gene_recep} % cours elec g�n�rateurs, r�cepteurs

%\inclure{elec/tp_carac_gene} % tp caract�ristique d'un g�n�rateur
%\inclure{elec/tp_carac_recep} % tp caract�ristique d'un
                                % r�cepteur (�lectrolyseur)


\inclure{elec/tp_puissance_energie} % tp puissance �nergie (continu)



\chapitre{Optique g�om�trique}
\inclure{opt/tp_reflexion_refraction}

\inclure{opt/cours_lentilles_minces}
\inclure{opt/doc_constructions_lentilles}
\inclure{opt/tp_lentilles_minces_conv_conjug} % Lentilles (Images, rel
                                % de conjugaison)
\inclure{opt/tp_loupe}

\inclure{ds_2005_2006_2nde_pcl/interro_opt_lcv}
\inclure{ds_2005_2006_2nde_pcl/ds_opt_lcv}

\inclure{opt/tp_lentilles_minces_conv_foco} % Focom�trie lentilles
                                % minces

% lunette astronomique





\chapitre{M�canique}
%\inclure{meca/tp_poids_ressort_archimede}
\inclure{meca/tp_poids}
\inclure{meca/tp_ressort}
\inclure{meca/tp_archimede}

\inclure{meca/tp_equilibre_solide_3_forces}
\inclure{meca/doc_rapporteurs}

\inclure{meca/tp_moment_force_axe}

\inclure{meca/tp_poulies_palans}




%\chapitre{Thermique}
