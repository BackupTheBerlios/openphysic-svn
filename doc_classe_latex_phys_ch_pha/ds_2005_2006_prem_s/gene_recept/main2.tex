%\doc{}


\renewcommand{\reproduire}{%
%\doc{\'Electricit� en courant continu}


\titredoc{\'Electricit� en courant continu}



\vressort{1}

\nomprenomclasse

\vressort{1}

%\setcounter{numexercice}{0}
\begin{exercice}{\'Etude d'un �lectrolyseur}\\
Une pile de force �lectromotrice $E = 6~V$ et de r�sistance $r =
2~\Ohm$ est associ�e en s�rie avec un �lectrolyseur de force contre
�lectromotrice $E' = 2~V$ et de r�sistance $r' = 0,1~\Ohm$.

\begin{enumerate}
\item Faire le sch�ma du montage.

\item Simplifier le sch�ma du montage en rempla�ant la pile et
  l'�lectrolyseur par deux g�n�rateurs id�aux de tension ($E$ et $E'$)
  et deux r�sistances ($r$ et $r'$).

\item Calculer l'intensit� du courant dans ce circuit.

\item Exprimer litt�ralement puis calculer :

  \begin{enumerate}
  \item la puissance chimique stock�e par la pile
  \item la puissance �lectrique disponible aux bornes de la pile
  \item la puissance �lectrique re�ue par l'�lectrolyseur
  \item la puissance �lectrique utile, utilis�e pour r�aliser des
    transformations chimiques

  \end{enumerate}

\item D�finir et calculer le rendement de l'�lectrolyseur.

\item D�finir et calculer le rendement du montage dans son ensemble.
\end{enumerate}
\end{exercice}

\vressort{1}

\begin{exercice}{Association de r�sistances}\\

\begin{center}
\begin{pspicture}(0,-2)(10,3)
\wire[intensitylabel=$I$](-1,0)(A)

\pnode(0,0){A}
\pnode(4,0){B}
\pnode(6,0){C}
\pnode(10,0){D}

\wire(D)(11,0)

\resistor[parallel,parallelarm=1.5,intensitylabel=$I_1$](A)(B){$R_1$}
\resistor[parallel,parallelarm=-1.5,intensitylabel=$I_2$](A)(B){$R_2$}

\wire(B)(C)

\resistor[parallel,parallelarm=1.5,intensitylabel=$I_3$](C)(D){$R_3$}
\resistor[parallel,parallelarm=-1.5,intensitylabel=$I_4$](C)(D){$R_4$}

\psline{<-}(-1,3)(11,3) \rput{0}(5,3.25){$U$}
\end{pspicture}
\end{center}

\donnees{
\item $U = 12~V$
\item $R_1 = 8~\Ohm$
\item $R_2 = 2~\Ohm$
\item $R_3 = 8~\Ohm$
\item $R_4 = 12~\Ohm$
}

\begin{enumerate}
\item D�terminer $R_{\mbox{\'eq1}}$ association de $R_1$ et $R_2$.
\item D�terminer $R_{\mbox{\'eq2}}$ association de $R_3$ et $R_4$ . 
\item Calculer l'intensit� $I$ du courant dans la branche principale.
\item D�terminer la tension $U_1$ aux bornes de  $R_{\mbox{\'eq1}}$.
\item D�terminer la tension $U_2$ aux bornes de  $R_{\mbox{\'eq2}}$.
\item D�terminer les intensit�s des courants $I_1$, $I_2$, $I_3$ et $I_4$.
\item Calculer la puissance dissip�e par effet Joule dans chaque
  r�sistance et indiquer dans laquelle elle est la plus grande.
\end{enumerate}
\end{exercice}
\setcounter{numexercice}{0}
}

%\setcouter{exo}{0}

\thispagestyle{empty}

\reproduire

%\vressort{1}

%\reproduire

\newpage
\thispagestyle{empty}

\reproduire

%\vressort{1}

%\reproduire

