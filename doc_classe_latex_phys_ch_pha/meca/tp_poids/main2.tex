\tp{\'Etude du poids d'un corps}

%\begin{multicols}{2}

\vressort{2}

\objectifs{
\item faire le lien entre l'intensit� du poids et la masse d'un corps
}

\materiel{
\item dynamom�tre
\item masselottes
}

%\bigskip

%\end{multicols}

\vressort{2}


%\section{\'Etude du poids d'un corps}

\section{Montage}

Le poids, comme toutes les forces se mesure avec un dynamom�tre. Il en existe en spirale et des longitudinaux.


Chaque masse au repos exerce une force sur le corps qui la retient (elle est en �quilibre). Ce corps peut �tre un support, une ficelle... La force qu'exerce la masse est son \emph{poids}.

\vressort{2}

\section{Mesures}
Mesurer les poids de quelques masselottes et compl�ter le tableau suivant :

\begin{arraydata}{9}
\hline
\rule[-0.4cm]{0cm}{1cm}%
$m$ ($kg$) & & & & & & & & & \\ \hline
\rule[-0.4cm]{0cm}{1cm}%
$P$ ($N$)  & & & & & & & & & \\ \hline
\end{arraydata}


\vressort{2}

\section{Exploitation, conclusion}
\begin{enumerate}

\item Construire une courbe o� figureront en abscisse la masse $m$ (en $kg$) et en ordonn�e le poids $P$ (en $N$).

\vressort{1}

\item Exploiter les param�tres de la courbe (Est-ce une droite�? Passe-t-elle par l'origine�? Quel est son coefficient directeur�? Quelle est l'unit� de ce coefficient directeur ?).

\vressort{1}

\item Conclure sur la relation liant l'intensit� $P$ du poids et la masse $m$.
\end{enumerate}

% \begin{center}
% \begin{pspicture}(-7,-4)(7,4)
%    \psmilli % papier milli noir
% \end{pspicture}
% \end{center}


%\noindent
%\papiermilli{11}{6}

% \begin{center}
% \scalebox{1.414}{
% \begin{pspicture}(0,0)(11,6)
%    \psmilli % papier milli noir
% \end{pspicture}
% }
% \end{center}

