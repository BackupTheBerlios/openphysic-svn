\tp{\'Etude de la tension d'un ressort}

%\begin{multicols}{2}

\vressort{2}

\objectifs{
\item relier la tension qu'exerce un ressort avec son allongement
}

\materiel{
\item dynamom�tre
\item masselottes
\item ressort
\item �prouvette gradu�e remplie d'eau (puis d'huile)
}

%\bigskip

%\end{multicols}

\vressort{2}


\section{Montage}

On dispose d'un ressort � spires non jointives, d'une potence comportant une r�gle gradu�e verticale et d'une boites de masses marqu�es � crochets.

\begin{multicols}{2}

%\begin{tabularx}{\textwidth}{XX}

%\begin{center}

\begin{figure}[H]
\begin{pspicture}(-1,-5)(7,0)
%\psgrid[subgriddiv=1,griddots=10]
% http://tex.loria.fr/graph-pack/pstricks/pst-usr4.pdf
%\pscoil(0,2)(0,-1)

% ----- 
% ////
%\psline(-0.5,0)(0.5,0)
%\multiput(0,0)(0.25,0){5}{\psline(-0.5,0)(-0.25,0.25)}

% potence
\psline[linewidth=0.1](0,0)(-1,0)
\psline[linewidth=0.1](-1,0)(-1,-5)
\psline[linewidth=0.1](-2,-5)(0,-5)
\multiput(0,0)(0.25,0){8}{\psline(-1.75,-5)(-2,-5.25)}

%\pszigzag[coilarm=.1]{-}(0,0)(0,-2.5) % linearc=0.1
\pszigzag[coilarm=.1,coilheight=0.5,coilwidth=0.5]{-}(0,0)(0,-2.5) % linearc=0.1

% \pszigzag[coilarm=.1,coilheight=1.4]{-}(0,0)(0,-3.5)
% The distance along the axes for each period is equal to colheight x coilwidth
% 1.4 = 3.5/2.5


\psline{|->}(1,0)(1,-2.5)
\rput(1.5,-1.25){$l_0$}
\psline{|->}(1,-2.5)(1,-3.5)
\rput(1.5,-3){$\Delta l$}

\rput(0.5,-4.25){Ressort � vide}

\rput(4,0){% Translation
%\psline(-0.5,0)(0.5,0)
%\multiput(0,0)(0.25,0){5}{\psline(-0.5,0)(-0.25,0.25)}

% potence
\psline[linewidth=0.1](0,0)(-1,0)
\psline[linewidth=0.1](-1,0)(-1,-5)
\psline[linewidth=0.1](-2,-5)(0,-5)
\multiput(0,0)(0.25,0){8}{\psline(-1.75,-5)(-2,-5.25)}

\pszigzag[coilarm=.1,coilheight=0.7,coilwidth=0.5]{-}(0,0)(0,-3.5)
% coilheight=0.7  =   0.5 (gauche) * 3.5 / 2.5
\psline{|->}(1,0)(1,-3.5)
\rput(1.5,-1.75){$l$}

\psframe*(-0.25,-3.5)(0.25,-4) % [fillstyle=vlines]

\rput(0.5,-4.25){Ressort en charge}
}
\end{pspicture}
\caption{Ressort � vide et en charge}
\end{figure}

%\end{center}

%&

%\begin{center}
% \begin{pspicture}(-1,-2)(2,2)
% \psgrid[subgriddiv=1,griddots=10]
% %   \psmilli % papier milli noir
% \pszigzag[coilarm=.1]{|-}(0,2)(0,-1.5)
% \end{pspicture}
%\end{center}

%\end{tabular}


%&

\rule{-1cm}{1.5cm}

\begin{itemize}
\item $l_0$ : longueur � vide du ressort.
\item $l$ : longueur du ressort en charge.
\item $\Delta l = l - l_0$ : allongement du ressort.
\end{itemize}

\end{multicols}

%\end{tabularx}


\vressort{2}

\section{\'Etude}

Le syst�me m�canique �tudi� est la masse $m$.
\begin{enumerate}
\item Faire le bilan des forces (ext�rieures) agissant sur la masse $m$.

\vressort{1}

\item Repr�senter les forces sur un sch�ma.

\vressort{1}

\item \'Enoncer la loi d'�quilibre de la masse $m$. En d�duire une relation liant l'intensit� $P$ du poids et la tension $T$ du ressort.

\end{enumerate}


\vressort{2}


\section{Mesures}

\begin{enumerate}
\item Mesurer la longueur $l_0$ du ressort � vide.

\vressort{1}

\item Poser diff�rentes masses $m$ � l'extr�mit� du ressort, mesurer la longueur du ressort en charge, d�duire l'allongement $\Delta l = l - l_0$ de celui-ci.

\vressort{1}

\item Noter les r�sultats exp�rimentaux dans le tableau suivant :


\begin{arraydata}{9}
\hline
\rule[-0.4cm]{0cm}{1cm}%
$m$ ($kg$) & & & & & & & & & \\ \hline
\rule[-0.4cm]{0cm}{1cm}%
$P$ ($N$)  & & & & & & & & & \\ \hline
\rule[-0.4cm]{0cm}{1cm}%
$T$ ($N$)  & & & & & & & & & \\ \hline
\rule[-0.4cm]{0cm}{1cm}%
$\Delta l$ ($m$)  & & & & & & & & & \\ \hline
\end{arraydata}


\vressort{1}

\item Tracer la courbe $T = f(\Delta l)$.

\vressort{1}

\item Exploiter les param�tres de la courbe (Est-ce une droite�? Passe-t-elle par l'origine�? Quel est son coefficient directeur�? Quelle est l'unit� de ce coefficient directeur ?).

\vressort{1}

\item En d�duire la relation (vectorielle) liant la tension $\vect{T}$ du ressort avec l'allongement de celui-ci.

\end{enumerate}


%\noindent
%\papiermilli{11}{5}

% \begin{center}
% \scalebox{1.414}{
% \begin{pspicture}(0,0)(11,5)
%    \psmilli % papier milli noir
% \end{pspicture}
% }
% \end{center}

