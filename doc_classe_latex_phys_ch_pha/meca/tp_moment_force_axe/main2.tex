\tp[Moment d'une force par rapport � un axe]{Moment d'une force \\par rapport � un axe}

\objectifs{
\item Trouver la condition de mise en rotation d'un objet.
\item Trouver l'expression du moment d'une force par rapport � un axe.
}


%\materiel{
%\item masselottes
%\item r�gle
%}


\section{Mise en rotation d'un solide}

\begin{center}
\begin{pspicture}(-5,-0.5)(5,0.5)
\psframe(-5,-0.5)(5,0.5)

%\rput(0,0){$\otimes$} % contraire de odot

\pscircle(0,0){0.2}
\psline(-0.14,-0.14)(0.14,0.14)
\psline(0.14,-0.14)(-0.14,0.14)


\rput(0.5,0){$\Delta$}
\end{pspicture}
\end{center}

\begin{enumerate}

\item Fixer l'axe de rotation $\Delta$ au milieu de la barre, puis, essayer de faire tourner cette derni�re :

\begin{itemize}
 \item en appuyant dessus.
 \item en la poussant ou en la tirant dans le sens de la longueur.
 \item en la poussant ou en la tirant perpendiculairement � elle m�me.
\end{itemize}

\item Faites un sch�ma du montage en faisant figurer la barre�; son axe de rotation et la force appliqu�e.

\item En comparant la direction de la force appliqu�e avec la direction de l'axe de rotation, conclure sur la condition de mise en rotation d'un solide autour d'un axe fixe.


\end{enumerate}


\section{Moment d'une force - R�gle du bras de levier}


\begin{enumerate}

\item Reproduire le sch�ma suivant en indiquant $F$ et $F'$.

%Faites le montage suivant�:



\begin{center}
\begin{pspicture}(-5,-2)(5,1.5)
\psframe(-5,-0.5)(5,0.5)
%\rput(0,0){$\otimes$} % contraire de odot
\pscircle(0,0){0.2}
\psline(-0.14,-0.14)(0.14,0.14)
\psline(0.14,-0.14)(-0.14,0.14)

\rput(0.5,0){$\Delta$}


\pscircle(-4,0){0.2} % gauche
\rput(-2,1.25){$d$}
\psline{<->}(-4,1)(0,1)
\psline(-4,0)(-4,-2)
\psframe(-4.25,-2)(-3.75,-2.5)
\rput(-4,-2.75){$m = 20~g$}


\pscircle(3,0){0.2} % droite
\rput(1.25,1.25){$d'$}
\psline{<->}(3,1)(0,1)
\psline(3,0)(3,-2)
\psframe(2.75,-2)(3.25,-2.5)
\rput(3,-2.75){$m'$}

\end{pspicture}
\end{center}


\donnees{
\item $g = 10~N.kg^{-1}$
}


\medskip


\item Calculer la force $F$ exerc�e par la masse $m$ sur la demie barre gauche
\item Mesurer la distance $d$
\item Calculer alors le produit $F \cdot d$. Quelle est son unit�?


\item En laissant la masse $m$ � l'extr�mit� gauche�; suspendez sur les diff�rentes positions � une distance $d$ de l'axe de rotation une masse $m$ de telle sorte que la barre soit en �quilibre.

\item Dresser alors un tableau de r�sultats�exp�rimentaux :


\begin{arraydata}{3}
\hline
masse $m'$                           &   &   &   \\ \hline
Force $F'$ exerc�e par la masse $m'$ &   &   &   \\ \hline
$d'$                                 &   &   &   \\ \hline
$F' \cdot d'$                        &   &   &   \\ \hline
\end{arraydata}


\item Que pouvez vous dire du produit $F' \cdot d'$�?




On appelle moment de la force F par rapport � l'axe $\Delta$, le produit $M_{\Delta}(F) =  \troufixe{1.5cm}$ exprim� en $N.m$.

\item En choisissant le sens des aiguilles d'une montre comme sens de rotation positif, si la force $F$ tend � faire tourner le solide dans ce sens, alors son moment sera positif, si c'est le sens inverse son moment sera n�gatif.

En raisonnant sur la barre, et sur les moments des forces qui s'appliquent dessus, donner la condition d'�quilibre d'un solide soumis � un couple de moments.

\end{enumerate}


\newpage

\null

