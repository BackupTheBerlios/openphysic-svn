\cours{Champ magn\'etique}

\Section{Qu'est-ce qu'un aimant}
Les aimants sont des objets capables d'attirer des morceaux de fer, de nickel, ainsi que leurs alliages. Il existe des interactions \`a distance entre quelques m\'etaux et certains mat\'eriaux dits magn\'etiques (naturellement aimant\'es ou susceptibles de s'aimanter sous l'effet d'un aimant).

\Section{Propri\'et\'e d'un aimant}


Chaque aimant poss\`ede deux p\^oles diff\'erents. Ils sont d\'esign\'es par nord et sud. Il existe des forces de r\'epulsions et des forces d'attraction.

\begin{itemize}
\item deux p\^oles de m\^eme nom se repoussent
\item deux p\^oles de noms contraires s'attirent
\end{itemize}

\Section{Champ magn\'etique d'un aimant}
\Subsection{Notion de champ magn\'etique}
On appelle champ magn\'etique la r\'egion de l'espace qui entoure un aimant.
(revoir def)

\Subsection{Le vecteur champ magn\'etique}
Une aiguille aimant\'ee, plac\'ee dans un champ magn\'etique d'un aimant s'oriente de facon bien d\'etermin\'ee.




Le champ magn\'etique est repr\'esente par le vecteur $\vect{B}$ : il a pour caract\'eristiques :

\begin{itemize}
\item Direction : celle que prendrait une petite aiguille aimant\'ee mobile placee en ce point
\item Sens : Sud vers le Nord de cette aiguille aimant\'ee
\item Intensit\'e : la valeur du champ magn\'etique au point consid\'er\'e. Elle s'exprime en Tesla $T$.\\
Ce nom vient de l'ingenieur croate Nikola \textsc{Tesla} (1857-1943)
\end{itemize}

\Subsection{Les lignes de champ}
On appele ligne de champ magn\'etique, une ligne qui en chacun de ces points est tangente au vecteur champ magn\'etique $\vect{B}$.

\Subsubsection{Spectre magn\'etique d'un aimant droit}

\Subsubsection{Propri\'etes des lignes de champ magn\'etique}

\begin{itemize}
\item \`A l'ext\'erieur d'un aimant, les lignes de champ sortent par le p\^ole Nord et entrent par le p\^ole Sud de l'aimant. 
\item Dans le cas d'un aimant en U, le vecteur champ magn\'etique garde \`a l'interieur :
 \begin{itemize}
  \item la m\^eme valeur
  \item la m\^eme direction
  \item le m\^eme sens
 \end{itemize}
\end{itemize}

On dit qu'il est uniforme en tout point de cette r\'egion.



\Section{Spectres magn\'etiques de quelques circuits parcourus par un courant}

Tout conducteur m\'etallique, parcouru par un courant, cr\'ee un champ magn\'etique.

\Subsection{Spectre magn\'etique d'une bobine parcourue par un courant}


On constate que les lignes de champ sont ferm\'ees

\Subsection{Spectre magn\'etique cr\'e\'e par un courant rectiligne}

Les lignes de champ sont concentriques.


\Subsection{Spectre magn\'etique d'un sol\'enoide}

\`A l'int\'erieur d'un solenoide, le champ magn\'etique est uniforme et parall\`ele \`a l'axe du solenoide.


\Section{Champ magn\'etique terrestre}

En premi\`ere approximation, le champ magn\'etique terrestre est similaire \`a celui que cr\'eerait un gigantesque aimant droit inclin\'e d'environ $11,5°$ par rapport \`a l'axe de rotation terrestre.


La direction donn\'ee par une boussoule ne coincide pas tout \`a fait avec la direction Nord-Sud g\'eographique. L'\'ecart angulaire entre ces deux droite est variable (selon le lieu et selon le temps). C'est la d\'eclinaison.



Elle est d'environ xxxxxxxx en France. La valeur du champ magn\'etque terrestre varie selon le lieu. Elle est d'environ $30 \mu T$ \`a l'\'equateur magn\'etique, de $60 \mu T$ aux p\^oles, $20 \mu T$ en France.




\Section{Champ magn\'etique cr\'e\'e par un courant \'electrique}

\Section{Champ magn\'etique cr\'e\'e par un solenoide}
Un solenoide est une bobine longue (la longueur est grande par rapport \`a son diam\`etre). D'apr\`es l'exp\'erience on peut dire qu'\`a l'int\'erieur d'un solenoide, le champ magn\'etique est uniforme et parall\`ele \`a l'axe du solenoide. Le sens du champ peut-\^etre donn\'e par la r\`egle du bonhomme d'Amp\`ere. L'extr\'emit\'e de la bobine par laquelle sort le vecteur champ est l'extr\'emit\'e Nord de la bobine.
La valeur du champ magn\'etique du solenoide peut \^etre calcul\'ee.

\[B = \mu_0 \frac{N I}{l}\]

\begin{unites}
\item $B$ : champ magn\'etique en Tesla ($T$)
\item $I$ : intensit\'e du courant dans le solenoide e($A$)
\item $N$ : nombre de spires
\item $l$ : longueur du solenoide
\item $\mu_0$ : perm\'eabilit\'e du vide (constante valant $4 \pi . 10^{-7}$)
\end{unites}
