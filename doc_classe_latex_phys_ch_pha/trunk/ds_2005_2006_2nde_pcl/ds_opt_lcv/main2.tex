\ds{Devoir Surveill� }{
 \item Optique - lentilles minces
}

\nomprenomclasse
\notationfinale{20}{5cm}


%\begin{exercice}{Titre de l'exercice}
%
%\end{exercice}



\renewcommand{\reproduire}{
%\emph{Questions de cours : Optique - lentilles minces}\\
\begin{enumerate}
\item Rappelez le symbole d'une lentille mince convergente et celui d'une
lentille mince divergente.
\item Placez les deux foyers $F$ et $F'$ et le centre optique $O$.
\item Dans le cas d'une lentille mince convergente, repr�sentez les
trois rayons particuliers traversant la lentille.
\item M�me question pour une lentille mince divergente.
\item Rappelez la formule de conjugaison des lentilles minces.
\item Rappelez les deux formules de grandissement.
\item D�finissez la vergence d'une lentille et pr�cisez les unit�s.
\end{enumerate}
}


\vressort{1}

\begin{large}

\begin{exercice}{Questions de cours}
\reproduire
\end{exercice}


\vressort{1}


\begin{exercice}{\\}
On place un objet lumineux plan $AB$ de $1~cm$ de hauteur, � $6~cm$ en
avant d'une lentille $L$ convergente de centre optique $O$ et de
distance focale $f'=4~cm$.

\begin{enumerate}
\item D�terminez la vergence de la lentille.
\item Faites un sch�ma � l'�chelle $1$, trouvez la position de
  l'image $A'B'$ ainsi que sa taille $\ma{A'B'}$.
\item Donnez les caract�ristiques de l'image $A'B'$ (r�elle/virtuelle,
  m�me sens que l'objet/renvers�e, agrandie/r�tr�cie)
\item D�terminez par le calcul la position de l'image $A'B'$.
\item D�terminez par le calcul la grandeur de l'image $A'B'$.
\end{enumerate}
\end{exercice}


\vressort{1}

\begin{exercice}{\\}
Une lentille convergente a une distance focale $f' = 10~cm$. Un objet de $1~cm$ de haut se trouve � $5~cm$ en avant de la lentille.

\begin{enumerate}
\item Calculez la vergence de la lentille.
\item Notez sur un sch�ma � l'�chelle $1$ ses foyers $F$ et $F'$.
\item D�terminez par construction la position, la nature et la taille
  de l'image.
\item  D�terminez par calcul la position, la nature et la taille
  de l'image.
\end{enumerate}

\end{exercice}

\end{large}

\vressort{1}

\newpage

\null

