\ds{Devoir Surveill� }{
 \item Optique - lentilles minces
}

\nomprenomclasse
\notationfinale{20}{5cm}


%\begin{exercice}{Titre de l'exercice}
%
%\end{exercice}



\renewcommand{\reproduire}{
%\emph{Questions de cours : Optique - lentilles minces}\\
\begin{enumerate}
\item Rappelez le symbole d'une lentille mince convergente et celui d'une
lentille mince divergente.
\item Placez les deux foyers $F$ et $F'$ et le centre optique $O$.
\item Dans le cas d'une lentille mince convergente, repr�sentez les
trois rayons particuliers traversant la lentille.
\item M�me question pour une lentille mince divergente.
\item Rappelez la formule de conjugaison des lentilles minces.
\item Rappelez les deux formules de grandissement.
\item D�finissez la vergence d'une lentille et pr�cisez les unit�s.
\end{enumerate}
}


\begin{exercice}{Questions de cours}
\reproduire
\end{exercice}

\newpage

\null

