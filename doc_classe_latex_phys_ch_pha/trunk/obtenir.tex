 % Comment obtenir ce document ?


\vressort{1}

Ce document est r�alis� avec \LaTeX.

\section*{Obtenir une copie de ce document}
Vous pouvez obtenir une copie de ce document au format \verb+.pdf+
Acrobat Reader sur :
\begin{itemize}
\item \url{http://s.cls.free.fr/wikini/wakka.php?wiki=Enseignement}
\item
  \url{http://svn.berlios.de/wsvn/openphysic/doc_classe_latex_phys_ch_pha/trunk/main_dvips_suite_2.pdf}
\item \url{http://svn.berlios.de/wsvn/openphysic/doc_classe_latex_phys_ch_pha/trunk/main.pdf}
\end{itemize}


\section*{Obtenir les fichiers \LaTeX par Subversion pour les modifier}

Il est possible d'obtenir les fichiers \LaTeX � l'aide du gestionnaire
de version Subversion.

\subsection*{Acc�s web}
\url{http://svn.berlios.de/viewcvs/openphysic/doc_classe_latex_phys_ch_pha/trunk}

\subsection{Acc�s anonyme}
\url{svn checkout svn://svn.berlios.de/openphysic/doc_classe_latex_phys_ch_pha/trunk}

\subsection{Acc�s d�veloppeur}
\url{svn checkout svn+ssh://scls19fr@svn.berlios.de/svnroot/repos/openphysic/doc_classe_latex_phys_ch_pha/trunk}


\vressort{3}