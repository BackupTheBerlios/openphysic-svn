\tp{Caract�ristique d'une pile}

\nomprenomclasse

\begin{multicols}{2}


\objectifs{
\item Tracer la caract�ristique $U = f(I)$ d'une pile.
\item D�terminer sa force �lectromotrice $E$ et sa r�sistance interne $r$
}

\materiel{
\item pile $1,5~V$
\item 2 multim�tres
\item 1 interrupteur $K$
\item 1 rh�ostat $R_h \leq 100~\Omega$
\item 1 r�sistance de protection $R = 10~\Omega$
}

%\tiny
\section{Montage}

\begin{center}
\begin{pspicture}(0,0)(7,5)
%\psgrid[subgriddiv=1,griddots=10]
\pnode(0,4){B}
\pnode(7,4){A}
\battery[tensionlabel=$U$](B)(A){}
\circledipole[parallel,parallelarm=-1,parallelsep=0,labeloffset=0](2,4)(5,4){\large \textbf{V}}
\pnode(0,0.5){C}
\pnode(6,0.5){D}
\wire(C)(B)
\pnode(2,0.5){CD1}
\pnode(4,0.5){CD2}
\ammeter{C}{CD1}
\resistor[intensitylabel=$I$,dipoleconvention=receptor](CD1)(CD2){$R$}
\switch(CD2)(D){$K$}
\resistor(D)(6,3){$R_h$}
\psline(A)(7,1.75)
\psline{->}(7,1.75)(6.25,1.75)
\end{pspicture}
\end{center}


\end{multicols}

\section{Manipulations}
\small
\begin{enumerate}
\item Pr�ciser les bornes des appareils de mesure puis r�aliser le montage et demander au professeur de le v�rifier avant mise sous tension.
\item Ouvrir l'interrupteur $K$. Mesurer $U$.
\item Placer le rh�ostat sur sa plus grande valeur ${R_h}_{\mbox{max}}$ (plus grande longueur de conducteur).\\
Fermer l'interrupteur $K$.\\
Mesurer $U$ et $I$.\\
Renouveler ces 2 mesures pour des valeurs plus �lev�es de $I$ en diminuant $R_h$ (en diminuant la longueur de conducteur dans le rh�ostat).
\end{enumerate}


\begin{arraydata}{9}
\hline
$I$ ($mA$) &  0 & 10 & 20 & 30 & 40 & 50 & 60 & 70 & 80 \\ \hline
\rule[-0.3cm]{0cm}{0.7cm}
$U$ ($V$)  &    &    &    &    &    &    &    &    &    \\ \hline
\end{arraydata}


\section{R�sultats}
\begin{enumerate}
\item Tracer la caract�ristique $U = f(I)$ de la pile.
\item On obtient une droite d�croissante ne passant pas par l'origine (fonction affine)\\
On v�rifie la \emph{loi d'Ohm pour un g�n�rateur} \fbox{$U = E -r I$}.

$E$ est \trouligne{l'ordonn�e � l'origine de la droite.}

$r$ est \trouligne{le coefficient directeur de la droite.}


Lorsque $I = 0~A$, $U = \trou{\hspace{0.5cm}E}$ : c'est la \trou{force �lectromotrice de la pile}, not�e en abr�g� \trou{f.e.m.}. Elle vaut $E = \makebox[20mm][l]{\dotfill}~V$

\medskip

D�terminer le coefficient directeur de la droite :

\trouformule{ a = \frac{U_2 - U_1}{I_2 - I_1}}

En d�duire la r�sistance interne de la pile : $r = \troufixe{2cm}~\Omega$


\end{enumerate}

%\newpage


\normalsize

\pagepapiermilli


% \begin{changemargin}{-1}{-1}

% \begin{center}
% \scalebox{1.414}{
% \begin{pspicture}(0,0)(11,19)
% \psmilli
% \end{pspicture}
% }
% \end{center}

% \end{changemargin}

