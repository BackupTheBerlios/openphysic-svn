

\classe{Seconde\ \\
option Physique et Chimie\ \\
de Laboratoire\ \\
Partie Physique}{Seconde option PCL - Partie Physique}

\chapitre{Courant et tension �lectrique}
\ds{Devoir Surveill�}{
%
}

\nomprenomclasse

\setcounter{numexercice}{0}

%\renewcommand{\tabularx}[1]{>{\centering}m{#1}} 

%\newcommand{\tabularxc}[1]{\tabularx{>{\centering}m{#1}}}

\vressort{3}

\begin{exercice}{Connaissance sur l'atome}%\\
\begin{enumerate}
\item De quoi est compos� un atome ?
\item Que signifie les lettres $A$, $Z$ et $X$ dans la repr�sentation \noyau{X}{Z}{A} ?
\item Comment trouve-t-on le nombre de neutrons d'un atome de l'�l�ment pr�c�dent.
\item Si un atome a $5$ protons, combien-a-t-il d'�lectrons ? Pourquoi ?
\item Qu'est-ce qui caract�rise un �l�ment chimique ?
\item Qu'est-ce qu'un isotope ?
\end{enumerate}
\end{exercice}



\vressort{3}



\begin{exercice}{Composition des atomes}\\
En vous aidant du tableau p�riodique des �l�ments,
compl�ter le tableau suivant :

\medskip

\noindent
%\begin{tabularx}{\textwidth}{|>{\centering}X|>{\centering}X|>{\centering}X|>{\centering}X|>{\centering}X|}
% \begin{tabularx}{\linewidth}{|X|X|X|X|X|}
% \hline
% \emph{nom}       & \emph{symbole}  & \emph{protons} & \emph{neutrons}
% & \emph{nucl�ons} \tbnl
% carbone   & \noyau{C}{6}{14}   &         &          & \rule[-0.5cm]{0cm}{1cm}         \tbnl
% fluor     & \noyau{F}{9}{19}   &         &          & \rule[-0.5cm]{0cm}{1cm}         \tbnl
% sodium    & \noyau{Na}{11}{23} &         &          & \rule[-0.5cm]{0cm}{1cm}         \tbnl
% oxyg�ne   & \noyau{O}{8}{16}   &         &          & \rule[-0.5cm]{0cm}{1cm}         \tbnl
% hydrog�ne &          &         & 0        & \rule[-0.5cm]{0cm}{1cm}         \tbnl
%           & \noyau{Cl}{17}{35} &         &          &  \rule[-0.5cm]{0cm}{1cm}        \tbnl
%           &          & 8       & \rule[-0.5cm]{0cm}{1cm}         & 16       \tbnl
% \end{tabularx}



\begin{tabularx}{\linewidth}{|>{\mystrut}X|X|X|X|X|}
\hline
% multicolumn pour faire dispara�tre le \mystrut
\multicolumn{1}{|X|}{\emph{nom}} & \emph{symbole}  &
\emph{protons} & \emph{neutrons} & \emph{nucl�ons} \tbnl
carbone   & \noyau{C}{6}{14}   &   &   &    \tbnl
fluor     & \noyau{F}{9}{19}   &   &   &    \tbnl
sodium    & \noyau{Na}{11}{23} &   &   &    \tbnl
oxyg�ne   & \noyau{O}{8}{16}   &   &   &    \tbnl
hydrog�ne &                    &   & 0 &    \tbnl
          & \noyau{Cl}{17}{35} &   &   &    \tbnl
          &                    & 8 &   & 16 \tbnl
\end{tabularx}


\end{exercice}


\vressort{3}


\begin{exercice}{Masse d'un atome de carbone 12}\\
Soit le carbone $12$ not� \noyau{C}{6}{12}.
\begin{enumerate}
\item L'�l�ment carbone peut-il avoir $5$ protons ? Pourquoi ?
\item Calculer la masse du noyau d'un atome de carbone $12$
sachant que la masse d'un nucl�on est $m_n = 1,67.10^{-27}~kg$
\item Calculer la masse des �lectrons de l'atome de carbone 12
sachant que la masse d'un �lectron vaut $m_e = 9,1.10^{-31}~kg$
\item Comparer la masse des �lectrons de l'atome � la masse du noyau.
Que concluez-vous ?
\item En d�duire, sans nouveau calcul, la masse de l'atome de carbone
  $12$.
\end{enumerate}
\end{exercice}

\newpage

\vressort{1}

\begin{exercice}{Couches �lectroniques}\\
Dans l'�tat le plus stable de l'atome, appel� �tat fondamental,
les �lectrons occupent successivement les couches,
en commen�ant par celles qui sont les plus proches du noyau : 
d'abord $K$ puis $L$ puis $M$.

Lorsqu'une couche est pleine, ou encore satur�e, on passe � la suivante.

La derni�re couche occup�e est appel�e couche externe.\\
Toutes les autres sont appel�es couches internes.

\medskip

\noindent
%\begin{tabularx}{\textwidth}{|>{\centering}X|>{\centering}X|>{\centering}X|>{\centering}X|}
\begin{tabularx}{\textwidth}{|>{\mystrut}X|X|X|X|}
\hline
\multicolumn{1}{|X|}{\emph{Symbole de la couche}}       & $K$ & $L$ & $M$  \tbnl
\emph{Nombre maximal d'�lectrons} & $2$ & $8$ & $18$ \tbnl
\end{tabularx}

\medskip

Ainsi, par exemple, l'atome de chlore ($Z=17$) a la configuration �lectronique :
$(K)^2(L)^8(M)^{7}$.

\begin{enumerate}
\item Indiquez le nombre d'�lectrons et donnez la configuration des atomes suivants :
  \begin{enumerate}
  \item \noyau{H}{1}{}
  \item \noyau{O}{8}{}
  \item \noyau{C}{6}{}
  \item \noyau{Ne}{10}{}
  \end{enumerate}
\item Parmi les ions ci-desssous, pr�cisez s'il s'agit d'anions ou de cations.
Indiquez le nombre d'�lectrons et donnez la configuration �lectronique
des ions suivants :
  \begin{enumerate}
  \item $Be^{2+}$ ($Z=4$)
  \item $Al^{3+}$ ($Z=13$)
  \item $O^{2-}$ ($Z=8$)
  \item $F^{-}$ ($Z=9$)
  \end{enumerate}
\end{enumerate}

\end{exercice}


\vressort{5}


\begin{exercice}{Charge d'un atome de Zinc}%\\
\begin{enumerate}
\item Combien de protons l'atome de zinc \noyau{Zn}{30}{65} contient-il ?
\item Combien d'�lectrons comporte-t-il ?
\item Calculer la charge totale des protons
sachant qu'un proton a pour charge $e = 1,6.10^{-19}~C$.
\item Calculer la charge totale des �lectrons
sachant qu'un �lectron a pour charge $-e = -1,6.10^{-19}~C$.
\item En d�duire la charge de l'atome de Zinc.
\item Ce r�sultat est-il identique pour tous les atomes ?
\item A l'issue d'une r�action dite d'oxydation, un atome de zinc $Zn$
  se transforme en un ion $Zn^{2+}$.
  \begin{enumerate}
  \item Donnez l'�quation de cette r�action
  (en faisant intervenir un ou plusieurs �lectrons not�s $e^-$).
  \item Indiquez la charge (en coulomb $C$) de cet ion.
  \end{enumerate}
\end{enumerate}
\end{exercice}

\vressort{3} % cours elec circuit �lectrique

\ds{Devoir Surveill�}{
%
}

\nomprenomclasse

\setcounter{numexercice}{0}

%\renewcommand{\tabularx}[1]{>{\centering}m{#1}} 

%\newcommand{\tabularxc}[1]{\tabularx{>{\centering}m{#1}}}

\vressort{3}

\begin{exercice}{Connaissance sur l'atome}%\\
\begin{enumerate}
\item De quoi est compos� un atome ?
\item Que signifie les lettres $A$, $Z$ et $X$ dans la repr�sentation \noyau{X}{Z}{A} ?
\item Comment trouve-t-on le nombre de neutrons d'un atome de l'�l�ment pr�c�dent.
\item Si un atome a $5$ protons, combien-a-t-il d'�lectrons ? Pourquoi ?
\item Qu'est-ce qui caract�rise un �l�ment chimique ?
\item Qu'est-ce qu'un isotope ?
\end{enumerate}
\end{exercice}



\vressort{3}



\begin{exercice}{Composition des atomes}\\
En vous aidant du tableau p�riodique des �l�ments,
compl�ter le tableau suivant :

\medskip

\noindent
%\begin{tabularx}{\textwidth}{|>{\centering}X|>{\centering}X|>{\centering}X|>{\centering}X|>{\centering}X|}
% \begin{tabularx}{\linewidth}{|X|X|X|X|X|}
% \hline
% \emph{nom}       & \emph{symbole}  & \emph{protons} & \emph{neutrons}
% & \emph{nucl�ons} \tbnl
% carbone   & \noyau{C}{6}{14}   &         &          & \rule[-0.5cm]{0cm}{1cm}         \tbnl
% fluor     & \noyau{F}{9}{19}   &         &          & \rule[-0.5cm]{0cm}{1cm}         \tbnl
% sodium    & \noyau{Na}{11}{23} &         &          & \rule[-0.5cm]{0cm}{1cm}         \tbnl
% oxyg�ne   & \noyau{O}{8}{16}   &         &          & \rule[-0.5cm]{0cm}{1cm}         \tbnl
% hydrog�ne &          &         & 0        & \rule[-0.5cm]{0cm}{1cm}         \tbnl
%           & \noyau{Cl}{17}{35} &         &          &  \rule[-0.5cm]{0cm}{1cm}        \tbnl
%           &          & 8       & \rule[-0.5cm]{0cm}{1cm}         & 16       \tbnl
% \end{tabularx}



\begin{tabularx}{\linewidth}{|>{\mystrut}X|X|X|X|X|}
\hline
% multicolumn pour faire dispara�tre le \mystrut
\multicolumn{1}{|X|}{\emph{nom}} & \emph{symbole}  &
\emph{protons} & \emph{neutrons} & \emph{nucl�ons} \tbnl
carbone   & \noyau{C}{6}{14}   &   &   &    \tbnl
fluor     & \noyau{F}{9}{19}   &   &   &    \tbnl
sodium    & \noyau{Na}{11}{23} &   &   &    \tbnl
oxyg�ne   & \noyau{O}{8}{16}   &   &   &    \tbnl
hydrog�ne &                    &   & 0 &    \tbnl
          & \noyau{Cl}{17}{35} &   &   &    \tbnl
          &                    & 8 &   & 16 \tbnl
\end{tabularx}


\end{exercice}


\vressort{3}


\begin{exercice}{Masse d'un atome de carbone 12}\\
Soit le carbone $12$ not� \noyau{C}{6}{12}.
\begin{enumerate}
\item L'�l�ment carbone peut-il avoir $5$ protons ? Pourquoi ?
\item Calculer la masse du noyau d'un atome de carbone $12$
sachant que la masse d'un nucl�on est $m_n = 1,67.10^{-27}~kg$
\item Calculer la masse des �lectrons de l'atome de carbone 12
sachant que la masse d'un �lectron vaut $m_e = 9,1.10^{-31}~kg$
\item Comparer la masse des �lectrons de l'atome � la masse du noyau.
Que concluez-vous ?
\item En d�duire, sans nouveau calcul, la masse de l'atome de carbone
  $12$.
\end{enumerate}
\end{exercice}

\newpage

\vressort{1}

\begin{exercice}{Couches �lectroniques}\\
Dans l'�tat le plus stable de l'atome, appel� �tat fondamental,
les �lectrons occupent successivement les couches,
en commen�ant par celles qui sont les plus proches du noyau : 
d'abord $K$ puis $L$ puis $M$.

Lorsqu'une couche est pleine, ou encore satur�e, on passe � la suivante.

La derni�re couche occup�e est appel�e couche externe.\\
Toutes les autres sont appel�es couches internes.

\medskip

\noindent
%\begin{tabularx}{\textwidth}{|>{\centering}X|>{\centering}X|>{\centering}X|>{\centering}X|}
\begin{tabularx}{\textwidth}{|>{\mystrut}X|X|X|X|}
\hline
\multicolumn{1}{|X|}{\emph{Symbole de la couche}}       & $K$ & $L$ & $M$  \tbnl
\emph{Nombre maximal d'�lectrons} & $2$ & $8$ & $18$ \tbnl
\end{tabularx}

\medskip

Ainsi, par exemple, l'atome de chlore ($Z=17$) a la configuration �lectronique :
$(K)^2(L)^8(M)^{7}$.

\begin{enumerate}
\item Indiquez le nombre d'�lectrons et donnez la configuration des atomes suivants :
  \begin{enumerate}
  \item \noyau{H}{1}{}
  \item \noyau{O}{8}{}
  \item \noyau{C}{6}{}
  \item \noyau{Ne}{10}{}
  \end{enumerate}
\item Parmi les ions ci-desssous, pr�cisez s'il s'agit d'anions ou de cations.
Indiquez le nombre d'�lectrons et donnez la configuration �lectronique
des ions suivants :
  \begin{enumerate}
  \item $Be^{2+}$ ($Z=4$)
  \item $Al^{3+}$ ($Z=13$)
  \item $O^{2-}$ ($Z=8$)
  \item $F^{-}$ ($Z=9$)
  \end{enumerate}
\end{enumerate}

\end{exercice}


\vressort{5}


\begin{exercice}{Charge d'un atome de Zinc}%\\
\begin{enumerate}
\item Combien de protons l'atome de zinc \noyau{Zn}{30}{65} contient-il ?
\item Combien d'�lectrons comporte-t-il ?
\item Calculer la charge totale des protons
sachant qu'un proton a pour charge $e = 1,6.10^{-19}~C$.
\item Calculer la charge totale des �lectrons
sachant qu'un �lectron a pour charge $-e = -1,6.10^{-19}~C$.
\item En d�duire la charge de l'atome de Zinc.
\item Ce r�sultat est-il identique pour tous les atomes ?
\item A l'issue d'une r�action dite d'oxydation, un atome de zinc $Zn$
  se transforme en un ion $Zn^{2+}$.
  \begin{enumerate}
  \item Donnez l'�quation de cette r�action
  (en faisant intervenir un ou plusieurs �lectrons not�s $e^-$).
  \item Indiquez la charge (en coulomb $C$) de cet ion.
  \end{enumerate}
\end{enumerate}
\end{exercice}

\vressort{3} % tp mesure de tension et d'intensit�

\ds{Devoir Surveill�}{
%
}

\nomprenomclasse

\setcounter{numexercice}{0}

%\renewcommand{\tabularx}[1]{>{\centering}m{#1}} 

%\newcommand{\tabularxc}[1]{\tabularx{>{\centering}m{#1}}}

\vressort{3}

\begin{exercice}{Connaissance sur l'atome}%\\
\begin{enumerate}
\item De quoi est compos� un atome ?
\item Que signifie les lettres $A$, $Z$ et $X$ dans la repr�sentation \noyau{X}{Z}{A} ?
\item Comment trouve-t-on le nombre de neutrons d'un atome de l'�l�ment pr�c�dent.
\item Si un atome a $5$ protons, combien-a-t-il d'�lectrons ? Pourquoi ?
\item Qu'est-ce qui caract�rise un �l�ment chimique ?
\item Qu'est-ce qu'un isotope ?
\end{enumerate}
\end{exercice}



\vressort{3}



\begin{exercice}{Composition des atomes}\\
En vous aidant du tableau p�riodique des �l�ments,
compl�ter le tableau suivant :

\medskip

\noindent
%\begin{tabularx}{\textwidth}{|>{\centering}X|>{\centering}X|>{\centering}X|>{\centering}X|>{\centering}X|}
% \begin{tabularx}{\linewidth}{|X|X|X|X|X|}
% \hline
% \emph{nom}       & \emph{symbole}  & \emph{protons} & \emph{neutrons}
% & \emph{nucl�ons} \tbnl
% carbone   & \noyau{C}{6}{14}   &         &          & \rule[-0.5cm]{0cm}{1cm}         \tbnl
% fluor     & \noyau{F}{9}{19}   &         &          & \rule[-0.5cm]{0cm}{1cm}         \tbnl
% sodium    & \noyau{Na}{11}{23} &         &          & \rule[-0.5cm]{0cm}{1cm}         \tbnl
% oxyg�ne   & \noyau{O}{8}{16}   &         &          & \rule[-0.5cm]{0cm}{1cm}         \tbnl
% hydrog�ne &          &         & 0        & \rule[-0.5cm]{0cm}{1cm}         \tbnl
%           & \noyau{Cl}{17}{35} &         &          &  \rule[-0.5cm]{0cm}{1cm}        \tbnl
%           &          & 8       & \rule[-0.5cm]{0cm}{1cm}         & 16       \tbnl
% \end{tabularx}



\begin{tabularx}{\linewidth}{|>{\mystrut}X|X|X|X|X|}
\hline
% multicolumn pour faire dispara�tre le \mystrut
\multicolumn{1}{|X|}{\emph{nom}} & \emph{symbole}  &
\emph{protons} & \emph{neutrons} & \emph{nucl�ons} \tbnl
carbone   & \noyau{C}{6}{14}   &   &   &    \tbnl
fluor     & \noyau{F}{9}{19}   &   &   &    \tbnl
sodium    & \noyau{Na}{11}{23} &   &   &    \tbnl
oxyg�ne   & \noyau{O}{8}{16}   &   &   &    \tbnl
hydrog�ne &                    &   & 0 &    \tbnl
          & \noyau{Cl}{17}{35} &   &   &    \tbnl
          &                    & 8 &   & 16 \tbnl
\end{tabularx}


\end{exercice}


\vressort{3}


\begin{exercice}{Masse d'un atome de carbone 12}\\
Soit le carbone $12$ not� \noyau{C}{6}{12}.
\begin{enumerate}
\item L'�l�ment carbone peut-il avoir $5$ protons ? Pourquoi ?
\item Calculer la masse du noyau d'un atome de carbone $12$
sachant que la masse d'un nucl�on est $m_n = 1,67.10^{-27}~kg$
\item Calculer la masse des �lectrons de l'atome de carbone 12
sachant que la masse d'un �lectron vaut $m_e = 9,1.10^{-31}~kg$
\item Comparer la masse des �lectrons de l'atome � la masse du noyau.
Que concluez-vous ?
\item En d�duire, sans nouveau calcul, la masse de l'atome de carbone
  $12$.
\end{enumerate}
\end{exercice}

\newpage

\vressort{1}

\begin{exercice}{Couches �lectroniques}\\
Dans l'�tat le plus stable de l'atome, appel� �tat fondamental,
les �lectrons occupent successivement les couches,
en commen�ant par celles qui sont les plus proches du noyau : 
d'abord $K$ puis $L$ puis $M$.

Lorsqu'une couche est pleine, ou encore satur�e, on passe � la suivante.

La derni�re couche occup�e est appel�e couche externe.\\
Toutes les autres sont appel�es couches internes.

\medskip

\noindent
%\begin{tabularx}{\textwidth}{|>{\centering}X|>{\centering}X|>{\centering}X|>{\centering}X|}
\begin{tabularx}{\textwidth}{|>{\mystrut}X|X|X|X|}
\hline
\multicolumn{1}{|X|}{\emph{Symbole de la couche}}       & $K$ & $L$ & $M$  \tbnl
\emph{Nombre maximal d'�lectrons} & $2$ & $8$ & $18$ \tbnl
\end{tabularx}

\medskip

Ainsi, par exemple, l'atome de chlore ($Z=17$) a la configuration �lectronique :
$(K)^2(L)^8(M)^{7}$.

\begin{enumerate}
\item Indiquez le nombre d'�lectrons et donnez la configuration des atomes suivants :
  \begin{enumerate}
  \item \noyau{H}{1}{}
  \item \noyau{O}{8}{}
  \item \noyau{C}{6}{}
  \item \noyau{Ne}{10}{}
  \end{enumerate}
\item Parmi les ions ci-desssous, pr�cisez s'il s'agit d'anions ou de cations.
Indiquez le nombre d'�lectrons et donnez la configuration �lectronique
des ions suivants :
  \begin{enumerate}
  \item $Be^{2+}$ ($Z=4$)
  \item $Al^{3+}$ ($Z=13$)
  \item $O^{2-}$ ($Z=8$)
  \item $F^{-}$ ($Z=9$)
  \end{enumerate}
\end{enumerate}

\end{exercice}


\vressort{5}


\begin{exercice}{Charge d'un atome de Zinc}%\\
\begin{enumerate}
\item Combien de protons l'atome de zinc \noyau{Zn}{30}{65} contient-il ?
\item Combien d'�lectrons comporte-t-il ?
\item Calculer la charge totale des protons
sachant qu'un proton a pour charge $e = 1,6.10^{-19}~C$.
\item Calculer la charge totale des �lectrons
sachant qu'un �lectron a pour charge $-e = -1,6.10^{-19}~C$.
\item En d�duire la charge de l'atome de Zinc.
\item Ce r�sultat est-il identique pour tous les atomes ?
\item A l'issue d'une r�action dite d'oxydation, un atome de zinc $Zn$
  se transforme en un ion $Zn^{2+}$.
  \begin{enumerate}
  \item Donnez l'�quation de cette r�action
  (en faisant intervenir un ou plusieurs �lectrons not�s $e^-$).
  \item Indiquez la charge (en coulomb $C$) de cet ion.
  \end{enumerate}
\end{enumerate}
\end{exercice}

\vressort{3} % document fiche
    % m�thode montage
    % utilisation d'un multim�tre, ...

\ds{Devoir Surveill�}{
%
}

\nomprenomclasse

\setcounter{numexercice}{0}

%\renewcommand{\tabularx}[1]{>{\centering}m{#1}} 

%\newcommand{\tabularxc}[1]{\tabularx{>{\centering}m{#1}}}

\vressort{3}

\begin{exercice}{Connaissance sur l'atome}%\\
\begin{enumerate}
\item De quoi est compos� un atome ?
\item Que signifie les lettres $A$, $Z$ et $X$ dans la repr�sentation \noyau{X}{Z}{A} ?
\item Comment trouve-t-on le nombre de neutrons d'un atome de l'�l�ment pr�c�dent.
\item Si un atome a $5$ protons, combien-a-t-il d'�lectrons ? Pourquoi ?
\item Qu'est-ce qui caract�rise un �l�ment chimique ?
\item Qu'est-ce qu'un isotope ?
\end{enumerate}
\end{exercice}



\vressort{3}



\begin{exercice}{Composition des atomes}\\
En vous aidant du tableau p�riodique des �l�ments,
compl�ter le tableau suivant :

\medskip

\noindent
%\begin{tabularx}{\textwidth}{|>{\centering}X|>{\centering}X|>{\centering}X|>{\centering}X|>{\centering}X|}
% \begin{tabularx}{\linewidth}{|X|X|X|X|X|}
% \hline
% \emph{nom}       & \emph{symbole}  & \emph{protons} & \emph{neutrons}
% & \emph{nucl�ons} \tbnl
% carbone   & \noyau{C}{6}{14}   &         &          & \rule[-0.5cm]{0cm}{1cm}         \tbnl
% fluor     & \noyau{F}{9}{19}   &         &          & \rule[-0.5cm]{0cm}{1cm}         \tbnl
% sodium    & \noyau{Na}{11}{23} &         &          & \rule[-0.5cm]{0cm}{1cm}         \tbnl
% oxyg�ne   & \noyau{O}{8}{16}   &         &          & \rule[-0.5cm]{0cm}{1cm}         \tbnl
% hydrog�ne &          &         & 0        & \rule[-0.5cm]{0cm}{1cm}         \tbnl
%           & \noyau{Cl}{17}{35} &         &          &  \rule[-0.5cm]{0cm}{1cm}        \tbnl
%           &          & 8       & \rule[-0.5cm]{0cm}{1cm}         & 16       \tbnl
% \end{tabularx}



\begin{tabularx}{\linewidth}{|>{\mystrut}X|X|X|X|X|}
\hline
% multicolumn pour faire dispara�tre le \mystrut
\multicolumn{1}{|X|}{\emph{nom}} & \emph{symbole}  &
\emph{protons} & \emph{neutrons} & \emph{nucl�ons} \tbnl
carbone   & \noyau{C}{6}{14}   &   &   &    \tbnl
fluor     & \noyau{F}{9}{19}   &   &   &    \tbnl
sodium    & \noyau{Na}{11}{23} &   &   &    \tbnl
oxyg�ne   & \noyau{O}{8}{16}   &   &   &    \tbnl
hydrog�ne &                    &   & 0 &    \tbnl
          & \noyau{Cl}{17}{35} &   &   &    \tbnl
          &                    & 8 &   & 16 \tbnl
\end{tabularx}


\end{exercice}


\vressort{3}


\begin{exercice}{Masse d'un atome de carbone 12}\\
Soit le carbone $12$ not� \noyau{C}{6}{12}.
\begin{enumerate}
\item L'�l�ment carbone peut-il avoir $5$ protons ? Pourquoi ?
\item Calculer la masse du noyau d'un atome de carbone $12$
sachant que la masse d'un nucl�on est $m_n = 1,67.10^{-27}~kg$
\item Calculer la masse des �lectrons de l'atome de carbone 12
sachant que la masse d'un �lectron vaut $m_e = 9,1.10^{-31}~kg$
\item Comparer la masse des �lectrons de l'atome � la masse du noyau.
Que concluez-vous ?
\item En d�duire, sans nouveau calcul, la masse de l'atome de carbone
  $12$.
\end{enumerate}
\end{exercice}

\newpage

\vressort{1}

\begin{exercice}{Couches �lectroniques}\\
Dans l'�tat le plus stable de l'atome, appel� �tat fondamental,
les �lectrons occupent successivement les couches,
en commen�ant par celles qui sont les plus proches du noyau : 
d'abord $K$ puis $L$ puis $M$.

Lorsqu'une couche est pleine, ou encore satur�e, on passe � la suivante.

La derni�re couche occup�e est appel�e couche externe.\\
Toutes les autres sont appel�es couches internes.

\medskip

\noindent
%\begin{tabularx}{\textwidth}{|>{\centering}X|>{\centering}X|>{\centering}X|>{\centering}X|}
\begin{tabularx}{\textwidth}{|>{\mystrut}X|X|X|X|}
\hline
\multicolumn{1}{|X|}{\emph{Symbole de la couche}}       & $K$ & $L$ & $M$  \tbnl
\emph{Nombre maximal d'�lectrons} & $2$ & $8$ & $18$ \tbnl
\end{tabularx}

\medskip

Ainsi, par exemple, l'atome de chlore ($Z=17$) a la configuration �lectronique :
$(K)^2(L)^8(M)^{7}$.

\begin{enumerate}
\item Indiquez le nombre d'�lectrons et donnez la configuration des atomes suivants :
  \begin{enumerate}
  \item \noyau{H}{1}{}
  \item \noyau{O}{8}{}
  \item \noyau{C}{6}{}
  \item \noyau{Ne}{10}{}
  \end{enumerate}
\item Parmi les ions ci-desssous, pr�cisez s'il s'agit d'anions ou de cations.
Indiquez le nombre d'�lectrons et donnez la configuration �lectronique
des ions suivants :
  \begin{enumerate}
  \item $Be^{2+}$ ($Z=4$)
  \item $Al^{3+}$ ($Z=13$)
  \item $O^{2-}$ ($Z=8$)
  \item $F^{-}$ ($Z=9$)
  \end{enumerate}
\end{enumerate}

\end{exercice}


\vressort{5}


\begin{exercice}{Charge d'un atome de Zinc}%\\
\begin{enumerate}
\item Combien de protons l'atome de zinc \noyau{Zn}{30}{65} contient-il ?
\item Combien d'�lectrons comporte-t-il ?
\item Calculer la charge totale des protons
sachant qu'un proton a pour charge $e = 1,6.10^{-19}~C$.
\item Calculer la charge totale des �lectrons
sachant qu'un �lectron a pour charge $-e = -1,6.10^{-19}~C$.
\item En d�duire la charge de l'atome de Zinc.
\item Ce r�sultat est-il identique pour tous les atomes ?
\item A l'issue d'une r�action dite d'oxydation, un atome de zinc $Zn$
  se transforme en un ion $Zn^{2+}$.
  \begin{enumerate}
  \item Donnez l'�quation de cette r�action
  (en faisant intervenir un ou plusieurs �lectrons not�s $e^-$).
  \item Indiquez la charge (en coulomb $C$) de cet ion.
  \end{enumerate}
\end{enumerate}
\end{exercice}

\vressort{3} % tp loi d'ohm

\ds{Devoir Surveill�}{
%
}

\nomprenomclasse

\setcounter{numexercice}{0}

%\renewcommand{\tabularx}[1]{>{\centering}m{#1}} 

%\newcommand{\tabularxc}[1]{\tabularx{>{\centering}m{#1}}}

\vressort{3}

\begin{exercice}{Connaissance sur l'atome}%\\
\begin{enumerate}
\item De quoi est compos� un atome ?
\item Que signifie les lettres $A$, $Z$ et $X$ dans la repr�sentation \noyau{X}{Z}{A} ?
\item Comment trouve-t-on le nombre de neutrons d'un atome de l'�l�ment pr�c�dent.
\item Si un atome a $5$ protons, combien-a-t-il d'�lectrons ? Pourquoi ?
\item Qu'est-ce qui caract�rise un �l�ment chimique ?
\item Qu'est-ce qu'un isotope ?
\end{enumerate}
\end{exercice}



\vressort{3}



\begin{exercice}{Composition des atomes}\\
En vous aidant du tableau p�riodique des �l�ments,
compl�ter le tableau suivant :

\medskip

\noindent
%\begin{tabularx}{\textwidth}{|>{\centering}X|>{\centering}X|>{\centering}X|>{\centering}X|>{\centering}X|}
% \begin{tabularx}{\linewidth}{|X|X|X|X|X|}
% \hline
% \emph{nom}       & \emph{symbole}  & \emph{protons} & \emph{neutrons}
% & \emph{nucl�ons} \tbnl
% carbone   & \noyau{C}{6}{14}   &         &          & \rule[-0.5cm]{0cm}{1cm}         \tbnl
% fluor     & \noyau{F}{9}{19}   &         &          & \rule[-0.5cm]{0cm}{1cm}         \tbnl
% sodium    & \noyau{Na}{11}{23} &         &          & \rule[-0.5cm]{0cm}{1cm}         \tbnl
% oxyg�ne   & \noyau{O}{8}{16}   &         &          & \rule[-0.5cm]{0cm}{1cm}         \tbnl
% hydrog�ne &          &         & 0        & \rule[-0.5cm]{0cm}{1cm}         \tbnl
%           & \noyau{Cl}{17}{35} &         &          &  \rule[-0.5cm]{0cm}{1cm}        \tbnl
%           &          & 8       & \rule[-0.5cm]{0cm}{1cm}         & 16       \tbnl
% \end{tabularx}



\begin{tabularx}{\linewidth}{|>{\mystrut}X|X|X|X|X|}
\hline
% multicolumn pour faire dispara�tre le \mystrut
\multicolumn{1}{|X|}{\emph{nom}} & \emph{symbole}  &
\emph{protons} & \emph{neutrons} & \emph{nucl�ons} \tbnl
carbone   & \noyau{C}{6}{14}   &   &   &    \tbnl
fluor     & \noyau{F}{9}{19}   &   &   &    \tbnl
sodium    & \noyau{Na}{11}{23} &   &   &    \tbnl
oxyg�ne   & \noyau{O}{8}{16}   &   &   &    \tbnl
hydrog�ne &                    &   & 0 &    \tbnl
          & \noyau{Cl}{17}{35} &   &   &    \tbnl
          &                    & 8 &   & 16 \tbnl
\end{tabularx}


\end{exercice}


\vressort{3}


\begin{exercice}{Masse d'un atome de carbone 12}\\
Soit le carbone $12$ not� \noyau{C}{6}{12}.
\begin{enumerate}
\item L'�l�ment carbone peut-il avoir $5$ protons ? Pourquoi ?
\item Calculer la masse du noyau d'un atome de carbone $12$
sachant que la masse d'un nucl�on est $m_n = 1,67.10^{-27}~kg$
\item Calculer la masse des �lectrons de l'atome de carbone 12
sachant que la masse d'un �lectron vaut $m_e = 9,1.10^{-31}~kg$
\item Comparer la masse des �lectrons de l'atome � la masse du noyau.
Que concluez-vous ?
\item En d�duire, sans nouveau calcul, la masse de l'atome de carbone
  $12$.
\end{enumerate}
\end{exercice}

\newpage

\vressort{1}

\begin{exercice}{Couches �lectroniques}\\
Dans l'�tat le plus stable de l'atome, appel� �tat fondamental,
les �lectrons occupent successivement les couches,
en commen�ant par celles qui sont les plus proches du noyau : 
d'abord $K$ puis $L$ puis $M$.

Lorsqu'une couche est pleine, ou encore satur�e, on passe � la suivante.

La derni�re couche occup�e est appel�e couche externe.\\
Toutes les autres sont appel�es couches internes.

\medskip

\noindent
%\begin{tabularx}{\textwidth}{|>{\centering}X|>{\centering}X|>{\centering}X|>{\centering}X|}
\begin{tabularx}{\textwidth}{|>{\mystrut}X|X|X|X|}
\hline
\multicolumn{1}{|X|}{\emph{Symbole de la couche}}       & $K$ & $L$ & $M$  \tbnl
\emph{Nombre maximal d'�lectrons} & $2$ & $8$ & $18$ \tbnl
\end{tabularx}

\medskip

Ainsi, par exemple, l'atome de chlore ($Z=17$) a la configuration �lectronique :
$(K)^2(L)^8(M)^{7}$.

\begin{enumerate}
\item Indiquez le nombre d'�lectrons et donnez la configuration des atomes suivants :
  \begin{enumerate}
  \item \noyau{H}{1}{}
  \item \noyau{O}{8}{}
  \item \noyau{C}{6}{}
  \item \noyau{Ne}{10}{}
  \end{enumerate}
\item Parmi les ions ci-desssous, pr�cisez s'il s'agit d'anions ou de cations.
Indiquez le nombre d'�lectrons et donnez la configuration �lectronique
des ions suivants :
  \begin{enumerate}
  \item $Be^{2+}$ ($Z=4$)
  \item $Al^{3+}$ ($Z=13$)
  \item $O^{2-}$ ($Z=8$)
  \item $F^{-}$ ($Z=9$)
  \end{enumerate}
\end{enumerate}

\end{exercice}


\vressort{5}


\begin{exercice}{Charge d'un atome de Zinc}%\\
\begin{enumerate}
\item Combien de protons l'atome de zinc \noyau{Zn}{30}{65} contient-il ?
\item Combien d'�lectrons comporte-t-il ?
\item Calculer la charge totale des protons
sachant qu'un proton a pour charge $e = 1,6.10^{-19}~C$.
\item Calculer la charge totale des �lectrons
sachant qu'un �lectron a pour charge $-e = -1,6.10^{-19}~C$.
\item En d�duire la charge de l'atome de Zinc.
\item Ce r�sultat est-il identique pour tous les atomes ?
\item A l'issue d'une r�action dite d'oxydation, un atome de zinc $Zn$
  se transforme en un ion $Zn^{2+}$.
  \begin{enumerate}
  \item Donnez l'�quation de cette r�action
  (en faisant intervenir un ou plusieurs �lectrons not�s $e^-$).
  \item Indiquez la charge (en coulomb $C$) de cet ion.
  \end{enumerate}
\end{enumerate}
\end{exercice}

\vressort{3} % tp associations de R

% association r s�rie (trac� U en fonction de I et montrer qu'on ajoute U)
% association r parall�le (trac� U en fonction de I et montrer qu'on ajoute I)

%\chapitre{G�n�rateurs et r�cepteurs}
%\ds{Devoir Surveill�}{
%
}

\nomprenomclasse

\setcounter{numexercice}{0}

%\renewcommand{\tabularx}[1]{>{\centering}m{#1}} 

%\newcommand{\tabularxc}[1]{\tabularx{>{\centering}m{#1}}}

\vressort{3}

\begin{exercice}{Connaissance sur l'atome}%\\
\begin{enumerate}
\item De quoi est compos� un atome ?
\item Que signifie les lettres $A$, $Z$ et $X$ dans la repr�sentation \noyau{X}{Z}{A} ?
\item Comment trouve-t-on le nombre de neutrons d'un atome de l'�l�ment pr�c�dent.
\item Si un atome a $5$ protons, combien-a-t-il d'�lectrons ? Pourquoi ?
\item Qu'est-ce qui caract�rise un �l�ment chimique ?
\item Qu'est-ce qu'un isotope ?
\end{enumerate}
\end{exercice}



\vressort{3}



\begin{exercice}{Composition des atomes}\\
En vous aidant du tableau p�riodique des �l�ments,
compl�ter le tableau suivant :

\medskip

\noindent
%\begin{tabularx}{\textwidth}{|>{\centering}X|>{\centering}X|>{\centering}X|>{\centering}X|>{\centering}X|}
% \begin{tabularx}{\linewidth}{|X|X|X|X|X|}
% \hline
% \emph{nom}       & \emph{symbole}  & \emph{protons} & \emph{neutrons}
% & \emph{nucl�ons} \tbnl
% carbone   & \noyau{C}{6}{14}   &         &          & \rule[-0.5cm]{0cm}{1cm}         \tbnl
% fluor     & \noyau{F}{9}{19}   &         &          & \rule[-0.5cm]{0cm}{1cm}         \tbnl
% sodium    & \noyau{Na}{11}{23} &         &          & \rule[-0.5cm]{0cm}{1cm}         \tbnl
% oxyg�ne   & \noyau{O}{8}{16}   &         &          & \rule[-0.5cm]{0cm}{1cm}         \tbnl
% hydrog�ne &          &         & 0        & \rule[-0.5cm]{0cm}{1cm}         \tbnl
%           & \noyau{Cl}{17}{35} &         &          &  \rule[-0.5cm]{0cm}{1cm}        \tbnl
%           &          & 8       & \rule[-0.5cm]{0cm}{1cm}         & 16       \tbnl
% \end{tabularx}



\begin{tabularx}{\linewidth}{|>{\mystrut}X|X|X|X|X|}
\hline
% multicolumn pour faire dispara�tre le \mystrut
\multicolumn{1}{|X|}{\emph{nom}} & \emph{symbole}  &
\emph{protons} & \emph{neutrons} & \emph{nucl�ons} \tbnl
carbone   & \noyau{C}{6}{14}   &   &   &    \tbnl
fluor     & \noyau{F}{9}{19}   &   &   &    \tbnl
sodium    & \noyau{Na}{11}{23} &   &   &    \tbnl
oxyg�ne   & \noyau{O}{8}{16}   &   &   &    \tbnl
hydrog�ne &                    &   & 0 &    \tbnl
          & \noyau{Cl}{17}{35} &   &   &    \tbnl
          &                    & 8 &   & 16 \tbnl
\end{tabularx}


\end{exercice}


\vressort{3}


\begin{exercice}{Masse d'un atome de carbone 12}\\
Soit le carbone $12$ not� \noyau{C}{6}{12}.
\begin{enumerate}
\item L'�l�ment carbone peut-il avoir $5$ protons ? Pourquoi ?
\item Calculer la masse du noyau d'un atome de carbone $12$
sachant que la masse d'un nucl�on est $m_n = 1,67.10^{-27}~kg$
\item Calculer la masse des �lectrons de l'atome de carbone 12
sachant que la masse d'un �lectron vaut $m_e = 9,1.10^{-31}~kg$
\item Comparer la masse des �lectrons de l'atome � la masse du noyau.
Que concluez-vous ?
\item En d�duire, sans nouveau calcul, la masse de l'atome de carbone
  $12$.
\end{enumerate}
\end{exercice}

\newpage

\vressort{1}

\begin{exercice}{Couches �lectroniques}\\
Dans l'�tat le plus stable de l'atome, appel� �tat fondamental,
les �lectrons occupent successivement les couches,
en commen�ant par celles qui sont les plus proches du noyau : 
d'abord $K$ puis $L$ puis $M$.

Lorsqu'une couche est pleine, ou encore satur�e, on passe � la suivante.

La derni�re couche occup�e est appel�e couche externe.\\
Toutes les autres sont appel�es couches internes.

\medskip

\noindent
%\begin{tabularx}{\textwidth}{|>{\centering}X|>{\centering}X|>{\centering}X|>{\centering}X|}
\begin{tabularx}{\textwidth}{|>{\mystrut}X|X|X|X|}
\hline
\multicolumn{1}{|X|}{\emph{Symbole de la couche}}       & $K$ & $L$ & $M$  \tbnl
\emph{Nombre maximal d'�lectrons} & $2$ & $8$ & $18$ \tbnl
\end{tabularx}

\medskip

Ainsi, par exemple, l'atome de chlore ($Z=17$) a la configuration �lectronique :
$(K)^2(L)^8(M)^{7}$.

\begin{enumerate}
\item Indiquez le nombre d'�lectrons et donnez la configuration des atomes suivants :
  \begin{enumerate}
  \item \noyau{H}{1}{}
  \item \noyau{O}{8}{}
  \item \noyau{C}{6}{}
  \item \noyau{Ne}{10}{}
  \end{enumerate}
\item Parmi les ions ci-desssous, pr�cisez s'il s'agit d'anions ou de cations.
Indiquez le nombre d'�lectrons et donnez la configuration �lectronique
des ions suivants :
  \begin{enumerate}
  \item $Be^{2+}$ ($Z=4$)
  \item $Al^{3+}$ ($Z=13$)
  \item $O^{2-}$ ($Z=8$)
  \item $F^{-}$ ($Z=9$)
  \end{enumerate}
\end{enumerate}

\end{exercice}


\vressort{5}


\begin{exercice}{Charge d'un atome de Zinc}%\\
\begin{enumerate}
\item Combien de protons l'atome de zinc \noyau{Zn}{30}{65} contient-il ?
\item Combien d'�lectrons comporte-t-il ?
\item Calculer la charge totale des protons
sachant qu'un proton a pour charge $e = 1,6.10^{-19}~C$.
\item Calculer la charge totale des �lectrons
sachant qu'un �lectron a pour charge $-e = -1,6.10^{-19}~C$.
\item En d�duire la charge de l'atome de Zinc.
\item Ce r�sultat est-il identique pour tous les atomes ?
\item A l'issue d'une r�action dite d'oxydation, un atome de zinc $Zn$
  se transforme en un ion $Zn^{2+}$.
  \begin{enumerate}
  \item Donnez l'�quation de cette r�action
  (en faisant intervenir un ou plusieurs �lectrons not�s $e^-$).
  \item Indiquez la charge (en coulomb $C$) de cet ion.
  \end{enumerate}
\end{enumerate}
\end{exercice}

\vressort{3} % cours elec g�n�rateurs, r�cepteurs

%\ds{Devoir Surveill�}{
%
}

\nomprenomclasse

\setcounter{numexercice}{0}

%\renewcommand{\tabularx}[1]{>{\centering}m{#1}} 

%\newcommand{\tabularxc}[1]{\tabularx{>{\centering}m{#1}}}

\vressort{3}

\begin{exercice}{Connaissance sur l'atome}%\\
\begin{enumerate}
\item De quoi est compos� un atome ?
\item Que signifie les lettres $A$, $Z$ et $X$ dans la repr�sentation \noyau{X}{Z}{A} ?
\item Comment trouve-t-on le nombre de neutrons d'un atome de l'�l�ment pr�c�dent.
\item Si un atome a $5$ protons, combien-a-t-il d'�lectrons ? Pourquoi ?
\item Qu'est-ce qui caract�rise un �l�ment chimique ?
\item Qu'est-ce qu'un isotope ?
\end{enumerate}
\end{exercice}



\vressort{3}



\begin{exercice}{Composition des atomes}\\
En vous aidant du tableau p�riodique des �l�ments,
compl�ter le tableau suivant :

\medskip

\noindent
%\begin{tabularx}{\textwidth}{|>{\centering}X|>{\centering}X|>{\centering}X|>{\centering}X|>{\centering}X|}
% \begin{tabularx}{\linewidth}{|X|X|X|X|X|}
% \hline
% \emph{nom}       & \emph{symbole}  & \emph{protons} & \emph{neutrons}
% & \emph{nucl�ons} \tbnl
% carbone   & \noyau{C}{6}{14}   &         &          & \rule[-0.5cm]{0cm}{1cm}         \tbnl
% fluor     & \noyau{F}{9}{19}   &         &          & \rule[-0.5cm]{0cm}{1cm}         \tbnl
% sodium    & \noyau{Na}{11}{23} &         &          & \rule[-0.5cm]{0cm}{1cm}         \tbnl
% oxyg�ne   & \noyau{O}{8}{16}   &         &          & \rule[-0.5cm]{0cm}{1cm}         \tbnl
% hydrog�ne &          &         & 0        & \rule[-0.5cm]{0cm}{1cm}         \tbnl
%           & \noyau{Cl}{17}{35} &         &          &  \rule[-0.5cm]{0cm}{1cm}        \tbnl
%           &          & 8       & \rule[-0.5cm]{0cm}{1cm}         & 16       \tbnl
% \end{tabularx}



\begin{tabularx}{\linewidth}{|>{\mystrut}X|X|X|X|X|}
\hline
% multicolumn pour faire dispara�tre le \mystrut
\multicolumn{1}{|X|}{\emph{nom}} & \emph{symbole}  &
\emph{protons} & \emph{neutrons} & \emph{nucl�ons} \tbnl
carbone   & \noyau{C}{6}{14}   &   &   &    \tbnl
fluor     & \noyau{F}{9}{19}   &   &   &    \tbnl
sodium    & \noyau{Na}{11}{23} &   &   &    \tbnl
oxyg�ne   & \noyau{O}{8}{16}   &   &   &    \tbnl
hydrog�ne &                    &   & 0 &    \tbnl
          & \noyau{Cl}{17}{35} &   &   &    \tbnl
          &                    & 8 &   & 16 \tbnl
\end{tabularx}


\end{exercice}


\vressort{3}


\begin{exercice}{Masse d'un atome de carbone 12}\\
Soit le carbone $12$ not� \noyau{C}{6}{12}.
\begin{enumerate}
\item L'�l�ment carbone peut-il avoir $5$ protons ? Pourquoi ?
\item Calculer la masse du noyau d'un atome de carbone $12$
sachant que la masse d'un nucl�on est $m_n = 1,67.10^{-27}~kg$
\item Calculer la masse des �lectrons de l'atome de carbone 12
sachant que la masse d'un �lectron vaut $m_e = 9,1.10^{-31}~kg$
\item Comparer la masse des �lectrons de l'atome � la masse du noyau.
Que concluez-vous ?
\item En d�duire, sans nouveau calcul, la masse de l'atome de carbone
  $12$.
\end{enumerate}
\end{exercice}

\newpage

\vressort{1}

\begin{exercice}{Couches �lectroniques}\\
Dans l'�tat le plus stable de l'atome, appel� �tat fondamental,
les �lectrons occupent successivement les couches,
en commen�ant par celles qui sont les plus proches du noyau : 
d'abord $K$ puis $L$ puis $M$.

Lorsqu'une couche est pleine, ou encore satur�e, on passe � la suivante.

La derni�re couche occup�e est appel�e couche externe.\\
Toutes les autres sont appel�es couches internes.

\medskip

\noindent
%\begin{tabularx}{\textwidth}{|>{\centering}X|>{\centering}X|>{\centering}X|>{\centering}X|}
\begin{tabularx}{\textwidth}{|>{\mystrut}X|X|X|X|}
\hline
\multicolumn{1}{|X|}{\emph{Symbole de la couche}}       & $K$ & $L$ & $M$  \tbnl
\emph{Nombre maximal d'�lectrons} & $2$ & $8$ & $18$ \tbnl
\end{tabularx}

\medskip

Ainsi, par exemple, l'atome de chlore ($Z=17$) a la configuration �lectronique :
$(K)^2(L)^8(M)^{7}$.

\begin{enumerate}
\item Indiquez le nombre d'�lectrons et donnez la configuration des atomes suivants :
  \begin{enumerate}
  \item \noyau{H}{1}{}
  \item \noyau{O}{8}{}
  \item \noyau{C}{6}{}
  \item \noyau{Ne}{10}{}
  \end{enumerate}
\item Parmi les ions ci-desssous, pr�cisez s'il s'agit d'anions ou de cations.
Indiquez le nombre d'�lectrons et donnez la configuration �lectronique
des ions suivants :
  \begin{enumerate}
  \item $Be^{2+}$ ($Z=4$)
  \item $Al^{3+}$ ($Z=13$)
  \item $O^{2-}$ ($Z=8$)
  \item $F^{-}$ ($Z=9$)
  \end{enumerate}
\end{enumerate}

\end{exercice}


\vressort{5}


\begin{exercice}{Charge d'un atome de Zinc}%\\
\begin{enumerate}
\item Combien de protons l'atome de zinc \noyau{Zn}{30}{65} contient-il ?
\item Combien d'�lectrons comporte-t-il ?
\item Calculer la charge totale des protons
sachant qu'un proton a pour charge $e = 1,6.10^{-19}~C$.
\item Calculer la charge totale des �lectrons
sachant qu'un �lectron a pour charge $-e = -1,6.10^{-19}~C$.
\item En d�duire la charge de l'atome de Zinc.
\item Ce r�sultat est-il identique pour tous les atomes ?
\item A l'issue d'une r�action dite d'oxydation, un atome de zinc $Zn$
  se transforme en un ion $Zn^{2+}$.
  \begin{enumerate}
  \item Donnez l'�quation de cette r�action
  (en faisant intervenir un ou plusieurs �lectrons not�s $e^-$).
  \item Indiquez la charge (en coulomb $C$) de cet ion.
  \end{enumerate}
\end{enumerate}
\end{exercice}

\vressort{3} % tp caract�ristique d'un g�n�rateur
%\ds{Devoir Surveill�}{
%
}

\nomprenomclasse

\setcounter{numexercice}{0}

%\renewcommand{\tabularx}[1]{>{\centering}m{#1}} 

%\newcommand{\tabularxc}[1]{\tabularx{>{\centering}m{#1}}}

\vressort{3}

\begin{exercice}{Connaissance sur l'atome}%\\
\begin{enumerate}
\item De quoi est compos� un atome ?
\item Que signifie les lettres $A$, $Z$ et $X$ dans la repr�sentation \noyau{X}{Z}{A} ?
\item Comment trouve-t-on le nombre de neutrons d'un atome de l'�l�ment pr�c�dent.
\item Si un atome a $5$ protons, combien-a-t-il d'�lectrons ? Pourquoi ?
\item Qu'est-ce qui caract�rise un �l�ment chimique ?
\item Qu'est-ce qu'un isotope ?
\end{enumerate}
\end{exercice}



\vressort{3}



\begin{exercice}{Composition des atomes}\\
En vous aidant du tableau p�riodique des �l�ments,
compl�ter le tableau suivant :

\medskip

\noindent
%\begin{tabularx}{\textwidth}{|>{\centering}X|>{\centering}X|>{\centering}X|>{\centering}X|>{\centering}X|}
% \begin{tabularx}{\linewidth}{|X|X|X|X|X|}
% \hline
% \emph{nom}       & \emph{symbole}  & \emph{protons} & \emph{neutrons}
% & \emph{nucl�ons} \tbnl
% carbone   & \noyau{C}{6}{14}   &         &          & \rule[-0.5cm]{0cm}{1cm}         \tbnl
% fluor     & \noyau{F}{9}{19}   &         &          & \rule[-0.5cm]{0cm}{1cm}         \tbnl
% sodium    & \noyau{Na}{11}{23} &         &          & \rule[-0.5cm]{0cm}{1cm}         \tbnl
% oxyg�ne   & \noyau{O}{8}{16}   &         &          & \rule[-0.5cm]{0cm}{1cm}         \tbnl
% hydrog�ne &          &         & 0        & \rule[-0.5cm]{0cm}{1cm}         \tbnl
%           & \noyau{Cl}{17}{35} &         &          &  \rule[-0.5cm]{0cm}{1cm}        \tbnl
%           &          & 8       & \rule[-0.5cm]{0cm}{1cm}         & 16       \tbnl
% \end{tabularx}



\begin{tabularx}{\linewidth}{|>{\mystrut}X|X|X|X|X|}
\hline
% multicolumn pour faire dispara�tre le \mystrut
\multicolumn{1}{|X|}{\emph{nom}} & \emph{symbole}  &
\emph{protons} & \emph{neutrons} & \emph{nucl�ons} \tbnl
carbone   & \noyau{C}{6}{14}   &   &   &    \tbnl
fluor     & \noyau{F}{9}{19}   &   &   &    \tbnl
sodium    & \noyau{Na}{11}{23} &   &   &    \tbnl
oxyg�ne   & \noyau{O}{8}{16}   &   &   &    \tbnl
hydrog�ne &                    &   & 0 &    \tbnl
          & \noyau{Cl}{17}{35} &   &   &    \tbnl
          &                    & 8 &   & 16 \tbnl
\end{tabularx}


\end{exercice}


\vressort{3}


\begin{exercice}{Masse d'un atome de carbone 12}\\
Soit le carbone $12$ not� \noyau{C}{6}{12}.
\begin{enumerate}
\item L'�l�ment carbone peut-il avoir $5$ protons ? Pourquoi ?
\item Calculer la masse du noyau d'un atome de carbone $12$
sachant que la masse d'un nucl�on est $m_n = 1,67.10^{-27}~kg$
\item Calculer la masse des �lectrons de l'atome de carbone 12
sachant que la masse d'un �lectron vaut $m_e = 9,1.10^{-31}~kg$
\item Comparer la masse des �lectrons de l'atome � la masse du noyau.
Que concluez-vous ?
\item En d�duire, sans nouveau calcul, la masse de l'atome de carbone
  $12$.
\end{enumerate}
\end{exercice}

\newpage

\vressort{1}

\begin{exercice}{Couches �lectroniques}\\
Dans l'�tat le plus stable de l'atome, appel� �tat fondamental,
les �lectrons occupent successivement les couches,
en commen�ant par celles qui sont les plus proches du noyau : 
d'abord $K$ puis $L$ puis $M$.

Lorsqu'une couche est pleine, ou encore satur�e, on passe � la suivante.

La derni�re couche occup�e est appel�e couche externe.\\
Toutes les autres sont appel�es couches internes.

\medskip

\noindent
%\begin{tabularx}{\textwidth}{|>{\centering}X|>{\centering}X|>{\centering}X|>{\centering}X|}
\begin{tabularx}{\textwidth}{|>{\mystrut}X|X|X|X|}
\hline
\multicolumn{1}{|X|}{\emph{Symbole de la couche}}       & $K$ & $L$ & $M$  \tbnl
\emph{Nombre maximal d'�lectrons} & $2$ & $8$ & $18$ \tbnl
\end{tabularx}

\medskip

Ainsi, par exemple, l'atome de chlore ($Z=17$) a la configuration �lectronique :
$(K)^2(L)^8(M)^{7}$.

\begin{enumerate}
\item Indiquez le nombre d'�lectrons et donnez la configuration des atomes suivants :
  \begin{enumerate}
  \item \noyau{H}{1}{}
  \item \noyau{O}{8}{}
  \item \noyau{C}{6}{}
  \item \noyau{Ne}{10}{}
  \end{enumerate}
\item Parmi les ions ci-desssous, pr�cisez s'il s'agit d'anions ou de cations.
Indiquez le nombre d'�lectrons et donnez la configuration �lectronique
des ions suivants :
  \begin{enumerate}
  \item $Be^{2+}$ ($Z=4$)
  \item $Al^{3+}$ ($Z=13$)
  \item $O^{2-}$ ($Z=8$)
  \item $F^{-}$ ($Z=9$)
  \end{enumerate}
\end{enumerate}

\end{exercice}


\vressort{5}


\begin{exercice}{Charge d'un atome de Zinc}%\\
\begin{enumerate}
\item Combien de protons l'atome de zinc \noyau{Zn}{30}{65} contient-il ?
\item Combien d'�lectrons comporte-t-il ?
\item Calculer la charge totale des protons
sachant qu'un proton a pour charge $e = 1,6.10^{-19}~C$.
\item Calculer la charge totale des �lectrons
sachant qu'un �lectron a pour charge $-e = -1,6.10^{-19}~C$.
\item En d�duire la charge de l'atome de Zinc.
\item Ce r�sultat est-il identique pour tous les atomes ?
\item A l'issue d'une r�action dite d'oxydation, un atome de zinc $Zn$
  se transforme en un ion $Zn^{2+}$.
  \begin{enumerate}
  \item Donnez l'�quation de cette r�action
  (en faisant intervenir un ou plusieurs �lectrons not�s $e^-$).
  \item Indiquez la charge (en coulomb $C$) de cet ion.
  \end{enumerate}
\end{enumerate}
\end{exercice}

\vressort{3} % tp caract�ristique d'un
                                % r�cepteur (�lectrolyseur)


\ds{Devoir Surveill�}{
%
}

\nomprenomclasse

\setcounter{numexercice}{0}

%\renewcommand{\tabularx}[1]{>{\centering}m{#1}} 

%\newcommand{\tabularxc}[1]{\tabularx{>{\centering}m{#1}}}

\vressort{3}

\begin{exercice}{Connaissance sur l'atome}%\\
\begin{enumerate}
\item De quoi est compos� un atome ?
\item Que signifie les lettres $A$, $Z$ et $X$ dans la repr�sentation \noyau{X}{Z}{A} ?
\item Comment trouve-t-on le nombre de neutrons d'un atome de l'�l�ment pr�c�dent.
\item Si un atome a $5$ protons, combien-a-t-il d'�lectrons ? Pourquoi ?
\item Qu'est-ce qui caract�rise un �l�ment chimique ?
\item Qu'est-ce qu'un isotope ?
\end{enumerate}
\end{exercice}



\vressort{3}



\begin{exercice}{Composition des atomes}\\
En vous aidant du tableau p�riodique des �l�ments,
compl�ter le tableau suivant :

\medskip

\noindent
%\begin{tabularx}{\textwidth}{|>{\centering}X|>{\centering}X|>{\centering}X|>{\centering}X|>{\centering}X|}
% \begin{tabularx}{\linewidth}{|X|X|X|X|X|}
% \hline
% \emph{nom}       & \emph{symbole}  & \emph{protons} & \emph{neutrons}
% & \emph{nucl�ons} \tbnl
% carbone   & \noyau{C}{6}{14}   &         &          & \rule[-0.5cm]{0cm}{1cm}         \tbnl
% fluor     & \noyau{F}{9}{19}   &         &          & \rule[-0.5cm]{0cm}{1cm}         \tbnl
% sodium    & \noyau{Na}{11}{23} &         &          & \rule[-0.5cm]{0cm}{1cm}         \tbnl
% oxyg�ne   & \noyau{O}{8}{16}   &         &          & \rule[-0.5cm]{0cm}{1cm}         \tbnl
% hydrog�ne &          &         & 0        & \rule[-0.5cm]{0cm}{1cm}         \tbnl
%           & \noyau{Cl}{17}{35} &         &          &  \rule[-0.5cm]{0cm}{1cm}        \tbnl
%           &          & 8       & \rule[-0.5cm]{0cm}{1cm}         & 16       \tbnl
% \end{tabularx}



\begin{tabularx}{\linewidth}{|>{\mystrut}X|X|X|X|X|}
\hline
% multicolumn pour faire dispara�tre le \mystrut
\multicolumn{1}{|X|}{\emph{nom}} & \emph{symbole}  &
\emph{protons} & \emph{neutrons} & \emph{nucl�ons} \tbnl
carbone   & \noyau{C}{6}{14}   &   &   &    \tbnl
fluor     & \noyau{F}{9}{19}   &   &   &    \tbnl
sodium    & \noyau{Na}{11}{23} &   &   &    \tbnl
oxyg�ne   & \noyau{O}{8}{16}   &   &   &    \tbnl
hydrog�ne &                    &   & 0 &    \tbnl
          & \noyau{Cl}{17}{35} &   &   &    \tbnl
          &                    & 8 &   & 16 \tbnl
\end{tabularx}


\end{exercice}


\vressort{3}


\begin{exercice}{Masse d'un atome de carbone 12}\\
Soit le carbone $12$ not� \noyau{C}{6}{12}.
\begin{enumerate}
\item L'�l�ment carbone peut-il avoir $5$ protons ? Pourquoi ?
\item Calculer la masse du noyau d'un atome de carbone $12$
sachant que la masse d'un nucl�on est $m_n = 1,67.10^{-27}~kg$
\item Calculer la masse des �lectrons de l'atome de carbone 12
sachant que la masse d'un �lectron vaut $m_e = 9,1.10^{-31}~kg$
\item Comparer la masse des �lectrons de l'atome � la masse du noyau.
Que concluez-vous ?
\item En d�duire, sans nouveau calcul, la masse de l'atome de carbone
  $12$.
\end{enumerate}
\end{exercice}

\newpage

\vressort{1}

\begin{exercice}{Couches �lectroniques}\\
Dans l'�tat le plus stable de l'atome, appel� �tat fondamental,
les �lectrons occupent successivement les couches,
en commen�ant par celles qui sont les plus proches du noyau : 
d'abord $K$ puis $L$ puis $M$.

Lorsqu'une couche est pleine, ou encore satur�e, on passe � la suivante.

La derni�re couche occup�e est appel�e couche externe.\\
Toutes les autres sont appel�es couches internes.

\medskip

\noindent
%\begin{tabularx}{\textwidth}{|>{\centering}X|>{\centering}X|>{\centering}X|>{\centering}X|}
\begin{tabularx}{\textwidth}{|>{\mystrut}X|X|X|X|}
\hline
\multicolumn{1}{|X|}{\emph{Symbole de la couche}}       & $K$ & $L$ & $M$  \tbnl
\emph{Nombre maximal d'�lectrons} & $2$ & $8$ & $18$ \tbnl
\end{tabularx}

\medskip

Ainsi, par exemple, l'atome de chlore ($Z=17$) a la configuration �lectronique :
$(K)^2(L)^8(M)^{7}$.

\begin{enumerate}
\item Indiquez le nombre d'�lectrons et donnez la configuration des atomes suivants :
  \begin{enumerate}
  \item \noyau{H}{1}{}
  \item \noyau{O}{8}{}
  \item \noyau{C}{6}{}
  \item \noyau{Ne}{10}{}
  \end{enumerate}
\item Parmi les ions ci-desssous, pr�cisez s'il s'agit d'anions ou de cations.
Indiquez le nombre d'�lectrons et donnez la configuration �lectronique
des ions suivants :
  \begin{enumerate}
  \item $Be^{2+}$ ($Z=4$)
  \item $Al^{3+}$ ($Z=13$)
  \item $O^{2-}$ ($Z=8$)
  \item $F^{-}$ ($Z=9$)
  \end{enumerate}
\end{enumerate}

\end{exercice}


\vressort{5}


\begin{exercice}{Charge d'un atome de Zinc}%\\
\begin{enumerate}
\item Combien de protons l'atome de zinc \noyau{Zn}{30}{65} contient-il ?
\item Combien d'�lectrons comporte-t-il ?
\item Calculer la charge totale des protons
sachant qu'un proton a pour charge $e = 1,6.10^{-19}~C$.
\item Calculer la charge totale des �lectrons
sachant qu'un �lectron a pour charge $-e = -1,6.10^{-19}~C$.
\item En d�duire la charge de l'atome de Zinc.
\item Ce r�sultat est-il identique pour tous les atomes ?
\item A l'issue d'une r�action dite d'oxydation, un atome de zinc $Zn$
  se transforme en un ion $Zn^{2+}$.
  \begin{enumerate}
  \item Donnez l'�quation de cette r�action
  (en faisant intervenir un ou plusieurs �lectrons not�s $e^-$).
  \item Indiquez la charge (en coulomb $C$) de cet ion.
  \end{enumerate}
\end{enumerate}
\end{exercice}

\vressort{3} % tp puissance �nergie (continu)



\chapitre{Optique g�om�trique}
\ds{Devoir Surveill�}{
%
}

\nomprenomclasse

\setcounter{numexercice}{0}

%\renewcommand{\tabularx}[1]{>{\centering}m{#1}} 

%\newcommand{\tabularxc}[1]{\tabularx{>{\centering}m{#1}}}

\vressort{3}

\begin{exercice}{Connaissance sur l'atome}%\\
\begin{enumerate}
\item De quoi est compos� un atome ?
\item Que signifie les lettres $A$, $Z$ et $X$ dans la repr�sentation \noyau{X}{Z}{A} ?
\item Comment trouve-t-on le nombre de neutrons d'un atome de l'�l�ment pr�c�dent.
\item Si un atome a $5$ protons, combien-a-t-il d'�lectrons ? Pourquoi ?
\item Qu'est-ce qui caract�rise un �l�ment chimique ?
\item Qu'est-ce qu'un isotope ?
\end{enumerate}
\end{exercice}



\vressort{3}



\begin{exercice}{Composition des atomes}\\
En vous aidant du tableau p�riodique des �l�ments,
compl�ter le tableau suivant :

\medskip

\noindent
%\begin{tabularx}{\textwidth}{|>{\centering}X|>{\centering}X|>{\centering}X|>{\centering}X|>{\centering}X|}
% \begin{tabularx}{\linewidth}{|X|X|X|X|X|}
% \hline
% \emph{nom}       & \emph{symbole}  & \emph{protons} & \emph{neutrons}
% & \emph{nucl�ons} \tbnl
% carbone   & \noyau{C}{6}{14}   &         &          & \rule[-0.5cm]{0cm}{1cm}         \tbnl
% fluor     & \noyau{F}{9}{19}   &         &          & \rule[-0.5cm]{0cm}{1cm}         \tbnl
% sodium    & \noyau{Na}{11}{23} &         &          & \rule[-0.5cm]{0cm}{1cm}         \tbnl
% oxyg�ne   & \noyau{O}{8}{16}   &         &          & \rule[-0.5cm]{0cm}{1cm}         \tbnl
% hydrog�ne &          &         & 0        & \rule[-0.5cm]{0cm}{1cm}         \tbnl
%           & \noyau{Cl}{17}{35} &         &          &  \rule[-0.5cm]{0cm}{1cm}        \tbnl
%           &          & 8       & \rule[-0.5cm]{0cm}{1cm}         & 16       \tbnl
% \end{tabularx}



\begin{tabularx}{\linewidth}{|>{\mystrut}X|X|X|X|X|}
\hline
% multicolumn pour faire dispara�tre le \mystrut
\multicolumn{1}{|X|}{\emph{nom}} & \emph{symbole}  &
\emph{protons} & \emph{neutrons} & \emph{nucl�ons} \tbnl
carbone   & \noyau{C}{6}{14}   &   &   &    \tbnl
fluor     & \noyau{F}{9}{19}   &   &   &    \tbnl
sodium    & \noyau{Na}{11}{23} &   &   &    \tbnl
oxyg�ne   & \noyau{O}{8}{16}   &   &   &    \tbnl
hydrog�ne &                    &   & 0 &    \tbnl
          & \noyau{Cl}{17}{35} &   &   &    \tbnl
          &                    & 8 &   & 16 \tbnl
\end{tabularx}


\end{exercice}


\vressort{3}


\begin{exercice}{Masse d'un atome de carbone 12}\\
Soit le carbone $12$ not� \noyau{C}{6}{12}.
\begin{enumerate}
\item L'�l�ment carbone peut-il avoir $5$ protons ? Pourquoi ?
\item Calculer la masse du noyau d'un atome de carbone $12$
sachant que la masse d'un nucl�on est $m_n = 1,67.10^{-27}~kg$
\item Calculer la masse des �lectrons de l'atome de carbone 12
sachant que la masse d'un �lectron vaut $m_e = 9,1.10^{-31}~kg$
\item Comparer la masse des �lectrons de l'atome � la masse du noyau.
Que concluez-vous ?
\item En d�duire, sans nouveau calcul, la masse de l'atome de carbone
  $12$.
\end{enumerate}
\end{exercice}

\newpage

\vressort{1}

\begin{exercice}{Couches �lectroniques}\\
Dans l'�tat le plus stable de l'atome, appel� �tat fondamental,
les �lectrons occupent successivement les couches,
en commen�ant par celles qui sont les plus proches du noyau : 
d'abord $K$ puis $L$ puis $M$.

Lorsqu'une couche est pleine, ou encore satur�e, on passe � la suivante.

La derni�re couche occup�e est appel�e couche externe.\\
Toutes les autres sont appel�es couches internes.

\medskip

\noindent
%\begin{tabularx}{\textwidth}{|>{\centering}X|>{\centering}X|>{\centering}X|>{\centering}X|}
\begin{tabularx}{\textwidth}{|>{\mystrut}X|X|X|X|}
\hline
\multicolumn{1}{|X|}{\emph{Symbole de la couche}}       & $K$ & $L$ & $M$  \tbnl
\emph{Nombre maximal d'�lectrons} & $2$ & $8$ & $18$ \tbnl
\end{tabularx}

\medskip

Ainsi, par exemple, l'atome de chlore ($Z=17$) a la configuration �lectronique :
$(K)^2(L)^8(M)^{7}$.

\begin{enumerate}
\item Indiquez le nombre d'�lectrons et donnez la configuration des atomes suivants :
  \begin{enumerate}
  \item \noyau{H}{1}{}
  \item \noyau{O}{8}{}
  \item \noyau{C}{6}{}
  \item \noyau{Ne}{10}{}
  \end{enumerate}
\item Parmi les ions ci-desssous, pr�cisez s'il s'agit d'anions ou de cations.
Indiquez le nombre d'�lectrons et donnez la configuration �lectronique
des ions suivants :
  \begin{enumerate}
  \item $Be^{2+}$ ($Z=4$)
  \item $Al^{3+}$ ($Z=13$)
  \item $O^{2-}$ ($Z=8$)
  \item $F^{-}$ ($Z=9$)
  \end{enumerate}
\end{enumerate}

\end{exercice}


\vressort{5}


\begin{exercice}{Charge d'un atome de Zinc}%\\
\begin{enumerate}
\item Combien de protons l'atome de zinc \noyau{Zn}{30}{65} contient-il ?
\item Combien d'�lectrons comporte-t-il ?
\item Calculer la charge totale des protons
sachant qu'un proton a pour charge $e = 1,6.10^{-19}~C$.
\item Calculer la charge totale des �lectrons
sachant qu'un �lectron a pour charge $-e = -1,6.10^{-19}~C$.
\item En d�duire la charge de l'atome de Zinc.
\item Ce r�sultat est-il identique pour tous les atomes ?
\item A l'issue d'une r�action dite d'oxydation, un atome de zinc $Zn$
  se transforme en un ion $Zn^{2+}$.
  \begin{enumerate}
  \item Donnez l'�quation de cette r�action
  (en faisant intervenir un ou plusieurs �lectrons not�s $e^-$).
  \item Indiquez la charge (en coulomb $C$) de cet ion.
  \end{enumerate}
\end{enumerate}
\end{exercice}

\vressort{3}

\ds{Devoir Surveill�}{
%
}

\nomprenomclasse

\setcounter{numexercice}{0}

%\renewcommand{\tabularx}[1]{>{\centering}m{#1}} 

%\newcommand{\tabularxc}[1]{\tabularx{>{\centering}m{#1}}}

\vressort{3}

\begin{exercice}{Connaissance sur l'atome}%\\
\begin{enumerate}
\item De quoi est compos� un atome ?
\item Que signifie les lettres $A$, $Z$ et $X$ dans la repr�sentation \noyau{X}{Z}{A} ?
\item Comment trouve-t-on le nombre de neutrons d'un atome de l'�l�ment pr�c�dent.
\item Si un atome a $5$ protons, combien-a-t-il d'�lectrons ? Pourquoi ?
\item Qu'est-ce qui caract�rise un �l�ment chimique ?
\item Qu'est-ce qu'un isotope ?
\end{enumerate}
\end{exercice}



\vressort{3}



\begin{exercice}{Composition des atomes}\\
En vous aidant du tableau p�riodique des �l�ments,
compl�ter le tableau suivant :

\medskip

\noindent
%\begin{tabularx}{\textwidth}{|>{\centering}X|>{\centering}X|>{\centering}X|>{\centering}X|>{\centering}X|}
% \begin{tabularx}{\linewidth}{|X|X|X|X|X|}
% \hline
% \emph{nom}       & \emph{symbole}  & \emph{protons} & \emph{neutrons}
% & \emph{nucl�ons} \tbnl
% carbone   & \noyau{C}{6}{14}   &         &          & \rule[-0.5cm]{0cm}{1cm}         \tbnl
% fluor     & \noyau{F}{9}{19}   &         &          & \rule[-0.5cm]{0cm}{1cm}         \tbnl
% sodium    & \noyau{Na}{11}{23} &         &          & \rule[-0.5cm]{0cm}{1cm}         \tbnl
% oxyg�ne   & \noyau{O}{8}{16}   &         &          & \rule[-0.5cm]{0cm}{1cm}         \tbnl
% hydrog�ne &          &         & 0        & \rule[-0.5cm]{0cm}{1cm}         \tbnl
%           & \noyau{Cl}{17}{35} &         &          &  \rule[-0.5cm]{0cm}{1cm}        \tbnl
%           &          & 8       & \rule[-0.5cm]{0cm}{1cm}         & 16       \tbnl
% \end{tabularx}



\begin{tabularx}{\linewidth}{|>{\mystrut}X|X|X|X|X|}
\hline
% multicolumn pour faire dispara�tre le \mystrut
\multicolumn{1}{|X|}{\emph{nom}} & \emph{symbole}  &
\emph{protons} & \emph{neutrons} & \emph{nucl�ons} \tbnl
carbone   & \noyau{C}{6}{14}   &   &   &    \tbnl
fluor     & \noyau{F}{9}{19}   &   &   &    \tbnl
sodium    & \noyau{Na}{11}{23} &   &   &    \tbnl
oxyg�ne   & \noyau{O}{8}{16}   &   &   &    \tbnl
hydrog�ne &                    &   & 0 &    \tbnl
          & \noyau{Cl}{17}{35} &   &   &    \tbnl
          &                    & 8 &   & 16 \tbnl
\end{tabularx}


\end{exercice}


\vressort{3}


\begin{exercice}{Masse d'un atome de carbone 12}\\
Soit le carbone $12$ not� \noyau{C}{6}{12}.
\begin{enumerate}
\item L'�l�ment carbone peut-il avoir $5$ protons ? Pourquoi ?
\item Calculer la masse du noyau d'un atome de carbone $12$
sachant que la masse d'un nucl�on est $m_n = 1,67.10^{-27}~kg$
\item Calculer la masse des �lectrons de l'atome de carbone 12
sachant que la masse d'un �lectron vaut $m_e = 9,1.10^{-31}~kg$
\item Comparer la masse des �lectrons de l'atome � la masse du noyau.
Que concluez-vous ?
\item En d�duire, sans nouveau calcul, la masse de l'atome de carbone
  $12$.
\end{enumerate}
\end{exercice}

\newpage

\vressort{1}

\begin{exercice}{Couches �lectroniques}\\
Dans l'�tat le plus stable de l'atome, appel� �tat fondamental,
les �lectrons occupent successivement les couches,
en commen�ant par celles qui sont les plus proches du noyau : 
d'abord $K$ puis $L$ puis $M$.

Lorsqu'une couche est pleine, ou encore satur�e, on passe � la suivante.

La derni�re couche occup�e est appel�e couche externe.\\
Toutes les autres sont appel�es couches internes.

\medskip

\noindent
%\begin{tabularx}{\textwidth}{|>{\centering}X|>{\centering}X|>{\centering}X|>{\centering}X|}
\begin{tabularx}{\textwidth}{|>{\mystrut}X|X|X|X|}
\hline
\multicolumn{1}{|X|}{\emph{Symbole de la couche}}       & $K$ & $L$ & $M$  \tbnl
\emph{Nombre maximal d'�lectrons} & $2$ & $8$ & $18$ \tbnl
\end{tabularx}

\medskip

Ainsi, par exemple, l'atome de chlore ($Z=17$) a la configuration �lectronique :
$(K)^2(L)^8(M)^{7}$.

\begin{enumerate}
\item Indiquez le nombre d'�lectrons et donnez la configuration des atomes suivants :
  \begin{enumerate}
  \item \noyau{H}{1}{}
  \item \noyau{O}{8}{}
  \item \noyau{C}{6}{}
  \item \noyau{Ne}{10}{}
  \end{enumerate}
\item Parmi les ions ci-desssous, pr�cisez s'il s'agit d'anions ou de cations.
Indiquez le nombre d'�lectrons et donnez la configuration �lectronique
des ions suivants :
  \begin{enumerate}
  \item $Be^{2+}$ ($Z=4$)
  \item $Al^{3+}$ ($Z=13$)
  \item $O^{2-}$ ($Z=8$)
  \item $F^{-}$ ($Z=9$)
  \end{enumerate}
\end{enumerate}

\end{exercice}


\vressort{5}


\begin{exercice}{Charge d'un atome de Zinc}%\\
\begin{enumerate}
\item Combien de protons l'atome de zinc \noyau{Zn}{30}{65} contient-il ?
\item Combien d'�lectrons comporte-t-il ?
\item Calculer la charge totale des protons
sachant qu'un proton a pour charge $e = 1,6.10^{-19}~C$.
\item Calculer la charge totale des �lectrons
sachant qu'un �lectron a pour charge $-e = -1,6.10^{-19}~C$.
\item En d�duire la charge de l'atome de Zinc.
\item Ce r�sultat est-il identique pour tous les atomes ?
\item A l'issue d'une r�action dite d'oxydation, un atome de zinc $Zn$
  se transforme en un ion $Zn^{2+}$.
  \begin{enumerate}
  \item Donnez l'�quation de cette r�action
  (en faisant intervenir un ou plusieurs �lectrons not�s $e^-$).
  \item Indiquez la charge (en coulomb $C$) de cet ion.
  \end{enumerate}
\end{enumerate}
\end{exercice}

\vressort{3}
\ds{Devoir Surveill�}{
%
}

\nomprenomclasse

\setcounter{numexercice}{0}

%\renewcommand{\tabularx}[1]{>{\centering}m{#1}} 

%\newcommand{\tabularxc}[1]{\tabularx{>{\centering}m{#1}}}

\vressort{3}

\begin{exercice}{Connaissance sur l'atome}%\\
\begin{enumerate}
\item De quoi est compos� un atome ?
\item Que signifie les lettres $A$, $Z$ et $X$ dans la repr�sentation \noyau{X}{Z}{A} ?
\item Comment trouve-t-on le nombre de neutrons d'un atome de l'�l�ment pr�c�dent.
\item Si un atome a $5$ protons, combien-a-t-il d'�lectrons ? Pourquoi ?
\item Qu'est-ce qui caract�rise un �l�ment chimique ?
\item Qu'est-ce qu'un isotope ?
\end{enumerate}
\end{exercice}



\vressort{3}



\begin{exercice}{Composition des atomes}\\
En vous aidant du tableau p�riodique des �l�ments,
compl�ter le tableau suivant :

\medskip

\noindent
%\begin{tabularx}{\textwidth}{|>{\centering}X|>{\centering}X|>{\centering}X|>{\centering}X|>{\centering}X|}
% \begin{tabularx}{\linewidth}{|X|X|X|X|X|}
% \hline
% \emph{nom}       & \emph{symbole}  & \emph{protons} & \emph{neutrons}
% & \emph{nucl�ons} \tbnl
% carbone   & \noyau{C}{6}{14}   &         &          & \rule[-0.5cm]{0cm}{1cm}         \tbnl
% fluor     & \noyau{F}{9}{19}   &         &          & \rule[-0.5cm]{0cm}{1cm}         \tbnl
% sodium    & \noyau{Na}{11}{23} &         &          & \rule[-0.5cm]{0cm}{1cm}         \tbnl
% oxyg�ne   & \noyau{O}{8}{16}   &         &          & \rule[-0.5cm]{0cm}{1cm}         \tbnl
% hydrog�ne &          &         & 0        & \rule[-0.5cm]{0cm}{1cm}         \tbnl
%           & \noyau{Cl}{17}{35} &         &          &  \rule[-0.5cm]{0cm}{1cm}        \tbnl
%           &          & 8       & \rule[-0.5cm]{0cm}{1cm}         & 16       \tbnl
% \end{tabularx}



\begin{tabularx}{\linewidth}{|>{\mystrut}X|X|X|X|X|}
\hline
% multicolumn pour faire dispara�tre le \mystrut
\multicolumn{1}{|X|}{\emph{nom}} & \emph{symbole}  &
\emph{protons} & \emph{neutrons} & \emph{nucl�ons} \tbnl
carbone   & \noyau{C}{6}{14}   &   &   &    \tbnl
fluor     & \noyau{F}{9}{19}   &   &   &    \tbnl
sodium    & \noyau{Na}{11}{23} &   &   &    \tbnl
oxyg�ne   & \noyau{O}{8}{16}   &   &   &    \tbnl
hydrog�ne &                    &   & 0 &    \tbnl
          & \noyau{Cl}{17}{35} &   &   &    \tbnl
          &                    & 8 &   & 16 \tbnl
\end{tabularx}


\end{exercice}


\vressort{3}


\begin{exercice}{Masse d'un atome de carbone 12}\\
Soit le carbone $12$ not� \noyau{C}{6}{12}.
\begin{enumerate}
\item L'�l�ment carbone peut-il avoir $5$ protons ? Pourquoi ?
\item Calculer la masse du noyau d'un atome de carbone $12$
sachant que la masse d'un nucl�on est $m_n = 1,67.10^{-27}~kg$
\item Calculer la masse des �lectrons de l'atome de carbone 12
sachant que la masse d'un �lectron vaut $m_e = 9,1.10^{-31}~kg$
\item Comparer la masse des �lectrons de l'atome � la masse du noyau.
Que concluez-vous ?
\item En d�duire, sans nouveau calcul, la masse de l'atome de carbone
  $12$.
\end{enumerate}
\end{exercice}

\newpage

\vressort{1}

\begin{exercice}{Couches �lectroniques}\\
Dans l'�tat le plus stable de l'atome, appel� �tat fondamental,
les �lectrons occupent successivement les couches,
en commen�ant par celles qui sont les plus proches du noyau : 
d'abord $K$ puis $L$ puis $M$.

Lorsqu'une couche est pleine, ou encore satur�e, on passe � la suivante.

La derni�re couche occup�e est appel�e couche externe.\\
Toutes les autres sont appel�es couches internes.

\medskip

\noindent
%\begin{tabularx}{\textwidth}{|>{\centering}X|>{\centering}X|>{\centering}X|>{\centering}X|}
\begin{tabularx}{\textwidth}{|>{\mystrut}X|X|X|X|}
\hline
\multicolumn{1}{|X|}{\emph{Symbole de la couche}}       & $K$ & $L$ & $M$  \tbnl
\emph{Nombre maximal d'�lectrons} & $2$ & $8$ & $18$ \tbnl
\end{tabularx}

\medskip

Ainsi, par exemple, l'atome de chlore ($Z=17$) a la configuration �lectronique :
$(K)^2(L)^8(M)^{7}$.

\begin{enumerate}
\item Indiquez le nombre d'�lectrons et donnez la configuration des atomes suivants :
  \begin{enumerate}
  \item \noyau{H}{1}{}
  \item \noyau{O}{8}{}
  \item \noyau{C}{6}{}
  \item \noyau{Ne}{10}{}
  \end{enumerate}
\item Parmi les ions ci-desssous, pr�cisez s'il s'agit d'anions ou de cations.
Indiquez le nombre d'�lectrons et donnez la configuration �lectronique
des ions suivants :
  \begin{enumerate}
  \item $Be^{2+}$ ($Z=4$)
  \item $Al^{3+}$ ($Z=13$)
  \item $O^{2-}$ ($Z=8$)
  \item $F^{-}$ ($Z=9$)
  \end{enumerate}
\end{enumerate}

\end{exercice}


\vressort{5}


\begin{exercice}{Charge d'un atome de Zinc}%\\
\begin{enumerate}
\item Combien de protons l'atome de zinc \noyau{Zn}{30}{65} contient-il ?
\item Combien d'�lectrons comporte-t-il ?
\item Calculer la charge totale des protons
sachant qu'un proton a pour charge $e = 1,6.10^{-19}~C$.
\item Calculer la charge totale des �lectrons
sachant qu'un �lectron a pour charge $-e = -1,6.10^{-19}~C$.
\item En d�duire la charge de l'atome de Zinc.
\item Ce r�sultat est-il identique pour tous les atomes ?
\item A l'issue d'une r�action dite d'oxydation, un atome de zinc $Zn$
  se transforme en un ion $Zn^{2+}$.
  \begin{enumerate}
  \item Donnez l'�quation de cette r�action
  (en faisant intervenir un ou plusieurs �lectrons not�s $e^-$).
  \item Indiquez la charge (en coulomb $C$) de cet ion.
  \end{enumerate}
\end{enumerate}
\end{exercice}

\vressort{3}
\ds{Devoir Surveill�}{
%
}

\nomprenomclasse

\setcounter{numexercice}{0}

%\renewcommand{\tabularx}[1]{>{\centering}m{#1}} 

%\newcommand{\tabularxc}[1]{\tabularx{>{\centering}m{#1}}}

\vressort{3}

\begin{exercice}{Connaissance sur l'atome}%\\
\begin{enumerate}
\item De quoi est compos� un atome ?
\item Que signifie les lettres $A$, $Z$ et $X$ dans la repr�sentation \noyau{X}{Z}{A} ?
\item Comment trouve-t-on le nombre de neutrons d'un atome de l'�l�ment pr�c�dent.
\item Si un atome a $5$ protons, combien-a-t-il d'�lectrons ? Pourquoi ?
\item Qu'est-ce qui caract�rise un �l�ment chimique ?
\item Qu'est-ce qu'un isotope ?
\end{enumerate}
\end{exercice}



\vressort{3}



\begin{exercice}{Composition des atomes}\\
En vous aidant du tableau p�riodique des �l�ments,
compl�ter le tableau suivant :

\medskip

\noindent
%\begin{tabularx}{\textwidth}{|>{\centering}X|>{\centering}X|>{\centering}X|>{\centering}X|>{\centering}X|}
% \begin{tabularx}{\linewidth}{|X|X|X|X|X|}
% \hline
% \emph{nom}       & \emph{symbole}  & \emph{protons} & \emph{neutrons}
% & \emph{nucl�ons} \tbnl
% carbone   & \noyau{C}{6}{14}   &         &          & \rule[-0.5cm]{0cm}{1cm}         \tbnl
% fluor     & \noyau{F}{9}{19}   &         &          & \rule[-0.5cm]{0cm}{1cm}         \tbnl
% sodium    & \noyau{Na}{11}{23} &         &          & \rule[-0.5cm]{0cm}{1cm}         \tbnl
% oxyg�ne   & \noyau{O}{8}{16}   &         &          & \rule[-0.5cm]{0cm}{1cm}         \tbnl
% hydrog�ne &          &         & 0        & \rule[-0.5cm]{0cm}{1cm}         \tbnl
%           & \noyau{Cl}{17}{35} &         &          &  \rule[-0.5cm]{0cm}{1cm}        \tbnl
%           &          & 8       & \rule[-0.5cm]{0cm}{1cm}         & 16       \tbnl
% \end{tabularx}



\begin{tabularx}{\linewidth}{|>{\mystrut}X|X|X|X|X|}
\hline
% multicolumn pour faire dispara�tre le \mystrut
\multicolumn{1}{|X|}{\emph{nom}} & \emph{symbole}  &
\emph{protons} & \emph{neutrons} & \emph{nucl�ons} \tbnl
carbone   & \noyau{C}{6}{14}   &   &   &    \tbnl
fluor     & \noyau{F}{9}{19}   &   &   &    \tbnl
sodium    & \noyau{Na}{11}{23} &   &   &    \tbnl
oxyg�ne   & \noyau{O}{8}{16}   &   &   &    \tbnl
hydrog�ne &                    &   & 0 &    \tbnl
          & \noyau{Cl}{17}{35} &   &   &    \tbnl
          &                    & 8 &   & 16 \tbnl
\end{tabularx}


\end{exercice}


\vressort{3}


\begin{exercice}{Masse d'un atome de carbone 12}\\
Soit le carbone $12$ not� \noyau{C}{6}{12}.
\begin{enumerate}
\item L'�l�ment carbone peut-il avoir $5$ protons ? Pourquoi ?
\item Calculer la masse du noyau d'un atome de carbone $12$
sachant que la masse d'un nucl�on est $m_n = 1,67.10^{-27}~kg$
\item Calculer la masse des �lectrons de l'atome de carbone 12
sachant que la masse d'un �lectron vaut $m_e = 9,1.10^{-31}~kg$
\item Comparer la masse des �lectrons de l'atome � la masse du noyau.
Que concluez-vous ?
\item En d�duire, sans nouveau calcul, la masse de l'atome de carbone
  $12$.
\end{enumerate}
\end{exercice}

\newpage

\vressort{1}

\begin{exercice}{Couches �lectroniques}\\
Dans l'�tat le plus stable de l'atome, appel� �tat fondamental,
les �lectrons occupent successivement les couches,
en commen�ant par celles qui sont les plus proches du noyau : 
d'abord $K$ puis $L$ puis $M$.

Lorsqu'une couche est pleine, ou encore satur�e, on passe � la suivante.

La derni�re couche occup�e est appel�e couche externe.\\
Toutes les autres sont appel�es couches internes.

\medskip

\noindent
%\begin{tabularx}{\textwidth}{|>{\centering}X|>{\centering}X|>{\centering}X|>{\centering}X|}
\begin{tabularx}{\textwidth}{|>{\mystrut}X|X|X|X|}
\hline
\multicolumn{1}{|X|}{\emph{Symbole de la couche}}       & $K$ & $L$ & $M$  \tbnl
\emph{Nombre maximal d'�lectrons} & $2$ & $8$ & $18$ \tbnl
\end{tabularx}

\medskip

Ainsi, par exemple, l'atome de chlore ($Z=17$) a la configuration �lectronique :
$(K)^2(L)^8(M)^{7}$.

\begin{enumerate}
\item Indiquez le nombre d'�lectrons et donnez la configuration des atomes suivants :
  \begin{enumerate}
  \item \noyau{H}{1}{}
  \item \noyau{O}{8}{}
  \item \noyau{C}{6}{}
  \item \noyau{Ne}{10}{}
  \end{enumerate}
\item Parmi les ions ci-desssous, pr�cisez s'il s'agit d'anions ou de cations.
Indiquez le nombre d'�lectrons et donnez la configuration �lectronique
des ions suivants :
  \begin{enumerate}
  \item $Be^{2+}$ ($Z=4$)
  \item $Al^{3+}$ ($Z=13$)
  \item $O^{2-}$ ($Z=8$)
  \item $F^{-}$ ($Z=9$)
  \end{enumerate}
\end{enumerate}

\end{exercice}


\vressort{5}


\begin{exercice}{Charge d'un atome de Zinc}%\\
\begin{enumerate}
\item Combien de protons l'atome de zinc \noyau{Zn}{30}{65} contient-il ?
\item Combien d'�lectrons comporte-t-il ?
\item Calculer la charge totale des protons
sachant qu'un proton a pour charge $e = 1,6.10^{-19}~C$.
\item Calculer la charge totale des �lectrons
sachant qu'un �lectron a pour charge $-e = -1,6.10^{-19}~C$.
\item En d�duire la charge de l'atome de Zinc.
\item Ce r�sultat est-il identique pour tous les atomes ?
\item A l'issue d'une r�action dite d'oxydation, un atome de zinc $Zn$
  se transforme en un ion $Zn^{2+}$.
  \begin{enumerate}
  \item Donnez l'�quation de cette r�action
  (en faisant intervenir un ou plusieurs �lectrons not�s $e^-$).
  \item Indiquez la charge (en coulomb $C$) de cet ion.
  \end{enumerate}
\end{enumerate}
\end{exercice}

\vressort{3} % Lentilles (Images, rel
                                % de conjugaison)
\ds{Devoir Surveill�}{
%
}

\nomprenomclasse

\setcounter{numexercice}{0}

%\renewcommand{\tabularx}[1]{>{\centering}m{#1}} 

%\newcommand{\tabularxc}[1]{\tabularx{>{\centering}m{#1}}}

\vressort{3}

\begin{exercice}{Connaissance sur l'atome}%\\
\begin{enumerate}
\item De quoi est compos� un atome ?
\item Que signifie les lettres $A$, $Z$ et $X$ dans la repr�sentation \noyau{X}{Z}{A} ?
\item Comment trouve-t-on le nombre de neutrons d'un atome de l'�l�ment pr�c�dent.
\item Si un atome a $5$ protons, combien-a-t-il d'�lectrons ? Pourquoi ?
\item Qu'est-ce qui caract�rise un �l�ment chimique ?
\item Qu'est-ce qu'un isotope ?
\end{enumerate}
\end{exercice}



\vressort{3}



\begin{exercice}{Composition des atomes}\\
En vous aidant du tableau p�riodique des �l�ments,
compl�ter le tableau suivant :

\medskip

\noindent
%\begin{tabularx}{\textwidth}{|>{\centering}X|>{\centering}X|>{\centering}X|>{\centering}X|>{\centering}X|}
% \begin{tabularx}{\linewidth}{|X|X|X|X|X|}
% \hline
% \emph{nom}       & \emph{symbole}  & \emph{protons} & \emph{neutrons}
% & \emph{nucl�ons} \tbnl
% carbone   & \noyau{C}{6}{14}   &         &          & \rule[-0.5cm]{0cm}{1cm}         \tbnl
% fluor     & \noyau{F}{9}{19}   &         &          & \rule[-0.5cm]{0cm}{1cm}         \tbnl
% sodium    & \noyau{Na}{11}{23} &         &          & \rule[-0.5cm]{0cm}{1cm}         \tbnl
% oxyg�ne   & \noyau{O}{8}{16}   &         &          & \rule[-0.5cm]{0cm}{1cm}         \tbnl
% hydrog�ne &          &         & 0        & \rule[-0.5cm]{0cm}{1cm}         \tbnl
%           & \noyau{Cl}{17}{35} &         &          &  \rule[-0.5cm]{0cm}{1cm}        \tbnl
%           &          & 8       & \rule[-0.5cm]{0cm}{1cm}         & 16       \tbnl
% \end{tabularx}



\begin{tabularx}{\linewidth}{|>{\mystrut}X|X|X|X|X|}
\hline
% multicolumn pour faire dispara�tre le \mystrut
\multicolumn{1}{|X|}{\emph{nom}} & \emph{symbole}  &
\emph{protons} & \emph{neutrons} & \emph{nucl�ons} \tbnl
carbone   & \noyau{C}{6}{14}   &   &   &    \tbnl
fluor     & \noyau{F}{9}{19}   &   &   &    \tbnl
sodium    & \noyau{Na}{11}{23} &   &   &    \tbnl
oxyg�ne   & \noyau{O}{8}{16}   &   &   &    \tbnl
hydrog�ne &                    &   & 0 &    \tbnl
          & \noyau{Cl}{17}{35} &   &   &    \tbnl
          &                    & 8 &   & 16 \tbnl
\end{tabularx}


\end{exercice}


\vressort{3}


\begin{exercice}{Masse d'un atome de carbone 12}\\
Soit le carbone $12$ not� \noyau{C}{6}{12}.
\begin{enumerate}
\item L'�l�ment carbone peut-il avoir $5$ protons ? Pourquoi ?
\item Calculer la masse du noyau d'un atome de carbone $12$
sachant que la masse d'un nucl�on est $m_n = 1,67.10^{-27}~kg$
\item Calculer la masse des �lectrons de l'atome de carbone 12
sachant que la masse d'un �lectron vaut $m_e = 9,1.10^{-31}~kg$
\item Comparer la masse des �lectrons de l'atome � la masse du noyau.
Que concluez-vous ?
\item En d�duire, sans nouveau calcul, la masse de l'atome de carbone
  $12$.
\end{enumerate}
\end{exercice}

\newpage

\vressort{1}

\begin{exercice}{Couches �lectroniques}\\
Dans l'�tat le plus stable de l'atome, appel� �tat fondamental,
les �lectrons occupent successivement les couches,
en commen�ant par celles qui sont les plus proches du noyau : 
d'abord $K$ puis $L$ puis $M$.

Lorsqu'une couche est pleine, ou encore satur�e, on passe � la suivante.

La derni�re couche occup�e est appel�e couche externe.\\
Toutes les autres sont appel�es couches internes.

\medskip

\noindent
%\begin{tabularx}{\textwidth}{|>{\centering}X|>{\centering}X|>{\centering}X|>{\centering}X|}
\begin{tabularx}{\textwidth}{|>{\mystrut}X|X|X|X|}
\hline
\multicolumn{1}{|X|}{\emph{Symbole de la couche}}       & $K$ & $L$ & $M$  \tbnl
\emph{Nombre maximal d'�lectrons} & $2$ & $8$ & $18$ \tbnl
\end{tabularx}

\medskip

Ainsi, par exemple, l'atome de chlore ($Z=17$) a la configuration �lectronique :
$(K)^2(L)^8(M)^{7}$.

\begin{enumerate}
\item Indiquez le nombre d'�lectrons et donnez la configuration des atomes suivants :
  \begin{enumerate}
  \item \noyau{H}{1}{}
  \item \noyau{O}{8}{}
  \item \noyau{C}{6}{}
  \item \noyau{Ne}{10}{}
  \end{enumerate}
\item Parmi les ions ci-desssous, pr�cisez s'il s'agit d'anions ou de cations.
Indiquez le nombre d'�lectrons et donnez la configuration �lectronique
des ions suivants :
  \begin{enumerate}
  \item $Be^{2+}$ ($Z=4$)
  \item $Al^{3+}$ ($Z=13$)
  \item $O^{2-}$ ($Z=8$)
  \item $F^{-}$ ($Z=9$)
  \end{enumerate}
\end{enumerate}

\end{exercice}


\vressort{5}


\begin{exercice}{Charge d'un atome de Zinc}%\\
\begin{enumerate}
\item Combien de protons l'atome de zinc \noyau{Zn}{30}{65} contient-il ?
\item Combien d'�lectrons comporte-t-il ?
\item Calculer la charge totale des protons
sachant qu'un proton a pour charge $e = 1,6.10^{-19}~C$.
\item Calculer la charge totale des �lectrons
sachant qu'un �lectron a pour charge $-e = -1,6.10^{-19}~C$.
\item En d�duire la charge de l'atome de Zinc.
\item Ce r�sultat est-il identique pour tous les atomes ?
\item A l'issue d'une r�action dite d'oxydation, un atome de zinc $Zn$
  se transforme en un ion $Zn^{2+}$.
  \begin{enumerate}
  \item Donnez l'�quation de cette r�action
  (en faisant intervenir un ou plusieurs �lectrons not�s $e^-$).
  \item Indiquez la charge (en coulomb $C$) de cet ion.
  \end{enumerate}
\end{enumerate}
\end{exercice}

\vressort{3}

\ds{Devoir Surveill�}{
%
}

\nomprenomclasse

\setcounter{numexercice}{0}

%\renewcommand{\tabularx}[1]{>{\centering}m{#1}} 

%\newcommand{\tabularxc}[1]{\tabularx{>{\centering}m{#1}}}

\vressort{3}

\begin{exercice}{Connaissance sur l'atome}%\\
\begin{enumerate}
\item De quoi est compos� un atome ?
\item Que signifie les lettres $A$, $Z$ et $X$ dans la repr�sentation \noyau{X}{Z}{A} ?
\item Comment trouve-t-on le nombre de neutrons d'un atome de l'�l�ment pr�c�dent.
\item Si un atome a $5$ protons, combien-a-t-il d'�lectrons ? Pourquoi ?
\item Qu'est-ce qui caract�rise un �l�ment chimique ?
\item Qu'est-ce qu'un isotope ?
\end{enumerate}
\end{exercice}



\vressort{3}



\begin{exercice}{Composition des atomes}\\
En vous aidant du tableau p�riodique des �l�ments,
compl�ter le tableau suivant :

\medskip

\noindent
%\begin{tabularx}{\textwidth}{|>{\centering}X|>{\centering}X|>{\centering}X|>{\centering}X|>{\centering}X|}
% \begin{tabularx}{\linewidth}{|X|X|X|X|X|}
% \hline
% \emph{nom}       & \emph{symbole}  & \emph{protons} & \emph{neutrons}
% & \emph{nucl�ons} \tbnl
% carbone   & \noyau{C}{6}{14}   &         &          & \rule[-0.5cm]{0cm}{1cm}         \tbnl
% fluor     & \noyau{F}{9}{19}   &         &          & \rule[-0.5cm]{0cm}{1cm}         \tbnl
% sodium    & \noyau{Na}{11}{23} &         &          & \rule[-0.5cm]{0cm}{1cm}         \tbnl
% oxyg�ne   & \noyau{O}{8}{16}   &         &          & \rule[-0.5cm]{0cm}{1cm}         \tbnl
% hydrog�ne &          &         & 0        & \rule[-0.5cm]{0cm}{1cm}         \tbnl
%           & \noyau{Cl}{17}{35} &         &          &  \rule[-0.5cm]{0cm}{1cm}        \tbnl
%           &          & 8       & \rule[-0.5cm]{0cm}{1cm}         & 16       \tbnl
% \end{tabularx}



\begin{tabularx}{\linewidth}{|>{\mystrut}X|X|X|X|X|}
\hline
% multicolumn pour faire dispara�tre le \mystrut
\multicolumn{1}{|X|}{\emph{nom}} & \emph{symbole}  &
\emph{protons} & \emph{neutrons} & \emph{nucl�ons} \tbnl
carbone   & \noyau{C}{6}{14}   &   &   &    \tbnl
fluor     & \noyau{F}{9}{19}   &   &   &    \tbnl
sodium    & \noyau{Na}{11}{23} &   &   &    \tbnl
oxyg�ne   & \noyau{O}{8}{16}   &   &   &    \tbnl
hydrog�ne &                    &   & 0 &    \tbnl
          & \noyau{Cl}{17}{35} &   &   &    \tbnl
          &                    & 8 &   & 16 \tbnl
\end{tabularx}


\end{exercice}


\vressort{3}


\begin{exercice}{Masse d'un atome de carbone 12}\\
Soit le carbone $12$ not� \noyau{C}{6}{12}.
\begin{enumerate}
\item L'�l�ment carbone peut-il avoir $5$ protons ? Pourquoi ?
\item Calculer la masse du noyau d'un atome de carbone $12$
sachant que la masse d'un nucl�on est $m_n = 1,67.10^{-27}~kg$
\item Calculer la masse des �lectrons de l'atome de carbone 12
sachant que la masse d'un �lectron vaut $m_e = 9,1.10^{-31}~kg$
\item Comparer la masse des �lectrons de l'atome � la masse du noyau.
Que concluez-vous ?
\item En d�duire, sans nouveau calcul, la masse de l'atome de carbone
  $12$.
\end{enumerate}
\end{exercice}

\newpage

\vressort{1}

\begin{exercice}{Couches �lectroniques}\\
Dans l'�tat le plus stable de l'atome, appel� �tat fondamental,
les �lectrons occupent successivement les couches,
en commen�ant par celles qui sont les plus proches du noyau : 
d'abord $K$ puis $L$ puis $M$.

Lorsqu'une couche est pleine, ou encore satur�e, on passe � la suivante.

La derni�re couche occup�e est appel�e couche externe.\\
Toutes les autres sont appel�es couches internes.

\medskip

\noindent
%\begin{tabularx}{\textwidth}{|>{\centering}X|>{\centering}X|>{\centering}X|>{\centering}X|}
\begin{tabularx}{\textwidth}{|>{\mystrut}X|X|X|X|}
\hline
\multicolumn{1}{|X|}{\emph{Symbole de la couche}}       & $K$ & $L$ & $M$  \tbnl
\emph{Nombre maximal d'�lectrons} & $2$ & $8$ & $18$ \tbnl
\end{tabularx}

\medskip

Ainsi, par exemple, l'atome de chlore ($Z=17$) a la configuration �lectronique :
$(K)^2(L)^8(M)^{7}$.

\begin{enumerate}
\item Indiquez le nombre d'�lectrons et donnez la configuration des atomes suivants :
  \begin{enumerate}
  \item \noyau{H}{1}{}
  \item \noyau{O}{8}{}
  \item \noyau{C}{6}{}
  \item \noyau{Ne}{10}{}
  \end{enumerate}
\item Parmi les ions ci-desssous, pr�cisez s'il s'agit d'anions ou de cations.
Indiquez le nombre d'�lectrons et donnez la configuration �lectronique
des ions suivants :
  \begin{enumerate}
  \item $Be^{2+}$ ($Z=4$)
  \item $Al^{3+}$ ($Z=13$)
  \item $O^{2-}$ ($Z=8$)
  \item $F^{-}$ ($Z=9$)
  \end{enumerate}
\end{enumerate}

\end{exercice}


\vressort{5}


\begin{exercice}{Charge d'un atome de Zinc}%\\
\begin{enumerate}
\item Combien de protons l'atome de zinc \noyau{Zn}{30}{65} contient-il ?
\item Combien d'�lectrons comporte-t-il ?
\item Calculer la charge totale des protons
sachant qu'un proton a pour charge $e = 1,6.10^{-19}~C$.
\item Calculer la charge totale des �lectrons
sachant qu'un �lectron a pour charge $-e = -1,6.10^{-19}~C$.
\item En d�duire la charge de l'atome de Zinc.
\item Ce r�sultat est-il identique pour tous les atomes ?
\item A l'issue d'une r�action dite d'oxydation, un atome de zinc $Zn$
  se transforme en un ion $Zn^{2+}$.
  \begin{enumerate}
  \item Donnez l'�quation de cette r�action
  (en faisant intervenir un ou plusieurs �lectrons not�s $e^-$).
  \item Indiquez la charge (en coulomb $C$) de cet ion.
  \end{enumerate}
\end{enumerate}
\end{exercice}

\vressort{3} % Focom�trie lentilles
                                % minces

% lunette astronomique





\chapitre{M�canique}
\ds{Devoir Surveill�}{
%
}

\nomprenomclasse

\setcounter{numexercice}{0}

%\renewcommand{\tabularx}[1]{>{\centering}m{#1}} 

%\newcommand{\tabularxc}[1]{\tabularx{>{\centering}m{#1}}}

\vressort{3}

\begin{exercice}{Connaissance sur l'atome}%\\
\begin{enumerate}
\item De quoi est compos� un atome ?
\item Que signifie les lettres $A$, $Z$ et $X$ dans la repr�sentation \noyau{X}{Z}{A} ?
\item Comment trouve-t-on le nombre de neutrons d'un atome de l'�l�ment pr�c�dent.
\item Si un atome a $5$ protons, combien-a-t-il d'�lectrons ? Pourquoi ?
\item Qu'est-ce qui caract�rise un �l�ment chimique ?
\item Qu'est-ce qu'un isotope ?
\end{enumerate}
\end{exercice}



\vressort{3}



\begin{exercice}{Composition des atomes}\\
En vous aidant du tableau p�riodique des �l�ments,
compl�ter le tableau suivant :

\medskip

\noindent
%\begin{tabularx}{\textwidth}{|>{\centering}X|>{\centering}X|>{\centering}X|>{\centering}X|>{\centering}X|}
% \begin{tabularx}{\linewidth}{|X|X|X|X|X|}
% \hline
% \emph{nom}       & \emph{symbole}  & \emph{protons} & \emph{neutrons}
% & \emph{nucl�ons} \tbnl
% carbone   & \noyau{C}{6}{14}   &         &          & \rule[-0.5cm]{0cm}{1cm}         \tbnl
% fluor     & \noyau{F}{9}{19}   &         &          & \rule[-0.5cm]{0cm}{1cm}         \tbnl
% sodium    & \noyau{Na}{11}{23} &         &          & \rule[-0.5cm]{0cm}{1cm}         \tbnl
% oxyg�ne   & \noyau{O}{8}{16}   &         &          & \rule[-0.5cm]{0cm}{1cm}         \tbnl
% hydrog�ne &          &         & 0        & \rule[-0.5cm]{0cm}{1cm}         \tbnl
%           & \noyau{Cl}{17}{35} &         &          &  \rule[-0.5cm]{0cm}{1cm}        \tbnl
%           &          & 8       & \rule[-0.5cm]{0cm}{1cm}         & 16       \tbnl
% \end{tabularx}



\begin{tabularx}{\linewidth}{|>{\mystrut}X|X|X|X|X|}
\hline
% multicolumn pour faire dispara�tre le \mystrut
\multicolumn{1}{|X|}{\emph{nom}} & \emph{symbole}  &
\emph{protons} & \emph{neutrons} & \emph{nucl�ons} \tbnl
carbone   & \noyau{C}{6}{14}   &   &   &    \tbnl
fluor     & \noyau{F}{9}{19}   &   &   &    \tbnl
sodium    & \noyau{Na}{11}{23} &   &   &    \tbnl
oxyg�ne   & \noyau{O}{8}{16}   &   &   &    \tbnl
hydrog�ne &                    &   & 0 &    \tbnl
          & \noyau{Cl}{17}{35} &   &   &    \tbnl
          &                    & 8 &   & 16 \tbnl
\end{tabularx}


\end{exercice}


\vressort{3}


\begin{exercice}{Masse d'un atome de carbone 12}\\
Soit le carbone $12$ not� \noyau{C}{6}{12}.
\begin{enumerate}
\item L'�l�ment carbone peut-il avoir $5$ protons ? Pourquoi ?
\item Calculer la masse du noyau d'un atome de carbone $12$
sachant que la masse d'un nucl�on est $m_n = 1,67.10^{-27}~kg$
\item Calculer la masse des �lectrons de l'atome de carbone 12
sachant que la masse d'un �lectron vaut $m_e = 9,1.10^{-31}~kg$
\item Comparer la masse des �lectrons de l'atome � la masse du noyau.
Que concluez-vous ?
\item En d�duire, sans nouveau calcul, la masse de l'atome de carbone
  $12$.
\end{enumerate}
\end{exercice}

\newpage

\vressort{1}

\begin{exercice}{Couches �lectroniques}\\
Dans l'�tat le plus stable de l'atome, appel� �tat fondamental,
les �lectrons occupent successivement les couches,
en commen�ant par celles qui sont les plus proches du noyau : 
d'abord $K$ puis $L$ puis $M$.

Lorsqu'une couche est pleine, ou encore satur�e, on passe � la suivante.

La derni�re couche occup�e est appel�e couche externe.\\
Toutes les autres sont appel�es couches internes.

\medskip

\noindent
%\begin{tabularx}{\textwidth}{|>{\centering}X|>{\centering}X|>{\centering}X|>{\centering}X|}
\begin{tabularx}{\textwidth}{|>{\mystrut}X|X|X|X|}
\hline
\multicolumn{1}{|X|}{\emph{Symbole de la couche}}       & $K$ & $L$ & $M$  \tbnl
\emph{Nombre maximal d'�lectrons} & $2$ & $8$ & $18$ \tbnl
\end{tabularx}

\medskip

Ainsi, par exemple, l'atome de chlore ($Z=17$) a la configuration �lectronique :
$(K)^2(L)^8(M)^{7}$.

\begin{enumerate}
\item Indiquez le nombre d'�lectrons et donnez la configuration des atomes suivants :
  \begin{enumerate}
  \item \noyau{H}{1}{}
  \item \noyau{O}{8}{}
  \item \noyau{C}{6}{}
  \item \noyau{Ne}{10}{}
  \end{enumerate}
\item Parmi les ions ci-desssous, pr�cisez s'il s'agit d'anions ou de cations.
Indiquez le nombre d'�lectrons et donnez la configuration �lectronique
des ions suivants :
  \begin{enumerate}
  \item $Be^{2+}$ ($Z=4$)
  \item $Al^{3+}$ ($Z=13$)
  \item $O^{2-}$ ($Z=8$)
  \item $F^{-}$ ($Z=9$)
  \end{enumerate}
\end{enumerate}

\end{exercice}


\vressort{5}


\begin{exercice}{Charge d'un atome de Zinc}%\\
\begin{enumerate}
\item Combien de protons l'atome de zinc \noyau{Zn}{30}{65} contient-il ?
\item Combien d'�lectrons comporte-t-il ?
\item Calculer la charge totale des protons
sachant qu'un proton a pour charge $e = 1,6.10^{-19}~C$.
\item Calculer la charge totale des �lectrons
sachant qu'un �lectron a pour charge $-e = -1,6.10^{-19}~C$.
\item En d�duire la charge de l'atome de Zinc.
\item Ce r�sultat est-il identique pour tous les atomes ?
\item A l'issue d'une r�action dite d'oxydation, un atome de zinc $Zn$
  se transforme en un ion $Zn^{2+}$.
  \begin{enumerate}
  \item Donnez l'�quation de cette r�action
  (en faisant intervenir un ou plusieurs �lectrons not�s $e^-$).
  \item Indiquez la charge (en coulomb $C$) de cet ion.
  \end{enumerate}
\end{enumerate}
\end{exercice}

\vressort{3}


\ds{Devoir Surveill�}{
%
}

\nomprenomclasse

\setcounter{numexercice}{0}

%\renewcommand{\tabularx}[1]{>{\centering}m{#1}} 

%\newcommand{\tabularxc}[1]{\tabularx{>{\centering}m{#1}}}

\vressort{3}

\begin{exercice}{Connaissance sur l'atome}%\\
\begin{enumerate}
\item De quoi est compos� un atome ?
\item Que signifie les lettres $A$, $Z$ et $X$ dans la repr�sentation \noyau{X}{Z}{A} ?
\item Comment trouve-t-on le nombre de neutrons d'un atome de l'�l�ment pr�c�dent.
\item Si un atome a $5$ protons, combien-a-t-il d'�lectrons ? Pourquoi ?
\item Qu'est-ce qui caract�rise un �l�ment chimique ?
\item Qu'est-ce qu'un isotope ?
\end{enumerate}
\end{exercice}



\vressort{3}



\begin{exercice}{Composition des atomes}\\
En vous aidant du tableau p�riodique des �l�ments,
compl�ter le tableau suivant :

\medskip

\noindent
%\begin{tabularx}{\textwidth}{|>{\centering}X|>{\centering}X|>{\centering}X|>{\centering}X|>{\centering}X|}
% \begin{tabularx}{\linewidth}{|X|X|X|X|X|}
% \hline
% \emph{nom}       & \emph{symbole}  & \emph{protons} & \emph{neutrons}
% & \emph{nucl�ons} \tbnl
% carbone   & \noyau{C}{6}{14}   &         &          & \rule[-0.5cm]{0cm}{1cm}         \tbnl
% fluor     & \noyau{F}{9}{19}   &         &          & \rule[-0.5cm]{0cm}{1cm}         \tbnl
% sodium    & \noyau{Na}{11}{23} &         &          & \rule[-0.5cm]{0cm}{1cm}         \tbnl
% oxyg�ne   & \noyau{O}{8}{16}   &         &          & \rule[-0.5cm]{0cm}{1cm}         \tbnl
% hydrog�ne &          &         & 0        & \rule[-0.5cm]{0cm}{1cm}         \tbnl
%           & \noyau{Cl}{17}{35} &         &          &  \rule[-0.5cm]{0cm}{1cm}        \tbnl
%           &          & 8       & \rule[-0.5cm]{0cm}{1cm}         & 16       \tbnl
% \end{tabularx}



\begin{tabularx}{\linewidth}{|>{\mystrut}X|X|X|X|X|}
\hline
% multicolumn pour faire dispara�tre le \mystrut
\multicolumn{1}{|X|}{\emph{nom}} & \emph{symbole}  &
\emph{protons} & \emph{neutrons} & \emph{nucl�ons} \tbnl
carbone   & \noyau{C}{6}{14}   &   &   &    \tbnl
fluor     & \noyau{F}{9}{19}   &   &   &    \tbnl
sodium    & \noyau{Na}{11}{23} &   &   &    \tbnl
oxyg�ne   & \noyau{O}{8}{16}   &   &   &    \tbnl
hydrog�ne &                    &   & 0 &    \tbnl
          & \noyau{Cl}{17}{35} &   &   &    \tbnl
          &                    & 8 &   & 16 \tbnl
\end{tabularx}


\end{exercice}


\vressort{3}


\begin{exercice}{Masse d'un atome de carbone 12}\\
Soit le carbone $12$ not� \noyau{C}{6}{12}.
\begin{enumerate}
\item L'�l�ment carbone peut-il avoir $5$ protons ? Pourquoi ?
\item Calculer la masse du noyau d'un atome de carbone $12$
sachant que la masse d'un nucl�on est $m_n = 1,67.10^{-27}~kg$
\item Calculer la masse des �lectrons de l'atome de carbone 12
sachant que la masse d'un �lectron vaut $m_e = 9,1.10^{-31}~kg$
\item Comparer la masse des �lectrons de l'atome � la masse du noyau.
Que concluez-vous ?
\item En d�duire, sans nouveau calcul, la masse de l'atome de carbone
  $12$.
\end{enumerate}
\end{exercice}

\newpage

\vressort{1}

\begin{exercice}{Couches �lectroniques}\\
Dans l'�tat le plus stable de l'atome, appel� �tat fondamental,
les �lectrons occupent successivement les couches,
en commen�ant par celles qui sont les plus proches du noyau : 
d'abord $K$ puis $L$ puis $M$.

Lorsqu'une couche est pleine, ou encore satur�e, on passe � la suivante.

La derni�re couche occup�e est appel�e couche externe.\\
Toutes les autres sont appel�es couches internes.

\medskip

\noindent
%\begin{tabularx}{\textwidth}{|>{\centering}X|>{\centering}X|>{\centering}X|>{\centering}X|}
\begin{tabularx}{\textwidth}{|>{\mystrut}X|X|X|X|}
\hline
\multicolumn{1}{|X|}{\emph{Symbole de la couche}}       & $K$ & $L$ & $M$  \tbnl
\emph{Nombre maximal d'�lectrons} & $2$ & $8$ & $18$ \tbnl
\end{tabularx}

\medskip

Ainsi, par exemple, l'atome de chlore ($Z=17$) a la configuration �lectronique :
$(K)^2(L)^8(M)^{7}$.

\begin{enumerate}
\item Indiquez le nombre d'�lectrons et donnez la configuration des atomes suivants :
  \begin{enumerate}
  \item \noyau{H}{1}{}
  \item \noyau{O}{8}{}
  \item \noyau{C}{6}{}
  \item \noyau{Ne}{10}{}
  \end{enumerate}
\item Parmi les ions ci-desssous, pr�cisez s'il s'agit d'anions ou de cations.
Indiquez le nombre d'�lectrons et donnez la configuration �lectronique
des ions suivants :
  \begin{enumerate}
  \item $Be^{2+}$ ($Z=4$)
  \item $Al^{3+}$ ($Z=13$)
  \item $O^{2-}$ ($Z=8$)
  \item $F^{-}$ ($Z=9$)
  \end{enumerate}
\end{enumerate}

\end{exercice}


\vressort{5}


\begin{exercice}{Charge d'un atome de Zinc}%\\
\begin{enumerate}
\item Combien de protons l'atome de zinc \noyau{Zn}{30}{65} contient-il ?
\item Combien d'�lectrons comporte-t-il ?
\item Calculer la charge totale des protons
sachant qu'un proton a pour charge $e = 1,6.10^{-19}~C$.
\item Calculer la charge totale des �lectrons
sachant qu'un �lectron a pour charge $-e = -1,6.10^{-19}~C$.
\item En d�duire la charge de l'atome de Zinc.
\item Ce r�sultat est-il identique pour tous les atomes ?
\item A l'issue d'une r�action dite d'oxydation, un atome de zinc $Zn$
  se transforme en un ion $Zn^{2+}$.
  \begin{enumerate}
  \item Donnez l'�quation de cette r�action
  (en faisant intervenir un ou plusieurs �lectrons not�s $e^-$).
  \item Indiquez la charge (en coulomb $C$) de cet ion.
  \end{enumerate}
\end{enumerate}
\end{exercice}

\vressort{3}
\ds{Devoir Surveill�}{
%
}

\nomprenomclasse

\setcounter{numexercice}{0}

%\renewcommand{\tabularx}[1]{>{\centering}m{#1}} 

%\newcommand{\tabularxc}[1]{\tabularx{>{\centering}m{#1}}}

\vressort{3}

\begin{exercice}{Connaissance sur l'atome}%\\
\begin{enumerate}
\item De quoi est compos� un atome ?
\item Que signifie les lettres $A$, $Z$ et $X$ dans la repr�sentation \noyau{X}{Z}{A} ?
\item Comment trouve-t-on le nombre de neutrons d'un atome de l'�l�ment pr�c�dent.
\item Si un atome a $5$ protons, combien-a-t-il d'�lectrons ? Pourquoi ?
\item Qu'est-ce qui caract�rise un �l�ment chimique ?
\item Qu'est-ce qu'un isotope ?
\end{enumerate}
\end{exercice}



\vressort{3}



\begin{exercice}{Composition des atomes}\\
En vous aidant du tableau p�riodique des �l�ments,
compl�ter le tableau suivant :

\medskip

\noindent
%\begin{tabularx}{\textwidth}{|>{\centering}X|>{\centering}X|>{\centering}X|>{\centering}X|>{\centering}X|}
% \begin{tabularx}{\linewidth}{|X|X|X|X|X|}
% \hline
% \emph{nom}       & \emph{symbole}  & \emph{protons} & \emph{neutrons}
% & \emph{nucl�ons} \tbnl
% carbone   & \noyau{C}{6}{14}   &         &          & \rule[-0.5cm]{0cm}{1cm}         \tbnl
% fluor     & \noyau{F}{9}{19}   &         &          & \rule[-0.5cm]{0cm}{1cm}         \tbnl
% sodium    & \noyau{Na}{11}{23} &         &          & \rule[-0.5cm]{0cm}{1cm}         \tbnl
% oxyg�ne   & \noyau{O}{8}{16}   &         &          & \rule[-0.5cm]{0cm}{1cm}         \tbnl
% hydrog�ne &          &         & 0        & \rule[-0.5cm]{0cm}{1cm}         \tbnl
%           & \noyau{Cl}{17}{35} &         &          &  \rule[-0.5cm]{0cm}{1cm}        \tbnl
%           &          & 8       & \rule[-0.5cm]{0cm}{1cm}         & 16       \tbnl
% \end{tabularx}



\begin{tabularx}{\linewidth}{|>{\mystrut}X|X|X|X|X|}
\hline
% multicolumn pour faire dispara�tre le \mystrut
\multicolumn{1}{|X|}{\emph{nom}} & \emph{symbole}  &
\emph{protons} & \emph{neutrons} & \emph{nucl�ons} \tbnl
carbone   & \noyau{C}{6}{14}   &   &   &    \tbnl
fluor     & \noyau{F}{9}{19}   &   &   &    \tbnl
sodium    & \noyau{Na}{11}{23} &   &   &    \tbnl
oxyg�ne   & \noyau{O}{8}{16}   &   &   &    \tbnl
hydrog�ne &                    &   & 0 &    \tbnl
          & \noyau{Cl}{17}{35} &   &   &    \tbnl
          &                    & 8 &   & 16 \tbnl
\end{tabularx}


\end{exercice}


\vressort{3}


\begin{exercice}{Masse d'un atome de carbone 12}\\
Soit le carbone $12$ not� \noyau{C}{6}{12}.
\begin{enumerate}
\item L'�l�ment carbone peut-il avoir $5$ protons ? Pourquoi ?
\item Calculer la masse du noyau d'un atome de carbone $12$
sachant que la masse d'un nucl�on est $m_n = 1,67.10^{-27}~kg$
\item Calculer la masse des �lectrons de l'atome de carbone 12
sachant que la masse d'un �lectron vaut $m_e = 9,1.10^{-31}~kg$
\item Comparer la masse des �lectrons de l'atome � la masse du noyau.
Que concluez-vous ?
\item En d�duire, sans nouveau calcul, la masse de l'atome de carbone
  $12$.
\end{enumerate}
\end{exercice}

\newpage

\vressort{1}

\begin{exercice}{Couches �lectroniques}\\
Dans l'�tat le plus stable de l'atome, appel� �tat fondamental,
les �lectrons occupent successivement les couches,
en commen�ant par celles qui sont les plus proches du noyau : 
d'abord $K$ puis $L$ puis $M$.

Lorsqu'une couche est pleine, ou encore satur�e, on passe � la suivante.

La derni�re couche occup�e est appel�e couche externe.\\
Toutes les autres sont appel�es couches internes.

\medskip

\noindent
%\begin{tabularx}{\textwidth}{|>{\centering}X|>{\centering}X|>{\centering}X|>{\centering}X|}
\begin{tabularx}{\textwidth}{|>{\mystrut}X|X|X|X|}
\hline
\multicolumn{1}{|X|}{\emph{Symbole de la couche}}       & $K$ & $L$ & $M$  \tbnl
\emph{Nombre maximal d'�lectrons} & $2$ & $8$ & $18$ \tbnl
\end{tabularx}

\medskip

Ainsi, par exemple, l'atome de chlore ($Z=17$) a la configuration �lectronique :
$(K)^2(L)^8(M)^{7}$.

\begin{enumerate}
\item Indiquez le nombre d'�lectrons et donnez la configuration des atomes suivants :
  \begin{enumerate}
  \item \noyau{H}{1}{}
  \item \noyau{O}{8}{}
  \item \noyau{C}{6}{}
  \item \noyau{Ne}{10}{}
  \end{enumerate}
\item Parmi les ions ci-desssous, pr�cisez s'il s'agit d'anions ou de cations.
Indiquez le nombre d'�lectrons et donnez la configuration �lectronique
des ions suivants :
  \begin{enumerate}
  \item $Be^{2+}$ ($Z=4$)
  \item $Al^{3+}$ ($Z=13$)
  \item $O^{2-}$ ($Z=8$)
  \item $F^{-}$ ($Z=9$)
  \end{enumerate}
\end{enumerate}

\end{exercice}


\vressort{5}


\begin{exercice}{Charge d'un atome de Zinc}%\\
\begin{enumerate}
\item Combien de protons l'atome de zinc \noyau{Zn}{30}{65} contient-il ?
\item Combien d'�lectrons comporte-t-il ?
\item Calculer la charge totale des protons
sachant qu'un proton a pour charge $e = 1,6.10^{-19}~C$.
\item Calculer la charge totale des �lectrons
sachant qu'un �lectron a pour charge $-e = -1,6.10^{-19}~C$.
\item En d�duire la charge de l'atome de Zinc.
\item Ce r�sultat est-il identique pour tous les atomes ?
\item A l'issue d'une r�action dite d'oxydation, un atome de zinc $Zn$
  se transforme en un ion $Zn^{2+}$.
  \begin{enumerate}
  \item Donnez l'�quation de cette r�action
  (en faisant intervenir un ou plusieurs �lectrons not�s $e^-$).
  \item Indiquez la charge (en coulomb $C$) de cet ion.
  \end{enumerate}
\end{enumerate}
\end{exercice}

\vressort{3}

\ds{Devoir Surveill�}{
%
}

\nomprenomclasse

\setcounter{numexercice}{0}

%\renewcommand{\tabularx}[1]{>{\centering}m{#1}} 

%\newcommand{\tabularxc}[1]{\tabularx{>{\centering}m{#1}}}

\vressort{3}

\begin{exercice}{Connaissance sur l'atome}%\\
\begin{enumerate}
\item De quoi est compos� un atome ?
\item Que signifie les lettres $A$, $Z$ et $X$ dans la repr�sentation \noyau{X}{Z}{A} ?
\item Comment trouve-t-on le nombre de neutrons d'un atome de l'�l�ment pr�c�dent.
\item Si un atome a $5$ protons, combien-a-t-il d'�lectrons ? Pourquoi ?
\item Qu'est-ce qui caract�rise un �l�ment chimique ?
\item Qu'est-ce qu'un isotope ?
\end{enumerate}
\end{exercice}



\vressort{3}



\begin{exercice}{Composition des atomes}\\
En vous aidant du tableau p�riodique des �l�ments,
compl�ter le tableau suivant :

\medskip

\noindent
%\begin{tabularx}{\textwidth}{|>{\centering}X|>{\centering}X|>{\centering}X|>{\centering}X|>{\centering}X|}
% \begin{tabularx}{\linewidth}{|X|X|X|X|X|}
% \hline
% \emph{nom}       & \emph{symbole}  & \emph{protons} & \emph{neutrons}
% & \emph{nucl�ons} \tbnl
% carbone   & \noyau{C}{6}{14}   &         &          & \rule[-0.5cm]{0cm}{1cm}         \tbnl
% fluor     & \noyau{F}{9}{19}   &         &          & \rule[-0.5cm]{0cm}{1cm}         \tbnl
% sodium    & \noyau{Na}{11}{23} &         &          & \rule[-0.5cm]{0cm}{1cm}         \tbnl
% oxyg�ne   & \noyau{O}{8}{16}   &         &          & \rule[-0.5cm]{0cm}{1cm}         \tbnl
% hydrog�ne &          &         & 0        & \rule[-0.5cm]{0cm}{1cm}         \tbnl
%           & \noyau{Cl}{17}{35} &         &          &  \rule[-0.5cm]{0cm}{1cm}        \tbnl
%           &          & 8       & \rule[-0.5cm]{0cm}{1cm}         & 16       \tbnl
% \end{tabularx}



\begin{tabularx}{\linewidth}{|>{\mystrut}X|X|X|X|X|}
\hline
% multicolumn pour faire dispara�tre le \mystrut
\multicolumn{1}{|X|}{\emph{nom}} & \emph{symbole}  &
\emph{protons} & \emph{neutrons} & \emph{nucl�ons} \tbnl
carbone   & \noyau{C}{6}{14}   &   &   &    \tbnl
fluor     & \noyau{F}{9}{19}   &   &   &    \tbnl
sodium    & \noyau{Na}{11}{23} &   &   &    \tbnl
oxyg�ne   & \noyau{O}{8}{16}   &   &   &    \tbnl
hydrog�ne &                    &   & 0 &    \tbnl
          & \noyau{Cl}{17}{35} &   &   &    \tbnl
          &                    & 8 &   & 16 \tbnl
\end{tabularx}


\end{exercice}


\vressort{3}


\begin{exercice}{Masse d'un atome de carbone 12}\\
Soit le carbone $12$ not� \noyau{C}{6}{12}.
\begin{enumerate}
\item L'�l�ment carbone peut-il avoir $5$ protons ? Pourquoi ?
\item Calculer la masse du noyau d'un atome de carbone $12$
sachant que la masse d'un nucl�on est $m_n = 1,67.10^{-27}~kg$
\item Calculer la masse des �lectrons de l'atome de carbone 12
sachant que la masse d'un �lectron vaut $m_e = 9,1.10^{-31}~kg$
\item Comparer la masse des �lectrons de l'atome � la masse du noyau.
Que concluez-vous ?
\item En d�duire, sans nouveau calcul, la masse de l'atome de carbone
  $12$.
\end{enumerate}
\end{exercice}

\newpage

\vressort{1}

\begin{exercice}{Couches �lectroniques}\\
Dans l'�tat le plus stable de l'atome, appel� �tat fondamental,
les �lectrons occupent successivement les couches,
en commen�ant par celles qui sont les plus proches du noyau : 
d'abord $K$ puis $L$ puis $M$.

Lorsqu'une couche est pleine, ou encore satur�e, on passe � la suivante.

La derni�re couche occup�e est appel�e couche externe.\\
Toutes les autres sont appel�es couches internes.

\medskip

\noindent
%\begin{tabularx}{\textwidth}{|>{\centering}X|>{\centering}X|>{\centering}X|>{\centering}X|}
\begin{tabularx}{\textwidth}{|>{\mystrut}X|X|X|X|}
\hline
\multicolumn{1}{|X|}{\emph{Symbole de la couche}}       & $K$ & $L$ & $M$  \tbnl
\emph{Nombre maximal d'�lectrons} & $2$ & $8$ & $18$ \tbnl
\end{tabularx}

\medskip

Ainsi, par exemple, l'atome de chlore ($Z=17$) a la configuration �lectronique :
$(K)^2(L)^8(M)^{7}$.

\begin{enumerate}
\item Indiquez le nombre d'�lectrons et donnez la configuration des atomes suivants :
  \begin{enumerate}
  \item \noyau{H}{1}{}
  \item \noyau{O}{8}{}
  \item \noyau{C}{6}{}
  \item \noyau{Ne}{10}{}
  \end{enumerate}
\item Parmi les ions ci-desssous, pr�cisez s'il s'agit d'anions ou de cations.
Indiquez le nombre d'�lectrons et donnez la configuration �lectronique
des ions suivants :
  \begin{enumerate}
  \item $Be^{2+}$ ($Z=4$)
  \item $Al^{3+}$ ($Z=13$)
  \item $O^{2-}$ ($Z=8$)
  \item $F^{-}$ ($Z=9$)
  \end{enumerate}
\end{enumerate}

\end{exercice}


\vressort{5}


\begin{exercice}{Charge d'un atome de Zinc}%\\
\begin{enumerate}
\item Combien de protons l'atome de zinc \noyau{Zn}{30}{65} contient-il ?
\item Combien d'�lectrons comporte-t-il ?
\item Calculer la charge totale des protons
sachant qu'un proton a pour charge $e = 1,6.10^{-19}~C$.
\item Calculer la charge totale des �lectrons
sachant qu'un �lectron a pour charge $-e = -1,6.10^{-19}~C$.
\item En d�duire la charge de l'atome de Zinc.
\item Ce r�sultat est-il identique pour tous les atomes ?
\item A l'issue d'une r�action dite d'oxydation, un atome de zinc $Zn$
  se transforme en un ion $Zn^{2+}$.
  \begin{enumerate}
  \item Donnez l'�quation de cette r�action
  (en faisant intervenir un ou plusieurs �lectrons not�s $e^-$).
  \item Indiquez la charge (en coulomb $C$) de cet ion.
  \end{enumerate}
\end{enumerate}
\end{exercice}

\vressort{3}





%\chapitre{Thermique}


\classe{Premi�re\ \\
Scientifique\ \\
Partie Physique}{Premi�re Scientifique - Partie Physique}



\chapitre{Int�ractions fondamentales}
\inclure{interactions/cours_particules_elementaires}    % Mettre les fig
\inclure{interactions/cours_interactions_fondamentales} % Mettre les fig

\inclure{electrostat/doc_electrostat}



\chapitre{Vitesses et mouvements}
\inclure{meca/cours_cinematique_intro} % Mettre les fig
\inclure{tp_prem_s_phys/tp02_mouvement}
\inclure{tp_prem_s_phys/tp02_mouvement_doc}
\inclure{tp_prem_s_phys/tp03_rotation}
\inclure{tp_prem_s_phys/tp04_mvt_parabolique}


\chapitre{Forces}
\inclure{meca/cours_forces} % Mettre les fig
\inclure{meca/tp_poids_ressort_archimede}
\inclure{meca/tp_equilibre_solide}
\inclure{meca/doc_rapporteurs}


\chapitre{Lois de Newton}
\inclure{meca/cours_lois_newton}  % Mettre les fig
\inclure{tp_prem_s_phys/tp08_deuxieme_loi_newton_micromega}
\inclure{tp_prem_s_phys/tp08_deuxieme_loi_newton_parabole_video}


\chapitre{Travail d'une force}
\inclure{meca/cours_travail_force}  % Mettre les fig


\chapitre{\'Energie cin�tique}
\inclure{meca/cours_energie_cinetique}  % Mettre les fig
\inclure{tp_prem_s_phys/tp09_theo_energie_cinetique}


\chapitre{\'Energie m�canique}
\inclure{meca/cours_energie_potentielle_mecanique}  % Mettre les fig
\inclure{tp_prem_s_phys/tp10_theo_energie_mecanique}


\chapitre{Transferts thermiques} %Calorim�trie}
\inclure{thermodynamique/cours_transferts_energie}
\inclure{thermodynamique/tp1_transferts_thermiques_capa_therm_calo}
\inclure{thermodynamique/tp2_capa_thermique_massique_metal}
\inclure{thermodynamique/tp3_chaleur_latente_fusion_glace}
\inclure{thermodynamique/doc_balance_metler_p1200}


\chapitre{\'Electricit�}
% Cours : Rappels de coll�ge
\inclure{elec/cours_rappel_college}

\inclure{elec/fiche_methode_montage} % document fiche
    % m�thode montage
    % utilisation d'un multim�tre, ...


% TP Potentiel le long d'un circuit ?


\inclure{elec/cours_elec_gene_recep} % cours elec g�n�rateurs, r�cepteurs
\inclure{elec/tp_carac_gene} % tp caract�ristique d'un g�n�rateur
\inclure{elec/tp_carac_recep} % tp caract�ristique d'un
                                % r�cepteur (�lectrolyseur)

% Cours : Circuits �lectriques




\chapitre{Optique g�om�trique}
\inclure{opt/cours_intro_opt_geom} % conditions de visibilit�
\inclure{opt/cours_miroir_plan} % miroir plan
\inclure{opt/cours_lentilles_minces} % lentilles minces
\inclure{opt/doc_constructions_lentilles}
\inclure{opt/tp_lentilles_minces_conv_conjug} % Lentilles (Images, rel de conjugaison)
\inclure{opt/tp_lentilles_minces_conv_foco} % Focom�trie lentilles minces
% lunette astronomique
% autres instruments d'optique ?




\chapitre{\'Electromagn�tisme}
\inclure{electromag/tp_champ_magnetique} % Champ magn�tique
\inclure{electromag/tp_bobines_helmholtz} % Bobines de Helmholtz
\inclure{electromag/tp_force_laplace} % Force de Laplace
\inclure{electromag/tp_force_lorentz} % Force de Lorentz





\chapitre{Devoir Surveill�} % 1S
\inclure{ds_2005_2006_prem_s/ds1}
\inclure{ds_2005_2006_prem_s/ds2}


% � ajouter si le dernier document contient un nb impair de pages
%\newpage
\classe{Premi�re\ \\
Scientifique\ \\
Partie Chimie}{Premi�re Scientifique - Partie Chimie}


\chapitre{Grandeurs physiques et quantit� de mati�re}
\ds{Devoir Surveill�}{
%
}

\nomprenomclasse

\setcounter{numexercice}{0}

%\renewcommand{\tabularx}[1]{>{\centering}m{#1}} 

%\newcommand{\tabularxc}[1]{\tabularx{>{\centering}m{#1}}}

\vressort{3}

\begin{exercice}{Connaissance sur l'atome}%\\
\begin{enumerate}
\item De quoi est compos� un atome ?
\item Que signifie les lettres $A$, $Z$ et $X$ dans la repr�sentation \noyau{X}{Z}{A} ?
\item Comment trouve-t-on le nombre de neutrons d'un atome de l'�l�ment pr�c�dent.
\item Si un atome a $5$ protons, combien-a-t-il d'�lectrons ? Pourquoi ?
\item Qu'est-ce qui caract�rise un �l�ment chimique ?
\item Qu'est-ce qu'un isotope ?
\end{enumerate}
\end{exercice}



\vressort{3}



\begin{exercice}{Composition des atomes}\\
En vous aidant du tableau p�riodique des �l�ments,
compl�ter le tableau suivant :

\medskip

\noindent
%\begin{tabularx}{\textwidth}{|>{\centering}X|>{\centering}X|>{\centering}X|>{\centering}X|>{\centering}X|}
% \begin{tabularx}{\linewidth}{|X|X|X|X|X|}
% \hline
% \emph{nom}       & \emph{symbole}  & \emph{protons} & \emph{neutrons}
% & \emph{nucl�ons} \tbnl
% carbone   & \noyau{C}{6}{14}   &         &          & \rule[-0.5cm]{0cm}{1cm}         \tbnl
% fluor     & \noyau{F}{9}{19}   &         &          & \rule[-0.5cm]{0cm}{1cm}         \tbnl
% sodium    & \noyau{Na}{11}{23} &         &          & \rule[-0.5cm]{0cm}{1cm}         \tbnl
% oxyg�ne   & \noyau{O}{8}{16}   &         &          & \rule[-0.5cm]{0cm}{1cm}         \tbnl
% hydrog�ne &          &         & 0        & \rule[-0.5cm]{0cm}{1cm}         \tbnl
%           & \noyau{Cl}{17}{35} &         &          &  \rule[-0.5cm]{0cm}{1cm}        \tbnl
%           &          & 8       & \rule[-0.5cm]{0cm}{1cm}         & 16       \tbnl
% \end{tabularx}



\begin{tabularx}{\linewidth}{|>{\mystrut}X|X|X|X|X|}
\hline
% multicolumn pour faire dispara�tre le \mystrut
\multicolumn{1}{|X|}{\emph{nom}} & \emph{symbole}  &
\emph{protons} & \emph{neutrons} & \emph{nucl�ons} \tbnl
carbone   & \noyau{C}{6}{14}   &   &   &    \tbnl
fluor     & \noyau{F}{9}{19}   &   &   &    \tbnl
sodium    & \noyau{Na}{11}{23} &   &   &    \tbnl
oxyg�ne   & \noyau{O}{8}{16}   &   &   &    \tbnl
hydrog�ne &                    &   & 0 &    \tbnl
          & \noyau{Cl}{17}{35} &   &   &    \tbnl
          &                    & 8 &   & 16 \tbnl
\end{tabularx}


\end{exercice}


\vressort{3}


\begin{exercice}{Masse d'un atome de carbone 12}\\
Soit le carbone $12$ not� \noyau{C}{6}{12}.
\begin{enumerate}
\item L'�l�ment carbone peut-il avoir $5$ protons ? Pourquoi ?
\item Calculer la masse du noyau d'un atome de carbone $12$
sachant que la masse d'un nucl�on est $m_n = 1,67.10^{-27}~kg$
\item Calculer la masse des �lectrons de l'atome de carbone 12
sachant que la masse d'un �lectron vaut $m_e = 9,1.10^{-31}~kg$
\item Comparer la masse des �lectrons de l'atome � la masse du noyau.
Que concluez-vous ?
\item En d�duire, sans nouveau calcul, la masse de l'atome de carbone
  $12$.
\end{enumerate}
\end{exercice}

\newpage

\vressort{1}

\begin{exercice}{Couches �lectroniques}\\
Dans l'�tat le plus stable de l'atome, appel� �tat fondamental,
les �lectrons occupent successivement les couches,
en commen�ant par celles qui sont les plus proches du noyau : 
d'abord $K$ puis $L$ puis $M$.

Lorsqu'une couche est pleine, ou encore satur�e, on passe � la suivante.

La derni�re couche occup�e est appel�e couche externe.\\
Toutes les autres sont appel�es couches internes.

\medskip

\noindent
%\begin{tabularx}{\textwidth}{|>{\centering}X|>{\centering}X|>{\centering}X|>{\centering}X|}
\begin{tabularx}{\textwidth}{|>{\mystrut}X|X|X|X|}
\hline
\multicolumn{1}{|X|}{\emph{Symbole de la couche}}       & $K$ & $L$ & $M$  \tbnl
\emph{Nombre maximal d'�lectrons} & $2$ & $8$ & $18$ \tbnl
\end{tabularx}

\medskip

Ainsi, par exemple, l'atome de chlore ($Z=17$) a la configuration �lectronique :
$(K)^2(L)^8(M)^{7}$.

\begin{enumerate}
\item Indiquez le nombre d'�lectrons et donnez la configuration des atomes suivants :
  \begin{enumerate}
  \item \noyau{H}{1}{}
  \item \noyau{O}{8}{}
  \item \noyau{C}{6}{}
  \item \noyau{Ne}{10}{}
  \end{enumerate}
\item Parmi les ions ci-desssous, pr�cisez s'il s'agit d'anions ou de cations.
Indiquez le nombre d'�lectrons et donnez la configuration �lectronique
des ions suivants :
  \begin{enumerate}
  \item $Be^{2+}$ ($Z=4$)
  \item $Al^{3+}$ ($Z=13$)
  \item $O^{2-}$ ($Z=8$)
  \item $F^{-}$ ($Z=9$)
  \end{enumerate}
\end{enumerate}

\end{exercice}


\vressort{5}


\begin{exercice}{Charge d'un atome de Zinc}%\\
\begin{enumerate}
\item Combien de protons l'atome de zinc \noyau{Zn}{30}{65} contient-il ?
\item Combien d'�lectrons comporte-t-il ?
\item Calculer la charge totale des protons
sachant qu'un proton a pour charge $e = 1,6.10^{-19}~C$.
\item Calculer la charge totale des �lectrons
sachant qu'un �lectron a pour charge $-e = -1,6.10^{-19}~C$.
\item En d�duire la charge de l'atome de Zinc.
\item Ce r�sultat est-il identique pour tous les atomes ?
\item A l'issue d'une r�action dite d'oxydation, un atome de zinc $Zn$
  se transforme en un ion $Zn^{2+}$.
  \begin{enumerate}
  \item Donnez l'�quation de cette r�action
  (en faisant intervenir un ou plusieurs �lectrons not�s $e^-$).
  \item Indiquez la charge (en coulomb $C$) de cet ion.
  \end{enumerate}
\end{enumerate}
\end{exercice}

\vressort{3}
\ds{Devoir Surveill�}{
%
}

\nomprenomclasse

\setcounter{numexercice}{0}

%\renewcommand{\tabularx}[1]{>{\centering}m{#1}} 

%\newcommand{\tabularxc}[1]{\tabularx{>{\centering}m{#1}}}

\vressort{3}

\begin{exercice}{Connaissance sur l'atome}%\\
\begin{enumerate}
\item De quoi est compos� un atome ?
\item Que signifie les lettres $A$, $Z$ et $X$ dans la repr�sentation \noyau{X}{Z}{A} ?
\item Comment trouve-t-on le nombre de neutrons d'un atome de l'�l�ment pr�c�dent.
\item Si un atome a $5$ protons, combien-a-t-il d'�lectrons ? Pourquoi ?
\item Qu'est-ce qui caract�rise un �l�ment chimique ?
\item Qu'est-ce qu'un isotope ?
\end{enumerate}
\end{exercice}



\vressort{3}



\begin{exercice}{Composition des atomes}\\
En vous aidant du tableau p�riodique des �l�ments,
compl�ter le tableau suivant :

\medskip

\noindent
%\begin{tabularx}{\textwidth}{|>{\centering}X|>{\centering}X|>{\centering}X|>{\centering}X|>{\centering}X|}
% \begin{tabularx}{\linewidth}{|X|X|X|X|X|}
% \hline
% \emph{nom}       & \emph{symbole}  & \emph{protons} & \emph{neutrons}
% & \emph{nucl�ons} \tbnl
% carbone   & \noyau{C}{6}{14}   &         &          & \rule[-0.5cm]{0cm}{1cm}         \tbnl
% fluor     & \noyau{F}{9}{19}   &         &          & \rule[-0.5cm]{0cm}{1cm}         \tbnl
% sodium    & \noyau{Na}{11}{23} &         &          & \rule[-0.5cm]{0cm}{1cm}         \tbnl
% oxyg�ne   & \noyau{O}{8}{16}   &         &          & \rule[-0.5cm]{0cm}{1cm}         \tbnl
% hydrog�ne &          &         & 0        & \rule[-0.5cm]{0cm}{1cm}         \tbnl
%           & \noyau{Cl}{17}{35} &         &          &  \rule[-0.5cm]{0cm}{1cm}        \tbnl
%           &          & 8       & \rule[-0.5cm]{0cm}{1cm}         & 16       \tbnl
% \end{tabularx}



\begin{tabularx}{\linewidth}{|>{\mystrut}X|X|X|X|X|}
\hline
% multicolumn pour faire dispara�tre le \mystrut
\multicolumn{1}{|X|}{\emph{nom}} & \emph{symbole}  &
\emph{protons} & \emph{neutrons} & \emph{nucl�ons} \tbnl
carbone   & \noyau{C}{6}{14}   &   &   &    \tbnl
fluor     & \noyau{F}{9}{19}   &   &   &    \tbnl
sodium    & \noyau{Na}{11}{23} &   &   &    \tbnl
oxyg�ne   & \noyau{O}{8}{16}   &   &   &    \tbnl
hydrog�ne &                    &   & 0 &    \tbnl
          & \noyau{Cl}{17}{35} &   &   &    \tbnl
          &                    & 8 &   & 16 \tbnl
\end{tabularx}


\end{exercice}


\vressort{3}


\begin{exercice}{Masse d'un atome de carbone 12}\\
Soit le carbone $12$ not� \noyau{C}{6}{12}.
\begin{enumerate}
\item L'�l�ment carbone peut-il avoir $5$ protons ? Pourquoi ?
\item Calculer la masse du noyau d'un atome de carbone $12$
sachant que la masse d'un nucl�on est $m_n = 1,67.10^{-27}~kg$
\item Calculer la masse des �lectrons de l'atome de carbone 12
sachant que la masse d'un �lectron vaut $m_e = 9,1.10^{-31}~kg$
\item Comparer la masse des �lectrons de l'atome � la masse du noyau.
Que concluez-vous ?
\item En d�duire, sans nouveau calcul, la masse de l'atome de carbone
  $12$.
\end{enumerate}
\end{exercice}

\newpage

\vressort{1}

\begin{exercice}{Couches �lectroniques}\\
Dans l'�tat le plus stable de l'atome, appel� �tat fondamental,
les �lectrons occupent successivement les couches,
en commen�ant par celles qui sont les plus proches du noyau : 
d'abord $K$ puis $L$ puis $M$.

Lorsqu'une couche est pleine, ou encore satur�e, on passe � la suivante.

La derni�re couche occup�e est appel�e couche externe.\\
Toutes les autres sont appel�es couches internes.

\medskip

\noindent
%\begin{tabularx}{\textwidth}{|>{\centering}X|>{\centering}X|>{\centering}X|>{\centering}X|}
\begin{tabularx}{\textwidth}{|>{\mystrut}X|X|X|X|}
\hline
\multicolumn{1}{|X|}{\emph{Symbole de la couche}}       & $K$ & $L$ & $M$  \tbnl
\emph{Nombre maximal d'�lectrons} & $2$ & $8$ & $18$ \tbnl
\end{tabularx}

\medskip

Ainsi, par exemple, l'atome de chlore ($Z=17$) a la configuration �lectronique :
$(K)^2(L)^8(M)^{7}$.

\begin{enumerate}
\item Indiquez le nombre d'�lectrons et donnez la configuration des atomes suivants :
  \begin{enumerate}
  \item \noyau{H}{1}{}
  \item \noyau{O}{8}{}
  \item \noyau{C}{6}{}
  \item \noyau{Ne}{10}{}
  \end{enumerate}
\item Parmi les ions ci-desssous, pr�cisez s'il s'agit d'anions ou de cations.
Indiquez le nombre d'�lectrons et donnez la configuration �lectronique
des ions suivants :
  \begin{enumerate}
  \item $Be^{2+}$ ($Z=4$)
  \item $Al^{3+}$ ($Z=13$)
  \item $O^{2-}$ ($Z=8$)
  \item $F^{-}$ ($Z=9$)
  \end{enumerate}
\end{enumerate}

\end{exercice}


\vressort{5}


\begin{exercice}{Charge d'un atome de Zinc}%\\
\begin{enumerate}
\item Combien de protons l'atome de zinc \noyau{Zn}{30}{65} contient-il ?
\item Combien d'�lectrons comporte-t-il ?
\item Calculer la charge totale des protons
sachant qu'un proton a pour charge $e = 1,6.10^{-19}~C$.
\item Calculer la charge totale des �lectrons
sachant qu'un �lectron a pour charge $-e = -1,6.10^{-19}~C$.
\item En d�duire la charge de l'atome de Zinc.
\item Ce r�sultat est-il identique pour tous les atomes ?
\item A l'issue d'une r�action dite d'oxydation, un atome de zinc $Zn$
  se transforme en un ion $Zn^{2+}$.
  \begin{enumerate}
  \item Donnez l'�quation de cette r�action
  (en faisant intervenir un ou plusieurs �lectrons not�s $e^-$).
  \item Indiquez la charge (en coulomb $C$) de cet ion.
  \end{enumerate}
\end{enumerate}
\end{exercice}

\vressort{3}
% pas de tp 3 (caract�risation des ions)
\ds{Devoir Surveill�}{
%
}

\nomprenomclasse

\setcounter{numexercice}{0}

%\renewcommand{\tabularx}[1]{>{\centering}m{#1}} 

%\newcommand{\tabularxc}[1]{\tabularx{>{\centering}m{#1}}}

\vressort{3}

\begin{exercice}{Connaissance sur l'atome}%\\
\begin{enumerate}
\item De quoi est compos� un atome ?
\item Que signifie les lettres $A$, $Z$ et $X$ dans la repr�sentation \noyau{X}{Z}{A} ?
\item Comment trouve-t-on le nombre de neutrons d'un atome de l'�l�ment pr�c�dent.
\item Si un atome a $5$ protons, combien-a-t-il d'�lectrons ? Pourquoi ?
\item Qu'est-ce qui caract�rise un �l�ment chimique ?
\item Qu'est-ce qu'un isotope ?
\end{enumerate}
\end{exercice}



\vressort{3}



\begin{exercice}{Composition des atomes}\\
En vous aidant du tableau p�riodique des �l�ments,
compl�ter le tableau suivant :

\medskip

\noindent
%\begin{tabularx}{\textwidth}{|>{\centering}X|>{\centering}X|>{\centering}X|>{\centering}X|>{\centering}X|}
% \begin{tabularx}{\linewidth}{|X|X|X|X|X|}
% \hline
% \emph{nom}       & \emph{symbole}  & \emph{protons} & \emph{neutrons}
% & \emph{nucl�ons} \tbnl
% carbone   & \noyau{C}{6}{14}   &         &          & \rule[-0.5cm]{0cm}{1cm}         \tbnl
% fluor     & \noyau{F}{9}{19}   &         &          & \rule[-0.5cm]{0cm}{1cm}         \tbnl
% sodium    & \noyau{Na}{11}{23} &         &          & \rule[-0.5cm]{0cm}{1cm}         \tbnl
% oxyg�ne   & \noyau{O}{8}{16}   &         &          & \rule[-0.5cm]{0cm}{1cm}         \tbnl
% hydrog�ne &          &         & 0        & \rule[-0.5cm]{0cm}{1cm}         \tbnl
%           & \noyau{Cl}{17}{35} &         &          &  \rule[-0.5cm]{0cm}{1cm}        \tbnl
%           &          & 8       & \rule[-0.5cm]{0cm}{1cm}         & 16       \tbnl
% \end{tabularx}



\begin{tabularx}{\linewidth}{|>{\mystrut}X|X|X|X|X|}
\hline
% multicolumn pour faire dispara�tre le \mystrut
\multicolumn{1}{|X|}{\emph{nom}} & \emph{symbole}  &
\emph{protons} & \emph{neutrons} & \emph{nucl�ons} \tbnl
carbone   & \noyau{C}{6}{14}   &   &   &    \tbnl
fluor     & \noyau{F}{9}{19}   &   &   &    \tbnl
sodium    & \noyau{Na}{11}{23} &   &   &    \tbnl
oxyg�ne   & \noyau{O}{8}{16}   &   &   &    \tbnl
hydrog�ne &                    &   & 0 &    \tbnl
          & \noyau{Cl}{17}{35} &   &   &    \tbnl
          &                    & 8 &   & 16 \tbnl
\end{tabularx}


\end{exercice}


\vressort{3}


\begin{exercice}{Masse d'un atome de carbone 12}\\
Soit le carbone $12$ not� \noyau{C}{6}{12}.
\begin{enumerate}
\item L'�l�ment carbone peut-il avoir $5$ protons ? Pourquoi ?
\item Calculer la masse du noyau d'un atome de carbone $12$
sachant que la masse d'un nucl�on est $m_n = 1,67.10^{-27}~kg$
\item Calculer la masse des �lectrons de l'atome de carbone 12
sachant que la masse d'un �lectron vaut $m_e = 9,1.10^{-31}~kg$
\item Comparer la masse des �lectrons de l'atome � la masse du noyau.
Que concluez-vous ?
\item En d�duire, sans nouveau calcul, la masse de l'atome de carbone
  $12$.
\end{enumerate}
\end{exercice}

\newpage

\vressort{1}

\begin{exercice}{Couches �lectroniques}\\
Dans l'�tat le plus stable de l'atome, appel� �tat fondamental,
les �lectrons occupent successivement les couches,
en commen�ant par celles qui sont les plus proches du noyau : 
d'abord $K$ puis $L$ puis $M$.

Lorsqu'une couche est pleine, ou encore satur�e, on passe � la suivante.

La derni�re couche occup�e est appel�e couche externe.\\
Toutes les autres sont appel�es couches internes.

\medskip

\noindent
%\begin{tabularx}{\textwidth}{|>{\centering}X|>{\centering}X|>{\centering}X|>{\centering}X|}
\begin{tabularx}{\textwidth}{|>{\mystrut}X|X|X|X|}
\hline
\multicolumn{1}{|X|}{\emph{Symbole de la couche}}       & $K$ & $L$ & $M$  \tbnl
\emph{Nombre maximal d'�lectrons} & $2$ & $8$ & $18$ \tbnl
\end{tabularx}

\medskip

Ainsi, par exemple, l'atome de chlore ($Z=17$) a la configuration �lectronique :
$(K)^2(L)^8(M)^{7}$.

\begin{enumerate}
\item Indiquez le nombre d'�lectrons et donnez la configuration des atomes suivants :
  \begin{enumerate}
  \item \noyau{H}{1}{}
  \item \noyau{O}{8}{}
  \item \noyau{C}{6}{}
  \item \noyau{Ne}{10}{}
  \end{enumerate}
\item Parmi les ions ci-desssous, pr�cisez s'il s'agit d'anions ou de cations.
Indiquez le nombre d'�lectrons et donnez la configuration �lectronique
des ions suivants :
  \begin{enumerate}
  \item $Be^{2+}$ ($Z=4$)
  \item $Al^{3+}$ ($Z=13$)
  \item $O^{2-}$ ($Z=8$)
  \item $F^{-}$ ($Z=9$)
  \end{enumerate}
\end{enumerate}

\end{exercice}


\vressort{5}


\begin{exercice}{Charge d'un atome de Zinc}%\\
\begin{enumerate}
\item Combien de protons l'atome de zinc \noyau{Zn}{30}{65} contient-il ?
\item Combien d'�lectrons comporte-t-il ?
\item Calculer la charge totale des protons
sachant qu'un proton a pour charge $e = 1,6.10^{-19}~C$.
\item Calculer la charge totale des �lectrons
sachant qu'un �lectron a pour charge $-e = -1,6.10^{-19}~C$.
\item En d�duire la charge de l'atome de Zinc.
\item Ce r�sultat est-il identique pour tous les atomes ?
\item A l'issue d'une r�action dite d'oxydation, un atome de zinc $Zn$
  se transforme en un ion $Zn^{2+}$.
  \begin{enumerate}
  \item Donnez l'�quation de cette r�action
  (en faisant intervenir un ou plusieurs �lectrons not�s $e^-$).
  \item Indiquez la charge (en coulomb $C$) de cet ion.
  \end{enumerate}
\end{enumerate}
\end{exercice}

\vressort{3}

\chapitre{Solutions �lectrolytiques}
\ds{Devoir Surveill�}{
%
}

\nomprenomclasse

\setcounter{numexercice}{0}

%\renewcommand{\tabularx}[1]{>{\centering}m{#1}} 

%\newcommand{\tabularxc}[1]{\tabularx{>{\centering}m{#1}}}

\vressort{3}

\begin{exercice}{Connaissance sur l'atome}%\\
\begin{enumerate}
\item De quoi est compos� un atome ?
\item Que signifie les lettres $A$, $Z$ et $X$ dans la repr�sentation \noyau{X}{Z}{A} ?
\item Comment trouve-t-on le nombre de neutrons d'un atome de l'�l�ment pr�c�dent.
\item Si un atome a $5$ protons, combien-a-t-il d'�lectrons ? Pourquoi ?
\item Qu'est-ce qui caract�rise un �l�ment chimique ?
\item Qu'est-ce qu'un isotope ?
\end{enumerate}
\end{exercice}



\vressort{3}



\begin{exercice}{Composition des atomes}\\
En vous aidant du tableau p�riodique des �l�ments,
compl�ter le tableau suivant :

\medskip

\noindent
%\begin{tabularx}{\textwidth}{|>{\centering}X|>{\centering}X|>{\centering}X|>{\centering}X|>{\centering}X|}
% \begin{tabularx}{\linewidth}{|X|X|X|X|X|}
% \hline
% \emph{nom}       & \emph{symbole}  & \emph{protons} & \emph{neutrons}
% & \emph{nucl�ons} \tbnl
% carbone   & \noyau{C}{6}{14}   &         &          & \rule[-0.5cm]{0cm}{1cm}         \tbnl
% fluor     & \noyau{F}{9}{19}   &         &          & \rule[-0.5cm]{0cm}{1cm}         \tbnl
% sodium    & \noyau{Na}{11}{23} &         &          & \rule[-0.5cm]{0cm}{1cm}         \tbnl
% oxyg�ne   & \noyau{O}{8}{16}   &         &          & \rule[-0.5cm]{0cm}{1cm}         \tbnl
% hydrog�ne &          &         & 0        & \rule[-0.5cm]{0cm}{1cm}         \tbnl
%           & \noyau{Cl}{17}{35} &         &          &  \rule[-0.5cm]{0cm}{1cm}        \tbnl
%           &          & 8       & \rule[-0.5cm]{0cm}{1cm}         & 16       \tbnl
% \end{tabularx}



\begin{tabularx}{\linewidth}{|>{\mystrut}X|X|X|X|X|}
\hline
% multicolumn pour faire dispara�tre le \mystrut
\multicolumn{1}{|X|}{\emph{nom}} & \emph{symbole}  &
\emph{protons} & \emph{neutrons} & \emph{nucl�ons} \tbnl
carbone   & \noyau{C}{6}{14}   &   &   &    \tbnl
fluor     & \noyau{F}{9}{19}   &   &   &    \tbnl
sodium    & \noyau{Na}{11}{23} &   &   &    \tbnl
oxyg�ne   & \noyau{O}{8}{16}   &   &   &    \tbnl
hydrog�ne &                    &   & 0 &    \tbnl
          & \noyau{Cl}{17}{35} &   &   &    \tbnl
          &                    & 8 &   & 16 \tbnl
\end{tabularx}


\end{exercice}


\vressort{3}


\begin{exercice}{Masse d'un atome de carbone 12}\\
Soit le carbone $12$ not� \noyau{C}{6}{12}.
\begin{enumerate}
\item L'�l�ment carbone peut-il avoir $5$ protons ? Pourquoi ?
\item Calculer la masse du noyau d'un atome de carbone $12$
sachant que la masse d'un nucl�on est $m_n = 1,67.10^{-27}~kg$
\item Calculer la masse des �lectrons de l'atome de carbone 12
sachant que la masse d'un �lectron vaut $m_e = 9,1.10^{-31}~kg$
\item Comparer la masse des �lectrons de l'atome � la masse du noyau.
Que concluez-vous ?
\item En d�duire, sans nouveau calcul, la masse de l'atome de carbone
  $12$.
\end{enumerate}
\end{exercice}

\newpage

\vressort{1}

\begin{exercice}{Couches �lectroniques}\\
Dans l'�tat le plus stable de l'atome, appel� �tat fondamental,
les �lectrons occupent successivement les couches,
en commen�ant par celles qui sont les plus proches du noyau : 
d'abord $K$ puis $L$ puis $M$.

Lorsqu'une couche est pleine, ou encore satur�e, on passe � la suivante.

La derni�re couche occup�e est appel�e couche externe.\\
Toutes les autres sont appel�es couches internes.

\medskip

\noindent
%\begin{tabularx}{\textwidth}{|>{\centering}X|>{\centering}X|>{\centering}X|>{\centering}X|}
\begin{tabularx}{\textwidth}{|>{\mystrut}X|X|X|X|}
\hline
\multicolumn{1}{|X|}{\emph{Symbole de la couche}}       & $K$ & $L$ & $M$  \tbnl
\emph{Nombre maximal d'�lectrons} & $2$ & $8$ & $18$ \tbnl
\end{tabularx}

\medskip

Ainsi, par exemple, l'atome de chlore ($Z=17$) a la configuration �lectronique :
$(K)^2(L)^8(M)^{7}$.

\begin{enumerate}
\item Indiquez le nombre d'�lectrons et donnez la configuration des atomes suivants :
  \begin{enumerate}
  \item \noyau{H}{1}{}
  \item \noyau{O}{8}{}
  \item \noyau{C}{6}{}
  \item \noyau{Ne}{10}{}
  \end{enumerate}
\item Parmi les ions ci-desssous, pr�cisez s'il s'agit d'anions ou de cations.
Indiquez le nombre d'�lectrons et donnez la configuration �lectronique
des ions suivants :
  \begin{enumerate}
  \item $Be^{2+}$ ($Z=4$)
  \item $Al^{3+}$ ($Z=13$)
  \item $O^{2-}$ ($Z=8$)
  \item $F^{-}$ ($Z=9$)
  \end{enumerate}
\end{enumerate}

\end{exercice}


\vressort{5}


\begin{exercice}{Charge d'un atome de Zinc}%\\
\begin{enumerate}
\item Combien de protons l'atome de zinc \noyau{Zn}{30}{65} contient-il ?
\item Combien d'�lectrons comporte-t-il ?
\item Calculer la charge totale des protons
sachant qu'un proton a pour charge $e = 1,6.10^{-19}~C$.
\item Calculer la charge totale des �lectrons
sachant qu'un �lectron a pour charge $-e = -1,6.10^{-19}~C$.
\item En d�duire la charge de l'atome de Zinc.
\item Ce r�sultat est-il identique pour tous les atomes ?
\item A l'issue d'une r�action dite d'oxydation, un atome de zinc $Zn$
  se transforme en un ion $Zn^{2+}$.
  \begin{enumerate}
  \item Donnez l'�quation de cette r�action
  (en faisant intervenir un ou plusieurs �lectrons not�s $e^-$).
  \item Indiquez la charge (en coulomb $C$) de cet ion.
  \end{enumerate}
\end{enumerate}
\end{exercice}

\vressort{3}
% doc : liste des ions � connaitre

\chapitre{Conductim�trie}
\ds{Devoir Surveill�}{
%
}

\nomprenomclasse

\setcounter{numexercice}{0}

%\renewcommand{\tabularx}[1]{>{\centering}m{#1}} 

%\newcommand{\tabularxc}[1]{\tabularx{>{\centering}m{#1}}}

\vressort{3}

\begin{exercice}{Connaissance sur l'atome}%\\
\begin{enumerate}
\item De quoi est compos� un atome ?
\item Que signifie les lettres $A$, $Z$ et $X$ dans la repr�sentation \noyau{X}{Z}{A} ?
\item Comment trouve-t-on le nombre de neutrons d'un atome de l'�l�ment pr�c�dent.
\item Si un atome a $5$ protons, combien-a-t-il d'�lectrons ? Pourquoi ?
\item Qu'est-ce qui caract�rise un �l�ment chimique ?
\item Qu'est-ce qu'un isotope ?
\end{enumerate}
\end{exercice}



\vressort{3}



\begin{exercice}{Composition des atomes}\\
En vous aidant du tableau p�riodique des �l�ments,
compl�ter le tableau suivant :

\medskip

\noindent
%\begin{tabularx}{\textwidth}{|>{\centering}X|>{\centering}X|>{\centering}X|>{\centering}X|>{\centering}X|}
% \begin{tabularx}{\linewidth}{|X|X|X|X|X|}
% \hline
% \emph{nom}       & \emph{symbole}  & \emph{protons} & \emph{neutrons}
% & \emph{nucl�ons} \tbnl
% carbone   & \noyau{C}{6}{14}   &         &          & \rule[-0.5cm]{0cm}{1cm}         \tbnl
% fluor     & \noyau{F}{9}{19}   &         &          & \rule[-0.5cm]{0cm}{1cm}         \tbnl
% sodium    & \noyau{Na}{11}{23} &         &          & \rule[-0.5cm]{0cm}{1cm}         \tbnl
% oxyg�ne   & \noyau{O}{8}{16}   &         &          & \rule[-0.5cm]{0cm}{1cm}         \tbnl
% hydrog�ne &          &         & 0        & \rule[-0.5cm]{0cm}{1cm}         \tbnl
%           & \noyau{Cl}{17}{35} &         &          &  \rule[-0.5cm]{0cm}{1cm}        \tbnl
%           &          & 8       & \rule[-0.5cm]{0cm}{1cm}         & 16       \tbnl
% \end{tabularx}



\begin{tabularx}{\linewidth}{|>{\mystrut}X|X|X|X|X|}
\hline
% multicolumn pour faire dispara�tre le \mystrut
\multicolumn{1}{|X|}{\emph{nom}} & \emph{symbole}  &
\emph{protons} & \emph{neutrons} & \emph{nucl�ons} \tbnl
carbone   & \noyau{C}{6}{14}   &   &   &    \tbnl
fluor     & \noyau{F}{9}{19}   &   &   &    \tbnl
sodium    & \noyau{Na}{11}{23} &   &   &    \tbnl
oxyg�ne   & \noyau{O}{8}{16}   &   &   &    \tbnl
hydrog�ne &                    &   & 0 &    \tbnl
          & \noyau{Cl}{17}{35} &   &   &    \tbnl
          &                    & 8 &   & 16 \tbnl
\end{tabularx}


\end{exercice}


\vressort{3}


\begin{exercice}{Masse d'un atome de carbone 12}\\
Soit le carbone $12$ not� \noyau{C}{6}{12}.
\begin{enumerate}
\item L'�l�ment carbone peut-il avoir $5$ protons ? Pourquoi ?
\item Calculer la masse du noyau d'un atome de carbone $12$
sachant que la masse d'un nucl�on est $m_n = 1,67.10^{-27}~kg$
\item Calculer la masse des �lectrons de l'atome de carbone 12
sachant que la masse d'un �lectron vaut $m_e = 9,1.10^{-31}~kg$
\item Comparer la masse des �lectrons de l'atome � la masse du noyau.
Que concluez-vous ?
\item En d�duire, sans nouveau calcul, la masse de l'atome de carbone
  $12$.
\end{enumerate}
\end{exercice}

\newpage

\vressort{1}

\begin{exercice}{Couches �lectroniques}\\
Dans l'�tat le plus stable de l'atome, appel� �tat fondamental,
les �lectrons occupent successivement les couches,
en commen�ant par celles qui sont les plus proches du noyau : 
d'abord $K$ puis $L$ puis $M$.

Lorsqu'une couche est pleine, ou encore satur�e, on passe � la suivante.

La derni�re couche occup�e est appel�e couche externe.\\
Toutes les autres sont appel�es couches internes.

\medskip

\noindent
%\begin{tabularx}{\textwidth}{|>{\centering}X|>{\centering}X|>{\centering}X|>{\centering}X|}
\begin{tabularx}{\textwidth}{|>{\mystrut}X|X|X|X|}
\hline
\multicolumn{1}{|X|}{\emph{Symbole de la couche}}       & $K$ & $L$ & $M$  \tbnl
\emph{Nombre maximal d'�lectrons} & $2$ & $8$ & $18$ \tbnl
\end{tabularx}

\medskip

Ainsi, par exemple, l'atome de chlore ($Z=17$) a la configuration �lectronique :
$(K)^2(L)^8(M)^{7}$.

\begin{enumerate}
\item Indiquez le nombre d'�lectrons et donnez la configuration des atomes suivants :
  \begin{enumerate}
  \item \noyau{H}{1}{}
  \item \noyau{O}{8}{}
  \item \noyau{C}{6}{}
  \item \noyau{Ne}{10}{}
  \end{enumerate}
\item Parmi les ions ci-desssous, pr�cisez s'il s'agit d'anions ou de cations.
Indiquez le nombre d'�lectrons et donnez la configuration �lectronique
des ions suivants :
  \begin{enumerate}
  \item $Be^{2+}$ ($Z=4$)
  \item $Al^{3+}$ ($Z=13$)
  \item $O^{2-}$ ($Z=8$)
  \item $F^{-}$ ($Z=9$)
  \end{enumerate}
\end{enumerate}

\end{exercice}


\vressort{5}


\begin{exercice}{Charge d'un atome de Zinc}%\\
\begin{enumerate}
\item Combien de protons l'atome de zinc \noyau{Zn}{30}{65} contient-il ?
\item Combien d'�lectrons comporte-t-il ?
\item Calculer la charge totale des protons
sachant qu'un proton a pour charge $e = 1,6.10^{-19}~C$.
\item Calculer la charge totale des �lectrons
sachant qu'un �lectron a pour charge $-e = -1,6.10^{-19}~C$.
\item En d�duire la charge de l'atome de Zinc.
\item Ce r�sultat est-il identique pour tous les atomes ?
\item A l'issue d'une r�action dite d'oxydation, un atome de zinc $Zn$
  se transforme en un ion $Zn^{2+}$.
  \begin{enumerate}
  \item Donnez l'�quation de cette r�action
  (en faisant intervenir un ou plusieurs �lectrons not�s $e^-$).
  \item Indiquez la charge (en coulomb $C$) de cet ion.
  \end{enumerate}
\end{enumerate}
\end{exercice}

\vressort{3}
\ds{Devoir Surveill�}{
%
}

\nomprenomclasse

\setcounter{numexercice}{0}

%\renewcommand{\tabularx}[1]{>{\centering}m{#1}} 

%\newcommand{\tabularxc}[1]{\tabularx{>{\centering}m{#1}}}

\vressort{3}

\begin{exercice}{Connaissance sur l'atome}%\\
\begin{enumerate}
\item De quoi est compos� un atome ?
\item Que signifie les lettres $A$, $Z$ et $X$ dans la repr�sentation \noyau{X}{Z}{A} ?
\item Comment trouve-t-on le nombre de neutrons d'un atome de l'�l�ment pr�c�dent.
\item Si un atome a $5$ protons, combien-a-t-il d'�lectrons ? Pourquoi ?
\item Qu'est-ce qui caract�rise un �l�ment chimique ?
\item Qu'est-ce qu'un isotope ?
\end{enumerate}
\end{exercice}



\vressort{3}



\begin{exercice}{Composition des atomes}\\
En vous aidant du tableau p�riodique des �l�ments,
compl�ter le tableau suivant :

\medskip

\noindent
%\begin{tabularx}{\textwidth}{|>{\centering}X|>{\centering}X|>{\centering}X|>{\centering}X|>{\centering}X|}
% \begin{tabularx}{\linewidth}{|X|X|X|X|X|}
% \hline
% \emph{nom}       & \emph{symbole}  & \emph{protons} & \emph{neutrons}
% & \emph{nucl�ons} \tbnl
% carbone   & \noyau{C}{6}{14}   &         &          & \rule[-0.5cm]{0cm}{1cm}         \tbnl
% fluor     & \noyau{F}{9}{19}   &         &          & \rule[-0.5cm]{0cm}{1cm}         \tbnl
% sodium    & \noyau{Na}{11}{23} &         &          & \rule[-0.5cm]{0cm}{1cm}         \tbnl
% oxyg�ne   & \noyau{O}{8}{16}   &         &          & \rule[-0.5cm]{0cm}{1cm}         \tbnl
% hydrog�ne &          &         & 0        & \rule[-0.5cm]{0cm}{1cm}         \tbnl
%           & \noyau{Cl}{17}{35} &         &          &  \rule[-0.5cm]{0cm}{1cm}        \tbnl
%           &          & 8       & \rule[-0.5cm]{0cm}{1cm}         & 16       \tbnl
% \end{tabularx}



\begin{tabularx}{\linewidth}{|>{\mystrut}X|X|X|X|X|}
\hline
% multicolumn pour faire dispara�tre le \mystrut
\multicolumn{1}{|X|}{\emph{nom}} & \emph{symbole}  &
\emph{protons} & \emph{neutrons} & \emph{nucl�ons} \tbnl
carbone   & \noyau{C}{6}{14}   &   &   &    \tbnl
fluor     & \noyau{F}{9}{19}   &   &   &    \tbnl
sodium    & \noyau{Na}{11}{23} &   &   &    \tbnl
oxyg�ne   & \noyau{O}{8}{16}   &   &   &    \tbnl
hydrog�ne &                    &   & 0 &    \tbnl
          & \noyau{Cl}{17}{35} &   &   &    \tbnl
          &                    & 8 &   & 16 \tbnl
\end{tabularx}


\end{exercice}


\vressort{3}


\begin{exercice}{Masse d'un atome de carbone 12}\\
Soit le carbone $12$ not� \noyau{C}{6}{12}.
\begin{enumerate}
\item L'�l�ment carbone peut-il avoir $5$ protons ? Pourquoi ?
\item Calculer la masse du noyau d'un atome de carbone $12$
sachant que la masse d'un nucl�on est $m_n = 1,67.10^{-27}~kg$
\item Calculer la masse des �lectrons de l'atome de carbone 12
sachant que la masse d'un �lectron vaut $m_e = 9,1.10^{-31}~kg$
\item Comparer la masse des �lectrons de l'atome � la masse du noyau.
Que concluez-vous ?
\item En d�duire, sans nouveau calcul, la masse de l'atome de carbone
  $12$.
\end{enumerate}
\end{exercice}

\newpage

\vressort{1}

\begin{exercice}{Couches �lectroniques}\\
Dans l'�tat le plus stable de l'atome, appel� �tat fondamental,
les �lectrons occupent successivement les couches,
en commen�ant par celles qui sont les plus proches du noyau : 
d'abord $K$ puis $L$ puis $M$.

Lorsqu'une couche est pleine, ou encore satur�e, on passe � la suivante.

La derni�re couche occup�e est appel�e couche externe.\\
Toutes les autres sont appel�es couches internes.

\medskip

\noindent
%\begin{tabularx}{\textwidth}{|>{\centering}X|>{\centering}X|>{\centering}X|>{\centering}X|}
\begin{tabularx}{\textwidth}{|>{\mystrut}X|X|X|X|}
\hline
\multicolumn{1}{|X|}{\emph{Symbole de la couche}}       & $K$ & $L$ & $M$  \tbnl
\emph{Nombre maximal d'�lectrons} & $2$ & $8$ & $18$ \tbnl
\end{tabularx}

\medskip

Ainsi, par exemple, l'atome de chlore ($Z=17$) a la configuration �lectronique :
$(K)^2(L)^8(M)^{7}$.

\begin{enumerate}
\item Indiquez le nombre d'�lectrons et donnez la configuration des atomes suivants :
  \begin{enumerate}
  \item \noyau{H}{1}{}
  \item \noyau{O}{8}{}
  \item \noyau{C}{6}{}
  \item \noyau{Ne}{10}{}
  \end{enumerate}
\item Parmi les ions ci-desssous, pr�cisez s'il s'agit d'anions ou de cations.
Indiquez le nombre d'�lectrons et donnez la configuration �lectronique
des ions suivants :
  \begin{enumerate}
  \item $Be^{2+}$ ($Z=4$)
  \item $Al^{3+}$ ($Z=13$)
  \item $O^{2-}$ ($Z=8$)
  \item $F^{-}$ ($Z=9$)
  \end{enumerate}
\end{enumerate}

\end{exercice}


\vressort{5}


\begin{exercice}{Charge d'un atome de Zinc}%\\
\begin{enumerate}
\item Combien de protons l'atome de zinc \noyau{Zn}{30}{65} contient-il ?
\item Combien d'�lectrons comporte-t-il ?
\item Calculer la charge totale des protons
sachant qu'un proton a pour charge $e = 1,6.10^{-19}~C$.
\item Calculer la charge totale des �lectrons
sachant qu'un �lectron a pour charge $-e = -1,6.10^{-19}~C$.
\item En d�duire la charge de l'atome de Zinc.
\item Ce r�sultat est-il identique pour tous les atomes ?
\item A l'issue d'une r�action dite d'oxydation, un atome de zinc $Zn$
  se transforme en un ion $Zn^{2+}$.
  \begin{enumerate}
  \item Donnez l'�quation de cette r�action
  (en faisant intervenir un ou plusieurs �lectrons not�s $e^-$).
  \item Indiquez la charge (en coulomb $C$) de cet ion.
  \end{enumerate}
\end{enumerate}
\end{exercice}

\vressort{3}
% doc : conductivit� molaires ioniques


%\chapitre{R�actions acido-basiques}
% doc : indicateurs color�s


%\chapitre{R�actions d'oxydo-r�duction}
% doc : quelques couple r�dox




\classe{Terminale\\
Sciences et Technologie de Laboratoire\\
Biochimie - G�nie Biologique}




\chapitre{Courant et tension �lectrique}
\ds{Devoir Surveill�}{
%
}

\nomprenomclasse

\setcounter{numexercice}{0}

%\renewcommand{\tabularx}[1]{>{\centering}m{#1}} 

%\newcommand{\tabularxc}[1]{\tabularx{>{\centering}m{#1}}}

\vressort{3}

\begin{exercice}{Connaissance sur l'atome}%\\
\begin{enumerate}
\item De quoi est compos� un atome ?
\item Que signifie les lettres $A$, $Z$ et $X$ dans la repr�sentation \noyau{X}{Z}{A} ?
\item Comment trouve-t-on le nombre de neutrons d'un atome de l'�l�ment pr�c�dent.
\item Si un atome a $5$ protons, combien-a-t-il d'�lectrons ? Pourquoi ?
\item Qu'est-ce qui caract�rise un �l�ment chimique ?
\item Qu'est-ce qu'un isotope ?
\end{enumerate}
\end{exercice}



\vressort{3}



\begin{exercice}{Composition des atomes}\\
En vous aidant du tableau p�riodique des �l�ments,
compl�ter le tableau suivant :

\medskip

\noindent
%\begin{tabularx}{\textwidth}{|>{\centering}X|>{\centering}X|>{\centering}X|>{\centering}X|>{\centering}X|}
% \begin{tabularx}{\linewidth}{|X|X|X|X|X|}
% \hline
% \emph{nom}       & \emph{symbole}  & \emph{protons} & \emph{neutrons}
% & \emph{nucl�ons} \tbnl
% carbone   & \noyau{C}{6}{14}   &         &          & \rule[-0.5cm]{0cm}{1cm}         \tbnl
% fluor     & \noyau{F}{9}{19}   &         &          & \rule[-0.5cm]{0cm}{1cm}         \tbnl
% sodium    & \noyau{Na}{11}{23} &         &          & \rule[-0.5cm]{0cm}{1cm}         \tbnl
% oxyg�ne   & \noyau{O}{8}{16}   &         &          & \rule[-0.5cm]{0cm}{1cm}         \tbnl
% hydrog�ne &          &         & 0        & \rule[-0.5cm]{0cm}{1cm}         \tbnl
%           & \noyau{Cl}{17}{35} &         &          &  \rule[-0.5cm]{0cm}{1cm}        \tbnl
%           &          & 8       & \rule[-0.5cm]{0cm}{1cm}         & 16       \tbnl
% \end{tabularx}



\begin{tabularx}{\linewidth}{|>{\mystrut}X|X|X|X|X|}
\hline
% multicolumn pour faire dispara�tre le \mystrut
\multicolumn{1}{|X|}{\emph{nom}} & \emph{symbole}  &
\emph{protons} & \emph{neutrons} & \emph{nucl�ons} \tbnl
carbone   & \noyau{C}{6}{14}   &   &   &    \tbnl
fluor     & \noyau{F}{9}{19}   &   &   &    \tbnl
sodium    & \noyau{Na}{11}{23} &   &   &    \tbnl
oxyg�ne   & \noyau{O}{8}{16}   &   &   &    \tbnl
hydrog�ne &                    &   & 0 &    \tbnl
          & \noyau{Cl}{17}{35} &   &   &    \tbnl
          &                    & 8 &   & 16 \tbnl
\end{tabularx}


\end{exercice}


\vressort{3}


\begin{exercice}{Masse d'un atome de carbone 12}\\
Soit le carbone $12$ not� \noyau{C}{6}{12}.
\begin{enumerate}
\item L'�l�ment carbone peut-il avoir $5$ protons ? Pourquoi ?
\item Calculer la masse du noyau d'un atome de carbone $12$
sachant que la masse d'un nucl�on est $m_n = 1,67.10^{-27}~kg$
\item Calculer la masse des �lectrons de l'atome de carbone 12
sachant que la masse d'un �lectron vaut $m_e = 9,1.10^{-31}~kg$
\item Comparer la masse des �lectrons de l'atome � la masse du noyau.
Que concluez-vous ?
\item En d�duire, sans nouveau calcul, la masse de l'atome de carbone
  $12$.
\end{enumerate}
\end{exercice}

\newpage

\vressort{1}

\begin{exercice}{Couches �lectroniques}\\
Dans l'�tat le plus stable de l'atome, appel� �tat fondamental,
les �lectrons occupent successivement les couches,
en commen�ant par celles qui sont les plus proches du noyau : 
d'abord $K$ puis $L$ puis $M$.

Lorsqu'une couche est pleine, ou encore satur�e, on passe � la suivante.

La derni�re couche occup�e est appel�e couche externe.\\
Toutes les autres sont appel�es couches internes.

\medskip

\noindent
%\begin{tabularx}{\textwidth}{|>{\centering}X|>{\centering}X|>{\centering}X|>{\centering}X|}
\begin{tabularx}{\textwidth}{|>{\mystrut}X|X|X|X|}
\hline
\multicolumn{1}{|X|}{\emph{Symbole de la couche}}       & $K$ & $L$ & $M$  \tbnl
\emph{Nombre maximal d'�lectrons} & $2$ & $8$ & $18$ \tbnl
\end{tabularx}

\medskip

Ainsi, par exemple, l'atome de chlore ($Z=17$) a la configuration �lectronique :
$(K)^2(L)^8(M)^{7}$.

\begin{enumerate}
\item Indiquez le nombre d'�lectrons et donnez la configuration des atomes suivants :
  \begin{enumerate}
  \item \noyau{H}{1}{}
  \item \noyau{O}{8}{}
  \item \noyau{C}{6}{}
  \item \noyau{Ne}{10}{}
  \end{enumerate}
\item Parmi les ions ci-desssous, pr�cisez s'il s'agit d'anions ou de cations.
Indiquez le nombre d'�lectrons et donnez la configuration �lectronique
des ions suivants :
  \begin{enumerate}
  \item $Be^{2+}$ ($Z=4$)
  \item $Al^{3+}$ ($Z=13$)
  \item $O^{2-}$ ($Z=8$)
  \item $F^{-}$ ($Z=9$)
  \end{enumerate}
\end{enumerate}

\end{exercice}


\vressort{5}


\begin{exercice}{Charge d'un atome de Zinc}%\\
\begin{enumerate}
\item Combien de protons l'atome de zinc \noyau{Zn}{30}{65} contient-il ?
\item Combien d'�lectrons comporte-t-il ?
\item Calculer la charge totale des protons
sachant qu'un proton a pour charge $e = 1,6.10^{-19}~C$.
\item Calculer la charge totale des �lectrons
sachant qu'un �lectron a pour charge $-e = -1,6.10^{-19}~C$.
\item En d�duire la charge de l'atome de Zinc.
\item Ce r�sultat est-il identique pour tous les atomes ?
\item A l'issue d'une r�action dite d'oxydation, un atome de zinc $Zn$
  se transforme en un ion $Zn^{2+}$.
  \begin{enumerate}
  \item Donnez l'�quation de cette r�action
  (en faisant intervenir un ou plusieurs �lectrons not�s $e^-$).
  \item Indiquez la charge (en coulomb $C$) de cet ion.
  \end{enumerate}
\end{enumerate}
\end{exercice}

\vressort{3} % cours elec circuit �lectrique

\ds{Devoir Surveill�}{
%
}

\nomprenomclasse

\setcounter{numexercice}{0}

%\renewcommand{\tabularx}[1]{>{\centering}m{#1}} 

%\newcommand{\tabularxc}[1]{\tabularx{>{\centering}m{#1}}}

\vressort{3}

\begin{exercice}{Connaissance sur l'atome}%\\
\begin{enumerate}
\item De quoi est compos� un atome ?
\item Que signifie les lettres $A$, $Z$ et $X$ dans la repr�sentation \noyau{X}{Z}{A} ?
\item Comment trouve-t-on le nombre de neutrons d'un atome de l'�l�ment pr�c�dent.
\item Si un atome a $5$ protons, combien-a-t-il d'�lectrons ? Pourquoi ?
\item Qu'est-ce qui caract�rise un �l�ment chimique ?
\item Qu'est-ce qu'un isotope ?
\end{enumerate}
\end{exercice}



\vressort{3}



\begin{exercice}{Composition des atomes}\\
En vous aidant du tableau p�riodique des �l�ments,
compl�ter le tableau suivant :

\medskip

\noindent
%\begin{tabularx}{\textwidth}{|>{\centering}X|>{\centering}X|>{\centering}X|>{\centering}X|>{\centering}X|}
% \begin{tabularx}{\linewidth}{|X|X|X|X|X|}
% \hline
% \emph{nom}       & \emph{symbole}  & \emph{protons} & \emph{neutrons}
% & \emph{nucl�ons} \tbnl
% carbone   & \noyau{C}{6}{14}   &         &          & \rule[-0.5cm]{0cm}{1cm}         \tbnl
% fluor     & \noyau{F}{9}{19}   &         &          & \rule[-0.5cm]{0cm}{1cm}         \tbnl
% sodium    & \noyau{Na}{11}{23} &         &          & \rule[-0.5cm]{0cm}{1cm}         \tbnl
% oxyg�ne   & \noyau{O}{8}{16}   &         &          & \rule[-0.5cm]{0cm}{1cm}         \tbnl
% hydrog�ne &          &         & 0        & \rule[-0.5cm]{0cm}{1cm}         \tbnl
%           & \noyau{Cl}{17}{35} &         &          &  \rule[-0.5cm]{0cm}{1cm}        \tbnl
%           &          & 8       & \rule[-0.5cm]{0cm}{1cm}         & 16       \tbnl
% \end{tabularx}



\begin{tabularx}{\linewidth}{|>{\mystrut}X|X|X|X|X|}
\hline
% multicolumn pour faire dispara�tre le \mystrut
\multicolumn{1}{|X|}{\emph{nom}} & \emph{symbole}  &
\emph{protons} & \emph{neutrons} & \emph{nucl�ons} \tbnl
carbone   & \noyau{C}{6}{14}   &   &   &    \tbnl
fluor     & \noyau{F}{9}{19}   &   &   &    \tbnl
sodium    & \noyau{Na}{11}{23} &   &   &    \tbnl
oxyg�ne   & \noyau{O}{8}{16}   &   &   &    \tbnl
hydrog�ne &                    &   & 0 &    \tbnl
          & \noyau{Cl}{17}{35} &   &   &    \tbnl
          &                    & 8 &   & 16 \tbnl
\end{tabularx}


\end{exercice}


\vressort{3}


\begin{exercice}{Masse d'un atome de carbone 12}\\
Soit le carbone $12$ not� \noyau{C}{6}{12}.
\begin{enumerate}
\item L'�l�ment carbone peut-il avoir $5$ protons ? Pourquoi ?
\item Calculer la masse du noyau d'un atome de carbone $12$
sachant que la masse d'un nucl�on est $m_n = 1,67.10^{-27}~kg$
\item Calculer la masse des �lectrons de l'atome de carbone 12
sachant que la masse d'un �lectron vaut $m_e = 9,1.10^{-31}~kg$
\item Comparer la masse des �lectrons de l'atome � la masse du noyau.
Que concluez-vous ?
\item En d�duire, sans nouveau calcul, la masse de l'atome de carbone
  $12$.
\end{enumerate}
\end{exercice}

\newpage

\vressort{1}

\begin{exercice}{Couches �lectroniques}\\
Dans l'�tat le plus stable de l'atome, appel� �tat fondamental,
les �lectrons occupent successivement les couches,
en commen�ant par celles qui sont les plus proches du noyau : 
d'abord $K$ puis $L$ puis $M$.

Lorsqu'une couche est pleine, ou encore satur�e, on passe � la suivante.

La derni�re couche occup�e est appel�e couche externe.\\
Toutes les autres sont appel�es couches internes.

\medskip

\noindent
%\begin{tabularx}{\textwidth}{|>{\centering}X|>{\centering}X|>{\centering}X|>{\centering}X|}
\begin{tabularx}{\textwidth}{|>{\mystrut}X|X|X|X|}
\hline
\multicolumn{1}{|X|}{\emph{Symbole de la couche}}       & $K$ & $L$ & $M$  \tbnl
\emph{Nombre maximal d'�lectrons} & $2$ & $8$ & $18$ \tbnl
\end{tabularx}

\medskip

Ainsi, par exemple, l'atome de chlore ($Z=17$) a la configuration �lectronique :
$(K)^2(L)^8(M)^{7}$.

\begin{enumerate}
\item Indiquez le nombre d'�lectrons et donnez la configuration des atomes suivants :
  \begin{enumerate}
  \item \noyau{H}{1}{}
  \item \noyau{O}{8}{}
  \item \noyau{C}{6}{}
  \item \noyau{Ne}{10}{}
  \end{enumerate}
\item Parmi les ions ci-desssous, pr�cisez s'il s'agit d'anions ou de cations.
Indiquez le nombre d'�lectrons et donnez la configuration �lectronique
des ions suivants :
  \begin{enumerate}
  \item $Be^{2+}$ ($Z=4$)
  \item $Al^{3+}$ ($Z=13$)
  \item $O^{2-}$ ($Z=8$)
  \item $F^{-}$ ($Z=9$)
  \end{enumerate}
\end{enumerate}

\end{exercice}


\vressort{5}


\begin{exercice}{Charge d'un atome de Zinc}%\\
\begin{enumerate}
\item Combien de protons l'atome de zinc \noyau{Zn}{30}{65} contient-il ?
\item Combien d'�lectrons comporte-t-il ?
\item Calculer la charge totale des protons
sachant qu'un proton a pour charge $e = 1,6.10^{-19}~C$.
\item Calculer la charge totale des �lectrons
sachant qu'un �lectron a pour charge $-e = -1,6.10^{-19}~C$.
\item En d�duire la charge de l'atome de Zinc.
\item Ce r�sultat est-il identique pour tous les atomes ?
\item A l'issue d'une r�action dite d'oxydation, un atome de zinc $Zn$
  se transforme en un ion $Zn^{2+}$.
  \begin{enumerate}
  \item Donnez l'�quation de cette r�action
  (en faisant intervenir un ou plusieurs �lectrons not�s $e^-$).
  \item Indiquez la charge (en coulomb $C$) de cet ion.
  \end{enumerate}
\end{enumerate}
\end{exercice}

\vressort{3} % tp mesure de tension et d'intensit�

\ds{Devoir Surveill�}{
%
}

\nomprenomclasse

\setcounter{numexercice}{0}

%\renewcommand{\tabularx}[1]{>{\centering}m{#1}} 

%\newcommand{\tabularxc}[1]{\tabularx{>{\centering}m{#1}}}

\vressort{3}

\begin{exercice}{Connaissance sur l'atome}%\\
\begin{enumerate}
\item De quoi est compos� un atome ?
\item Que signifie les lettres $A$, $Z$ et $X$ dans la repr�sentation \noyau{X}{Z}{A} ?
\item Comment trouve-t-on le nombre de neutrons d'un atome de l'�l�ment pr�c�dent.
\item Si un atome a $5$ protons, combien-a-t-il d'�lectrons ? Pourquoi ?
\item Qu'est-ce qui caract�rise un �l�ment chimique ?
\item Qu'est-ce qu'un isotope ?
\end{enumerate}
\end{exercice}



\vressort{3}



\begin{exercice}{Composition des atomes}\\
En vous aidant du tableau p�riodique des �l�ments,
compl�ter le tableau suivant :

\medskip

\noindent
%\begin{tabularx}{\textwidth}{|>{\centering}X|>{\centering}X|>{\centering}X|>{\centering}X|>{\centering}X|}
% \begin{tabularx}{\linewidth}{|X|X|X|X|X|}
% \hline
% \emph{nom}       & \emph{symbole}  & \emph{protons} & \emph{neutrons}
% & \emph{nucl�ons} \tbnl
% carbone   & \noyau{C}{6}{14}   &         &          & \rule[-0.5cm]{0cm}{1cm}         \tbnl
% fluor     & \noyau{F}{9}{19}   &         &          & \rule[-0.5cm]{0cm}{1cm}         \tbnl
% sodium    & \noyau{Na}{11}{23} &         &          & \rule[-0.5cm]{0cm}{1cm}         \tbnl
% oxyg�ne   & \noyau{O}{8}{16}   &         &          & \rule[-0.5cm]{0cm}{1cm}         \tbnl
% hydrog�ne &          &         & 0        & \rule[-0.5cm]{0cm}{1cm}         \tbnl
%           & \noyau{Cl}{17}{35} &         &          &  \rule[-0.5cm]{0cm}{1cm}        \tbnl
%           &          & 8       & \rule[-0.5cm]{0cm}{1cm}         & 16       \tbnl
% \end{tabularx}



\begin{tabularx}{\linewidth}{|>{\mystrut}X|X|X|X|X|}
\hline
% multicolumn pour faire dispara�tre le \mystrut
\multicolumn{1}{|X|}{\emph{nom}} & \emph{symbole}  &
\emph{protons} & \emph{neutrons} & \emph{nucl�ons} \tbnl
carbone   & \noyau{C}{6}{14}   &   &   &    \tbnl
fluor     & \noyau{F}{9}{19}   &   &   &    \tbnl
sodium    & \noyau{Na}{11}{23} &   &   &    \tbnl
oxyg�ne   & \noyau{O}{8}{16}   &   &   &    \tbnl
hydrog�ne &                    &   & 0 &    \tbnl
          & \noyau{Cl}{17}{35} &   &   &    \tbnl
          &                    & 8 &   & 16 \tbnl
\end{tabularx}


\end{exercice}


\vressort{3}


\begin{exercice}{Masse d'un atome de carbone 12}\\
Soit le carbone $12$ not� \noyau{C}{6}{12}.
\begin{enumerate}
\item L'�l�ment carbone peut-il avoir $5$ protons ? Pourquoi ?
\item Calculer la masse du noyau d'un atome de carbone $12$
sachant que la masse d'un nucl�on est $m_n = 1,67.10^{-27}~kg$
\item Calculer la masse des �lectrons de l'atome de carbone 12
sachant que la masse d'un �lectron vaut $m_e = 9,1.10^{-31}~kg$
\item Comparer la masse des �lectrons de l'atome � la masse du noyau.
Que concluez-vous ?
\item En d�duire, sans nouveau calcul, la masse de l'atome de carbone
  $12$.
\end{enumerate}
\end{exercice}

\newpage

\vressort{1}

\begin{exercice}{Couches �lectroniques}\\
Dans l'�tat le plus stable de l'atome, appel� �tat fondamental,
les �lectrons occupent successivement les couches,
en commen�ant par celles qui sont les plus proches du noyau : 
d'abord $K$ puis $L$ puis $M$.

Lorsqu'une couche est pleine, ou encore satur�e, on passe � la suivante.

La derni�re couche occup�e est appel�e couche externe.\\
Toutes les autres sont appel�es couches internes.

\medskip

\noindent
%\begin{tabularx}{\textwidth}{|>{\centering}X|>{\centering}X|>{\centering}X|>{\centering}X|}
\begin{tabularx}{\textwidth}{|>{\mystrut}X|X|X|X|}
\hline
\multicolumn{1}{|X|}{\emph{Symbole de la couche}}       & $K$ & $L$ & $M$  \tbnl
\emph{Nombre maximal d'�lectrons} & $2$ & $8$ & $18$ \tbnl
\end{tabularx}

\medskip

Ainsi, par exemple, l'atome de chlore ($Z=17$) a la configuration �lectronique :
$(K)^2(L)^8(M)^{7}$.

\begin{enumerate}
\item Indiquez le nombre d'�lectrons et donnez la configuration des atomes suivants :
  \begin{enumerate}
  \item \noyau{H}{1}{}
  \item \noyau{O}{8}{}
  \item \noyau{C}{6}{}
  \item \noyau{Ne}{10}{}
  \end{enumerate}
\item Parmi les ions ci-desssous, pr�cisez s'il s'agit d'anions ou de cations.
Indiquez le nombre d'�lectrons et donnez la configuration �lectronique
des ions suivants :
  \begin{enumerate}
  \item $Be^{2+}$ ($Z=4$)
  \item $Al^{3+}$ ($Z=13$)
  \item $O^{2-}$ ($Z=8$)
  \item $F^{-}$ ($Z=9$)
  \end{enumerate}
\end{enumerate}

\end{exercice}


\vressort{5}


\begin{exercice}{Charge d'un atome de Zinc}%\\
\begin{enumerate}
\item Combien de protons l'atome de zinc \noyau{Zn}{30}{65} contient-il ?
\item Combien d'�lectrons comporte-t-il ?
\item Calculer la charge totale des protons
sachant qu'un proton a pour charge $e = 1,6.10^{-19}~C$.
\item Calculer la charge totale des �lectrons
sachant qu'un �lectron a pour charge $-e = -1,6.10^{-19}~C$.
\item En d�duire la charge de l'atome de Zinc.
\item Ce r�sultat est-il identique pour tous les atomes ?
\item A l'issue d'une r�action dite d'oxydation, un atome de zinc $Zn$
  se transforme en un ion $Zn^{2+}$.
  \begin{enumerate}
  \item Donnez l'�quation de cette r�action
  (en faisant intervenir un ou plusieurs �lectrons not�s $e^-$).
  \item Indiquez la charge (en coulomb $C$) de cet ion.
  \end{enumerate}
\end{enumerate}
\end{exercice}

\vressort{3} % document fiche
    % m�thode montage
    % utilisation d'un multim�tre, ...

\ds{Devoir Surveill�}{
%
}

\nomprenomclasse

\setcounter{numexercice}{0}

%\renewcommand{\tabularx}[1]{>{\centering}m{#1}} 

%\newcommand{\tabularxc}[1]{\tabularx{>{\centering}m{#1}}}

\vressort{3}

\begin{exercice}{Connaissance sur l'atome}%\\
\begin{enumerate}
\item De quoi est compos� un atome ?
\item Que signifie les lettres $A$, $Z$ et $X$ dans la repr�sentation \noyau{X}{Z}{A} ?
\item Comment trouve-t-on le nombre de neutrons d'un atome de l'�l�ment pr�c�dent.
\item Si un atome a $5$ protons, combien-a-t-il d'�lectrons ? Pourquoi ?
\item Qu'est-ce qui caract�rise un �l�ment chimique ?
\item Qu'est-ce qu'un isotope ?
\end{enumerate}
\end{exercice}



\vressort{3}



\begin{exercice}{Composition des atomes}\\
En vous aidant du tableau p�riodique des �l�ments,
compl�ter le tableau suivant :

\medskip

\noindent
%\begin{tabularx}{\textwidth}{|>{\centering}X|>{\centering}X|>{\centering}X|>{\centering}X|>{\centering}X|}
% \begin{tabularx}{\linewidth}{|X|X|X|X|X|}
% \hline
% \emph{nom}       & \emph{symbole}  & \emph{protons} & \emph{neutrons}
% & \emph{nucl�ons} \tbnl
% carbone   & \noyau{C}{6}{14}   &         &          & \rule[-0.5cm]{0cm}{1cm}         \tbnl
% fluor     & \noyau{F}{9}{19}   &         &          & \rule[-0.5cm]{0cm}{1cm}         \tbnl
% sodium    & \noyau{Na}{11}{23} &         &          & \rule[-0.5cm]{0cm}{1cm}         \tbnl
% oxyg�ne   & \noyau{O}{8}{16}   &         &          & \rule[-0.5cm]{0cm}{1cm}         \tbnl
% hydrog�ne &          &         & 0        & \rule[-0.5cm]{0cm}{1cm}         \tbnl
%           & \noyau{Cl}{17}{35} &         &          &  \rule[-0.5cm]{0cm}{1cm}        \tbnl
%           &          & 8       & \rule[-0.5cm]{0cm}{1cm}         & 16       \tbnl
% \end{tabularx}



\begin{tabularx}{\linewidth}{|>{\mystrut}X|X|X|X|X|}
\hline
% multicolumn pour faire dispara�tre le \mystrut
\multicolumn{1}{|X|}{\emph{nom}} & \emph{symbole}  &
\emph{protons} & \emph{neutrons} & \emph{nucl�ons} \tbnl
carbone   & \noyau{C}{6}{14}   &   &   &    \tbnl
fluor     & \noyau{F}{9}{19}   &   &   &    \tbnl
sodium    & \noyau{Na}{11}{23} &   &   &    \tbnl
oxyg�ne   & \noyau{O}{8}{16}   &   &   &    \tbnl
hydrog�ne &                    &   & 0 &    \tbnl
          & \noyau{Cl}{17}{35} &   &   &    \tbnl
          &                    & 8 &   & 16 \tbnl
\end{tabularx}


\end{exercice}


\vressort{3}


\begin{exercice}{Masse d'un atome de carbone 12}\\
Soit le carbone $12$ not� \noyau{C}{6}{12}.
\begin{enumerate}
\item L'�l�ment carbone peut-il avoir $5$ protons ? Pourquoi ?
\item Calculer la masse du noyau d'un atome de carbone $12$
sachant que la masse d'un nucl�on est $m_n = 1,67.10^{-27}~kg$
\item Calculer la masse des �lectrons de l'atome de carbone 12
sachant que la masse d'un �lectron vaut $m_e = 9,1.10^{-31}~kg$
\item Comparer la masse des �lectrons de l'atome � la masse du noyau.
Que concluez-vous ?
\item En d�duire, sans nouveau calcul, la masse de l'atome de carbone
  $12$.
\end{enumerate}
\end{exercice}

\newpage

\vressort{1}

\begin{exercice}{Couches �lectroniques}\\
Dans l'�tat le plus stable de l'atome, appel� �tat fondamental,
les �lectrons occupent successivement les couches,
en commen�ant par celles qui sont les plus proches du noyau : 
d'abord $K$ puis $L$ puis $M$.

Lorsqu'une couche est pleine, ou encore satur�e, on passe � la suivante.

La derni�re couche occup�e est appel�e couche externe.\\
Toutes les autres sont appel�es couches internes.

\medskip

\noindent
%\begin{tabularx}{\textwidth}{|>{\centering}X|>{\centering}X|>{\centering}X|>{\centering}X|}
\begin{tabularx}{\textwidth}{|>{\mystrut}X|X|X|X|}
\hline
\multicolumn{1}{|X|}{\emph{Symbole de la couche}}       & $K$ & $L$ & $M$  \tbnl
\emph{Nombre maximal d'�lectrons} & $2$ & $8$ & $18$ \tbnl
\end{tabularx}

\medskip

Ainsi, par exemple, l'atome de chlore ($Z=17$) a la configuration �lectronique :
$(K)^2(L)^8(M)^{7}$.

\begin{enumerate}
\item Indiquez le nombre d'�lectrons et donnez la configuration des atomes suivants :
  \begin{enumerate}
  \item \noyau{H}{1}{}
  \item \noyau{O}{8}{}
  \item \noyau{C}{6}{}
  \item \noyau{Ne}{10}{}
  \end{enumerate}
\item Parmi les ions ci-desssous, pr�cisez s'il s'agit d'anions ou de cations.
Indiquez le nombre d'�lectrons et donnez la configuration �lectronique
des ions suivants :
  \begin{enumerate}
  \item $Be^{2+}$ ($Z=4$)
  \item $Al^{3+}$ ($Z=13$)
  \item $O^{2-}$ ($Z=8$)
  \item $F^{-}$ ($Z=9$)
  \end{enumerate}
\end{enumerate}

\end{exercice}


\vressort{5}


\begin{exercice}{Charge d'un atome de Zinc}%\\
\begin{enumerate}
\item Combien de protons l'atome de zinc \noyau{Zn}{30}{65} contient-il ?
\item Combien d'�lectrons comporte-t-il ?
\item Calculer la charge totale des protons
sachant qu'un proton a pour charge $e = 1,6.10^{-19}~C$.
\item Calculer la charge totale des �lectrons
sachant qu'un �lectron a pour charge $-e = -1,6.10^{-19}~C$.
\item En d�duire la charge de l'atome de Zinc.
\item Ce r�sultat est-il identique pour tous les atomes ?
\item A l'issue d'une r�action dite d'oxydation, un atome de zinc $Zn$
  se transforme en un ion $Zn^{2+}$.
  \begin{enumerate}
  \item Donnez l'�quation de cette r�action
  (en faisant intervenir un ou plusieurs �lectrons not�s $e^-$).
  \item Indiquez la charge (en coulomb $C$) de cet ion.
  \end{enumerate}
\end{enumerate}
\end{exercice}

\vressort{3} % tp loi d'ohm

\ds{Devoir Surveill�}{
%
}

\nomprenomclasse

\setcounter{numexercice}{0}

%\renewcommand{\tabularx}[1]{>{\centering}m{#1}} 

%\newcommand{\tabularxc}[1]{\tabularx{>{\centering}m{#1}}}

\vressort{3}

\begin{exercice}{Connaissance sur l'atome}%\\
\begin{enumerate}
\item De quoi est compos� un atome ?
\item Que signifie les lettres $A$, $Z$ et $X$ dans la repr�sentation \noyau{X}{Z}{A} ?
\item Comment trouve-t-on le nombre de neutrons d'un atome de l'�l�ment pr�c�dent.
\item Si un atome a $5$ protons, combien-a-t-il d'�lectrons ? Pourquoi ?
\item Qu'est-ce qui caract�rise un �l�ment chimique ?
\item Qu'est-ce qu'un isotope ?
\end{enumerate}
\end{exercice}



\vressort{3}



\begin{exercice}{Composition des atomes}\\
En vous aidant du tableau p�riodique des �l�ments,
compl�ter le tableau suivant :

\medskip

\noindent
%\begin{tabularx}{\textwidth}{|>{\centering}X|>{\centering}X|>{\centering}X|>{\centering}X|>{\centering}X|}
% \begin{tabularx}{\linewidth}{|X|X|X|X|X|}
% \hline
% \emph{nom}       & \emph{symbole}  & \emph{protons} & \emph{neutrons}
% & \emph{nucl�ons} \tbnl
% carbone   & \noyau{C}{6}{14}   &         &          & \rule[-0.5cm]{0cm}{1cm}         \tbnl
% fluor     & \noyau{F}{9}{19}   &         &          & \rule[-0.5cm]{0cm}{1cm}         \tbnl
% sodium    & \noyau{Na}{11}{23} &         &          & \rule[-0.5cm]{0cm}{1cm}         \tbnl
% oxyg�ne   & \noyau{O}{8}{16}   &         &          & \rule[-0.5cm]{0cm}{1cm}         \tbnl
% hydrog�ne &          &         & 0        & \rule[-0.5cm]{0cm}{1cm}         \tbnl
%           & \noyau{Cl}{17}{35} &         &          &  \rule[-0.5cm]{0cm}{1cm}        \tbnl
%           &          & 8       & \rule[-0.5cm]{0cm}{1cm}         & 16       \tbnl
% \end{tabularx}



\begin{tabularx}{\linewidth}{|>{\mystrut}X|X|X|X|X|}
\hline
% multicolumn pour faire dispara�tre le \mystrut
\multicolumn{1}{|X|}{\emph{nom}} & \emph{symbole}  &
\emph{protons} & \emph{neutrons} & \emph{nucl�ons} \tbnl
carbone   & \noyau{C}{6}{14}   &   &   &    \tbnl
fluor     & \noyau{F}{9}{19}   &   &   &    \tbnl
sodium    & \noyau{Na}{11}{23} &   &   &    \tbnl
oxyg�ne   & \noyau{O}{8}{16}   &   &   &    \tbnl
hydrog�ne &                    &   & 0 &    \tbnl
          & \noyau{Cl}{17}{35} &   &   &    \tbnl
          &                    & 8 &   & 16 \tbnl
\end{tabularx}


\end{exercice}


\vressort{3}


\begin{exercice}{Masse d'un atome de carbone 12}\\
Soit le carbone $12$ not� \noyau{C}{6}{12}.
\begin{enumerate}
\item L'�l�ment carbone peut-il avoir $5$ protons ? Pourquoi ?
\item Calculer la masse du noyau d'un atome de carbone $12$
sachant que la masse d'un nucl�on est $m_n = 1,67.10^{-27}~kg$
\item Calculer la masse des �lectrons de l'atome de carbone 12
sachant que la masse d'un �lectron vaut $m_e = 9,1.10^{-31}~kg$
\item Comparer la masse des �lectrons de l'atome � la masse du noyau.
Que concluez-vous ?
\item En d�duire, sans nouveau calcul, la masse de l'atome de carbone
  $12$.
\end{enumerate}
\end{exercice}

\newpage

\vressort{1}

\begin{exercice}{Couches �lectroniques}\\
Dans l'�tat le plus stable de l'atome, appel� �tat fondamental,
les �lectrons occupent successivement les couches,
en commen�ant par celles qui sont les plus proches du noyau : 
d'abord $K$ puis $L$ puis $M$.

Lorsqu'une couche est pleine, ou encore satur�e, on passe � la suivante.

La derni�re couche occup�e est appel�e couche externe.\\
Toutes les autres sont appel�es couches internes.

\medskip

\noindent
%\begin{tabularx}{\textwidth}{|>{\centering}X|>{\centering}X|>{\centering}X|>{\centering}X|}
\begin{tabularx}{\textwidth}{|>{\mystrut}X|X|X|X|}
\hline
\multicolumn{1}{|X|}{\emph{Symbole de la couche}}       & $K$ & $L$ & $M$  \tbnl
\emph{Nombre maximal d'�lectrons} & $2$ & $8$ & $18$ \tbnl
\end{tabularx}

\medskip

Ainsi, par exemple, l'atome de chlore ($Z=17$) a la configuration �lectronique :
$(K)^2(L)^8(M)^{7}$.

\begin{enumerate}
\item Indiquez le nombre d'�lectrons et donnez la configuration des atomes suivants :
  \begin{enumerate}
  \item \noyau{H}{1}{}
  \item \noyau{O}{8}{}
  \item \noyau{C}{6}{}
  \item \noyau{Ne}{10}{}
  \end{enumerate}
\item Parmi les ions ci-desssous, pr�cisez s'il s'agit d'anions ou de cations.
Indiquez le nombre d'�lectrons et donnez la configuration �lectronique
des ions suivants :
  \begin{enumerate}
  \item $Be^{2+}$ ($Z=4$)
  \item $Al^{3+}$ ($Z=13$)
  \item $O^{2-}$ ($Z=8$)
  \item $F^{-}$ ($Z=9$)
  \end{enumerate}
\end{enumerate}

\end{exercice}


\vressort{5}


\begin{exercice}{Charge d'un atome de Zinc}%\\
\begin{enumerate}
\item Combien de protons l'atome de zinc \noyau{Zn}{30}{65} contient-il ?
\item Combien d'�lectrons comporte-t-il ?
\item Calculer la charge totale des protons
sachant qu'un proton a pour charge $e = 1,6.10^{-19}~C$.
\item Calculer la charge totale des �lectrons
sachant qu'un �lectron a pour charge $-e = -1,6.10^{-19}~C$.
\item En d�duire la charge de l'atome de Zinc.
\item Ce r�sultat est-il identique pour tous les atomes ?
\item A l'issue d'une r�action dite d'oxydation, un atome de zinc $Zn$
  se transforme en un ion $Zn^{2+}$.
  \begin{enumerate}
  \item Donnez l'�quation de cette r�action
  (en faisant intervenir un ou plusieurs �lectrons not�s $e^-$).
  \item Indiquez la charge (en coulomb $C$) de cet ion.
  \end{enumerate}
\end{enumerate}
\end{exercice}

\vressort{3} % tp associations de R

% association r s�rie (trac� U en fonction de I et montrer qu'on ajoute U)
% association r parall�le (trac� U en fonction de I et montrer qu'on ajoute I)



\chapitre{G�n�rateurs et r�cepteurs}
\ds{Devoir Surveill�}{
%
}

\nomprenomclasse

\setcounter{numexercice}{0}

%\renewcommand{\tabularx}[1]{>{\centering}m{#1}} 

%\newcommand{\tabularxc}[1]{\tabularx{>{\centering}m{#1}}}

\vressort{3}

\begin{exercice}{Connaissance sur l'atome}%\\
\begin{enumerate}
\item De quoi est compos� un atome ?
\item Que signifie les lettres $A$, $Z$ et $X$ dans la repr�sentation \noyau{X}{Z}{A} ?
\item Comment trouve-t-on le nombre de neutrons d'un atome de l'�l�ment pr�c�dent.
\item Si un atome a $5$ protons, combien-a-t-il d'�lectrons ? Pourquoi ?
\item Qu'est-ce qui caract�rise un �l�ment chimique ?
\item Qu'est-ce qu'un isotope ?
\end{enumerate}
\end{exercice}



\vressort{3}



\begin{exercice}{Composition des atomes}\\
En vous aidant du tableau p�riodique des �l�ments,
compl�ter le tableau suivant :

\medskip

\noindent
%\begin{tabularx}{\textwidth}{|>{\centering}X|>{\centering}X|>{\centering}X|>{\centering}X|>{\centering}X|}
% \begin{tabularx}{\linewidth}{|X|X|X|X|X|}
% \hline
% \emph{nom}       & \emph{symbole}  & \emph{protons} & \emph{neutrons}
% & \emph{nucl�ons} \tbnl
% carbone   & \noyau{C}{6}{14}   &         &          & \rule[-0.5cm]{0cm}{1cm}         \tbnl
% fluor     & \noyau{F}{9}{19}   &         &          & \rule[-0.5cm]{0cm}{1cm}         \tbnl
% sodium    & \noyau{Na}{11}{23} &         &          & \rule[-0.5cm]{0cm}{1cm}         \tbnl
% oxyg�ne   & \noyau{O}{8}{16}   &         &          & \rule[-0.5cm]{0cm}{1cm}         \tbnl
% hydrog�ne &          &         & 0        & \rule[-0.5cm]{0cm}{1cm}         \tbnl
%           & \noyau{Cl}{17}{35} &         &          &  \rule[-0.5cm]{0cm}{1cm}        \tbnl
%           &          & 8       & \rule[-0.5cm]{0cm}{1cm}         & 16       \tbnl
% \end{tabularx}



\begin{tabularx}{\linewidth}{|>{\mystrut}X|X|X|X|X|}
\hline
% multicolumn pour faire dispara�tre le \mystrut
\multicolumn{1}{|X|}{\emph{nom}} & \emph{symbole}  &
\emph{protons} & \emph{neutrons} & \emph{nucl�ons} \tbnl
carbone   & \noyau{C}{6}{14}   &   &   &    \tbnl
fluor     & \noyau{F}{9}{19}   &   &   &    \tbnl
sodium    & \noyau{Na}{11}{23} &   &   &    \tbnl
oxyg�ne   & \noyau{O}{8}{16}   &   &   &    \tbnl
hydrog�ne &                    &   & 0 &    \tbnl
          & \noyau{Cl}{17}{35} &   &   &    \tbnl
          &                    & 8 &   & 16 \tbnl
\end{tabularx}


\end{exercice}


\vressort{3}


\begin{exercice}{Masse d'un atome de carbone 12}\\
Soit le carbone $12$ not� \noyau{C}{6}{12}.
\begin{enumerate}
\item L'�l�ment carbone peut-il avoir $5$ protons ? Pourquoi ?
\item Calculer la masse du noyau d'un atome de carbone $12$
sachant que la masse d'un nucl�on est $m_n = 1,67.10^{-27}~kg$
\item Calculer la masse des �lectrons de l'atome de carbone 12
sachant que la masse d'un �lectron vaut $m_e = 9,1.10^{-31}~kg$
\item Comparer la masse des �lectrons de l'atome � la masse du noyau.
Que concluez-vous ?
\item En d�duire, sans nouveau calcul, la masse de l'atome de carbone
  $12$.
\end{enumerate}
\end{exercice}

\newpage

\vressort{1}

\begin{exercice}{Couches �lectroniques}\\
Dans l'�tat le plus stable de l'atome, appel� �tat fondamental,
les �lectrons occupent successivement les couches,
en commen�ant par celles qui sont les plus proches du noyau : 
d'abord $K$ puis $L$ puis $M$.

Lorsqu'une couche est pleine, ou encore satur�e, on passe � la suivante.

La derni�re couche occup�e est appel�e couche externe.\\
Toutes les autres sont appel�es couches internes.

\medskip

\noindent
%\begin{tabularx}{\textwidth}{|>{\centering}X|>{\centering}X|>{\centering}X|>{\centering}X|}
\begin{tabularx}{\textwidth}{|>{\mystrut}X|X|X|X|}
\hline
\multicolumn{1}{|X|}{\emph{Symbole de la couche}}       & $K$ & $L$ & $M$  \tbnl
\emph{Nombre maximal d'�lectrons} & $2$ & $8$ & $18$ \tbnl
\end{tabularx}

\medskip

Ainsi, par exemple, l'atome de chlore ($Z=17$) a la configuration �lectronique :
$(K)^2(L)^8(M)^{7}$.

\begin{enumerate}
\item Indiquez le nombre d'�lectrons et donnez la configuration des atomes suivants :
  \begin{enumerate}
  \item \noyau{H}{1}{}
  \item \noyau{O}{8}{}
  \item \noyau{C}{6}{}
  \item \noyau{Ne}{10}{}
  \end{enumerate}
\item Parmi les ions ci-desssous, pr�cisez s'il s'agit d'anions ou de cations.
Indiquez le nombre d'�lectrons et donnez la configuration �lectronique
des ions suivants :
  \begin{enumerate}
  \item $Be^{2+}$ ($Z=4$)
  \item $Al^{3+}$ ($Z=13$)
  \item $O^{2-}$ ($Z=8$)
  \item $F^{-}$ ($Z=9$)
  \end{enumerate}
\end{enumerate}

\end{exercice}


\vressort{5}


\begin{exercice}{Charge d'un atome de Zinc}%\\
\begin{enumerate}
\item Combien de protons l'atome de zinc \noyau{Zn}{30}{65} contient-il ?
\item Combien d'�lectrons comporte-t-il ?
\item Calculer la charge totale des protons
sachant qu'un proton a pour charge $e = 1,6.10^{-19}~C$.
\item Calculer la charge totale des �lectrons
sachant qu'un �lectron a pour charge $-e = -1,6.10^{-19}~C$.
\item En d�duire la charge de l'atome de Zinc.
\item Ce r�sultat est-il identique pour tous les atomes ?
\item A l'issue d'une r�action dite d'oxydation, un atome de zinc $Zn$
  se transforme en un ion $Zn^{2+}$.
  \begin{enumerate}
  \item Donnez l'�quation de cette r�action
  (en faisant intervenir un ou plusieurs �lectrons not�s $e^-$).
  \item Indiquez la charge (en coulomb $C$) de cet ion.
  \end{enumerate}
\end{enumerate}
\end{exercice}

\vressort{3} % cours elec g�n�rateurs, r�cepteurs

\ds{Devoir Surveill�}{
%
}

\nomprenomclasse

\setcounter{numexercice}{0}

%\renewcommand{\tabularx}[1]{>{\centering}m{#1}} 

%\newcommand{\tabularxc}[1]{\tabularx{>{\centering}m{#1}}}

\vressort{3}

\begin{exercice}{Connaissance sur l'atome}%\\
\begin{enumerate}
\item De quoi est compos� un atome ?
\item Que signifie les lettres $A$, $Z$ et $X$ dans la repr�sentation \noyau{X}{Z}{A} ?
\item Comment trouve-t-on le nombre de neutrons d'un atome de l'�l�ment pr�c�dent.
\item Si un atome a $5$ protons, combien-a-t-il d'�lectrons ? Pourquoi ?
\item Qu'est-ce qui caract�rise un �l�ment chimique ?
\item Qu'est-ce qu'un isotope ?
\end{enumerate}
\end{exercice}



\vressort{3}



\begin{exercice}{Composition des atomes}\\
En vous aidant du tableau p�riodique des �l�ments,
compl�ter le tableau suivant :

\medskip

\noindent
%\begin{tabularx}{\textwidth}{|>{\centering}X|>{\centering}X|>{\centering}X|>{\centering}X|>{\centering}X|}
% \begin{tabularx}{\linewidth}{|X|X|X|X|X|}
% \hline
% \emph{nom}       & \emph{symbole}  & \emph{protons} & \emph{neutrons}
% & \emph{nucl�ons} \tbnl
% carbone   & \noyau{C}{6}{14}   &         &          & \rule[-0.5cm]{0cm}{1cm}         \tbnl
% fluor     & \noyau{F}{9}{19}   &         &          & \rule[-0.5cm]{0cm}{1cm}         \tbnl
% sodium    & \noyau{Na}{11}{23} &         &          & \rule[-0.5cm]{0cm}{1cm}         \tbnl
% oxyg�ne   & \noyau{O}{8}{16}   &         &          & \rule[-0.5cm]{0cm}{1cm}         \tbnl
% hydrog�ne &          &         & 0        & \rule[-0.5cm]{0cm}{1cm}         \tbnl
%           & \noyau{Cl}{17}{35} &         &          &  \rule[-0.5cm]{0cm}{1cm}        \tbnl
%           &          & 8       & \rule[-0.5cm]{0cm}{1cm}         & 16       \tbnl
% \end{tabularx}



\begin{tabularx}{\linewidth}{|>{\mystrut}X|X|X|X|X|}
\hline
% multicolumn pour faire dispara�tre le \mystrut
\multicolumn{1}{|X|}{\emph{nom}} & \emph{symbole}  &
\emph{protons} & \emph{neutrons} & \emph{nucl�ons} \tbnl
carbone   & \noyau{C}{6}{14}   &   &   &    \tbnl
fluor     & \noyau{F}{9}{19}   &   &   &    \tbnl
sodium    & \noyau{Na}{11}{23} &   &   &    \tbnl
oxyg�ne   & \noyau{O}{8}{16}   &   &   &    \tbnl
hydrog�ne &                    &   & 0 &    \tbnl
          & \noyau{Cl}{17}{35} &   &   &    \tbnl
          &                    & 8 &   & 16 \tbnl
\end{tabularx}


\end{exercice}


\vressort{3}


\begin{exercice}{Masse d'un atome de carbone 12}\\
Soit le carbone $12$ not� \noyau{C}{6}{12}.
\begin{enumerate}
\item L'�l�ment carbone peut-il avoir $5$ protons ? Pourquoi ?
\item Calculer la masse du noyau d'un atome de carbone $12$
sachant que la masse d'un nucl�on est $m_n = 1,67.10^{-27}~kg$
\item Calculer la masse des �lectrons de l'atome de carbone 12
sachant que la masse d'un �lectron vaut $m_e = 9,1.10^{-31}~kg$
\item Comparer la masse des �lectrons de l'atome � la masse du noyau.
Que concluez-vous ?
\item En d�duire, sans nouveau calcul, la masse de l'atome de carbone
  $12$.
\end{enumerate}
\end{exercice}

\newpage

\vressort{1}

\begin{exercice}{Couches �lectroniques}\\
Dans l'�tat le plus stable de l'atome, appel� �tat fondamental,
les �lectrons occupent successivement les couches,
en commen�ant par celles qui sont les plus proches du noyau : 
d'abord $K$ puis $L$ puis $M$.

Lorsqu'une couche est pleine, ou encore satur�e, on passe � la suivante.

La derni�re couche occup�e est appel�e couche externe.\\
Toutes les autres sont appel�es couches internes.

\medskip

\noindent
%\begin{tabularx}{\textwidth}{|>{\centering}X|>{\centering}X|>{\centering}X|>{\centering}X|}
\begin{tabularx}{\textwidth}{|>{\mystrut}X|X|X|X|}
\hline
\multicolumn{1}{|X|}{\emph{Symbole de la couche}}       & $K$ & $L$ & $M$  \tbnl
\emph{Nombre maximal d'�lectrons} & $2$ & $8$ & $18$ \tbnl
\end{tabularx}

\medskip

Ainsi, par exemple, l'atome de chlore ($Z=17$) a la configuration �lectronique :
$(K)^2(L)^8(M)^{7}$.

\begin{enumerate}
\item Indiquez le nombre d'�lectrons et donnez la configuration des atomes suivants :
  \begin{enumerate}
  \item \noyau{H}{1}{}
  \item \noyau{O}{8}{}
  \item \noyau{C}{6}{}
  \item \noyau{Ne}{10}{}
  \end{enumerate}
\item Parmi les ions ci-desssous, pr�cisez s'il s'agit d'anions ou de cations.
Indiquez le nombre d'�lectrons et donnez la configuration �lectronique
des ions suivants :
  \begin{enumerate}
  \item $Be^{2+}$ ($Z=4$)
  \item $Al^{3+}$ ($Z=13$)
  \item $O^{2-}$ ($Z=8$)
  \item $F^{-}$ ($Z=9$)
  \end{enumerate}
\end{enumerate}

\end{exercice}


\vressort{5}


\begin{exercice}{Charge d'un atome de Zinc}%\\
\begin{enumerate}
\item Combien de protons l'atome de zinc \noyau{Zn}{30}{65} contient-il ?
\item Combien d'�lectrons comporte-t-il ?
\item Calculer la charge totale des protons
sachant qu'un proton a pour charge $e = 1,6.10^{-19}~C$.
\item Calculer la charge totale des �lectrons
sachant qu'un �lectron a pour charge $-e = -1,6.10^{-19}~C$.
\item En d�duire la charge de l'atome de Zinc.
\item Ce r�sultat est-il identique pour tous les atomes ?
\item A l'issue d'une r�action dite d'oxydation, un atome de zinc $Zn$
  se transforme en un ion $Zn^{2+}$.
  \begin{enumerate}
  \item Donnez l'�quation de cette r�action
  (en faisant intervenir un ou plusieurs �lectrons not�s $e^-$).
  \item Indiquez la charge (en coulomb $C$) de cet ion.
  \end{enumerate}
\end{enumerate}
\end{exercice}

\vressort{3} % tp caract�ristique d'un g�n�rateur
\ds{Devoir Surveill�}{
%
}

\nomprenomclasse

\setcounter{numexercice}{0}

%\renewcommand{\tabularx}[1]{>{\centering}m{#1}} 

%\newcommand{\tabularxc}[1]{\tabularx{>{\centering}m{#1}}}

\vressort{3}

\begin{exercice}{Connaissance sur l'atome}%\\
\begin{enumerate}
\item De quoi est compos� un atome ?
\item Que signifie les lettres $A$, $Z$ et $X$ dans la repr�sentation \noyau{X}{Z}{A} ?
\item Comment trouve-t-on le nombre de neutrons d'un atome de l'�l�ment pr�c�dent.
\item Si un atome a $5$ protons, combien-a-t-il d'�lectrons ? Pourquoi ?
\item Qu'est-ce qui caract�rise un �l�ment chimique ?
\item Qu'est-ce qu'un isotope ?
\end{enumerate}
\end{exercice}



\vressort{3}



\begin{exercice}{Composition des atomes}\\
En vous aidant du tableau p�riodique des �l�ments,
compl�ter le tableau suivant :

\medskip

\noindent
%\begin{tabularx}{\textwidth}{|>{\centering}X|>{\centering}X|>{\centering}X|>{\centering}X|>{\centering}X|}
% \begin{tabularx}{\linewidth}{|X|X|X|X|X|}
% \hline
% \emph{nom}       & \emph{symbole}  & \emph{protons} & \emph{neutrons}
% & \emph{nucl�ons} \tbnl
% carbone   & \noyau{C}{6}{14}   &         &          & \rule[-0.5cm]{0cm}{1cm}         \tbnl
% fluor     & \noyau{F}{9}{19}   &         &          & \rule[-0.5cm]{0cm}{1cm}         \tbnl
% sodium    & \noyau{Na}{11}{23} &         &          & \rule[-0.5cm]{0cm}{1cm}         \tbnl
% oxyg�ne   & \noyau{O}{8}{16}   &         &          & \rule[-0.5cm]{0cm}{1cm}         \tbnl
% hydrog�ne &          &         & 0        & \rule[-0.5cm]{0cm}{1cm}         \tbnl
%           & \noyau{Cl}{17}{35} &         &          &  \rule[-0.5cm]{0cm}{1cm}        \tbnl
%           &          & 8       & \rule[-0.5cm]{0cm}{1cm}         & 16       \tbnl
% \end{tabularx}



\begin{tabularx}{\linewidth}{|>{\mystrut}X|X|X|X|X|}
\hline
% multicolumn pour faire dispara�tre le \mystrut
\multicolumn{1}{|X|}{\emph{nom}} & \emph{symbole}  &
\emph{protons} & \emph{neutrons} & \emph{nucl�ons} \tbnl
carbone   & \noyau{C}{6}{14}   &   &   &    \tbnl
fluor     & \noyau{F}{9}{19}   &   &   &    \tbnl
sodium    & \noyau{Na}{11}{23} &   &   &    \tbnl
oxyg�ne   & \noyau{O}{8}{16}   &   &   &    \tbnl
hydrog�ne &                    &   & 0 &    \tbnl
          & \noyau{Cl}{17}{35} &   &   &    \tbnl
          &                    & 8 &   & 16 \tbnl
\end{tabularx}


\end{exercice}


\vressort{3}


\begin{exercice}{Masse d'un atome de carbone 12}\\
Soit le carbone $12$ not� \noyau{C}{6}{12}.
\begin{enumerate}
\item L'�l�ment carbone peut-il avoir $5$ protons ? Pourquoi ?
\item Calculer la masse du noyau d'un atome de carbone $12$
sachant que la masse d'un nucl�on est $m_n = 1,67.10^{-27}~kg$
\item Calculer la masse des �lectrons de l'atome de carbone 12
sachant que la masse d'un �lectron vaut $m_e = 9,1.10^{-31}~kg$
\item Comparer la masse des �lectrons de l'atome � la masse du noyau.
Que concluez-vous ?
\item En d�duire, sans nouveau calcul, la masse de l'atome de carbone
  $12$.
\end{enumerate}
\end{exercice}

\newpage

\vressort{1}

\begin{exercice}{Couches �lectroniques}\\
Dans l'�tat le plus stable de l'atome, appel� �tat fondamental,
les �lectrons occupent successivement les couches,
en commen�ant par celles qui sont les plus proches du noyau : 
d'abord $K$ puis $L$ puis $M$.

Lorsqu'une couche est pleine, ou encore satur�e, on passe � la suivante.

La derni�re couche occup�e est appel�e couche externe.\\
Toutes les autres sont appel�es couches internes.

\medskip

\noindent
%\begin{tabularx}{\textwidth}{|>{\centering}X|>{\centering}X|>{\centering}X|>{\centering}X|}
\begin{tabularx}{\textwidth}{|>{\mystrut}X|X|X|X|}
\hline
\multicolumn{1}{|X|}{\emph{Symbole de la couche}}       & $K$ & $L$ & $M$  \tbnl
\emph{Nombre maximal d'�lectrons} & $2$ & $8$ & $18$ \tbnl
\end{tabularx}

\medskip

Ainsi, par exemple, l'atome de chlore ($Z=17$) a la configuration �lectronique :
$(K)^2(L)^8(M)^{7}$.

\begin{enumerate}
\item Indiquez le nombre d'�lectrons et donnez la configuration des atomes suivants :
  \begin{enumerate}
  \item \noyau{H}{1}{}
  \item \noyau{O}{8}{}
  \item \noyau{C}{6}{}
  \item \noyau{Ne}{10}{}
  \end{enumerate}
\item Parmi les ions ci-desssous, pr�cisez s'il s'agit d'anions ou de cations.
Indiquez le nombre d'�lectrons et donnez la configuration �lectronique
des ions suivants :
  \begin{enumerate}
  \item $Be^{2+}$ ($Z=4$)
  \item $Al^{3+}$ ($Z=13$)
  \item $O^{2-}$ ($Z=8$)
  \item $F^{-}$ ($Z=9$)
  \end{enumerate}
\end{enumerate}

\end{exercice}


\vressort{5}


\begin{exercice}{Charge d'un atome de Zinc}%\\
\begin{enumerate}
\item Combien de protons l'atome de zinc \noyau{Zn}{30}{65} contient-il ?
\item Combien d'�lectrons comporte-t-il ?
\item Calculer la charge totale des protons
sachant qu'un proton a pour charge $e = 1,6.10^{-19}~C$.
\item Calculer la charge totale des �lectrons
sachant qu'un �lectron a pour charge $-e = -1,6.10^{-19}~C$.
\item En d�duire la charge de l'atome de Zinc.
\item Ce r�sultat est-il identique pour tous les atomes ?
\item A l'issue d'une r�action dite d'oxydation, un atome de zinc $Zn$
  se transforme en un ion $Zn^{2+}$.
  \begin{enumerate}
  \item Donnez l'�quation de cette r�action
  (en faisant intervenir un ou plusieurs �lectrons not�s $e^-$).
  \item Indiquez la charge (en coulomb $C$) de cet ion.
  \end{enumerate}
\end{enumerate}
\end{exercice}

\vressort{3} % tp caract�ristique d'un
                                % r�cepteur (�lectrolyseur)



\chapitre{\'Electrostatique}
\ds{Devoir Surveill�}{
%
}

\nomprenomclasse

\setcounter{numexercice}{0}

%\renewcommand{\tabularx}[1]{>{\centering}m{#1}} 

%\newcommand{\tabularxc}[1]{\tabularx{>{\centering}m{#1}}}

\vressort{3}

\begin{exercice}{Connaissance sur l'atome}%\\
\begin{enumerate}
\item De quoi est compos� un atome ?
\item Que signifie les lettres $A$, $Z$ et $X$ dans la repr�sentation \noyau{X}{Z}{A} ?
\item Comment trouve-t-on le nombre de neutrons d'un atome de l'�l�ment pr�c�dent.
\item Si un atome a $5$ protons, combien-a-t-il d'�lectrons ? Pourquoi ?
\item Qu'est-ce qui caract�rise un �l�ment chimique ?
\item Qu'est-ce qu'un isotope ?
\end{enumerate}
\end{exercice}



\vressort{3}



\begin{exercice}{Composition des atomes}\\
En vous aidant du tableau p�riodique des �l�ments,
compl�ter le tableau suivant :

\medskip

\noindent
%\begin{tabularx}{\textwidth}{|>{\centering}X|>{\centering}X|>{\centering}X|>{\centering}X|>{\centering}X|}
% \begin{tabularx}{\linewidth}{|X|X|X|X|X|}
% \hline
% \emph{nom}       & \emph{symbole}  & \emph{protons} & \emph{neutrons}
% & \emph{nucl�ons} \tbnl
% carbone   & \noyau{C}{6}{14}   &         &          & \rule[-0.5cm]{0cm}{1cm}         \tbnl
% fluor     & \noyau{F}{9}{19}   &         &          & \rule[-0.5cm]{0cm}{1cm}         \tbnl
% sodium    & \noyau{Na}{11}{23} &         &          & \rule[-0.5cm]{0cm}{1cm}         \tbnl
% oxyg�ne   & \noyau{O}{8}{16}   &         &          & \rule[-0.5cm]{0cm}{1cm}         \tbnl
% hydrog�ne &          &         & 0        & \rule[-0.5cm]{0cm}{1cm}         \tbnl
%           & \noyau{Cl}{17}{35} &         &          &  \rule[-0.5cm]{0cm}{1cm}        \tbnl
%           &          & 8       & \rule[-0.5cm]{0cm}{1cm}         & 16       \tbnl
% \end{tabularx}



\begin{tabularx}{\linewidth}{|>{\mystrut}X|X|X|X|X|}
\hline
% multicolumn pour faire dispara�tre le \mystrut
\multicolumn{1}{|X|}{\emph{nom}} & \emph{symbole}  &
\emph{protons} & \emph{neutrons} & \emph{nucl�ons} \tbnl
carbone   & \noyau{C}{6}{14}   &   &   &    \tbnl
fluor     & \noyau{F}{9}{19}   &   &   &    \tbnl
sodium    & \noyau{Na}{11}{23} &   &   &    \tbnl
oxyg�ne   & \noyau{O}{8}{16}   &   &   &    \tbnl
hydrog�ne &                    &   & 0 &    \tbnl
          & \noyau{Cl}{17}{35} &   &   &    \tbnl
          &                    & 8 &   & 16 \tbnl
\end{tabularx}


\end{exercice}


\vressort{3}


\begin{exercice}{Masse d'un atome de carbone 12}\\
Soit le carbone $12$ not� \noyau{C}{6}{12}.
\begin{enumerate}
\item L'�l�ment carbone peut-il avoir $5$ protons ? Pourquoi ?
\item Calculer la masse du noyau d'un atome de carbone $12$
sachant que la masse d'un nucl�on est $m_n = 1,67.10^{-27}~kg$
\item Calculer la masse des �lectrons de l'atome de carbone 12
sachant que la masse d'un �lectron vaut $m_e = 9,1.10^{-31}~kg$
\item Comparer la masse des �lectrons de l'atome � la masse du noyau.
Que concluez-vous ?
\item En d�duire, sans nouveau calcul, la masse de l'atome de carbone
  $12$.
\end{enumerate}
\end{exercice}

\newpage

\vressort{1}

\begin{exercice}{Couches �lectroniques}\\
Dans l'�tat le plus stable de l'atome, appel� �tat fondamental,
les �lectrons occupent successivement les couches,
en commen�ant par celles qui sont les plus proches du noyau : 
d'abord $K$ puis $L$ puis $M$.

Lorsqu'une couche est pleine, ou encore satur�e, on passe � la suivante.

La derni�re couche occup�e est appel�e couche externe.\\
Toutes les autres sont appel�es couches internes.

\medskip

\noindent
%\begin{tabularx}{\textwidth}{|>{\centering}X|>{\centering}X|>{\centering}X|>{\centering}X|}
\begin{tabularx}{\textwidth}{|>{\mystrut}X|X|X|X|}
\hline
\multicolumn{1}{|X|}{\emph{Symbole de la couche}}       & $K$ & $L$ & $M$  \tbnl
\emph{Nombre maximal d'�lectrons} & $2$ & $8$ & $18$ \tbnl
\end{tabularx}

\medskip

Ainsi, par exemple, l'atome de chlore ($Z=17$) a la configuration �lectronique :
$(K)^2(L)^8(M)^{7}$.

\begin{enumerate}
\item Indiquez le nombre d'�lectrons et donnez la configuration des atomes suivants :
  \begin{enumerate}
  \item \noyau{H}{1}{}
  \item \noyau{O}{8}{}
  \item \noyau{C}{6}{}
  \item \noyau{Ne}{10}{}
  \end{enumerate}
\item Parmi les ions ci-desssous, pr�cisez s'il s'agit d'anions ou de cations.
Indiquez le nombre d'�lectrons et donnez la configuration �lectronique
des ions suivants :
  \begin{enumerate}
  \item $Be^{2+}$ ($Z=4$)
  \item $Al^{3+}$ ($Z=13$)
  \item $O^{2-}$ ($Z=8$)
  \item $F^{-}$ ($Z=9$)
  \end{enumerate}
\end{enumerate}

\end{exercice}


\vressort{5}


\begin{exercice}{Charge d'un atome de Zinc}%\\
\begin{enumerate}
\item Combien de protons l'atome de zinc \noyau{Zn}{30}{65} contient-il ?
\item Combien d'�lectrons comporte-t-il ?
\item Calculer la charge totale des protons
sachant qu'un proton a pour charge $e = 1,6.10^{-19}~C$.
\item Calculer la charge totale des �lectrons
sachant qu'un �lectron a pour charge $-e = -1,6.10^{-19}~C$.
\item En d�duire la charge de l'atome de Zinc.
\item Ce r�sultat est-il identique pour tous les atomes ?
\item A l'issue d'une r�action dite d'oxydation, un atome de zinc $Zn$
  se transforme en un ion $Zn^{2+}$.
  \begin{enumerate}
  \item Donnez l'�quation de cette r�action
  (en faisant intervenir un ou plusieurs �lectrons not�s $e^-$).
  \item Indiquez la charge (en coulomb $C$) de cet ion.
  \end{enumerate}
\end{enumerate}
\end{exercice}

\vressort{3}



\chapitre{Oxydo-r�duction}
\ds{Devoir Surveill�}{
%
}

\nomprenomclasse

\setcounter{numexercice}{0}

%\renewcommand{\tabularx}[1]{>{\centering}m{#1}} 

%\newcommand{\tabularxc}[1]{\tabularx{>{\centering}m{#1}}}

\vressort{3}

\begin{exercice}{Connaissance sur l'atome}%\\
\begin{enumerate}
\item De quoi est compos� un atome ?
\item Que signifie les lettres $A$, $Z$ et $X$ dans la repr�sentation \noyau{X}{Z}{A} ?
\item Comment trouve-t-on le nombre de neutrons d'un atome de l'�l�ment pr�c�dent.
\item Si un atome a $5$ protons, combien-a-t-il d'�lectrons ? Pourquoi ?
\item Qu'est-ce qui caract�rise un �l�ment chimique ?
\item Qu'est-ce qu'un isotope ?
\end{enumerate}
\end{exercice}



\vressort{3}



\begin{exercice}{Composition des atomes}\\
En vous aidant du tableau p�riodique des �l�ments,
compl�ter le tableau suivant :

\medskip

\noindent
%\begin{tabularx}{\textwidth}{|>{\centering}X|>{\centering}X|>{\centering}X|>{\centering}X|>{\centering}X|}
% \begin{tabularx}{\linewidth}{|X|X|X|X|X|}
% \hline
% \emph{nom}       & \emph{symbole}  & \emph{protons} & \emph{neutrons}
% & \emph{nucl�ons} \tbnl
% carbone   & \noyau{C}{6}{14}   &         &          & \rule[-0.5cm]{0cm}{1cm}         \tbnl
% fluor     & \noyau{F}{9}{19}   &         &          & \rule[-0.5cm]{0cm}{1cm}         \tbnl
% sodium    & \noyau{Na}{11}{23} &         &          & \rule[-0.5cm]{0cm}{1cm}         \tbnl
% oxyg�ne   & \noyau{O}{8}{16}   &         &          & \rule[-0.5cm]{0cm}{1cm}         \tbnl
% hydrog�ne &          &         & 0        & \rule[-0.5cm]{0cm}{1cm}         \tbnl
%           & \noyau{Cl}{17}{35} &         &          &  \rule[-0.5cm]{0cm}{1cm}        \tbnl
%           &          & 8       & \rule[-0.5cm]{0cm}{1cm}         & 16       \tbnl
% \end{tabularx}



\begin{tabularx}{\linewidth}{|>{\mystrut}X|X|X|X|X|}
\hline
% multicolumn pour faire dispara�tre le \mystrut
\multicolumn{1}{|X|}{\emph{nom}} & \emph{symbole}  &
\emph{protons} & \emph{neutrons} & \emph{nucl�ons} \tbnl
carbone   & \noyau{C}{6}{14}   &   &   &    \tbnl
fluor     & \noyau{F}{9}{19}   &   &   &    \tbnl
sodium    & \noyau{Na}{11}{23} &   &   &    \tbnl
oxyg�ne   & \noyau{O}{8}{16}   &   &   &    \tbnl
hydrog�ne &                    &   & 0 &    \tbnl
          & \noyau{Cl}{17}{35} &   &   &    \tbnl
          &                    & 8 &   & 16 \tbnl
\end{tabularx}


\end{exercice}


\vressort{3}


\begin{exercice}{Masse d'un atome de carbone 12}\\
Soit le carbone $12$ not� \noyau{C}{6}{12}.
\begin{enumerate}
\item L'�l�ment carbone peut-il avoir $5$ protons ? Pourquoi ?
\item Calculer la masse du noyau d'un atome de carbone $12$
sachant que la masse d'un nucl�on est $m_n = 1,67.10^{-27}~kg$
\item Calculer la masse des �lectrons de l'atome de carbone 12
sachant que la masse d'un �lectron vaut $m_e = 9,1.10^{-31}~kg$
\item Comparer la masse des �lectrons de l'atome � la masse du noyau.
Que concluez-vous ?
\item En d�duire, sans nouveau calcul, la masse de l'atome de carbone
  $12$.
\end{enumerate}
\end{exercice}

\newpage

\vressort{1}

\begin{exercice}{Couches �lectroniques}\\
Dans l'�tat le plus stable de l'atome, appel� �tat fondamental,
les �lectrons occupent successivement les couches,
en commen�ant par celles qui sont les plus proches du noyau : 
d'abord $K$ puis $L$ puis $M$.

Lorsqu'une couche est pleine, ou encore satur�e, on passe � la suivante.

La derni�re couche occup�e est appel�e couche externe.\\
Toutes les autres sont appel�es couches internes.

\medskip

\noindent
%\begin{tabularx}{\textwidth}{|>{\centering}X|>{\centering}X|>{\centering}X|>{\centering}X|}
\begin{tabularx}{\textwidth}{|>{\mystrut}X|X|X|X|}
\hline
\multicolumn{1}{|X|}{\emph{Symbole de la couche}}       & $K$ & $L$ & $M$  \tbnl
\emph{Nombre maximal d'�lectrons} & $2$ & $8$ & $18$ \tbnl
\end{tabularx}

\medskip

Ainsi, par exemple, l'atome de chlore ($Z=17$) a la configuration �lectronique :
$(K)^2(L)^8(M)^{7}$.

\begin{enumerate}
\item Indiquez le nombre d'�lectrons et donnez la configuration des atomes suivants :
  \begin{enumerate}
  \item \noyau{H}{1}{}
  \item \noyau{O}{8}{}
  \item \noyau{C}{6}{}
  \item \noyau{Ne}{10}{}
  \end{enumerate}
\item Parmi les ions ci-desssous, pr�cisez s'il s'agit d'anions ou de cations.
Indiquez le nombre d'�lectrons et donnez la configuration �lectronique
des ions suivants :
  \begin{enumerate}
  \item $Be^{2+}$ ($Z=4$)
  \item $Al^{3+}$ ($Z=13$)
  \item $O^{2-}$ ($Z=8$)
  \item $F^{-}$ ($Z=9$)
  \end{enumerate}
\end{enumerate}

\end{exercice}


\vressort{5}


\begin{exercice}{Charge d'un atome de Zinc}%\\
\begin{enumerate}
\item Combien de protons l'atome de zinc \noyau{Zn}{30}{65} contient-il ?
\item Combien d'�lectrons comporte-t-il ?
\item Calculer la charge totale des protons
sachant qu'un proton a pour charge $e = 1,6.10^{-19}~C$.
\item Calculer la charge totale des �lectrons
sachant qu'un �lectron a pour charge $-e = -1,6.10^{-19}~C$.
\item En d�duire la charge de l'atome de Zinc.
\item Ce r�sultat est-il identique pour tous les atomes ?
\item A l'issue d'une r�action dite d'oxydation, un atome de zinc $Zn$
  se transforme en un ion $Zn^{2+}$.
  \begin{enumerate}
  \item Donnez l'�quation de cette r�action
  (en faisant intervenir un ou plusieurs �lectrons not�s $e^-$).
  \item Indiquez la charge (en coulomb $C$) de cet ion.
  \end{enumerate}
\end{enumerate}
\end{exercice}

\vressort{3}


\chapitre{La lumi�re}
\ds{Devoir Surveill�}{
%
}

\nomprenomclasse

\setcounter{numexercice}{0}

%\renewcommand{\tabularx}[1]{>{\centering}m{#1}} 

%\newcommand{\tabularxc}[1]{\tabularx{>{\centering}m{#1}}}

\vressort{3}

\begin{exercice}{Connaissance sur l'atome}%\\
\begin{enumerate}
\item De quoi est compos� un atome ?
\item Que signifie les lettres $A$, $Z$ et $X$ dans la repr�sentation \noyau{X}{Z}{A} ?
\item Comment trouve-t-on le nombre de neutrons d'un atome de l'�l�ment pr�c�dent.
\item Si un atome a $5$ protons, combien-a-t-il d'�lectrons ? Pourquoi ?
\item Qu'est-ce qui caract�rise un �l�ment chimique ?
\item Qu'est-ce qu'un isotope ?
\end{enumerate}
\end{exercice}



\vressort{3}



\begin{exercice}{Composition des atomes}\\
En vous aidant du tableau p�riodique des �l�ments,
compl�ter le tableau suivant :

\medskip

\noindent
%\begin{tabularx}{\textwidth}{|>{\centering}X|>{\centering}X|>{\centering}X|>{\centering}X|>{\centering}X|}
% \begin{tabularx}{\linewidth}{|X|X|X|X|X|}
% \hline
% \emph{nom}       & \emph{symbole}  & \emph{protons} & \emph{neutrons}
% & \emph{nucl�ons} \tbnl
% carbone   & \noyau{C}{6}{14}   &         &          & \rule[-0.5cm]{0cm}{1cm}         \tbnl
% fluor     & \noyau{F}{9}{19}   &         &          & \rule[-0.5cm]{0cm}{1cm}         \tbnl
% sodium    & \noyau{Na}{11}{23} &         &          & \rule[-0.5cm]{0cm}{1cm}         \tbnl
% oxyg�ne   & \noyau{O}{8}{16}   &         &          & \rule[-0.5cm]{0cm}{1cm}         \tbnl
% hydrog�ne &          &         & 0        & \rule[-0.5cm]{0cm}{1cm}         \tbnl
%           & \noyau{Cl}{17}{35} &         &          &  \rule[-0.5cm]{0cm}{1cm}        \tbnl
%           &          & 8       & \rule[-0.5cm]{0cm}{1cm}         & 16       \tbnl
% \end{tabularx}



\begin{tabularx}{\linewidth}{|>{\mystrut}X|X|X|X|X|}
\hline
% multicolumn pour faire dispara�tre le \mystrut
\multicolumn{1}{|X|}{\emph{nom}} & \emph{symbole}  &
\emph{protons} & \emph{neutrons} & \emph{nucl�ons} \tbnl
carbone   & \noyau{C}{6}{14}   &   &   &    \tbnl
fluor     & \noyau{F}{9}{19}   &   &   &    \tbnl
sodium    & \noyau{Na}{11}{23} &   &   &    \tbnl
oxyg�ne   & \noyau{O}{8}{16}   &   &   &    \tbnl
hydrog�ne &                    &   & 0 &    \tbnl
          & \noyau{Cl}{17}{35} &   &   &    \tbnl
          &                    & 8 &   & 16 \tbnl
\end{tabularx}


\end{exercice}


\vressort{3}


\begin{exercice}{Masse d'un atome de carbone 12}\\
Soit le carbone $12$ not� \noyau{C}{6}{12}.
\begin{enumerate}
\item L'�l�ment carbone peut-il avoir $5$ protons ? Pourquoi ?
\item Calculer la masse du noyau d'un atome de carbone $12$
sachant que la masse d'un nucl�on est $m_n = 1,67.10^{-27}~kg$
\item Calculer la masse des �lectrons de l'atome de carbone 12
sachant que la masse d'un �lectron vaut $m_e = 9,1.10^{-31}~kg$
\item Comparer la masse des �lectrons de l'atome � la masse du noyau.
Que concluez-vous ?
\item En d�duire, sans nouveau calcul, la masse de l'atome de carbone
  $12$.
\end{enumerate}
\end{exercice}

\newpage

\vressort{1}

\begin{exercice}{Couches �lectroniques}\\
Dans l'�tat le plus stable de l'atome, appel� �tat fondamental,
les �lectrons occupent successivement les couches,
en commen�ant par celles qui sont les plus proches du noyau : 
d'abord $K$ puis $L$ puis $M$.

Lorsqu'une couche est pleine, ou encore satur�e, on passe � la suivante.

La derni�re couche occup�e est appel�e couche externe.\\
Toutes les autres sont appel�es couches internes.

\medskip

\noindent
%\begin{tabularx}{\textwidth}{|>{\centering}X|>{\centering}X|>{\centering}X|>{\centering}X|}
\begin{tabularx}{\textwidth}{|>{\mystrut}X|X|X|X|}
\hline
\multicolumn{1}{|X|}{\emph{Symbole de la couche}}       & $K$ & $L$ & $M$  \tbnl
\emph{Nombre maximal d'�lectrons} & $2$ & $8$ & $18$ \tbnl
\end{tabularx}

\medskip

Ainsi, par exemple, l'atome de chlore ($Z=17$) a la configuration �lectronique :
$(K)^2(L)^8(M)^{7}$.

\begin{enumerate}
\item Indiquez le nombre d'�lectrons et donnez la configuration des atomes suivants :
  \begin{enumerate}
  \item \noyau{H}{1}{}
  \item \noyau{O}{8}{}
  \item \noyau{C}{6}{}
  \item \noyau{Ne}{10}{}
  \end{enumerate}
\item Parmi les ions ci-desssous, pr�cisez s'il s'agit d'anions ou de cations.
Indiquez le nombre d'�lectrons et donnez la configuration �lectronique
des ions suivants :
  \begin{enumerate}
  \item $Be^{2+}$ ($Z=4$)
  \item $Al^{3+}$ ($Z=13$)
  \item $O^{2-}$ ($Z=8$)
  \item $F^{-}$ ($Z=9$)
  \end{enumerate}
\end{enumerate}

\end{exercice}


\vressort{5}


\begin{exercice}{Charge d'un atome de Zinc}%\\
\begin{enumerate}
\item Combien de protons l'atome de zinc \noyau{Zn}{30}{65} contient-il ?
\item Combien d'�lectrons comporte-t-il ?
\item Calculer la charge totale des protons
sachant qu'un proton a pour charge $e = 1,6.10^{-19}~C$.
\item Calculer la charge totale des �lectrons
sachant qu'un �lectron a pour charge $-e = -1,6.10^{-19}~C$.
\item En d�duire la charge de l'atome de Zinc.
\item Ce r�sultat est-il identique pour tous les atomes ?
\item A l'issue d'une r�action dite d'oxydation, un atome de zinc $Zn$
  se transforme en un ion $Zn^{2+}$.
  \begin{enumerate}
  \item Donnez l'�quation de cette r�action
  (en faisant intervenir un ou plusieurs �lectrons not�s $e^-$).
  \item Indiquez la charge (en coulomb $C$) de cet ion.
  \end{enumerate}
\end{enumerate}
\end{exercice}

\vressort{3}


% \chapitre{Spectroscopie}
% %\input{
 

% \chapitre{Rayons X}


\chapitre{Radioactivit�}
\ds{Devoir Surveill�}{
%
}

\nomprenomclasse

\setcounter{numexercice}{0}

%\renewcommand{\tabularx}[1]{>{\centering}m{#1}} 

%\newcommand{\tabularxc}[1]{\tabularx{>{\centering}m{#1}}}

\vressort{3}

\begin{exercice}{Connaissance sur l'atome}%\\
\begin{enumerate}
\item De quoi est compos� un atome ?
\item Que signifie les lettres $A$, $Z$ et $X$ dans la repr�sentation \noyau{X}{Z}{A} ?
\item Comment trouve-t-on le nombre de neutrons d'un atome de l'�l�ment pr�c�dent.
\item Si un atome a $5$ protons, combien-a-t-il d'�lectrons ? Pourquoi ?
\item Qu'est-ce qui caract�rise un �l�ment chimique ?
\item Qu'est-ce qu'un isotope ?
\end{enumerate}
\end{exercice}



\vressort{3}



\begin{exercice}{Composition des atomes}\\
En vous aidant du tableau p�riodique des �l�ments,
compl�ter le tableau suivant :

\medskip

\noindent
%\begin{tabularx}{\textwidth}{|>{\centering}X|>{\centering}X|>{\centering}X|>{\centering}X|>{\centering}X|}
% \begin{tabularx}{\linewidth}{|X|X|X|X|X|}
% \hline
% \emph{nom}       & \emph{symbole}  & \emph{protons} & \emph{neutrons}
% & \emph{nucl�ons} \tbnl
% carbone   & \noyau{C}{6}{14}   &         &          & \rule[-0.5cm]{0cm}{1cm}         \tbnl
% fluor     & \noyau{F}{9}{19}   &         &          & \rule[-0.5cm]{0cm}{1cm}         \tbnl
% sodium    & \noyau{Na}{11}{23} &         &          & \rule[-0.5cm]{0cm}{1cm}         \tbnl
% oxyg�ne   & \noyau{O}{8}{16}   &         &          & \rule[-0.5cm]{0cm}{1cm}         \tbnl
% hydrog�ne &          &         & 0        & \rule[-0.5cm]{0cm}{1cm}         \tbnl
%           & \noyau{Cl}{17}{35} &         &          &  \rule[-0.5cm]{0cm}{1cm}        \tbnl
%           &          & 8       & \rule[-0.5cm]{0cm}{1cm}         & 16       \tbnl
% \end{tabularx}



\begin{tabularx}{\linewidth}{|>{\mystrut}X|X|X|X|X|}
\hline
% multicolumn pour faire dispara�tre le \mystrut
\multicolumn{1}{|X|}{\emph{nom}} & \emph{symbole}  &
\emph{protons} & \emph{neutrons} & \emph{nucl�ons} \tbnl
carbone   & \noyau{C}{6}{14}   &   &   &    \tbnl
fluor     & \noyau{F}{9}{19}   &   &   &    \tbnl
sodium    & \noyau{Na}{11}{23} &   &   &    \tbnl
oxyg�ne   & \noyau{O}{8}{16}   &   &   &    \tbnl
hydrog�ne &                    &   & 0 &    \tbnl
          & \noyau{Cl}{17}{35} &   &   &    \tbnl
          &                    & 8 &   & 16 \tbnl
\end{tabularx}


\end{exercice}


\vressort{3}


\begin{exercice}{Masse d'un atome de carbone 12}\\
Soit le carbone $12$ not� \noyau{C}{6}{12}.
\begin{enumerate}
\item L'�l�ment carbone peut-il avoir $5$ protons ? Pourquoi ?
\item Calculer la masse du noyau d'un atome de carbone $12$
sachant que la masse d'un nucl�on est $m_n = 1,67.10^{-27}~kg$
\item Calculer la masse des �lectrons de l'atome de carbone 12
sachant que la masse d'un �lectron vaut $m_e = 9,1.10^{-31}~kg$
\item Comparer la masse des �lectrons de l'atome � la masse du noyau.
Que concluez-vous ?
\item En d�duire, sans nouveau calcul, la masse de l'atome de carbone
  $12$.
\end{enumerate}
\end{exercice}

\newpage

\vressort{1}

\begin{exercice}{Couches �lectroniques}\\
Dans l'�tat le plus stable de l'atome, appel� �tat fondamental,
les �lectrons occupent successivement les couches,
en commen�ant par celles qui sont les plus proches du noyau : 
d'abord $K$ puis $L$ puis $M$.

Lorsqu'une couche est pleine, ou encore satur�e, on passe � la suivante.

La derni�re couche occup�e est appel�e couche externe.\\
Toutes les autres sont appel�es couches internes.

\medskip

\noindent
%\begin{tabularx}{\textwidth}{|>{\centering}X|>{\centering}X|>{\centering}X|>{\centering}X|}
\begin{tabularx}{\textwidth}{|>{\mystrut}X|X|X|X|}
\hline
\multicolumn{1}{|X|}{\emph{Symbole de la couche}}       & $K$ & $L$ & $M$  \tbnl
\emph{Nombre maximal d'�lectrons} & $2$ & $8$ & $18$ \tbnl
\end{tabularx}

\medskip

Ainsi, par exemple, l'atome de chlore ($Z=17$) a la configuration �lectronique :
$(K)^2(L)^8(M)^{7}$.

\begin{enumerate}
\item Indiquez le nombre d'�lectrons et donnez la configuration des atomes suivants :
  \begin{enumerate}
  \item \noyau{H}{1}{}
  \item \noyau{O}{8}{}
  \item \noyau{C}{6}{}
  \item \noyau{Ne}{10}{}
  \end{enumerate}
\item Parmi les ions ci-desssous, pr�cisez s'il s'agit d'anions ou de cations.
Indiquez le nombre d'�lectrons et donnez la configuration �lectronique
des ions suivants :
  \begin{enumerate}
  \item $Be^{2+}$ ($Z=4$)
  \item $Al^{3+}$ ($Z=13$)
  \item $O^{2-}$ ($Z=8$)
  \item $F^{-}$ ($Z=9$)
  \end{enumerate}
\end{enumerate}

\end{exercice}


\vressort{5}


\begin{exercice}{Charge d'un atome de Zinc}%\\
\begin{enumerate}
\item Combien de protons l'atome de zinc \noyau{Zn}{30}{65} contient-il ?
\item Combien d'�lectrons comporte-t-il ?
\item Calculer la charge totale des protons
sachant qu'un proton a pour charge $e = 1,6.10^{-19}~C$.
\item Calculer la charge totale des �lectrons
sachant qu'un �lectron a pour charge $-e = -1,6.10^{-19}~C$.
\item En d�duire la charge de l'atome de Zinc.
\item Ce r�sultat est-il identique pour tous les atomes ?
\item A l'issue d'une r�action dite d'oxydation, un atome de zinc $Zn$
  se transforme en un ion $Zn^{2+}$.
  \begin{enumerate}
  \item Donnez l'�quation de cette r�action
  (en faisant intervenir un ou plusieurs �lectrons not�s $e^-$).
  \item Indiquez la charge (en coulomb $C$) de cet ion.
  \end{enumerate}
\end{enumerate}
\end{exercice}

\vressort{3}  % CRAB
% % Jeu avec des d�s : loi de d�croissance radioactive

% R�actions nucl�aires spontan�es

% R�actions nucl�aires provoqu�es




% \chapitre{Devoir Surveill�} % Term STL B
% \ds{Devoir Surveill�}{
%
}

\nomprenomclasse

\setcounter{numexercice}{0}

%\renewcommand{\tabularx}[1]{>{\centering}m{#1}} 

%\newcommand{\tabularxc}[1]{\tabularx{>{\centering}m{#1}}}

\vressort{3}

\begin{exercice}{Connaissance sur l'atome}%\\
\begin{enumerate}
\item De quoi est compos� un atome ?
\item Que signifie les lettres $A$, $Z$ et $X$ dans la repr�sentation \noyau{X}{Z}{A} ?
\item Comment trouve-t-on le nombre de neutrons d'un atome de l'�l�ment pr�c�dent.
\item Si un atome a $5$ protons, combien-a-t-il d'�lectrons ? Pourquoi ?
\item Qu'est-ce qui caract�rise un �l�ment chimique ?
\item Qu'est-ce qu'un isotope ?
\end{enumerate}
\end{exercice}



\vressort{3}



\begin{exercice}{Composition des atomes}\\
En vous aidant du tableau p�riodique des �l�ments,
compl�ter le tableau suivant :

\medskip

\noindent
%\begin{tabularx}{\textwidth}{|>{\centering}X|>{\centering}X|>{\centering}X|>{\centering}X|>{\centering}X|}
% \begin{tabularx}{\linewidth}{|X|X|X|X|X|}
% \hline
% \emph{nom}       & \emph{symbole}  & \emph{protons} & \emph{neutrons}
% & \emph{nucl�ons} \tbnl
% carbone   & \noyau{C}{6}{14}   &         &          & \rule[-0.5cm]{0cm}{1cm}         \tbnl
% fluor     & \noyau{F}{9}{19}   &         &          & \rule[-0.5cm]{0cm}{1cm}         \tbnl
% sodium    & \noyau{Na}{11}{23} &         &          & \rule[-0.5cm]{0cm}{1cm}         \tbnl
% oxyg�ne   & \noyau{O}{8}{16}   &         &          & \rule[-0.5cm]{0cm}{1cm}         \tbnl
% hydrog�ne &          &         & 0        & \rule[-0.5cm]{0cm}{1cm}         \tbnl
%           & \noyau{Cl}{17}{35} &         &          &  \rule[-0.5cm]{0cm}{1cm}        \tbnl
%           &          & 8       & \rule[-0.5cm]{0cm}{1cm}         & 16       \tbnl
% \end{tabularx}



\begin{tabularx}{\linewidth}{|>{\mystrut}X|X|X|X|X|}
\hline
% multicolumn pour faire dispara�tre le \mystrut
\multicolumn{1}{|X|}{\emph{nom}} & \emph{symbole}  &
\emph{protons} & \emph{neutrons} & \emph{nucl�ons} \tbnl
carbone   & \noyau{C}{6}{14}   &   &   &    \tbnl
fluor     & \noyau{F}{9}{19}   &   &   &    \tbnl
sodium    & \noyau{Na}{11}{23} &   &   &    \tbnl
oxyg�ne   & \noyau{O}{8}{16}   &   &   &    \tbnl
hydrog�ne &                    &   & 0 &    \tbnl
          & \noyau{Cl}{17}{35} &   &   &    \tbnl
          &                    & 8 &   & 16 \tbnl
\end{tabularx}


\end{exercice}


\vressort{3}


\begin{exercice}{Masse d'un atome de carbone 12}\\
Soit le carbone $12$ not� \noyau{C}{6}{12}.
\begin{enumerate}
\item L'�l�ment carbone peut-il avoir $5$ protons ? Pourquoi ?
\item Calculer la masse du noyau d'un atome de carbone $12$
sachant que la masse d'un nucl�on est $m_n = 1,67.10^{-27}~kg$
\item Calculer la masse des �lectrons de l'atome de carbone 12
sachant que la masse d'un �lectron vaut $m_e = 9,1.10^{-31}~kg$
\item Comparer la masse des �lectrons de l'atome � la masse du noyau.
Que concluez-vous ?
\item En d�duire, sans nouveau calcul, la masse de l'atome de carbone
  $12$.
\end{enumerate}
\end{exercice}

\newpage

\vressort{1}

\begin{exercice}{Couches �lectroniques}\\
Dans l'�tat le plus stable de l'atome, appel� �tat fondamental,
les �lectrons occupent successivement les couches,
en commen�ant par celles qui sont les plus proches du noyau : 
d'abord $K$ puis $L$ puis $M$.

Lorsqu'une couche est pleine, ou encore satur�e, on passe � la suivante.

La derni�re couche occup�e est appel�e couche externe.\\
Toutes les autres sont appel�es couches internes.

\medskip

\noindent
%\begin{tabularx}{\textwidth}{|>{\centering}X|>{\centering}X|>{\centering}X|>{\centering}X|}
\begin{tabularx}{\textwidth}{|>{\mystrut}X|X|X|X|}
\hline
\multicolumn{1}{|X|}{\emph{Symbole de la couche}}       & $K$ & $L$ & $M$  \tbnl
\emph{Nombre maximal d'�lectrons} & $2$ & $8$ & $18$ \tbnl
\end{tabularx}

\medskip

Ainsi, par exemple, l'atome de chlore ($Z=17$) a la configuration �lectronique :
$(K)^2(L)^8(M)^{7}$.

\begin{enumerate}
\item Indiquez le nombre d'�lectrons et donnez la configuration des atomes suivants :
  \begin{enumerate}
  \item \noyau{H}{1}{}
  \item \noyau{O}{8}{}
  \item \noyau{C}{6}{}
  \item \noyau{Ne}{10}{}
  \end{enumerate}
\item Parmi les ions ci-desssous, pr�cisez s'il s'agit d'anions ou de cations.
Indiquez le nombre d'�lectrons et donnez la configuration �lectronique
des ions suivants :
  \begin{enumerate}
  \item $Be^{2+}$ ($Z=4$)
  \item $Al^{3+}$ ($Z=13$)
  \item $O^{2-}$ ($Z=8$)
  \item $F^{-}$ ($Z=9$)
  \end{enumerate}
\end{enumerate}

\end{exercice}


\vressort{5}


\begin{exercice}{Charge d'un atome de Zinc}%\\
\begin{enumerate}
\item Combien de protons l'atome de zinc \noyau{Zn}{30}{65} contient-il ?
\item Combien d'�lectrons comporte-t-il ?
\item Calculer la charge totale des protons
sachant qu'un proton a pour charge $e = 1,6.10^{-19}~C$.
\item Calculer la charge totale des �lectrons
sachant qu'un �lectron a pour charge $-e = -1,6.10^{-19}~C$.
\item En d�duire la charge de l'atome de Zinc.
\item Ce r�sultat est-il identique pour tous les atomes ?
\item A l'issue d'une r�action dite d'oxydation, un atome de zinc $Zn$
  se transforme en un ion $Zn^{2+}$.
  \begin{enumerate}
  \item Donnez l'�quation de cette r�action
  (en faisant intervenir un ou plusieurs �lectrons not�s $e^-$).
  \item Indiquez la charge (en coulomb $C$) de cet ion.
  \end{enumerate}
\end{enumerate}
\end{exercice}

\vressort{3}




\classe[Premi�re STI G�nie \'Electrotechnique]{Premi�re\ \\
Sciences et Technologies Industrielles\ \\
G�nie \'Electrotechnique}





% % tp caract�ristique d'un r�cepteur (moteur � courant continu)

% % potentiel le long d'un circuit �lectrique 
% % tp �tude du montage potentiom�trique (diviseur de tension)

% % tp �tude de la diode PN
% % tp �tude de la diode Zener
% % tp �tude d'un r�gulateur int�gr� de tension 7805



% % tp th�or�me de th�venin / norton
% % tp th�or�me de superposition

\chapitre{Amplificateur Op�rationnel}
\inclure{elec/tp_ao_lin} % tp ao en regime lin�aire
\inclure{elec/tp_ao_lin_eval} % �valutation de tp
 % tp ao en r�gime non lineaire


% tp condensateur (charge � courant constant)


\chapitre{R�gimes variables}
\inclure{elec/tp_elec_oscillo_gbf} % tp regime variable, oscillo, gbf
\inclure{elec/cours_oscillo_gbf}
\inclure{elec/cours_grandeurs_periodiques} % cours grandeurs p�riodiques
\inclure{elec/cours_regimes_transitoires} % cours r�gimes transitoires

% TP RC RL
% TP RLC


% tp transfo, pont de diode, filtrage C, RIT


% tp force �lectromagn�tique (balance �lectromagn�tique ou haut-parleur)

% tp Charge � courant constant d'un condensateur

% tp transistor bipolaire (caract�ristiques)
% tp transistor bipolaire (utilisations)
% tp transistor bipolaire (commutation)

% tp champ magn�tique
% tp r�gime transitoire RC et RL
% tp r�gime transitoire RLC

% tp utilisation du gbf et de l'oscillo
% tp mesure de valeur moyenne et de valeur efficace d'un signal p�riodique




\chapitre{R�gimes sinuso�daux}
\inclure{elec/cours_elec_sin} % cours r�gimes sinuso�daux
\inclure{elec/cours_dipoles_lin_sin} % cours : dip�les lin�aires �l�ementaires en r�gime sinuso�dal

\inclure{elec/cours_asso_serie_dip_sin_reson} % cours : associations de dip�les, r�sonance

% tp : mesure de l'imp�dance et du d�phasage des dipoles �l�mentaires
% association s�rie de dipoles en r�gime sinusoidal
% tp : mesure de d�phasage pour des circuit RC et RL s�rie
% tp : �tude d'un circuit RLC en r�gime sinuso�dal

\inclure{elec/cours_puissance_sin} % cours : puissance en r�gime sinuso�dal monophas�

\inclure{elec/tp_puissance_sin_mono} % tp mesure de puissance en monophas�


% cours : syst�me triphas�s �quilibr�s



% \inclure{elec/tp_redressement_mono_non_comm} % tp redressement mono non command�
% % tp redressement mono command�
% % tp �tude d'une alim continue stabilis�e

% % tp fonctions de l'�lectronique : optocoupleur


% � ajouter si le dernier document contient un nb impair de pages
%\newpage

\classe{Math�matiques Sup�rieures\ \\
PCSI}{Math�matiques Sup�rieures PCSI}

\chapitre{Optique g�om�trique}



% tp_cours_instruments_optique

\inclure{opt/tp_lunette_collimateur} % tp_lunette_collimateur


\inclure{opt/cours_prisme} % cours_prisme


\inclure{opt/tp_prisme_angle} % tp_prisme_angle

\inclure{opt/tp_prisme_indice} % tp_prisme_indice

\inclure{opt/tp_prisme_dispersion} % tp_goniometre





% � ajouter si le dernier document contient un nb impair de pages
%\newpage




% \chapitre{Optique g�om�trique}
% \ds{Devoir Surveill�}{
%
}

\nomprenomclasse

\setcounter{numexercice}{0}

%\renewcommand{\tabularx}[1]{>{\centering}m{#1}} 

%\newcommand{\tabularxc}[1]{\tabularx{>{\centering}m{#1}}}

\vressort{3}

\begin{exercice}{Connaissance sur l'atome}%\\
\begin{enumerate}
\item De quoi est compos� un atome ?
\item Que signifie les lettres $A$, $Z$ et $X$ dans la repr�sentation \noyau{X}{Z}{A} ?
\item Comment trouve-t-on le nombre de neutrons d'un atome de l'�l�ment pr�c�dent.
\item Si un atome a $5$ protons, combien-a-t-il d'�lectrons ? Pourquoi ?
\item Qu'est-ce qui caract�rise un �l�ment chimique ?
\item Qu'est-ce qu'un isotope ?
\end{enumerate}
\end{exercice}



\vressort{3}



\begin{exercice}{Composition des atomes}\\
En vous aidant du tableau p�riodique des �l�ments,
compl�ter le tableau suivant :

\medskip

\noindent
%\begin{tabularx}{\textwidth}{|>{\centering}X|>{\centering}X|>{\centering}X|>{\centering}X|>{\centering}X|}
% \begin{tabularx}{\linewidth}{|X|X|X|X|X|}
% \hline
% \emph{nom}       & \emph{symbole}  & \emph{protons} & \emph{neutrons}
% & \emph{nucl�ons} \tbnl
% carbone   & \noyau{C}{6}{14}   &         &          & \rule[-0.5cm]{0cm}{1cm}         \tbnl
% fluor     & \noyau{F}{9}{19}   &         &          & \rule[-0.5cm]{0cm}{1cm}         \tbnl
% sodium    & \noyau{Na}{11}{23} &         &          & \rule[-0.5cm]{0cm}{1cm}         \tbnl
% oxyg�ne   & \noyau{O}{8}{16}   &         &          & \rule[-0.5cm]{0cm}{1cm}         \tbnl
% hydrog�ne &          &         & 0        & \rule[-0.5cm]{0cm}{1cm}         \tbnl
%           & \noyau{Cl}{17}{35} &         &          &  \rule[-0.5cm]{0cm}{1cm}        \tbnl
%           &          & 8       & \rule[-0.5cm]{0cm}{1cm}         & 16       \tbnl
% \end{tabularx}



\begin{tabularx}{\linewidth}{|>{\mystrut}X|X|X|X|X|}
\hline
% multicolumn pour faire dispara�tre le \mystrut
\multicolumn{1}{|X|}{\emph{nom}} & \emph{symbole}  &
\emph{protons} & \emph{neutrons} & \emph{nucl�ons} \tbnl
carbone   & \noyau{C}{6}{14}   &   &   &    \tbnl
fluor     & \noyau{F}{9}{19}   &   &   &    \tbnl
sodium    & \noyau{Na}{11}{23} &   &   &    \tbnl
oxyg�ne   & \noyau{O}{8}{16}   &   &   &    \tbnl
hydrog�ne &                    &   & 0 &    \tbnl
          & \noyau{Cl}{17}{35} &   &   &    \tbnl
          &                    & 8 &   & 16 \tbnl
\end{tabularx}


\end{exercice}


\vressort{3}


\begin{exercice}{Masse d'un atome de carbone 12}\\
Soit le carbone $12$ not� \noyau{C}{6}{12}.
\begin{enumerate}
\item L'�l�ment carbone peut-il avoir $5$ protons ? Pourquoi ?
\item Calculer la masse du noyau d'un atome de carbone $12$
sachant que la masse d'un nucl�on est $m_n = 1,67.10^{-27}~kg$
\item Calculer la masse des �lectrons de l'atome de carbone 12
sachant que la masse d'un �lectron vaut $m_e = 9,1.10^{-31}~kg$
\item Comparer la masse des �lectrons de l'atome � la masse du noyau.
Que concluez-vous ?
\item En d�duire, sans nouveau calcul, la masse de l'atome de carbone
  $12$.
\end{enumerate}
\end{exercice}

\newpage

\vressort{1}

\begin{exercice}{Couches �lectroniques}\\
Dans l'�tat le plus stable de l'atome, appel� �tat fondamental,
les �lectrons occupent successivement les couches,
en commen�ant par celles qui sont les plus proches du noyau : 
d'abord $K$ puis $L$ puis $M$.

Lorsqu'une couche est pleine, ou encore satur�e, on passe � la suivante.

La derni�re couche occup�e est appel�e couche externe.\\
Toutes les autres sont appel�es couches internes.

\medskip

\noindent
%\begin{tabularx}{\textwidth}{|>{\centering}X|>{\centering}X|>{\centering}X|>{\centering}X|}
\begin{tabularx}{\textwidth}{|>{\mystrut}X|X|X|X|}
\hline
\multicolumn{1}{|X|}{\emph{Symbole de la couche}}       & $K$ & $L$ & $M$  \tbnl
\emph{Nombre maximal d'�lectrons} & $2$ & $8$ & $18$ \tbnl
\end{tabularx}

\medskip

Ainsi, par exemple, l'atome de chlore ($Z=17$) a la configuration �lectronique :
$(K)^2(L)^8(M)^{7}$.

\begin{enumerate}
\item Indiquez le nombre d'�lectrons et donnez la configuration des atomes suivants :
  \begin{enumerate}
  \item \noyau{H}{1}{}
  \item \noyau{O}{8}{}
  \item \noyau{C}{6}{}
  \item \noyau{Ne}{10}{}
  \end{enumerate}
\item Parmi les ions ci-desssous, pr�cisez s'il s'agit d'anions ou de cations.
Indiquez le nombre d'�lectrons et donnez la configuration �lectronique
des ions suivants :
  \begin{enumerate}
  \item $Be^{2+}$ ($Z=4$)
  \item $Al^{3+}$ ($Z=13$)
  \item $O^{2-}$ ($Z=8$)
  \item $F^{-}$ ($Z=9$)
  \end{enumerate}
\end{enumerate}

\end{exercice}


\vressort{5}


\begin{exercice}{Charge d'un atome de Zinc}%\\
\begin{enumerate}
\item Combien de protons l'atome de zinc \noyau{Zn}{30}{65} contient-il ?
\item Combien d'�lectrons comporte-t-il ?
\item Calculer la charge totale des protons
sachant qu'un proton a pour charge $e = 1,6.10^{-19}~C$.
\item Calculer la charge totale des �lectrons
sachant qu'un �lectron a pour charge $-e = -1,6.10^{-19}~C$.
\item En d�duire la charge de l'atome de Zinc.
\item Ce r�sultat est-il identique pour tous les atomes ?
\item A l'issue d'une r�action dite d'oxydation, un atome de zinc $Zn$
  se transforme en un ion $Zn^{2+}$.
  \begin{enumerate}
  \item Donnez l'�quation de cette r�action
  (en faisant intervenir un ou plusieurs �lectrons not�s $e^-$).
  \item Indiquez la charge (en coulomb $C$) de cet ion.
  \end{enumerate}
\end{enumerate}
\end{exercice}

\vressort{3}
%  R�fractom�tre (cf TP Thierry + CDI Chimie)
% Spectroscope (Prisme + goniom�tre)
% Spectrophotom�tre
% Polarim�tre
% Spectre R�sonance Magn�tique Nucl�aire (liaison) http://www.scienceofspectroscopy.info
% TP sup optique ?
% Prisme Mesure d'indice (gonio)
% Prisme Mesure d'angle

%\chapitre{Optique ondulatoire}
% Diffraction (mesure du diam�tre d'un cheveu)
% Interf�rences (Michelson ?)



% TP de MPI (portes logiques, etc..)
% cf A LEQUITTE ENCPB http://www.educnet.education.fr/rnchimie/sommaire.htm







\appendix % annexes

\chapitre{Documents divers}
\ds{Devoir Surveill�}{
%
}

\nomprenomclasse

\setcounter{numexercice}{0}

%\renewcommand{\tabularx}[1]{>{\centering}m{#1}} 

%\newcommand{\tabularxc}[1]{\tabularx{>{\centering}m{#1}}}

\vressort{3}

\begin{exercice}{Connaissance sur l'atome}%\\
\begin{enumerate}
\item De quoi est compos� un atome ?
\item Que signifie les lettres $A$, $Z$ et $X$ dans la repr�sentation \noyau{X}{Z}{A} ?
\item Comment trouve-t-on le nombre de neutrons d'un atome de l'�l�ment pr�c�dent.
\item Si un atome a $5$ protons, combien-a-t-il d'�lectrons ? Pourquoi ?
\item Qu'est-ce qui caract�rise un �l�ment chimique ?
\item Qu'est-ce qu'un isotope ?
\end{enumerate}
\end{exercice}



\vressort{3}



\begin{exercice}{Composition des atomes}\\
En vous aidant du tableau p�riodique des �l�ments,
compl�ter le tableau suivant :

\medskip

\noindent
%\begin{tabularx}{\textwidth}{|>{\centering}X|>{\centering}X|>{\centering}X|>{\centering}X|>{\centering}X|}
% \begin{tabularx}{\linewidth}{|X|X|X|X|X|}
% \hline
% \emph{nom}       & \emph{symbole}  & \emph{protons} & \emph{neutrons}
% & \emph{nucl�ons} \tbnl
% carbone   & \noyau{C}{6}{14}   &         &          & \rule[-0.5cm]{0cm}{1cm}         \tbnl
% fluor     & \noyau{F}{9}{19}   &         &          & \rule[-0.5cm]{0cm}{1cm}         \tbnl
% sodium    & \noyau{Na}{11}{23} &         &          & \rule[-0.5cm]{0cm}{1cm}         \tbnl
% oxyg�ne   & \noyau{O}{8}{16}   &         &          & \rule[-0.5cm]{0cm}{1cm}         \tbnl
% hydrog�ne &          &         & 0        & \rule[-0.5cm]{0cm}{1cm}         \tbnl
%           & \noyau{Cl}{17}{35} &         &          &  \rule[-0.5cm]{0cm}{1cm}        \tbnl
%           &          & 8       & \rule[-0.5cm]{0cm}{1cm}         & 16       \tbnl
% \end{tabularx}



\begin{tabularx}{\linewidth}{|>{\mystrut}X|X|X|X|X|}
\hline
% multicolumn pour faire dispara�tre le \mystrut
\multicolumn{1}{|X|}{\emph{nom}} & \emph{symbole}  &
\emph{protons} & \emph{neutrons} & \emph{nucl�ons} \tbnl
carbone   & \noyau{C}{6}{14}   &   &   &    \tbnl
fluor     & \noyau{F}{9}{19}   &   &   &    \tbnl
sodium    & \noyau{Na}{11}{23} &   &   &    \tbnl
oxyg�ne   & \noyau{O}{8}{16}   &   &   &    \tbnl
hydrog�ne &                    &   & 0 &    \tbnl
          & \noyau{Cl}{17}{35} &   &   &    \tbnl
          &                    & 8 &   & 16 \tbnl
\end{tabularx}


\end{exercice}


\vressort{3}


\begin{exercice}{Masse d'un atome de carbone 12}\\
Soit le carbone $12$ not� \noyau{C}{6}{12}.
\begin{enumerate}
\item L'�l�ment carbone peut-il avoir $5$ protons ? Pourquoi ?
\item Calculer la masse du noyau d'un atome de carbone $12$
sachant que la masse d'un nucl�on est $m_n = 1,67.10^{-27}~kg$
\item Calculer la masse des �lectrons de l'atome de carbone 12
sachant que la masse d'un �lectron vaut $m_e = 9,1.10^{-31}~kg$
\item Comparer la masse des �lectrons de l'atome � la masse du noyau.
Que concluez-vous ?
\item En d�duire, sans nouveau calcul, la masse de l'atome de carbone
  $12$.
\end{enumerate}
\end{exercice}

\newpage

\vressort{1}

\begin{exercice}{Couches �lectroniques}\\
Dans l'�tat le plus stable de l'atome, appel� �tat fondamental,
les �lectrons occupent successivement les couches,
en commen�ant par celles qui sont les plus proches du noyau : 
d'abord $K$ puis $L$ puis $M$.

Lorsqu'une couche est pleine, ou encore satur�e, on passe � la suivante.

La derni�re couche occup�e est appel�e couche externe.\\
Toutes les autres sont appel�es couches internes.

\medskip

\noindent
%\begin{tabularx}{\textwidth}{|>{\centering}X|>{\centering}X|>{\centering}X|>{\centering}X|}
\begin{tabularx}{\textwidth}{|>{\mystrut}X|X|X|X|}
\hline
\multicolumn{1}{|X|}{\emph{Symbole de la couche}}       & $K$ & $L$ & $M$  \tbnl
\emph{Nombre maximal d'�lectrons} & $2$ & $8$ & $18$ \tbnl
\end{tabularx}

\medskip

Ainsi, par exemple, l'atome de chlore ($Z=17$) a la configuration �lectronique :
$(K)^2(L)^8(M)^{7}$.

\begin{enumerate}
\item Indiquez le nombre d'�lectrons et donnez la configuration des atomes suivants :
  \begin{enumerate}
  \item \noyau{H}{1}{}
  \item \noyau{O}{8}{}
  \item \noyau{C}{6}{}
  \item \noyau{Ne}{10}{}
  \end{enumerate}
\item Parmi les ions ci-desssous, pr�cisez s'il s'agit d'anions ou de cations.
Indiquez le nombre d'�lectrons et donnez la configuration �lectronique
des ions suivants :
  \begin{enumerate}
  \item $Be^{2+}$ ($Z=4$)
  \item $Al^{3+}$ ($Z=13$)
  \item $O^{2-}$ ($Z=8$)
  \item $F^{-}$ ($Z=9$)
  \end{enumerate}
\end{enumerate}

\end{exercice}


\vressort{5}


\begin{exercice}{Charge d'un atome de Zinc}%\\
\begin{enumerate}
\item Combien de protons l'atome de zinc \noyau{Zn}{30}{65} contient-il ?
\item Combien d'�lectrons comporte-t-il ?
\item Calculer la charge totale des protons
sachant qu'un proton a pour charge $e = 1,6.10^{-19}~C$.
\item Calculer la charge totale des �lectrons
sachant qu'un �lectron a pour charge $-e = -1,6.10^{-19}~C$.
\item En d�duire la charge de l'atome de Zinc.
\item Ce r�sultat est-il identique pour tous les atomes ?
\item A l'issue d'une r�action dite d'oxydation, un atome de zinc $Zn$
  se transforme en un ion $Zn^{2+}$.
  \begin{enumerate}
  \item Donnez l'�quation de cette r�action
  (en faisant intervenir un ou plusieurs �lectrons not�s $e^-$).
  \item Indiquez la charge (en coulomb $C$) de cet ion.
  \end{enumerate}
\end{enumerate}
\end{exercice}

\vressort{3}
\ds{Devoir Surveill�}{
%
}

\nomprenomclasse

\setcounter{numexercice}{0}

%\renewcommand{\tabularx}[1]{>{\centering}m{#1}} 

%\newcommand{\tabularxc}[1]{\tabularx{>{\centering}m{#1}}}

\vressort{3}

\begin{exercice}{Connaissance sur l'atome}%\\
\begin{enumerate}
\item De quoi est compos� un atome ?
\item Que signifie les lettres $A$, $Z$ et $X$ dans la repr�sentation \noyau{X}{Z}{A} ?
\item Comment trouve-t-on le nombre de neutrons d'un atome de l'�l�ment pr�c�dent.
\item Si un atome a $5$ protons, combien-a-t-il d'�lectrons ? Pourquoi ?
\item Qu'est-ce qui caract�rise un �l�ment chimique ?
\item Qu'est-ce qu'un isotope ?
\end{enumerate}
\end{exercice}



\vressort{3}



\begin{exercice}{Composition des atomes}\\
En vous aidant du tableau p�riodique des �l�ments,
compl�ter le tableau suivant :

\medskip

\noindent
%\begin{tabularx}{\textwidth}{|>{\centering}X|>{\centering}X|>{\centering}X|>{\centering}X|>{\centering}X|}
% \begin{tabularx}{\linewidth}{|X|X|X|X|X|}
% \hline
% \emph{nom}       & \emph{symbole}  & \emph{protons} & \emph{neutrons}
% & \emph{nucl�ons} \tbnl
% carbone   & \noyau{C}{6}{14}   &         &          & \rule[-0.5cm]{0cm}{1cm}         \tbnl
% fluor     & \noyau{F}{9}{19}   &         &          & \rule[-0.5cm]{0cm}{1cm}         \tbnl
% sodium    & \noyau{Na}{11}{23} &         &          & \rule[-0.5cm]{0cm}{1cm}         \tbnl
% oxyg�ne   & \noyau{O}{8}{16}   &         &          & \rule[-0.5cm]{0cm}{1cm}         \tbnl
% hydrog�ne &          &         & 0        & \rule[-0.5cm]{0cm}{1cm}         \tbnl
%           & \noyau{Cl}{17}{35} &         &          &  \rule[-0.5cm]{0cm}{1cm}        \tbnl
%           &          & 8       & \rule[-0.5cm]{0cm}{1cm}         & 16       \tbnl
% \end{tabularx}



\begin{tabularx}{\linewidth}{|>{\mystrut}X|X|X|X|X|}
\hline
% multicolumn pour faire dispara�tre le \mystrut
\multicolumn{1}{|X|}{\emph{nom}} & \emph{symbole}  &
\emph{protons} & \emph{neutrons} & \emph{nucl�ons} \tbnl
carbone   & \noyau{C}{6}{14}   &   &   &    \tbnl
fluor     & \noyau{F}{9}{19}   &   &   &    \tbnl
sodium    & \noyau{Na}{11}{23} &   &   &    \tbnl
oxyg�ne   & \noyau{O}{8}{16}   &   &   &    \tbnl
hydrog�ne &                    &   & 0 &    \tbnl
          & \noyau{Cl}{17}{35} &   &   &    \tbnl
          &                    & 8 &   & 16 \tbnl
\end{tabularx}


\end{exercice}


\vressort{3}


\begin{exercice}{Masse d'un atome de carbone 12}\\
Soit le carbone $12$ not� \noyau{C}{6}{12}.
\begin{enumerate}
\item L'�l�ment carbone peut-il avoir $5$ protons ? Pourquoi ?
\item Calculer la masse du noyau d'un atome de carbone $12$
sachant que la masse d'un nucl�on est $m_n = 1,67.10^{-27}~kg$
\item Calculer la masse des �lectrons de l'atome de carbone 12
sachant que la masse d'un �lectron vaut $m_e = 9,1.10^{-31}~kg$
\item Comparer la masse des �lectrons de l'atome � la masse du noyau.
Que concluez-vous ?
\item En d�duire, sans nouveau calcul, la masse de l'atome de carbone
  $12$.
\end{enumerate}
\end{exercice}

\newpage

\vressort{1}

\begin{exercice}{Couches �lectroniques}\\
Dans l'�tat le plus stable de l'atome, appel� �tat fondamental,
les �lectrons occupent successivement les couches,
en commen�ant par celles qui sont les plus proches du noyau : 
d'abord $K$ puis $L$ puis $M$.

Lorsqu'une couche est pleine, ou encore satur�e, on passe � la suivante.

La derni�re couche occup�e est appel�e couche externe.\\
Toutes les autres sont appel�es couches internes.

\medskip

\noindent
%\begin{tabularx}{\textwidth}{|>{\centering}X|>{\centering}X|>{\centering}X|>{\centering}X|}
\begin{tabularx}{\textwidth}{|>{\mystrut}X|X|X|X|}
\hline
\multicolumn{1}{|X|}{\emph{Symbole de la couche}}       & $K$ & $L$ & $M$  \tbnl
\emph{Nombre maximal d'�lectrons} & $2$ & $8$ & $18$ \tbnl
\end{tabularx}

\medskip

Ainsi, par exemple, l'atome de chlore ($Z=17$) a la configuration �lectronique :
$(K)^2(L)^8(M)^{7}$.

\begin{enumerate}
\item Indiquez le nombre d'�lectrons et donnez la configuration des atomes suivants :
  \begin{enumerate}
  \item \noyau{H}{1}{}
  \item \noyau{O}{8}{}
  \item \noyau{C}{6}{}
  \item \noyau{Ne}{10}{}
  \end{enumerate}
\item Parmi les ions ci-desssous, pr�cisez s'il s'agit d'anions ou de cations.
Indiquez le nombre d'�lectrons et donnez la configuration �lectronique
des ions suivants :
  \begin{enumerate}
  \item $Be^{2+}$ ($Z=4$)
  \item $Al^{3+}$ ($Z=13$)
  \item $O^{2-}$ ($Z=8$)
  \item $F^{-}$ ($Z=9$)
  \end{enumerate}
\end{enumerate}

\end{exercice}


\vressort{5}


\begin{exercice}{Charge d'un atome de Zinc}%\\
\begin{enumerate}
\item Combien de protons l'atome de zinc \noyau{Zn}{30}{65} contient-il ?
\item Combien d'�lectrons comporte-t-il ?
\item Calculer la charge totale des protons
sachant qu'un proton a pour charge $e = 1,6.10^{-19}~C$.
\item Calculer la charge totale des �lectrons
sachant qu'un �lectron a pour charge $-e = -1,6.10^{-19}~C$.
\item En d�duire la charge de l'atome de Zinc.
\item Ce r�sultat est-il identique pour tous les atomes ?
\item A l'issue d'une r�action dite d'oxydation, un atome de zinc $Zn$
  se transforme en un ion $Zn^{2+}$.
  \begin{enumerate}
  \item Donnez l'�quation de cette r�action
  (en faisant intervenir un ou plusieurs �lectrons not�s $e^-$).
  \item Indiquez la charge (en coulomb $C$) de cet ion.
  \end{enumerate}
\end{enumerate}
\end{exercice}

\vressort{3}
\ds{Devoir Surveill�}{
%
}

\nomprenomclasse

\setcounter{numexercice}{0}

%\renewcommand{\tabularx}[1]{>{\centering}m{#1}} 

%\newcommand{\tabularxc}[1]{\tabularx{>{\centering}m{#1}}}

\vressort{3}

\begin{exercice}{Connaissance sur l'atome}%\\
\begin{enumerate}
\item De quoi est compos� un atome ?
\item Que signifie les lettres $A$, $Z$ et $X$ dans la repr�sentation \noyau{X}{Z}{A} ?
\item Comment trouve-t-on le nombre de neutrons d'un atome de l'�l�ment pr�c�dent.
\item Si un atome a $5$ protons, combien-a-t-il d'�lectrons ? Pourquoi ?
\item Qu'est-ce qui caract�rise un �l�ment chimique ?
\item Qu'est-ce qu'un isotope ?
\end{enumerate}
\end{exercice}



\vressort{3}



\begin{exercice}{Composition des atomes}\\
En vous aidant du tableau p�riodique des �l�ments,
compl�ter le tableau suivant :

\medskip

\noindent
%\begin{tabularx}{\textwidth}{|>{\centering}X|>{\centering}X|>{\centering}X|>{\centering}X|>{\centering}X|}
% \begin{tabularx}{\linewidth}{|X|X|X|X|X|}
% \hline
% \emph{nom}       & \emph{symbole}  & \emph{protons} & \emph{neutrons}
% & \emph{nucl�ons} \tbnl
% carbone   & \noyau{C}{6}{14}   &         &          & \rule[-0.5cm]{0cm}{1cm}         \tbnl
% fluor     & \noyau{F}{9}{19}   &         &          & \rule[-0.5cm]{0cm}{1cm}         \tbnl
% sodium    & \noyau{Na}{11}{23} &         &          & \rule[-0.5cm]{0cm}{1cm}         \tbnl
% oxyg�ne   & \noyau{O}{8}{16}   &         &          & \rule[-0.5cm]{0cm}{1cm}         \tbnl
% hydrog�ne &          &         & 0        & \rule[-0.5cm]{0cm}{1cm}         \tbnl
%           & \noyau{Cl}{17}{35} &         &          &  \rule[-0.5cm]{0cm}{1cm}        \tbnl
%           &          & 8       & \rule[-0.5cm]{0cm}{1cm}         & 16       \tbnl
% \end{tabularx}



\begin{tabularx}{\linewidth}{|>{\mystrut}X|X|X|X|X|}
\hline
% multicolumn pour faire dispara�tre le \mystrut
\multicolumn{1}{|X|}{\emph{nom}} & \emph{symbole}  &
\emph{protons} & \emph{neutrons} & \emph{nucl�ons} \tbnl
carbone   & \noyau{C}{6}{14}   &   &   &    \tbnl
fluor     & \noyau{F}{9}{19}   &   &   &    \tbnl
sodium    & \noyau{Na}{11}{23} &   &   &    \tbnl
oxyg�ne   & \noyau{O}{8}{16}   &   &   &    \tbnl
hydrog�ne &                    &   & 0 &    \tbnl
          & \noyau{Cl}{17}{35} &   &   &    \tbnl
          &                    & 8 &   & 16 \tbnl
\end{tabularx}


\end{exercice}


\vressort{3}


\begin{exercice}{Masse d'un atome de carbone 12}\\
Soit le carbone $12$ not� \noyau{C}{6}{12}.
\begin{enumerate}
\item L'�l�ment carbone peut-il avoir $5$ protons ? Pourquoi ?
\item Calculer la masse du noyau d'un atome de carbone $12$
sachant que la masse d'un nucl�on est $m_n = 1,67.10^{-27}~kg$
\item Calculer la masse des �lectrons de l'atome de carbone 12
sachant que la masse d'un �lectron vaut $m_e = 9,1.10^{-31}~kg$
\item Comparer la masse des �lectrons de l'atome � la masse du noyau.
Que concluez-vous ?
\item En d�duire, sans nouveau calcul, la masse de l'atome de carbone
  $12$.
\end{enumerate}
\end{exercice}

\newpage

\vressort{1}

\begin{exercice}{Couches �lectroniques}\\
Dans l'�tat le plus stable de l'atome, appel� �tat fondamental,
les �lectrons occupent successivement les couches,
en commen�ant par celles qui sont les plus proches du noyau : 
d'abord $K$ puis $L$ puis $M$.

Lorsqu'une couche est pleine, ou encore satur�e, on passe � la suivante.

La derni�re couche occup�e est appel�e couche externe.\\
Toutes les autres sont appel�es couches internes.

\medskip

\noindent
%\begin{tabularx}{\textwidth}{|>{\centering}X|>{\centering}X|>{\centering}X|>{\centering}X|}
\begin{tabularx}{\textwidth}{|>{\mystrut}X|X|X|X|}
\hline
\multicolumn{1}{|X|}{\emph{Symbole de la couche}}       & $K$ & $L$ & $M$  \tbnl
\emph{Nombre maximal d'�lectrons} & $2$ & $8$ & $18$ \tbnl
\end{tabularx}

\medskip

Ainsi, par exemple, l'atome de chlore ($Z=17$) a la configuration �lectronique :
$(K)^2(L)^8(M)^{7}$.

\begin{enumerate}
\item Indiquez le nombre d'�lectrons et donnez la configuration des atomes suivants :
  \begin{enumerate}
  \item \noyau{H}{1}{}
  \item \noyau{O}{8}{}
  \item \noyau{C}{6}{}
  \item \noyau{Ne}{10}{}
  \end{enumerate}
\item Parmi les ions ci-desssous, pr�cisez s'il s'agit d'anions ou de cations.
Indiquez le nombre d'�lectrons et donnez la configuration �lectronique
des ions suivants :
  \begin{enumerate}
  \item $Be^{2+}$ ($Z=4$)
  \item $Al^{3+}$ ($Z=13$)
  \item $O^{2-}$ ($Z=8$)
  \item $F^{-}$ ($Z=9$)
  \end{enumerate}
\end{enumerate}

\end{exercice}


\vressort{5}


\begin{exercice}{Charge d'un atome de Zinc}%\\
\begin{enumerate}
\item Combien de protons l'atome de zinc \noyau{Zn}{30}{65} contient-il ?
\item Combien d'�lectrons comporte-t-il ?
\item Calculer la charge totale des protons
sachant qu'un proton a pour charge $e = 1,6.10^{-19}~C$.
\item Calculer la charge totale des �lectrons
sachant qu'un �lectron a pour charge $-e = -1,6.10^{-19}~C$.
\item En d�duire la charge de l'atome de Zinc.
\item Ce r�sultat est-il identique pour tous les atomes ?
\item A l'issue d'une r�action dite d'oxydation, un atome de zinc $Zn$
  se transforme en un ion $Zn^{2+}$.
  \begin{enumerate}
  \item Donnez l'�quation de cette r�action
  (en faisant intervenir un ou plusieurs �lectrons not�s $e^-$).
  \item Indiquez la charge (en coulomb $C$) de cet ion.
  \end{enumerate}
\end{enumerate}
\end{exercice}

\vressort{3}
\ds{Devoir Surveill�}{
%
}

\nomprenomclasse

\setcounter{numexercice}{0}

%\renewcommand{\tabularx}[1]{>{\centering}m{#1}} 

%\newcommand{\tabularxc}[1]{\tabularx{>{\centering}m{#1}}}

\vressort{3}

\begin{exercice}{Connaissance sur l'atome}%\\
\begin{enumerate}
\item De quoi est compos� un atome ?
\item Que signifie les lettres $A$, $Z$ et $X$ dans la repr�sentation \noyau{X}{Z}{A} ?
\item Comment trouve-t-on le nombre de neutrons d'un atome de l'�l�ment pr�c�dent.
\item Si un atome a $5$ protons, combien-a-t-il d'�lectrons ? Pourquoi ?
\item Qu'est-ce qui caract�rise un �l�ment chimique ?
\item Qu'est-ce qu'un isotope ?
\end{enumerate}
\end{exercice}



\vressort{3}



\begin{exercice}{Composition des atomes}\\
En vous aidant du tableau p�riodique des �l�ments,
compl�ter le tableau suivant :

\medskip

\noindent
%\begin{tabularx}{\textwidth}{|>{\centering}X|>{\centering}X|>{\centering}X|>{\centering}X|>{\centering}X|}
% \begin{tabularx}{\linewidth}{|X|X|X|X|X|}
% \hline
% \emph{nom}       & \emph{symbole}  & \emph{protons} & \emph{neutrons}
% & \emph{nucl�ons} \tbnl
% carbone   & \noyau{C}{6}{14}   &         &          & \rule[-0.5cm]{0cm}{1cm}         \tbnl
% fluor     & \noyau{F}{9}{19}   &         &          & \rule[-0.5cm]{0cm}{1cm}         \tbnl
% sodium    & \noyau{Na}{11}{23} &         &          & \rule[-0.5cm]{0cm}{1cm}         \tbnl
% oxyg�ne   & \noyau{O}{8}{16}   &         &          & \rule[-0.5cm]{0cm}{1cm}         \tbnl
% hydrog�ne &          &         & 0        & \rule[-0.5cm]{0cm}{1cm}         \tbnl
%           & \noyau{Cl}{17}{35} &         &          &  \rule[-0.5cm]{0cm}{1cm}        \tbnl
%           &          & 8       & \rule[-0.5cm]{0cm}{1cm}         & 16       \tbnl
% \end{tabularx}



\begin{tabularx}{\linewidth}{|>{\mystrut}X|X|X|X|X|}
\hline
% multicolumn pour faire dispara�tre le \mystrut
\multicolumn{1}{|X|}{\emph{nom}} & \emph{symbole}  &
\emph{protons} & \emph{neutrons} & \emph{nucl�ons} \tbnl
carbone   & \noyau{C}{6}{14}   &   &   &    \tbnl
fluor     & \noyau{F}{9}{19}   &   &   &    \tbnl
sodium    & \noyau{Na}{11}{23} &   &   &    \tbnl
oxyg�ne   & \noyau{O}{8}{16}   &   &   &    \tbnl
hydrog�ne &                    &   & 0 &    \tbnl
          & \noyau{Cl}{17}{35} &   &   &    \tbnl
          &                    & 8 &   & 16 \tbnl
\end{tabularx}


\end{exercice}


\vressort{3}


\begin{exercice}{Masse d'un atome de carbone 12}\\
Soit le carbone $12$ not� \noyau{C}{6}{12}.
\begin{enumerate}
\item L'�l�ment carbone peut-il avoir $5$ protons ? Pourquoi ?
\item Calculer la masse du noyau d'un atome de carbone $12$
sachant que la masse d'un nucl�on est $m_n = 1,67.10^{-27}~kg$
\item Calculer la masse des �lectrons de l'atome de carbone 12
sachant que la masse d'un �lectron vaut $m_e = 9,1.10^{-31}~kg$
\item Comparer la masse des �lectrons de l'atome � la masse du noyau.
Que concluez-vous ?
\item En d�duire, sans nouveau calcul, la masse de l'atome de carbone
  $12$.
\end{enumerate}
\end{exercice}

\newpage

\vressort{1}

\begin{exercice}{Couches �lectroniques}\\
Dans l'�tat le plus stable de l'atome, appel� �tat fondamental,
les �lectrons occupent successivement les couches,
en commen�ant par celles qui sont les plus proches du noyau : 
d'abord $K$ puis $L$ puis $M$.

Lorsqu'une couche est pleine, ou encore satur�e, on passe � la suivante.

La derni�re couche occup�e est appel�e couche externe.\\
Toutes les autres sont appel�es couches internes.

\medskip

\noindent
%\begin{tabularx}{\textwidth}{|>{\centering}X|>{\centering}X|>{\centering}X|>{\centering}X|}
\begin{tabularx}{\textwidth}{|>{\mystrut}X|X|X|X|}
\hline
\multicolumn{1}{|X|}{\emph{Symbole de la couche}}       & $K$ & $L$ & $M$  \tbnl
\emph{Nombre maximal d'�lectrons} & $2$ & $8$ & $18$ \tbnl
\end{tabularx}

\medskip

Ainsi, par exemple, l'atome de chlore ($Z=17$) a la configuration �lectronique :
$(K)^2(L)^8(M)^{7}$.

\begin{enumerate}
\item Indiquez le nombre d'�lectrons et donnez la configuration des atomes suivants :
  \begin{enumerate}
  \item \noyau{H}{1}{}
  \item \noyau{O}{8}{}
  \item \noyau{C}{6}{}
  \item \noyau{Ne}{10}{}
  \end{enumerate}
\item Parmi les ions ci-desssous, pr�cisez s'il s'agit d'anions ou de cations.
Indiquez le nombre d'�lectrons et donnez la configuration �lectronique
des ions suivants :
  \begin{enumerate}
  \item $Be^{2+}$ ($Z=4$)
  \item $Al^{3+}$ ($Z=13$)
  \item $O^{2-}$ ($Z=8$)
  \item $F^{-}$ ($Z=9$)
  \end{enumerate}
\end{enumerate}

\end{exercice}


\vressort{5}


\begin{exercice}{Charge d'un atome de Zinc}%\\
\begin{enumerate}
\item Combien de protons l'atome de zinc \noyau{Zn}{30}{65} contient-il ?
\item Combien d'�lectrons comporte-t-il ?
\item Calculer la charge totale des protons
sachant qu'un proton a pour charge $e = 1,6.10^{-19}~C$.
\item Calculer la charge totale des �lectrons
sachant qu'un �lectron a pour charge $-e = -1,6.10^{-19}~C$.
\item En d�duire la charge de l'atome de Zinc.
\item Ce r�sultat est-il identique pour tous les atomes ?
\item A l'issue d'une r�action dite d'oxydation, un atome de zinc $Zn$
  se transforme en un ion $Zn^{2+}$.
  \begin{enumerate}
  \item Donnez l'�quation de cette r�action
  (en faisant intervenir un ou plusieurs �lectrons not�s $e^-$).
  \item Indiquez la charge (en coulomb $C$) de cet ion.
  \end{enumerate}
\end{enumerate}
\end{exercice}

\vressort{3} % Alphabet grec
\ds{Devoir Surveill�}{
%
}

\nomprenomclasse

\setcounter{numexercice}{0}

%\renewcommand{\tabularx}[1]{>{\centering}m{#1}} 

%\newcommand{\tabularxc}[1]{\tabularx{>{\centering}m{#1}}}

\vressort{3}

\begin{exercice}{Connaissance sur l'atome}%\\
\begin{enumerate}
\item De quoi est compos� un atome ?
\item Que signifie les lettres $A$, $Z$ et $X$ dans la repr�sentation \noyau{X}{Z}{A} ?
\item Comment trouve-t-on le nombre de neutrons d'un atome de l'�l�ment pr�c�dent.
\item Si un atome a $5$ protons, combien-a-t-il d'�lectrons ? Pourquoi ?
\item Qu'est-ce qui caract�rise un �l�ment chimique ?
\item Qu'est-ce qu'un isotope ?
\end{enumerate}
\end{exercice}



\vressort{3}



\begin{exercice}{Composition des atomes}\\
En vous aidant du tableau p�riodique des �l�ments,
compl�ter le tableau suivant :

\medskip

\noindent
%\begin{tabularx}{\textwidth}{|>{\centering}X|>{\centering}X|>{\centering}X|>{\centering}X|>{\centering}X|}
% \begin{tabularx}{\linewidth}{|X|X|X|X|X|}
% \hline
% \emph{nom}       & \emph{symbole}  & \emph{protons} & \emph{neutrons}
% & \emph{nucl�ons} \tbnl
% carbone   & \noyau{C}{6}{14}   &         &          & \rule[-0.5cm]{0cm}{1cm}         \tbnl
% fluor     & \noyau{F}{9}{19}   &         &          & \rule[-0.5cm]{0cm}{1cm}         \tbnl
% sodium    & \noyau{Na}{11}{23} &         &          & \rule[-0.5cm]{0cm}{1cm}         \tbnl
% oxyg�ne   & \noyau{O}{8}{16}   &         &          & \rule[-0.5cm]{0cm}{1cm}         \tbnl
% hydrog�ne &          &         & 0        & \rule[-0.5cm]{0cm}{1cm}         \tbnl
%           & \noyau{Cl}{17}{35} &         &          &  \rule[-0.5cm]{0cm}{1cm}        \tbnl
%           &          & 8       & \rule[-0.5cm]{0cm}{1cm}         & 16       \tbnl
% \end{tabularx}



\begin{tabularx}{\linewidth}{|>{\mystrut}X|X|X|X|X|}
\hline
% multicolumn pour faire dispara�tre le \mystrut
\multicolumn{1}{|X|}{\emph{nom}} & \emph{symbole}  &
\emph{protons} & \emph{neutrons} & \emph{nucl�ons} \tbnl
carbone   & \noyau{C}{6}{14}   &   &   &    \tbnl
fluor     & \noyau{F}{9}{19}   &   &   &    \tbnl
sodium    & \noyau{Na}{11}{23} &   &   &    \tbnl
oxyg�ne   & \noyau{O}{8}{16}   &   &   &    \tbnl
hydrog�ne &                    &   & 0 &    \tbnl
          & \noyau{Cl}{17}{35} &   &   &    \tbnl
          &                    & 8 &   & 16 \tbnl
\end{tabularx}


\end{exercice}


\vressort{3}


\begin{exercice}{Masse d'un atome de carbone 12}\\
Soit le carbone $12$ not� \noyau{C}{6}{12}.
\begin{enumerate}
\item L'�l�ment carbone peut-il avoir $5$ protons ? Pourquoi ?
\item Calculer la masse du noyau d'un atome de carbone $12$
sachant que la masse d'un nucl�on est $m_n = 1,67.10^{-27}~kg$
\item Calculer la masse des �lectrons de l'atome de carbone 12
sachant que la masse d'un �lectron vaut $m_e = 9,1.10^{-31}~kg$
\item Comparer la masse des �lectrons de l'atome � la masse du noyau.
Que concluez-vous ?
\item En d�duire, sans nouveau calcul, la masse de l'atome de carbone
  $12$.
\end{enumerate}
\end{exercice}

\newpage

\vressort{1}

\begin{exercice}{Couches �lectroniques}\\
Dans l'�tat le plus stable de l'atome, appel� �tat fondamental,
les �lectrons occupent successivement les couches,
en commen�ant par celles qui sont les plus proches du noyau : 
d'abord $K$ puis $L$ puis $M$.

Lorsqu'une couche est pleine, ou encore satur�e, on passe � la suivante.

La derni�re couche occup�e est appel�e couche externe.\\
Toutes les autres sont appel�es couches internes.

\medskip

\noindent
%\begin{tabularx}{\textwidth}{|>{\centering}X|>{\centering}X|>{\centering}X|>{\centering}X|}
\begin{tabularx}{\textwidth}{|>{\mystrut}X|X|X|X|}
\hline
\multicolumn{1}{|X|}{\emph{Symbole de la couche}}       & $K$ & $L$ & $M$  \tbnl
\emph{Nombre maximal d'�lectrons} & $2$ & $8$ & $18$ \tbnl
\end{tabularx}

\medskip

Ainsi, par exemple, l'atome de chlore ($Z=17$) a la configuration �lectronique :
$(K)^2(L)^8(M)^{7}$.

\begin{enumerate}
\item Indiquez le nombre d'�lectrons et donnez la configuration des atomes suivants :
  \begin{enumerate}
  \item \noyau{H}{1}{}
  \item \noyau{O}{8}{}
  \item \noyau{C}{6}{}
  \item \noyau{Ne}{10}{}
  \end{enumerate}
\item Parmi les ions ci-desssous, pr�cisez s'il s'agit d'anions ou de cations.
Indiquez le nombre d'�lectrons et donnez la configuration �lectronique
des ions suivants :
  \begin{enumerate}
  \item $Be^{2+}$ ($Z=4$)
  \item $Al^{3+}$ ($Z=13$)
  \item $O^{2-}$ ($Z=8$)
  \item $F^{-}$ ($Z=9$)
  \end{enumerate}
\end{enumerate}

\end{exercice}


\vressort{5}


\begin{exercice}{Charge d'un atome de Zinc}%\\
\begin{enumerate}
\item Combien de protons l'atome de zinc \noyau{Zn}{30}{65} contient-il ?
\item Combien d'�lectrons comporte-t-il ?
\item Calculer la charge totale des protons
sachant qu'un proton a pour charge $e = 1,6.10^{-19}~C$.
\item Calculer la charge totale des �lectrons
sachant qu'un �lectron a pour charge $-e = -1,6.10^{-19}~C$.
\item En d�duire la charge de l'atome de Zinc.
\item Ce r�sultat est-il identique pour tous les atomes ?
\item A l'issue d'une r�action dite d'oxydation, un atome de zinc $Zn$
  se transforme en un ion $Zn^{2+}$.
  \begin{enumerate}
  \item Donnez l'�quation de cette r�action
  (en faisant intervenir un ou plusieurs �lectrons not�s $e^-$).
  \item Indiquez la charge (en coulomb $C$) de cet ion.
  \end{enumerate}
\end{enumerate}
\end{exercice}

\vressort{3} %  unit�s - Conversions longueur/surface/volume (faire 3 tableaux) (� compl�ter)
\ds{Devoir Surveill�}{
%
}

\nomprenomclasse

\setcounter{numexercice}{0}

%\renewcommand{\tabularx}[1]{>{\centering}m{#1}} 

%\newcommand{\tabularxc}[1]{\tabularx{>{\centering}m{#1}}}

\vressort{3}

\begin{exercice}{Connaissance sur l'atome}%\\
\begin{enumerate}
\item De quoi est compos� un atome ?
\item Que signifie les lettres $A$, $Z$ et $X$ dans la repr�sentation \noyau{X}{Z}{A} ?
\item Comment trouve-t-on le nombre de neutrons d'un atome de l'�l�ment pr�c�dent.
\item Si un atome a $5$ protons, combien-a-t-il d'�lectrons ? Pourquoi ?
\item Qu'est-ce qui caract�rise un �l�ment chimique ?
\item Qu'est-ce qu'un isotope ?
\end{enumerate}
\end{exercice}



\vressort{3}



\begin{exercice}{Composition des atomes}\\
En vous aidant du tableau p�riodique des �l�ments,
compl�ter le tableau suivant :

\medskip

\noindent
%\begin{tabularx}{\textwidth}{|>{\centering}X|>{\centering}X|>{\centering}X|>{\centering}X|>{\centering}X|}
% \begin{tabularx}{\linewidth}{|X|X|X|X|X|}
% \hline
% \emph{nom}       & \emph{symbole}  & \emph{protons} & \emph{neutrons}
% & \emph{nucl�ons} \tbnl
% carbone   & \noyau{C}{6}{14}   &         &          & \rule[-0.5cm]{0cm}{1cm}         \tbnl
% fluor     & \noyau{F}{9}{19}   &         &          & \rule[-0.5cm]{0cm}{1cm}         \tbnl
% sodium    & \noyau{Na}{11}{23} &         &          & \rule[-0.5cm]{0cm}{1cm}         \tbnl
% oxyg�ne   & \noyau{O}{8}{16}   &         &          & \rule[-0.5cm]{0cm}{1cm}         \tbnl
% hydrog�ne &          &         & 0        & \rule[-0.5cm]{0cm}{1cm}         \tbnl
%           & \noyau{Cl}{17}{35} &         &          &  \rule[-0.5cm]{0cm}{1cm}        \tbnl
%           &          & 8       & \rule[-0.5cm]{0cm}{1cm}         & 16       \tbnl
% \end{tabularx}



\begin{tabularx}{\linewidth}{|>{\mystrut}X|X|X|X|X|}
\hline
% multicolumn pour faire dispara�tre le \mystrut
\multicolumn{1}{|X|}{\emph{nom}} & \emph{symbole}  &
\emph{protons} & \emph{neutrons} & \emph{nucl�ons} \tbnl
carbone   & \noyau{C}{6}{14}   &   &   &    \tbnl
fluor     & \noyau{F}{9}{19}   &   &   &    \tbnl
sodium    & \noyau{Na}{11}{23} &   &   &    \tbnl
oxyg�ne   & \noyau{O}{8}{16}   &   &   &    \tbnl
hydrog�ne &                    &   & 0 &    \tbnl
          & \noyau{Cl}{17}{35} &   &   &    \tbnl
          &                    & 8 &   & 16 \tbnl
\end{tabularx}


\end{exercice}


\vressort{3}


\begin{exercice}{Masse d'un atome de carbone 12}\\
Soit le carbone $12$ not� \noyau{C}{6}{12}.
\begin{enumerate}
\item L'�l�ment carbone peut-il avoir $5$ protons ? Pourquoi ?
\item Calculer la masse du noyau d'un atome de carbone $12$
sachant que la masse d'un nucl�on est $m_n = 1,67.10^{-27}~kg$
\item Calculer la masse des �lectrons de l'atome de carbone 12
sachant que la masse d'un �lectron vaut $m_e = 9,1.10^{-31}~kg$
\item Comparer la masse des �lectrons de l'atome � la masse du noyau.
Que concluez-vous ?
\item En d�duire, sans nouveau calcul, la masse de l'atome de carbone
  $12$.
\end{enumerate}
\end{exercice}

\newpage

\vressort{1}

\begin{exercice}{Couches �lectroniques}\\
Dans l'�tat le plus stable de l'atome, appel� �tat fondamental,
les �lectrons occupent successivement les couches,
en commen�ant par celles qui sont les plus proches du noyau : 
d'abord $K$ puis $L$ puis $M$.

Lorsqu'une couche est pleine, ou encore satur�e, on passe � la suivante.

La derni�re couche occup�e est appel�e couche externe.\\
Toutes les autres sont appel�es couches internes.

\medskip

\noindent
%\begin{tabularx}{\textwidth}{|>{\centering}X|>{\centering}X|>{\centering}X|>{\centering}X|}
\begin{tabularx}{\textwidth}{|>{\mystrut}X|X|X|X|}
\hline
\multicolumn{1}{|X|}{\emph{Symbole de la couche}}       & $K$ & $L$ & $M$  \tbnl
\emph{Nombre maximal d'�lectrons} & $2$ & $8$ & $18$ \tbnl
\end{tabularx}

\medskip

Ainsi, par exemple, l'atome de chlore ($Z=17$) a la configuration �lectronique :
$(K)^2(L)^8(M)^{7}$.

\begin{enumerate}
\item Indiquez le nombre d'�lectrons et donnez la configuration des atomes suivants :
  \begin{enumerate}
  \item \noyau{H}{1}{}
  \item \noyau{O}{8}{}
  \item \noyau{C}{6}{}
  \item \noyau{Ne}{10}{}
  \end{enumerate}
\item Parmi les ions ci-desssous, pr�cisez s'il s'agit d'anions ou de cations.
Indiquez le nombre d'�lectrons et donnez la configuration �lectronique
des ions suivants :
  \begin{enumerate}
  \item $Be^{2+}$ ($Z=4$)
  \item $Al^{3+}$ ($Z=13$)
  \item $O^{2-}$ ($Z=8$)
  \item $F^{-}$ ($Z=9$)
  \end{enumerate}
\end{enumerate}

\end{exercice}


\vressort{5}


\begin{exercice}{Charge d'un atome de Zinc}%\\
\begin{enumerate}
\item Combien de protons l'atome de zinc \noyau{Zn}{30}{65} contient-il ?
\item Combien d'�lectrons comporte-t-il ?
\item Calculer la charge totale des protons
sachant qu'un proton a pour charge $e = 1,6.10^{-19}~C$.
\item Calculer la charge totale des �lectrons
sachant qu'un �lectron a pour charge $-e = -1,6.10^{-19}~C$.
\item En d�duire la charge de l'atome de Zinc.
\item Ce r�sultat est-il identique pour tous les atomes ?
\item A l'issue d'une r�action dite d'oxydation, un atome de zinc $Zn$
  se transforme en un ion $Zn^{2+}$.
  \begin{enumerate}
  \item Donnez l'�quation de cette r�action
  (en faisant intervenir un ou plusieurs �lectrons not�s $e^-$).
  \item Indiquez la charge (en coulomb $C$) de cet ion.
  \end{enumerate}
\end{enumerate}
\end{exercice}

\vressort{3} % Multiples
\ds{Devoir Surveill�}{
%
}

\nomprenomclasse

\setcounter{numexercice}{0}

%\renewcommand{\tabularx}[1]{>{\centering}m{#1}} 

%\newcommand{\tabularxc}[1]{\tabularx{>{\centering}m{#1}}}

\vressort{3}

\begin{exercice}{Connaissance sur l'atome}%\\
\begin{enumerate}
\item De quoi est compos� un atome ?
\item Que signifie les lettres $A$, $Z$ et $X$ dans la repr�sentation \noyau{X}{Z}{A} ?
\item Comment trouve-t-on le nombre de neutrons d'un atome de l'�l�ment pr�c�dent.
\item Si un atome a $5$ protons, combien-a-t-il d'�lectrons ? Pourquoi ?
\item Qu'est-ce qui caract�rise un �l�ment chimique ?
\item Qu'est-ce qu'un isotope ?
\end{enumerate}
\end{exercice}



\vressort{3}



\begin{exercice}{Composition des atomes}\\
En vous aidant du tableau p�riodique des �l�ments,
compl�ter le tableau suivant :

\medskip

\noindent
%\begin{tabularx}{\textwidth}{|>{\centering}X|>{\centering}X|>{\centering}X|>{\centering}X|>{\centering}X|}
% \begin{tabularx}{\linewidth}{|X|X|X|X|X|}
% \hline
% \emph{nom}       & \emph{symbole}  & \emph{protons} & \emph{neutrons}
% & \emph{nucl�ons} \tbnl
% carbone   & \noyau{C}{6}{14}   &         &          & \rule[-0.5cm]{0cm}{1cm}         \tbnl
% fluor     & \noyau{F}{9}{19}   &         &          & \rule[-0.5cm]{0cm}{1cm}         \tbnl
% sodium    & \noyau{Na}{11}{23} &         &          & \rule[-0.5cm]{0cm}{1cm}         \tbnl
% oxyg�ne   & \noyau{O}{8}{16}   &         &          & \rule[-0.5cm]{0cm}{1cm}         \tbnl
% hydrog�ne &          &         & 0        & \rule[-0.5cm]{0cm}{1cm}         \tbnl
%           & \noyau{Cl}{17}{35} &         &          &  \rule[-0.5cm]{0cm}{1cm}        \tbnl
%           &          & 8       & \rule[-0.5cm]{0cm}{1cm}         & 16       \tbnl
% \end{tabularx}



\begin{tabularx}{\linewidth}{|>{\mystrut}X|X|X|X|X|}
\hline
% multicolumn pour faire dispara�tre le \mystrut
\multicolumn{1}{|X|}{\emph{nom}} & \emph{symbole}  &
\emph{protons} & \emph{neutrons} & \emph{nucl�ons} \tbnl
carbone   & \noyau{C}{6}{14}   &   &   &    \tbnl
fluor     & \noyau{F}{9}{19}   &   &   &    \tbnl
sodium    & \noyau{Na}{11}{23} &   &   &    \tbnl
oxyg�ne   & \noyau{O}{8}{16}   &   &   &    \tbnl
hydrog�ne &                    &   & 0 &    \tbnl
          & \noyau{Cl}{17}{35} &   &   &    \tbnl
          &                    & 8 &   & 16 \tbnl
\end{tabularx}


\end{exercice}


\vressort{3}


\begin{exercice}{Masse d'un atome de carbone 12}\\
Soit le carbone $12$ not� \noyau{C}{6}{12}.
\begin{enumerate}
\item L'�l�ment carbone peut-il avoir $5$ protons ? Pourquoi ?
\item Calculer la masse du noyau d'un atome de carbone $12$
sachant que la masse d'un nucl�on est $m_n = 1,67.10^{-27}~kg$
\item Calculer la masse des �lectrons de l'atome de carbone 12
sachant que la masse d'un �lectron vaut $m_e = 9,1.10^{-31}~kg$
\item Comparer la masse des �lectrons de l'atome � la masse du noyau.
Que concluez-vous ?
\item En d�duire, sans nouveau calcul, la masse de l'atome de carbone
  $12$.
\end{enumerate}
\end{exercice}

\newpage

\vressort{1}

\begin{exercice}{Couches �lectroniques}\\
Dans l'�tat le plus stable de l'atome, appel� �tat fondamental,
les �lectrons occupent successivement les couches,
en commen�ant par celles qui sont les plus proches du noyau : 
d'abord $K$ puis $L$ puis $M$.

Lorsqu'une couche est pleine, ou encore satur�e, on passe � la suivante.

La derni�re couche occup�e est appel�e couche externe.\\
Toutes les autres sont appel�es couches internes.

\medskip

\noindent
%\begin{tabularx}{\textwidth}{|>{\centering}X|>{\centering}X|>{\centering}X|>{\centering}X|}
\begin{tabularx}{\textwidth}{|>{\mystrut}X|X|X|X|}
\hline
\multicolumn{1}{|X|}{\emph{Symbole de la couche}}       & $K$ & $L$ & $M$  \tbnl
\emph{Nombre maximal d'�lectrons} & $2$ & $8$ & $18$ \tbnl
\end{tabularx}

\medskip

Ainsi, par exemple, l'atome de chlore ($Z=17$) a la configuration �lectronique :
$(K)^2(L)^8(M)^{7}$.

\begin{enumerate}
\item Indiquez le nombre d'�lectrons et donnez la configuration des atomes suivants :
  \begin{enumerate}
  \item \noyau{H}{1}{}
  \item \noyau{O}{8}{}
  \item \noyau{C}{6}{}
  \item \noyau{Ne}{10}{}
  \end{enumerate}
\item Parmi les ions ci-desssous, pr�cisez s'il s'agit d'anions ou de cations.
Indiquez le nombre d'�lectrons et donnez la configuration �lectronique
des ions suivants :
  \begin{enumerate}
  \item $Be^{2+}$ ($Z=4$)
  \item $Al^{3+}$ ($Z=13$)
  \item $O^{2-}$ ($Z=8$)
  \item $F^{-}$ ($Z=9$)
  \end{enumerate}
\end{enumerate}

\end{exercice}


\vressort{5}


\begin{exercice}{Charge d'un atome de Zinc}%\\
\begin{enumerate}
\item Combien de protons l'atome de zinc \noyau{Zn}{30}{65} contient-il ?
\item Combien d'�lectrons comporte-t-il ?
\item Calculer la charge totale des protons
sachant qu'un proton a pour charge $e = 1,6.10^{-19}~C$.
\item Calculer la charge totale des �lectrons
sachant qu'un �lectron a pour charge $-e = -1,6.10^{-19}~C$.
\item En d�duire la charge de l'atome de Zinc.
\item Ce r�sultat est-il identique pour tous les atomes ?
\item A l'issue d'une r�action dite d'oxydation, un atome de zinc $Zn$
  se transforme en un ion $Zn^{2+}$.
  \begin{enumerate}
  \item Donnez l'�quation de cette r�action
  (en faisant intervenir un ou plusieurs �lectrons not�s $e^-$).
  \item Indiquez la charge (en coulomb $C$) de cet ion.
  \end{enumerate}
\end{enumerate}
\end{exercice}

\vressort{3} % Constantes fondamentales (� compl�ter)

% %\ds{Devoir Surveill�}{
%
}

\nomprenomclasse

\setcounter{numexercice}{0}

%\renewcommand{\tabularx}[1]{>{\centering}m{#1}} 

%\newcommand{\tabularxc}[1]{\tabularx{>{\centering}m{#1}}}

\vressort{3}

\begin{exercice}{Connaissance sur l'atome}%\\
\begin{enumerate}
\item De quoi est compos� un atome ?
\item Que signifie les lettres $A$, $Z$ et $X$ dans la repr�sentation \noyau{X}{Z}{A} ?
\item Comment trouve-t-on le nombre de neutrons d'un atome de l'�l�ment pr�c�dent.
\item Si un atome a $5$ protons, combien-a-t-il d'�lectrons ? Pourquoi ?
\item Qu'est-ce qui caract�rise un �l�ment chimique ?
\item Qu'est-ce qu'un isotope ?
\end{enumerate}
\end{exercice}



\vressort{3}



\begin{exercice}{Composition des atomes}\\
En vous aidant du tableau p�riodique des �l�ments,
compl�ter le tableau suivant :

\medskip

\noindent
%\begin{tabularx}{\textwidth}{|>{\centering}X|>{\centering}X|>{\centering}X|>{\centering}X|>{\centering}X|}
% \begin{tabularx}{\linewidth}{|X|X|X|X|X|}
% \hline
% \emph{nom}       & \emph{symbole}  & \emph{protons} & \emph{neutrons}
% & \emph{nucl�ons} \tbnl
% carbone   & \noyau{C}{6}{14}   &         &          & \rule[-0.5cm]{0cm}{1cm}         \tbnl
% fluor     & \noyau{F}{9}{19}   &         &          & \rule[-0.5cm]{0cm}{1cm}         \tbnl
% sodium    & \noyau{Na}{11}{23} &         &          & \rule[-0.5cm]{0cm}{1cm}         \tbnl
% oxyg�ne   & \noyau{O}{8}{16}   &         &          & \rule[-0.5cm]{0cm}{1cm}         \tbnl
% hydrog�ne &          &         & 0        & \rule[-0.5cm]{0cm}{1cm}         \tbnl
%           & \noyau{Cl}{17}{35} &         &          &  \rule[-0.5cm]{0cm}{1cm}        \tbnl
%           &          & 8       & \rule[-0.5cm]{0cm}{1cm}         & 16       \tbnl
% \end{tabularx}



\begin{tabularx}{\linewidth}{|>{\mystrut}X|X|X|X|X|}
\hline
% multicolumn pour faire dispara�tre le \mystrut
\multicolumn{1}{|X|}{\emph{nom}} & \emph{symbole}  &
\emph{protons} & \emph{neutrons} & \emph{nucl�ons} \tbnl
carbone   & \noyau{C}{6}{14}   &   &   &    \tbnl
fluor     & \noyau{F}{9}{19}   &   &   &    \tbnl
sodium    & \noyau{Na}{11}{23} &   &   &    \tbnl
oxyg�ne   & \noyau{O}{8}{16}   &   &   &    \tbnl
hydrog�ne &                    &   & 0 &    \tbnl
          & \noyau{Cl}{17}{35} &   &   &    \tbnl
          &                    & 8 &   & 16 \tbnl
\end{tabularx}


\end{exercice}


\vressort{3}


\begin{exercice}{Masse d'un atome de carbone 12}\\
Soit le carbone $12$ not� \noyau{C}{6}{12}.
\begin{enumerate}
\item L'�l�ment carbone peut-il avoir $5$ protons ? Pourquoi ?
\item Calculer la masse du noyau d'un atome de carbone $12$
sachant que la masse d'un nucl�on est $m_n = 1,67.10^{-27}~kg$
\item Calculer la masse des �lectrons de l'atome de carbone 12
sachant que la masse d'un �lectron vaut $m_e = 9,1.10^{-31}~kg$
\item Comparer la masse des �lectrons de l'atome � la masse du noyau.
Que concluez-vous ?
\item En d�duire, sans nouveau calcul, la masse de l'atome de carbone
  $12$.
\end{enumerate}
\end{exercice}

\newpage

\vressort{1}

\begin{exercice}{Couches �lectroniques}\\
Dans l'�tat le plus stable de l'atome, appel� �tat fondamental,
les �lectrons occupent successivement les couches,
en commen�ant par celles qui sont les plus proches du noyau : 
d'abord $K$ puis $L$ puis $M$.

Lorsqu'une couche est pleine, ou encore satur�e, on passe � la suivante.

La derni�re couche occup�e est appel�e couche externe.\\
Toutes les autres sont appel�es couches internes.

\medskip

\noindent
%\begin{tabularx}{\textwidth}{|>{\centering}X|>{\centering}X|>{\centering}X|>{\centering}X|}
\begin{tabularx}{\textwidth}{|>{\mystrut}X|X|X|X|}
\hline
\multicolumn{1}{|X|}{\emph{Symbole de la couche}}       & $K$ & $L$ & $M$  \tbnl
\emph{Nombre maximal d'�lectrons} & $2$ & $8$ & $18$ \tbnl
\end{tabularx}

\medskip

Ainsi, par exemple, l'atome de chlore ($Z=17$) a la configuration �lectronique :
$(K)^2(L)^8(M)^{7}$.

\begin{enumerate}
\item Indiquez le nombre d'�lectrons et donnez la configuration des atomes suivants :
  \begin{enumerate}
  \item \noyau{H}{1}{}
  \item \noyau{O}{8}{}
  \item \noyau{C}{6}{}
  \item \noyau{Ne}{10}{}
  \end{enumerate}
\item Parmi les ions ci-desssous, pr�cisez s'il s'agit d'anions ou de cations.
Indiquez le nombre d'�lectrons et donnez la configuration �lectronique
des ions suivants :
  \begin{enumerate}
  \item $Be^{2+}$ ($Z=4$)
  \item $Al^{3+}$ ($Z=13$)
  \item $O^{2-}$ ($Z=8$)
  \item $F^{-}$ ($Z=9$)
  \end{enumerate}
\end{enumerate}

\end{exercice}


\vressort{5}


\begin{exercice}{Charge d'un atome de Zinc}%\\
\begin{enumerate}
\item Combien de protons l'atome de zinc \noyau{Zn}{30}{65} contient-il ?
\item Combien d'�lectrons comporte-t-il ?
\item Calculer la charge totale des protons
sachant qu'un proton a pour charge $e = 1,6.10^{-19}~C$.
\item Calculer la charge totale des �lectrons
sachant qu'un �lectron a pour charge $-e = -1,6.10^{-19}~C$.
\item En d�duire la charge de l'atome de Zinc.
\item Ce r�sultat est-il identique pour tous les atomes ?
\item A l'issue d'une r�action dite d'oxydation, un atome de zinc $Zn$
  se transforme en un ion $Zn^{2+}$.
  \begin{enumerate}
  \item Donnez l'�quation de cette r�action
  (en faisant intervenir un ou plusieurs �lectrons not�s $e^-$).
  \item Indiquez la charge (en coulomb $C$) de cet ion.
  \end{enumerate}
\end{enumerate}
\end{exercice}

\vressort{3} % TO DO : Donn�es astro
\ds{Devoir Surveill�}{
%
}

\nomprenomclasse

\setcounter{numexercice}{0}

%\renewcommand{\tabularx}[1]{>{\centering}m{#1}} 

%\newcommand{\tabularxc}[1]{\tabularx{>{\centering}m{#1}}}

\vressort{3}

\begin{exercice}{Connaissance sur l'atome}%\\
\begin{enumerate}
\item De quoi est compos� un atome ?
\item Que signifie les lettres $A$, $Z$ et $X$ dans la repr�sentation \noyau{X}{Z}{A} ?
\item Comment trouve-t-on le nombre de neutrons d'un atome de l'�l�ment pr�c�dent.
\item Si un atome a $5$ protons, combien-a-t-il d'�lectrons ? Pourquoi ?
\item Qu'est-ce qui caract�rise un �l�ment chimique ?
\item Qu'est-ce qu'un isotope ?
\end{enumerate}
\end{exercice}



\vressort{3}



\begin{exercice}{Composition des atomes}\\
En vous aidant du tableau p�riodique des �l�ments,
compl�ter le tableau suivant :

\medskip

\noindent
%\begin{tabularx}{\textwidth}{|>{\centering}X|>{\centering}X|>{\centering}X|>{\centering}X|>{\centering}X|}
% \begin{tabularx}{\linewidth}{|X|X|X|X|X|}
% \hline
% \emph{nom}       & \emph{symbole}  & \emph{protons} & \emph{neutrons}
% & \emph{nucl�ons} \tbnl
% carbone   & \noyau{C}{6}{14}   &         &          & \rule[-0.5cm]{0cm}{1cm}         \tbnl
% fluor     & \noyau{F}{9}{19}   &         &          & \rule[-0.5cm]{0cm}{1cm}         \tbnl
% sodium    & \noyau{Na}{11}{23} &         &          & \rule[-0.5cm]{0cm}{1cm}         \tbnl
% oxyg�ne   & \noyau{O}{8}{16}   &         &          & \rule[-0.5cm]{0cm}{1cm}         \tbnl
% hydrog�ne &          &         & 0        & \rule[-0.5cm]{0cm}{1cm}         \tbnl
%           & \noyau{Cl}{17}{35} &         &          &  \rule[-0.5cm]{0cm}{1cm}        \tbnl
%           &          & 8       & \rule[-0.5cm]{0cm}{1cm}         & 16       \tbnl
% \end{tabularx}



\begin{tabularx}{\linewidth}{|>{\mystrut}X|X|X|X|X|}
\hline
% multicolumn pour faire dispara�tre le \mystrut
\multicolumn{1}{|X|}{\emph{nom}} & \emph{symbole}  &
\emph{protons} & \emph{neutrons} & \emph{nucl�ons} \tbnl
carbone   & \noyau{C}{6}{14}   &   &   &    \tbnl
fluor     & \noyau{F}{9}{19}   &   &   &    \tbnl
sodium    & \noyau{Na}{11}{23} &   &   &    \tbnl
oxyg�ne   & \noyau{O}{8}{16}   &   &   &    \tbnl
hydrog�ne &                    &   & 0 &    \tbnl
          & \noyau{Cl}{17}{35} &   &   &    \tbnl
          &                    & 8 &   & 16 \tbnl
\end{tabularx}


\end{exercice}


\vressort{3}


\begin{exercice}{Masse d'un atome de carbone 12}\\
Soit le carbone $12$ not� \noyau{C}{6}{12}.
\begin{enumerate}
\item L'�l�ment carbone peut-il avoir $5$ protons ? Pourquoi ?
\item Calculer la masse du noyau d'un atome de carbone $12$
sachant que la masse d'un nucl�on est $m_n = 1,67.10^{-27}~kg$
\item Calculer la masse des �lectrons de l'atome de carbone 12
sachant que la masse d'un �lectron vaut $m_e = 9,1.10^{-31}~kg$
\item Comparer la masse des �lectrons de l'atome � la masse du noyau.
Que concluez-vous ?
\item En d�duire, sans nouveau calcul, la masse de l'atome de carbone
  $12$.
\end{enumerate}
\end{exercice}

\newpage

\vressort{1}

\begin{exercice}{Couches �lectroniques}\\
Dans l'�tat le plus stable de l'atome, appel� �tat fondamental,
les �lectrons occupent successivement les couches,
en commen�ant par celles qui sont les plus proches du noyau : 
d'abord $K$ puis $L$ puis $M$.

Lorsqu'une couche est pleine, ou encore satur�e, on passe � la suivante.

La derni�re couche occup�e est appel�e couche externe.\\
Toutes les autres sont appel�es couches internes.

\medskip

\noindent
%\begin{tabularx}{\textwidth}{|>{\centering}X|>{\centering}X|>{\centering}X|>{\centering}X|}
\begin{tabularx}{\textwidth}{|>{\mystrut}X|X|X|X|}
\hline
\multicolumn{1}{|X|}{\emph{Symbole de la couche}}       & $K$ & $L$ & $M$  \tbnl
\emph{Nombre maximal d'�lectrons} & $2$ & $8$ & $18$ \tbnl
\end{tabularx}

\medskip

Ainsi, par exemple, l'atome de chlore ($Z=17$) a la configuration �lectronique :
$(K)^2(L)^8(M)^{7}$.

\begin{enumerate}
\item Indiquez le nombre d'�lectrons et donnez la configuration des atomes suivants :
  \begin{enumerate}
  \item \noyau{H}{1}{}
  \item \noyau{O}{8}{}
  \item \noyau{C}{6}{}
  \item \noyau{Ne}{10}{}
  \end{enumerate}
\item Parmi les ions ci-desssous, pr�cisez s'il s'agit d'anions ou de cations.
Indiquez le nombre d'�lectrons et donnez la configuration �lectronique
des ions suivants :
  \begin{enumerate}
  \item $Be^{2+}$ ($Z=4$)
  \item $Al^{3+}$ ($Z=13$)
  \item $O^{2-}$ ($Z=8$)
  \item $F^{-}$ ($Z=9$)
  \end{enumerate}
\end{enumerate}

\end{exercice}


\vressort{5}


\begin{exercice}{Charge d'un atome de Zinc}%\\
\begin{enumerate}
\item Combien de protons l'atome de zinc \noyau{Zn}{30}{65} contient-il ?
\item Combien d'�lectrons comporte-t-il ?
\item Calculer la charge totale des protons
sachant qu'un proton a pour charge $e = 1,6.10^{-19}~C$.
\item Calculer la charge totale des �lectrons
sachant qu'un �lectron a pour charge $-e = -1,6.10^{-19}~C$.
\item En d�duire la charge de l'atome de Zinc.
\item Ce r�sultat est-il identique pour tous les atomes ?
\item A l'issue d'une r�action dite d'oxydation, un atome de zinc $Zn$
  se transforme en un ion $Zn^{2+}$.
  \begin{enumerate}
  \item Donnez l'�quation de cette r�action
  (en faisant intervenir un ou plusieurs �lectrons not�s $e^-$).
  \item Indiquez la charge (en coulomb $C$) de cet ion.
  \end{enumerate}
\end{enumerate}
\end{exercice}

\vressort{3}

\ds{Devoir Surveill�}{
%
}

\nomprenomclasse

\setcounter{numexercice}{0}

%\renewcommand{\tabularx}[1]{>{\centering}m{#1}} 

%\newcommand{\tabularxc}[1]{\tabularx{>{\centering}m{#1}}}

\vressort{3}

\begin{exercice}{Connaissance sur l'atome}%\\
\begin{enumerate}
\item De quoi est compos� un atome ?
\item Que signifie les lettres $A$, $Z$ et $X$ dans la repr�sentation \noyau{X}{Z}{A} ?
\item Comment trouve-t-on le nombre de neutrons d'un atome de l'�l�ment pr�c�dent.
\item Si un atome a $5$ protons, combien-a-t-il d'�lectrons ? Pourquoi ?
\item Qu'est-ce qui caract�rise un �l�ment chimique ?
\item Qu'est-ce qu'un isotope ?
\end{enumerate}
\end{exercice}



\vressort{3}



\begin{exercice}{Composition des atomes}\\
En vous aidant du tableau p�riodique des �l�ments,
compl�ter le tableau suivant :

\medskip

\noindent
%\begin{tabularx}{\textwidth}{|>{\centering}X|>{\centering}X|>{\centering}X|>{\centering}X|>{\centering}X|}
% \begin{tabularx}{\linewidth}{|X|X|X|X|X|}
% \hline
% \emph{nom}       & \emph{symbole}  & \emph{protons} & \emph{neutrons}
% & \emph{nucl�ons} \tbnl
% carbone   & \noyau{C}{6}{14}   &         &          & \rule[-0.5cm]{0cm}{1cm}         \tbnl
% fluor     & \noyau{F}{9}{19}   &         &          & \rule[-0.5cm]{0cm}{1cm}         \tbnl
% sodium    & \noyau{Na}{11}{23} &         &          & \rule[-0.5cm]{0cm}{1cm}         \tbnl
% oxyg�ne   & \noyau{O}{8}{16}   &         &          & \rule[-0.5cm]{0cm}{1cm}         \tbnl
% hydrog�ne &          &         & 0        & \rule[-0.5cm]{0cm}{1cm}         \tbnl
%           & \noyau{Cl}{17}{35} &         &          &  \rule[-0.5cm]{0cm}{1cm}        \tbnl
%           &          & 8       & \rule[-0.5cm]{0cm}{1cm}         & 16       \tbnl
% \end{tabularx}



\begin{tabularx}{\linewidth}{|>{\mystrut}X|X|X|X|X|}
\hline
% multicolumn pour faire dispara�tre le \mystrut
\multicolumn{1}{|X|}{\emph{nom}} & \emph{symbole}  &
\emph{protons} & \emph{neutrons} & \emph{nucl�ons} \tbnl
carbone   & \noyau{C}{6}{14}   &   &   &    \tbnl
fluor     & \noyau{F}{9}{19}   &   &   &    \tbnl
sodium    & \noyau{Na}{11}{23} &   &   &    \tbnl
oxyg�ne   & \noyau{O}{8}{16}   &   &   &    \tbnl
hydrog�ne &                    &   & 0 &    \tbnl
          & \noyau{Cl}{17}{35} &   &   &    \tbnl
          &                    & 8 &   & 16 \tbnl
\end{tabularx}


\end{exercice}


\vressort{3}


\begin{exercice}{Masse d'un atome de carbone 12}\\
Soit le carbone $12$ not� \noyau{C}{6}{12}.
\begin{enumerate}
\item L'�l�ment carbone peut-il avoir $5$ protons ? Pourquoi ?
\item Calculer la masse du noyau d'un atome de carbone $12$
sachant que la masse d'un nucl�on est $m_n = 1,67.10^{-27}~kg$
\item Calculer la masse des �lectrons de l'atome de carbone 12
sachant que la masse d'un �lectron vaut $m_e = 9,1.10^{-31}~kg$
\item Comparer la masse des �lectrons de l'atome � la masse du noyau.
Que concluez-vous ?
\item En d�duire, sans nouveau calcul, la masse de l'atome de carbone
  $12$.
\end{enumerate}
\end{exercice}

\newpage

\vressort{1}

\begin{exercice}{Couches �lectroniques}\\
Dans l'�tat le plus stable de l'atome, appel� �tat fondamental,
les �lectrons occupent successivement les couches,
en commen�ant par celles qui sont les plus proches du noyau : 
d'abord $K$ puis $L$ puis $M$.

Lorsqu'une couche est pleine, ou encore satur�e, on passe � la suivante.

La derni�re couche occup�e est appel�e couche externe.\\
Toutes les autres sont appel�es couches internes.

\medskip

\noindent
%\begin{tabularx}{\textwidth}{|>{\centering}X|>{\centering}X|>{\centering}X|>{\centering}X|}
\begin{tabularx}{\textwidth}{|>{\mystrut}X|X|X|X|}
\hline
\multicolumn{1}{|X|}{\emph{Symbole de la couche}}       & $K$ & $L$ & $M$  \tbnl
\emph{Nombre maximal d'�lectrons} & $2$ & $8$ & $18$ \tbnl
\end{tabularx}

\medskip

Ainsi, par exemple, l'atome de chlore ($Z=17$) a la configuration �lectronique :
$(K)^2(L)^8(M)^{7}$.

\begin{enumerate}
\item Indiquez le nombre d'�lectrons et donnez la configuration des atomes suivants :
  \begin{enumerate}
  \item \noyau{H}{1}{}
  \item \noyau{O}{8}{}
  \item \noyau{C}{6}{}
  \item \noyau{Ne}{10}{}
  \end{enumerate}
\item Parmi les ions ci-desssous, pr�cisez s'il s'agit d'anions ou de cations.
Indiquez le nombre d'�lectrons et donnez la configuration �lectronique
des ions suivants :
  \begin{enumerate}
  \item $Be^{2+}$ ($Z=4$)
  \item $Al^{3+}$ ($Z=13$)
  \item $O^{2-}$ ($Z=8$)
  \item $F^{-}$ ($Z=9$)
  \end{enumerate}
\end{enumerate}

\end{exercice}


\vressort{5}


\begin{exercice}{Charge d'un atome de Zinc}%\\
\begin{enumerate}
\item Combien de protons l'atome de zinc \noyau{Zn}{30}{65} contient-il ?
\item Combien d'�lectrons comporte-t-il ?
\item Calculer la charge totale des protons
sachant qu'un proton a pour charge $e = 1,6.10^{-19}~C$.
\item Calculer la charge totale des �lectrons
sachant qu'un �lectron a pour charge $-e = -1,6.10^{-19}~C$.
\item En d�duire la charge de l'atome de Zinc.
\item Ce r�sultat est-il identique pour tous les atomes ?
\item A l'issue d'une r�action dite d'oxydation, un atome de zinc $Zn$
  se transforme en un ion $Zn^{2+}$.
  \begin{enumerate}
  \item Donnez l'�quation de cette r�action
  (en faisant intervenir un ou plusieurs �lectrons not�s $e^-$).
  \item Indiquez la charge (en coulomb $C$) de cet ion.
  \end{enumerate}
\end{enumerate}
\end{exercice}

\vressort{3}


\ds{Devoir Surveill�}{
%
}

\nomprenomclasse

\setcounter{numexercice}{0}

%\renewcommand{\tabularx}[1]{>{\centering}m{#1}} 

%\newcommand{\tabularxc}[1]{\tabularx{>{\centering}m{#1}}}

\vressort{3}

\begin{exercice}{Connaissance sur l'atome}%\\
\begin{enumerate}
\item De quoi est compos� un atome ?
\item Que signifie les lettres $A$, $Z$ et $X$ dans la repr�sentation \noyau{X}{Z}{A} ?
\item Comment trouve-t-on le nombre de neutrons d'un atome de l'�l�ment pr�c�dent.
\item Si un atome a $5$ protons, combien-a-t-il d'�lectrons ? Pourquoi ?
\item Qu'est-ce qui caract�rise un �l�ment chimique ?
\item Qu'est-ce qu'un isotope ?
\end{enumerate}
\end{exercice}



\vressort{3}



\begin{exercice}{Composition des atomes}\\
En vous aidant du tableau p�riodique des �l�ments,
compl�ter le tableau suivant :

\medskip

\noindent
%\begin{tabularx}{\textwidth}{|>{\centering}X|>{\centering}X|>{\centering}X|>{\centering}X|>{\centering}X|}
% \begin{tabularx}{\linewidth}{|X|X|X|X|X|}
% \hline
% \emph{nom}       & \emph{symbole}  & \emph{protons} & \emph{neutrons}
% & \emph{nucl�ons} \tbnl
% carbone   & \noyau{C}{6}{14}   &         &          & \rule[-0.5cm]{0cm}{1cm}         \tbnl
% fluor     & \noyau{F}{9}{19}   &         &          & \rule[-0.5cm]{0cm}{1cm}         \tbnl
% sodium    & \noyau{Na}{11}{23} &         &          & \rule[-0.5cm]{0cm}{1cm}         \tbnl
% oxyg�ne   & \noyau{O}{8}{16}   &         &          & \rule[-0.5cm]{0cm}{1cm}         \tbnl
% hydrog�ne &          &         & 0        & \rule[-0.5cm]{0cm}{1cm}         \tbnl
%           & \noyau{Cl}{17}{35} &         &          &  \rule[-0.5cm]{0cm}{1cm}        \tbnl
%           &          & 8       & \rule[-0.5cm]{0cm}{1cm}         & 16       \tbnl
% \end{tabularx}



\begin{tabularx}{\linewidth}{|>{\mystrut}X|X|X|X|X|}
\hline
% multicolumn pour faire dispara�tre le \mystrut
\multicolumn{1}{|X|}{\emph{nom}} & \emph{symbole}  &
\emph{protons} & \emph{neutrons} & \emph{nucl�ons} \tbnl
carbone   & \noyau{C}{6}{14}   &   &   &    \tbnl
fluor     & \noyau{F}{9}{19}   &   &   &    \tbnl
sodium    & \noyau{Na}{11}{23} &   &   &    \tbnl
oxyg�ne   & \noyau{O}{8}{16}   &   &   &    \tbnl
hydrog�ne &                    &   & 0 &    \tbnl
          & \noyau{Cl}{17}{35} &   &   &    \tbnl
          &                    & 8 &   & 16 \tbnl
\end{tabularx}


\end{exercice}


\vressort{3}


\begin{exercice}{Masse d'un atome de carbone 12}\\
Soit le carbone $12$ not� \noyau{C}{6}{12}.
\begin{enumerate}
\item L'�l�ment carbone peut-il avoir $5$ protons ? Pourquoi ?
\item Calculer la masse du noyau d'un atome de carbone $12$
sachant que la masse d'un nucl�on est $m_n = 1,67.10^{-27}~kg$
\item Calculer la masse des �lectrons de l'atome de carbone 12
sachant que la masse d'un �lectron vaut $m_e = 9,1.10^{-31}~kg$
\item Comparer la masse des �lectrons de l'atome � la masse du noyau.
Que concluez-vous ?
\item En d�duire, sans nouveau calcul, la masse de l'atome de carbone
  $12$.
\end{enumerate}
\end{exercice}

\newpage

\vressort{1}

\begin{exercice}{Couches �lectroniques}\\
Dans l'�tat le plus stable de l'atome, appel� �tat fondamental,
les �lectrons occupent successivement les couches,
en commen�ant par celles qui sont les plus proches du noyau : 
d'abord $K$ puis $L$ puis $M$.

Lorsqu'une couche est pleine, ou encore satur�e, on passe � la suivante.

La derni�re couche occup�e est appel�e couche externe.\\
Toutes les autres sont appel�es couches internes.

\medskip

\noindent
%\begin{tabularx}{\textwidth}{|>{\centering}X|>{\centering}X|>{\centering}X|>{\centering}X|}
\begin{tabularx}{\textwidth}{|>{\mystrut}X|X|X|X|}
\hline
\multicolumn{1}{|X|}{\emph{Symbole de la couche}}       & $K$ & $L$ & $M$  \tbnl
\emph{Nombre maximal d'�lectrons} & $2$ & $8$ & $18$ \tbnl
\end{tabularx}

\medskip

Ainsi, par exemple, l'atome de chlore ($Z=17$) a la configuration �lectronique :
$(K)^2(L)^8(M)^{7}$.

\begin{enumerate}
\item Indiquez le nombre d'�lectrons et donnez la configuration des atomes suivants :
  \begin{enumerate}
  \item \noyau{H}{1}{}
  \item \noyau{O}{8}{}
  \item \noyau{C}{6}{}
  \item \noyau{Ne}{10}{}
  \end{enumerate}
\item Parmi les ions ci-desssous, pr�cisez s'il s'agit d'anions ou de cations.
Indiquez le nombre d'�lectrons et donnez la configuration �lectronique
des ions suivants :
  \begin{enumerate}
  \item $Be^{2+}$ ($Z=4$)
  \item $Al^{3+}$ ($Z=13$)
  \item $O^{2-}$ ($Z=8$)
  \item $F^{-}$ ($Z=9$)
  \end{enumerate}
\end{enumerate}

\end{exercice}


\vressort{5}


\begin{exercice}{Charge d'un atome de Zinc}%\\
\begin{enumerate}
\item Combien de protons l'atome de zinc \noyau{Zn}{30}{65} contient-il ?
\item Combien d'�lectrons comporte-t-il ?
\item Calculer la charge totale des protons
sachant qu'un proton a pour charge $e = 1,6.10^{-19}~C$.
\item Calculer la charge totale des �lectrons
sachant qu'un �lectron a pour charge $-e = -1,6.10^{-19}~C$.
\item En d�duire la charge de l'atome de Zinc.
\item Ce r�sultat est-il identique pour tous les atomes ?
\item A l'issue d'une r�action dite d'oxydation, un atome de zinc $Zn$
  se transforme en un ion $Zn^{2+}$.
  \begin{enumerate}
  \item Donnez l'�quation de cette r�action
  (en faisant intervenir un ou plusieurs �lectrons not�s $e^-$).
  \item Indiquez la charge (en coulomb $C$) de cet ion.
  \end{enumerate}
\end{enumerate}
\end{exercice}

\vressort{3} % oscillogrammes


% Trigonom�trie
% Math (autres)

% Pr�cision d'une mesure ; nombre de chiffres significatifs


\chapitre{Devoir Surveill�} % Mod�le de DS
\ds{Devoir Surveill�}{
%
}

\nomprenomclasse

\setcounter{numexercice}{0}

%\renewcommand{\tabularx}[1]{>{\centering}m{#1}} 

%\newcommand{\tabularxc}[1]{\tabularx{>{\centering}m{#1}}}

\vressort{3}

\begin{exercice}{Connaissance sur l'atome}%\\
\begin{enumerate}
\item De quoi est compos� un atome ?
\item Que signifie les lettres $A$, $Z$ et $X$ dans la repr�sentation \noyau{X}{Z}{A} ?
\item Comment trouve-t-on le nombre de neutrons d'un atome de l'�l�ment pr�c�dent.
\item Si un atome a $5$ protons, combien-a-t-il d'�lectrons ? Pourquoi ?
\item Qu'est-ce qui caract�rise un �l�ment chimique ?
\item Qu'est-ce qu'un isotope ?
\end{enumerate}
\end{exercice}



\vressort{3}



\begin{exercice}{Composition des atomes}\\
En vous aidant du tableau p�riodique des �l�ments,
compl�ter le tableau suivant :

\medskip

\noindent
%\begin{tabularx}{\textwidth}{|>{\centering}X|>{\centering}X|>{\centering}X|>{\centering}X|>{\centering}X|}
% \begin{tabularx}{\linewidth}{|X|X|X|X|X|}
% \hline
% \emph{nom}       & \emph{symbole}  & \emph{protons} & \emph{neutrons}
% & \emph{nucl�ons} \tbnl
% carbone   & \noyau{C}{6}{14}   &         &          & \rule[-0.5cm]{0cm}{1cm}         \tbnl
% fluor     & \noyau{F}{9}{19}   &         &          & \rule[-0.5cm]{0cm}{1cm}         \tbnl
% sodium    & \noyau{Na}{11}{23} &         &          & \rule[-0.5cm]{0cm}{1cm}         \tbnl
% oxyg�ne   & \noyau{O}{8}{16}   &         &          & \rule[-0.5cm]{0cm}{1cm}         \tbnl
% hydrog�ne &          &         & 0        & \rule[-0.5cm]{0cm}{1cm}         \tbnl
%           & \noyau{Cl}{17}{35} &         &          &  \rule[-0.5cm]{0cm}{1cm}        \tbnl
%           &          & 8       & \rule[-0.5cm]{0cm}{1cm}         & 16       \tbnl
% \end{tabularx}



\begin{tabularx}{\linewidth}{|>{\mystrut}X|X|X|X|X|}
\hline
% multicolumn pour faire dispara�tre le \mystrut
\multicolumn{1}{|X|}{\emph{nom}} & \emph{symbole}  &
\emph{protons} & \emph{neutrons} & \emph{nucl�ons} \tbnl
carbone   & \noyau{C}{6}{14}   &   &   &    \tbnl
fluor     & \noyau{F}{9}{19}   &   &   &    \tbnl
sodium    & \noyau{Na}{11}{23} &   &   &    \tbnl
oxyg�ne   & \noyau{O}{8}{16}   &   &   &    \tbnl
hydrog�ne &                    &   & 0 &    \tbnl
          & \noyau{Cl}{17}{35} &   &   &    \tbnl
          &                    & 8 &   & 16 \tbnl
\end{tabularx}


\end{exercice}


\vressort{3}


\begin{exercice}{Masse d'un atome de carbone 12}\\
Soit le carbone $12$ not� \noyau{C}{6}{12}.
\begin{enumerate}
\item L'�l�ment carbone peut-il avoir $5$ protons ? Pourquoi ?
\item Calculer la masse du noyau d'un atome de carbone $12$
sachant que la masse d'un nucl�on est $m_n = 1,67.10^{-27}~kg$
\item Calculer la masse des �lectrons de l'atome de carbone 12
sachant que la masse d'un �lectron vaut $m_e = 9,1.10^{-31}~kg$
\item Comparer la masse des �lectrons de l'atome � la masse du noyau.
Que concluez-vous ?
\item En d�duire, sans nouveau calcul, la masse de l'atome de carbone
  $12$.
\end{enumerate}
\end{exercice}

\newpage

\vressort{1}

\begin{exercice}{Couches �lectroniques}\\
Dans l'�tat le plus stable de l'atome, appel� �tat fondamental,
les �lectrons occupent successivement les couches,
en commen�ant par celles qui sont les plus proches du noyau : 
d'abord $K$ puis $L$ puis $M$.

Lorsqu'une couche est pleine, ou encore satur�e, on passe � la suivante.

La derni�re couche occup�e est appel�e couche externe.\\
Toutes les autres sont appel�es couches internes.

\medskip

\noindent
%\begin{tabularx}{\textwidth}{|>{\centering}X|>{\centering}X|>{\centering}X|>{\centering}X|}
\begin{tabularx}{\textwidth}{|>{\mystrut}X|X|X|X|}
\hline
\multicolumn{1}{|X|}{\emph{Symbole de la couche}}       & $K$ & $L$ & $M$  \tbnl
\emph{Nombre maximal d'�lectrons} & $2$ & $8$ & $18$ \tbnl
\end{tabularx}

\medskip

Ainsi, par exemple, l'atome de chlore ($Z=17$) a la configuration �lectronique :
$(K)^2(L)^8(M)^{7}$.

\begin{enumerate}
\item Indiquez le nombre d'�lectrons et donnez la configuration des atomes suivants :
  \begin{enumerate}
  \item \noyau{H}{1}{}
  \item \noyau{O}{8}{}
  \item \noyau{C}{6}{}
  \item \noyau{Ne}{10}{}
  \end{enumerate}
\item Parmi les ions ci-desssous, pr�cisez s'il s'agit d'anions ou de cations.
Indiquez le nombre d'�lectrons et donnez la configuration �lectronique
des ions suivants :
  \begin{enumerate}
  \item $Be^{2+}$ ($Z=4$)
  \item $Al^{3+}$ ($Z=13$)
  \item $O^{2-}$ ($Z=8$)
  \item $F^{-}$ ($Z=9$)
  \end{enumerate}
\end{enumerate}

\end{exercice}


\vressort{5}


\begin{exercice}{Charge d'un atome de Zinc}%\\
\begin{enumerate}
\item Combien de protons l'atome de zinc \noyau{Zn}{30}{65} contient-il ?
\item Combien d'�lectrons comporte-t-il ?
\item Calculer la charge totale des protons
sachant qu'un proton a pour charge $e = 1,6.10^{-19}~C$.
\item Calculer la charge totale des �lectrons
sachant qu'un �lectron a pour charge $-e = -1,6.10^{-19}~C$.
\item En d�duire la charge de l'atome de Zinc.
\item Ce r�sultat est-il identique pour tous les atomes ?
\item A l'issue d'une r�action dite d'oxydation, un atome de zinc $Zn$
  se transforme en un ion $Zn^{2+}$.
  \begin{enumerate}
  \item Donnez l'�quation de cette r�action
  (en faisant intervenir un ou plusieurs �lectrons not�s $e^-$).
  \item Indiquez la charge (en coulomb $C$) de cet ion.
  \end{enumerate}
\end{enumerate}
\end{exercice}

\vressort{3}

