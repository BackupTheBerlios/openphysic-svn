% �~Simplifiez-vous la bibliographie sous LateX~�
% D'apr�s GNU/Linux Magazine France Mai 2006 n�83 
% Cyril Buttay et Florent Morel
% voir aussi
% http://www.lsv.ens-cachan.fr/~markey/bibla.php

%% Bibliographie incorpor�e avec l'environnement thebibliography

% \section{Essai de bibliographie incorpor�e}

% L'utilisation de l'environnement \texttt{thebibliography} est d�crite
% dans \cite{livre_de_chevet}. Quand nous serons pass�s � la post�rit�,
% le pr�sent article sera cit� comme suit \cite{cyril_et_florent}.


% \begin{thebibliography}{9}
% \bibitem[Lamport-86]{livre_de_chevet} Lamport, L,
%  {\it \LaTeX: A Document Preparation System},
%  Addison-Wesley, 1986.

% \bibitem{cyril_et_florent} Buttay, C. \& Morel, F.,
%  {\it Linux Pratique}, n�1457, d�cembre 2051.
% \end{thebibliography}



%% Bibliographie externe avec BibTeX

% Voir GNU/Linux Magazine France Mai 2006 (N�83)
% latex fichier.tex
% bibtex fichier     (utilise fichier.aux et donne fichier.bbl avec
%                     fichier = main ou only)
% latex fichier.tex
% latex fichier.tex

\section{Essai d'utilisation de BibTeX}
L'utilisation de BibTeX est d�crite dans \cite{lamport86}.
Quand nous serons pass�s � la post�rit�, le pr�sent article sera cit�
comme suit \cite{cyril_et_florent}.



% voir les styles de biblio BibTeX dans
% /usr/share/texmf/bibtex/bst
% http://www.ctan.org/tex-archive/biblio/bibtex/contrib/bib-fr
% Il faut ajouter les styles de biblio fran�ais (*-fr.bst) � tetex

\bibliographystyle{plain-fr}
% plain : classique, alphab�tique, �tiquette num�rique
% alpha : idem mais �tiquette 3 lettres auteur et 2 chiffre
% unsrt : �tiquette num�rique pas de classement (ordre de la biblio)
% apalike : �tiquette auteur-date




\bibliography{bibliographie/bibliographie}

