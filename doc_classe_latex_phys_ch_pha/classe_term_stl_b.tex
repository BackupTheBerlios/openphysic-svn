 \classe{Terminale\ \\
Sciences et Technologie de Laboratoire\ \\
Biochimie - G�nie Biologique}{Terminale STL Biochimie}




\chapitre{Courant et tension �lectrique}
\inclure{elec/cours_elec_intro} % cours elec circuit �lectrique

\inclure{elec/tp_i_u} % tp mesure de tension et d'intensit�

\inclure{elec/fiche_methode_montage} % document fiche
    % m�thode montage
    % utilisation d'un multim�tre, ...

\inclure{elec/tp_loi_ohm} % tp loi d'ohm

\inclure{elec/tp_asso_r} % tp associations de R

% association r s�rie (trac� U en fonction de I et montrer qu'on ajoute U)
% association r parall�le (trac� U en fonction de I et montrer qu'on ajoute I)



\chapitre{G�n�rateurs et r�cepteurs}
\inclure{elec/cours_elec_gene_recep} % cours elec g�n�rateurs, r�cepteurs

\inclure{elec/tp_carac_gene} % tp caract�ristique d'un g�n�rateur
\inclure{elec/tp_carac_recep} % tp caract�ristique d'un
                                % r�cepteur (�lectrolyseur)



\chapitre{\'Electrostatique}
\inclure{electrostat/cours_electrostat_intro}



\chapitre{Oxydo-r�duction}
\inclure{oxydo_reduction/tp_piles}



\chapitre{Radioactivit�}
\inclure{radioactivite/tp_crab}  % CRAB
% % Jeu avec des d�s : loi de d�croissance radioactive

% R�actions nucl�aires spontan�es

% R�actions nucl�aires provoqu�es


\inclure{ds_2005_2006_term_stl_b/ds_2005_2006_term_stl_b_electrolyseur_radioac}

\chapitre{La lumi�re}
\inclure{lumiere/cours_intro}


% \chapitre{Spectroscopie}
% %\inclure{
 

% \chapitre{Rayons X}








% \chapitre{Devoir Surveill�} % Term STL B
% \inclure{ds_2005_2006_term_stl_b/ds1}
