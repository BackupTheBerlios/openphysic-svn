\tp[\'Evaluation : Condensateur : charge � courant constant]{\'Evaluation\\ Condensateur : charge � courant constant}



\Section{Partie th�orique}
\Subsection{Question de cours}
\begin{enumerate}
\item Rappelez l'expression de l'�nergie emmagasin�e dans un condensateur.
\item Rappelez l'expression de la capacit� d'un condensateur plan.
\item Rappelez la relation qui lie la charge stock�e dans un
  condensateur � la tension � ses bornes.
\end{enumerate}
\Subsection{D�monstration de cours}
Soit deux condensateurs de capacit�  et .
\begin{enumerate}
\item (Re)d�montrez l'expression de la capacit� �quivalente  de ces deux condensateurs associ�s en parall�le.
\item (Re)d�montrez l'expression de la capacit� �quivalente  de ces
  deux condensateurs associ�s en s�rie.
\end{enumerate}

\Subsection{Charge d'un condensateur � courant constant}
�tudiez le montage donn� en partie pratique et exprimez  en fonction de ,  et . Pr�cisez � quoi servent le interrupteurs  et .


\Section{Partie pratique}
Vous avez � votre disposition deux condensateurs  et  de capacit�
inconnue.

\begin{enumerate}
\item D�terminez exp�rimentalement la capacit� du condensateur 1.
\item D�terminez exp�rimentalement la capacit� du condensateur 2.
\item D�terminez exp�rimentalement la capacit� �quivalente  de
  l'association des deux condensateurs en parall�le. V�rifiez la
  relation entre ,  et .
\end{enumerate}

Vous tracerez donc pour chaque cas  en pr�cisant la tension  que vous
avez choisi ainsi que la r�sistance R.


On rappelle ci-dessous le montage utilis� :



