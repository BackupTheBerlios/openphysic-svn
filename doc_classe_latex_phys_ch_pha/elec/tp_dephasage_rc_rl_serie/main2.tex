\tp{Mesure de d�phasages pour les circuits $R$-$C$ et $R$-$L$ s�rie}

\vressort{1}

\objectifs{
\item Mesurer � l'oscillosocpe le d�phasage entre le courant et la tension.
}

\vressort{3}

\Section{Circuit $R$-$C$}

\begin{enumerate}
\item R�gler, avec le GBF, $u(t)$ : signal sinuso�dal avec $\hat{u} = 4~V$ et $f = 1~kHz$

\item C�bler le montage ci-dessous hors tension, avec les connexions permettant de visualiser, en m�me temps, � l'oscilloscope $u_R(t)$ et $u(t)$ et \emph{faire v�rifier par le professeur} :

% montage

$C = 0,22~\mu F$


$R = 470~\Ohm$


\item Repr�senter $u_R(t)$ et $u(t)$ sur le m�me oscillogramme et expliquer le signe de $\varphi$.

\item Faire varier $f$ de $500~Hz$ � $5~kHz$ et mesurer $\hat{u_R}$, $\hat{u}$, $T$ et $\tau$. En d�duire $Z$ et $\varphi$. Comment �voluent $Z$ et $\varphi$ avec $f$ ? On rappelle que $\displaystyle \abs{\varphi} = 360 \frac{\tau}{T}$ et $\displaystyle Z = \frac{U_{eff}}{I_{eff}}$

\item En prenant $u_R(t)$ comme r�f�rence des phases, �crire les �quations de $u_R(t)$ et $u(t)$ pour $f = 1~kHz$. Repr�senter leurs vecteurs de Fresnel associ�s en prenant l'�chelle $1~cm \equivalent 0,5~V$.

\item En d�duire l'expression de $u_C(t)$.

\end{enumerate}




\vressort{3}




\Section{Circuit $R$-$L$}

Reprendre les m�me questions qu'au I. pour le montage suivant :



Attention pour le 4., on fera varier $f$ de $1~kHz$ � $10~kHz$ et au 6., on demande l'expression de $u_L(t)$. Peut-on consid�rer la bobine comme id�ale ?



% montage


$R = 100~\Ohm$


$L = 10~mH$



