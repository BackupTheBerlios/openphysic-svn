\tp{Mesure de d�phasages pour les circuits $R$-$C$ et $R$-$L$ s�rie}

\objectifs{
\item Mesurer � l'oscillosocpe le d�phasage entre le courant et la tension.
}

\Section{Circuit $R$-$C$}

\begin{enumerate}
\item R�gler, avec le GBF, $u(t)$ : signal sinuso�dal avec $\hat{u} = 4~V$ et $f = 1~kHz$

\item C�bler le montage ci-dessous hors tension, avec les connexions permettant de visualiser, en m�me temps, � l'oscilloscope $u_R(t)$ et $u(t)$ et \emph{faire v�rifier par le professeur} :

% montage

$C = 0,22~\mu F$


$R = 470~\Ohm$


\item Repr�senter $u_R(t)$ et $u(t)$ sur le m�me oscillogramme et expliquer le signe de $\varphi$.

\item z

\item z

\item z

\end{enumerate}




\Section{Circuit $R$-$L$}

\Section{Travail exp�rimental}

On effectuera les mesures pour les 3 cas suivants :

\Subsection{D = $R$ = bo�te de r�sistances $\times~10~\Ohm$}
\begin{enumerate}
\item x
\item y
\item z
\end{enumerate}


\Subsection{D = $C$ = bo�te de condensateurs $\times~1~\mu F$}
\begin{enumerate}
\item x
\item y
\item z
\end{enumerate}


\Subsection{D = $L$ = bobine � noyau variable : \emph{attention on prendra $r = 10~\Ohm$}}
\begin{enumerate}
\item x
\item y
\item z
\end{enumerate}



