\cours{Les r�gimes sinuso�daux}

\section{Visualisation d'une tension alternative sinuso�dale}

\subsection{Visualisation}

\subsection{Caract�ristiques}
\subsubsection{P�riode $T$}
\subsubsection{Fr�quence $f$}
\subsubsection{Valeur maximale $U_{m}$}

\subsection{Tension instantan�e}

\subsection{Relation entre $\omega$ et $T$}

\section{Intensit� instantan�e}

\subsection{Expression de l'intensit� en fonction du temps}

\subsection{D�phasage entre $i$ et $u$}

\section{Tension et intensit� efficaces}
\subsection{D�finition}

\section{Imp�dance d'un dip�le}
\subsection{Imp�dance d'une bobine}

\subsection{Imp�dance d'un dip�le R,L,C}

\subsection{Le ph�nom�ne de r�sonance}

\section{Le transformateur}
\subsection{Principe}

\subsection{Rapport de transformation}

\subsection{Int�r�t des transformateurs}