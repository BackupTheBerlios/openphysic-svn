\tp{Caract�ristiques statiques d'un transistor NPN}

\Section{Montage Emetteur Commun}

\materiel{
\item transistor 2N1613 (ou 2N1711)
\item $R_B = 100~k\Ohm$
\item $R_C = 1~k\Ohm$
}



% montage


\begin{itemize}
\item La jonction base �metteur est polaris�e en direct.
\item La jonction collecteur base est polaris�e en inverse.
\end{itemize}


\Section{Caract�ristique d'entr�e : $I_B = f(V_{BE})$ pour $V_{CE} = cste$}

\begin{enumerate}
\item \`A l'aide de $E_B$, on fait varier $I_B$ de $0$ � environ $110~\mu A$ et on rel�ve les valeurs de $V_{BE}$, $I_B$ et $I_C$ en maintenant $V_{CE} = 5~V$.

\item Tracer la caract�ristique $I_B = f(V_{BE})$ et en d�duite la r�sistance dynamique $\displaystyle r = \frac{\Delta V_{BE}}{\Delta I_B}$

\item Interpr�ter les r�sultats.

\end{enumerate}



\Section{Caract�ristique de transfert de courant : $I_C = f(I_B)$ pour $V_{CE} = cste$}

\begin{enumerate}
\item Tracer la caract�ristique $I_C = f(I_B)$ pour $V_{CE} = 5~V$.

\item Calculer le facteur d'amplification en courant : $\displaystyle \beta = \frac{\Delta I_C}{\Delta I_B}$.

\item Interpr�ter les r�sultats.
\end{enumerate}



\Section{Caract�ristique de sortie : $I_C = f(V_{CE})$ pour $I_B = cste$}
\begin{enumerate}
\item On maintient $I_B$ constant. On fait varier $V_{CE}$ et on rel�ve $I_C$ et $V_{CE}$.

 
\item Faire le tableau des mesures pour $I_B = 20\ 40\ 60\ 80\ 100~\mu A$


% tableau


\item Tracer les caract�ristiques $I_C = f(V_{CE})$ et calculer, dans la partie rectiligne de la premi�re et derni�re courbe, la r�sistance $\displaystyle R = \frac{\Delta V_{CE}}{\Delta I_C}$


\item Interpr�ter les r�sultats.

\end{enumerate}


\Section{Polarisation du transistor : Point de fonctionnement. Droite de charge.}


\Subsection{Montage}

Dans le montage pr�c�dent, dont on vient de d�terminer les caract�ristiques on remplace $R_B$ par une r�sistance fixe de $10~k\Ohm$ et deux boites A.O.I.P. ($\times~1~k\Ohm$ et $\times~10~k\Ohm$)


\Subsection{Mesures : d�termination exp�rimentale du point de fonctionnement}


\Subsection{D�termination graphique du point de fonctionnement}
\begin{enumerate}

\item x

\item x

\item x

\end{enumerate}