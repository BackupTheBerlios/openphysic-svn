\tp{Transistor en commutation}

\Section{Montage}

On veut mesurer $I_B$, $V_{BE}$, $I_C$ et $V_{CE}$. Faire un sch�ma de c�blage.



\montage{
\item transistor 2N1613 (ou 2N1711)
\item $R_C = 1~k\Ohm$
\itme $R_B$ : boite A.O.I.P. ($\times~100~k\Ohm$ et $\times~10k\Ohm$)
}


% montage


On veut polariser le transistor � :
\begin{itemize}
\item $I_C = 5~mA$
\item $V_{CE} = 5~V$
\end{itemize}


Calculer $E_C$.


\Section{Mesures}

On fait varier $R_B$ de $1~M\Ohm$ � $20~k\Ohm$. Relever $I_B$, $V_{BE}$, $I_C$ et $V_{CE}$.


\begin{center}
\emph{NE PAS DESCENDRE EN DESSOUS DE $20~k\Ohm$}
\end{center}


% tableau


\begin{enumerate}
\item Tracer les courbes $I_C = f(I_B)$ et $I_C = f(V_{CE})$.

\item Tracer sur le rep�re $I_C = f(V_{CE})$ la droite de charge th�orique.

\item Interpr�ter les r�sultats :

\begin{enumerate}
\item D�duire de la courbe d'amplification en courant $\beta$.

\item D�terminer, d'apr�s les mesures, la valeur de $I_B$ qui correspond au d�but de la saturation et en d�duire la valeur de $R_B$ correspondante.

\item Pr�ciser le comportement du transistor (en commutation la comparaison avec un interrupteur ferm� ou un interrupteur ouvert est-elle justifi�e).

\item O� est le point de fonctionnement lorsque $R_B$ varie ?
\end{enumerate}


\end{enumerate}




\newpage

\Section{\'Etude � l'oscilloscope}

Faire un sch�ma de c�blage, avec les voies de l'oscilloscope, sachant que l'on veut visualiser $V_e$ et $V_{CE}$.




% Montage



\begin{enumerate}
\item Calculer ${R_B}_{max}$ pour que le transistor soit satur� quand $V_e = {V_e}_{max} = 1~V$, en prenant les valeurs de saturation trouv�es pr�c�demment.

\item R�aliser le montage avec :\\
\begin{itemize}
\item ${V_e}_{max} = 1~V$ ; $f = 1~kHz$
\item $E_C = 10~V$
\item $R_C = 1~k\Ohm$
\end{itemize}

\item Relever, en concordance de temps, les courbes $V_e = f(t)$ et $V_{CE} = f(t)$.

\item Pr�ciser le comportement du transistor et interpr�ter.

\end{enumerate}