\tp{Utilisation d'un transistor NPN}


\Section{R�gulateur de tension par transistor et diode Zener}

\Subsection{Montage}

\materiel{
\item diode Z�ner $V_Z = 7,5~V$ (${I_Z}_{min} = 5~mA$ ; ${I_Z}_{max} = 60~mA$)
\item transistor 2N1613 ou 2N1711 (${I_C}_{max} = 150~mA$  ; $P_{max} = 800~mW$)
}



\Subsection{Pr�paration}

\begin{enumerate}
\item D�terminer l'expression de $V_s$ en fonction de $V_Z$ et $V_{BE}$.\\
Montrer que ce montage fournit une tension $V_S$ stabilis�e lorsque $E$ varie. Pr�ciser � quelles conditions.

\item Entre quelles limites peut-on faire varier $E$ si $R_B = 100~\Ohm$ ?

\item Quelle puissant faut-il choisir pour la r�sistance $R_B$ ?\\
Quelle valeur minimum doit-on attribuer � la r�sistance $R$ ?


\end{enumerate}

\Subsection{Mesures}
\begin{enumerate}
\item Avec $R = 220~\Ohm$ et $R_B = 100~\Ohm$, r�aliser le montage permettant de mesurer $E$, $V_S$, $I_S$ et $I_Z$.

\item On maintient $R$ constant. On fait varier $E$.\\
Relever et tracer $V_S = f(E)$ et $I_Z = f(E)$ sans d�passer les valeurs permises pour $I_Z$ et $I_S$. Noter la valeur de $V_{BE}$.



% tableau


\item Justifier  l'allure des courbes. Pr�ciser le coportement de la diode Z�ner.

\end{enumerate}


\newpage


\Section{R�alisation et �tude d'un g�n�rateur de courant}

\Subsection{Montage}

\materiel{
\item transistor 2N1613 (ou 2N1711)
\item Diode Z�ner $V_Z = 7,5~V$
\item $R_1 = 1~k\Ohm$
\item $R_C$ : r�sistance de charge (boites A.O.I.P. $\times~10~k\Ohm$ et $\times~1~k\Ohm$
\item $R_E = 6,8~k\Ohm$ (ou potentiom�tre de $10~k\Ohm$)
}


% montage




\Subsection{Pr�paration}

\begin{enumerate}
\item Exprimer la tension $V_{EM}$ en fonction de $V_Z$ et de $V_{BE}$. Calculer $V_{EM}$.

\item Montrer que $I_E$ ne d�pend que de $R_E$. Pr�ciser � quelles conditions ?

\item D�terminer $I_C$ et les limites de $R_C$ pour que le montage d�livre un courant $I_C$ constant si $R_E = 6,8~k\Ohm$, $C_{CC} = 20~V$ et ${V_{CE}}_SAT = 0,4~V$.
\end{enumerate}


\Subsection{Mesures}

\begin{enumerate}
\item Faire un sch�ma de c�blage sachant que l'on veut mesurer $I_C$.\\
R�aliser le montage en prenant $V_{CC} = 20~V$, $R_E = 6,8~k\Ohm$ et $R_C$ r�sistance variable.

\item Mesurer $I_C$ pour $R_C = 0~\Ohm$ (si n�cessaire remplacer $R_E$ par le potentiom�tre et r�gler $R_E$ pour obtenir $I_C \approx 1~mA$).\\
Noter $V_{BE}$ et $V_Z$ et comparer avec les valeurs utilis�es au B.1.

\item Faire varier $R_C$ de $0$ � $20~k\Ohm$ et relever $I_C$. Tracer $I_C = f(R_C)$.

\item Interpr�ter l'allure de la courbe obtenue et v�rifier les valeurs trouv�es au B.3.

\end{enumerate}