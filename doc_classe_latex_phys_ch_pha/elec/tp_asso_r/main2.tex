\tp[Association de conducteurs ohmiques]{Association de\\
conducteurs ohmiques}

\nomprenomclasse


\objectifs{
\item V�rifier les relations d'association de conducteurs ohmiques en s�rie et en parall�le.
}

%\section*{Mat�riel}
%2 conducteurs ohmique de r�sistance inconnue

\section{Association s�rie}
On r�alise l'association en s�rie de deux conducteurs ohmiques $R_1$ et $R_2$.\\
On alimente cette association par un g�n�rateur de tension r�glable $U$.\\
On mesure la tension $U_1$ aux bornes de $R_1$, $U_2$ aux bornes de $R_2$ ainsi que la tension $U$ aux bornes du g�n�rateur et le courant $I$ dans le circuit.

\begin{enumerate}

\item Faire le sch�ma du montage en faisant appara�tre les tensions et les courants avec les conventions habituelles, ainsi qu'en pr�cisant les bornes des appareils de mesure.

\cadre{4cm}

\item R�aliser le montage et le montrer au professeur avant mise sous tension. Mesurer les grandeurs indiqu�es plus haut.


\begin{arraydata}{11}
\hline
\rule[-0.4cm]{0cm}{1cm}
$U$ ($V$)       & & & & & & & & & & & \\ \hline
\rule[-0.4cm]{0cm}{1cm}
$I$ ($A$)       & & & & & & & & & & & \\ \hline
\rule[-0.4cm]{0cm}{1cm}
$U_1$ ($V$)     & & & & & & & & & & & \\ \hline
\rule[-0.4cm]{0cm}{1cm}
$U_2$ ($V$)     & & & & & & & & & & & \\ \hline
\hline
\rule[-0.4cm]{0cm}{1cm}
$U_1+U_2$       & & & & & & & & & & & \\ \hline
%$\frac{U_1}{I}$ & & & \\ \hline
%$\frac{U_2}{I}$ & & & \\ \hline
%$\frac{U}{I}$   & & & \\ \hline
\end{arraydata}


\item Calculer $U_1+U_2$. Comparer � $U$. Conclure.
%\item Calculer $\frac{U_1}{I}$, $\frac{U_2}{I}$ et $\frac{U}{I}$.
\item Tracer $U_1$, $U_2$, $U$ et $U_1 + U_2$ en fonction de $I$.
\item Donner la valeur de :
 \begin{itemize}
 \item $R_1 = ............$ 
 \item $R_2 = ............$
 \item $R = ............$
 \end{itemize}
%\item Quelle est la valeur de la r�sistance �quivalente � l'association en s�rie des deux conducteurs ohmiques ?
\item Calculer $R_1 + R_2$. Conclure.
\end{enumerate}


\newpage

\section{Association parall�le}
On r�alise l'association en parall�le de deux conducteurs ohmiques $R_1$ et $R_2$.\\
On alimente cette association par un g�n�rateur de tension r�glable $U$.\\
On mesure le courant $I_1$ traversant le conducteur ohmique de r�sistance $R_1$, $I_2$ celui traversant $R_2$ ainsi que la tension $U$ aux bornes du g�n�rateur et le courant $I$ qu'il d�bite.

\begin{enumerate}

\item Faire le sch�ma du montage en faisant appara�tre les tensions et les courants avec les conventions habituelles, ainsi qu'en pr�cisant les bornes des appareils de mesure.

\cadre{4cm}

\item R�aliser le montage et le montrer au professeur avant mise sous tension. Mesurer les grandeurs indiqu�es plus haut.


\begin{arraydata}{11}
\hline
\rule[-0.4cm]{0cm}{1cm}
$U$ ($V$)       & & & & & & & & & & & \\ \hline
\rule[-0.4cm]{0cm}{1cm}
$I$ ($A$)       & & & & & & & & & & & \\ \hline
\rule[-0.4cm]{0cm}{1cm}
$I_1$ ($A$)     & & & & & & & & & & & \\ \hline
\rule[-0.4cm]{0cm}{1cm}
$I_2$ ($A$)     & & & & & & & & & & & \\ \hline
\hline
\rule[-0.4cm]{0cm}{1cm}
$I_1+I_2$       & & & & & & & & & & & \\ \hline
%$\frac{U}{I_1}$ & & & \\ \hline
%$\frac{U}{I_2}$ & & & \\ \hline
%$\frac{U}{I}$   & & & \\ \hline
\end{arraydata}


\item Calculer $I_1+I_2$, comparer � I. Conclure.

\item Tracer $U$ en fonction de $I$, $U$ en fonction de $I_1$, $U$ en fonction de $I_2$ et $U$ en fonction de $I_1 + I_2$.

%\item Calculer $\frac{U}{I_1}$, $\frac{U}{I_2}$ et $\frac{U}{I}$.
\item Donner la valeur de :
 \begin{itemize}
 \item $R_1 = ............$ 
 \item $R_2 = ............$
 \item $R = ............$
 \end{itemize}
%\item Quelle est la valeur de la r�sistance �quivalente � l'association en parall�le des deux conducteurs ohmiques ?
\item Calculer $\frac{1}{R_1} + \frac{1}{R_2}$. Conclure.
\end{enumerate}


\section*{R�sum� : association s�rie ou parall�le de conducteurs ohmiques}
%\begin{tabularx}{\linewidth}{|c|c|}
%\hline
%Association s�rie & Association parall�le \\ \hline
%\end{tabularx}

\cadre{4cm}
