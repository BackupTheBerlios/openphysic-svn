\classe[Premi�re Scientifique - Partie Physique]{Premi�re\ \\
Scientifique\ \\
Partie Physique}



\chapitre{Int�ractions fondamentales}
\inclure{interactions/cours_particules_elementaires}    % Mettre les fig
\inclure{interactions/cours_interactions_fondamentales} % Mettre les fig

\inclure{electrostat/doc_electrostat}



\chapitre{Vitesses et mouvements}
\inclure{meca/cours_cinematique_intro} % Mettre les fig
\inclure{tp_prem_s_phys/tp02_mouvement}
\inclure{tp_prem_s_phys/tp02_mouvement_doc}
\inclure{tp_prem_s_phys/tp03_rotation}
\inclure{tp_prem_s_phys/tp04_mvt_parabolique}


\chapitre{Forces}
\inclure{meca/cours_forces} % Mettre les fig
\inclure{meca/tp_poids_ressort_archimede}
\inclure{meca/tp_equilibre_solide}
\inclure{meca/doc_rapporteurs}


\chapitre{Lois de Newton}
\inclure{meca/cours_lois_newton}  % Mettre les fig
\inclure{tp_prem_s_phys/tp08_deuxieme_loi_newton_micromega}
\inclure{tp_prem_s_phys/tp08_deuxieme_loi_newton_parabole_video}


\chapitre{Travail d'une force}
\inclure{meca/cours_travail_force}  % Mettre les fig


\chapitre{\'Energie cin�tique}
\inclure{meca/cours_energie_cinetique}  % Mettre les fig
\inclure{tp_prem_s_phys/tp09_theo_energie_cinetique}


\chapitre{\'Energie m�canique}
\inclure{meca/cours_energie_potentielle_mecanique}  % Mettre les fig
\inclure{tp_prem_s_phys/tp10_theo_energie_mecanique}


\chapitre{Transferts thermiques} %Calorim�trie}
\inclure{thermodynamique/cours_transferts_energie}
\inclure{thermodynamique/tp1_transferts_thermiques_capa_therm_calo}
\inclure{thermodynamique/tp2_capa_thermique_massique_metal}
\inclure{thermodynamique/tp3_chaleur_latente_fusion_glace}
\inclure{thermodynamique/doc_balance_metler_p1200}


\chapitre{\'Electricit�}
% Cours : Rappels de coll�ge
\inclure{elec/cours_rappel_college}

\inclure{elec/fiche_methode_montage} % document fiche
    % m�thode montage
    % utilisation d'un multim�tre, ...


% TP Potentiel le long d'un circuit ?


\inclure{elec/cours_elec_gene_recep} % cours elec g�n�rateurs, r�cepteurs
\inclure{elec/tp_carac_gene} % tp caract�ristique d'un g�n�rateur
\inclure{elec/tp_carac_recep} % tp caract�ristique d'un
                                % r�cepteur (�lectrolyseur)

% Cours : Circuits �lectriques




\chapitre{Optique g�om�trique}
\inclure{opt/cours_intro_opt_geom} % conditions de visibilit�
\inclure{opt/cours_miroir_plan} % miroir plan
\inclure{opt/cours_lentilles_minces} % lentilles minces
\inclure{opt/doc_constructions_lentilles}
\inclure{opt/tp_lentilles_minces_conv_conjug} % Lentilles (Images, rel de conjugaison)
\inclure{opt/tp_lentilles_minces_conv_foco} % Focom�trie lentilles minces
% lunette astronomique
% autres instruments d'optique ?




\chapitre{\'Electromagn�tisme}
\inclure{electromag/tp_champ_magnetique} % Champ magn�tique
\inclure{electromag/tp_bobines_helmholtz} % Bobines de Helmholtz
\inclure{electromag/tp_force_laplace} % Force de Laplace
\inclure{electromag/tp_force_lorentz} % Force de Lorentz





\chapitre{Devoir Surveill�} % 1S
\inclure{ds_2005_2006_prem_s/ds1}
\inclure{ds_2005_2006_prem_s/ds2}


% � ajouter si le dernier document contient un nb impair de pages
%\newpage