\classe{Premi�re\ \\
Scientifique\ \\
Partie Physique}{Premi�re Scientifique - Partie Physique}



\chapitre{Int�ractions fondamentales}
\ds{Devoir Surveill�}{
%
}

\nomprenomclasse

\setcounter{numexercice}{0}

%\renewcommand{\tabularx}[1]{>{\centering}m{#1}} 

%\newcommand{\tabularxc}[1]{\tabularx{>{\centering}m{#1}}}

\vressort{3}

\begin{exercice}{Connaissance sur l'atome}%\\
\begin{enumerate}
\item De quoi est compos� un atome ?
\item Que signifie les lettres $A$, $Z$ et $X$ dans la repr�sentation \noyau{X}{Z}{A} ?
\item Comment trouve-t-on le nombre de neutrons d'un atome de l'�l�ment pr�c�dent.
\item Si un atome a $5$ protons, combien-a-t-il d'�lectrons ? Pourquoi ?
\item Qu'est-ce qui caract�rise un �l�ment chimique ?
\item Qu'est-ce qu'un isotope ?
\end{enumerate}
\end{exercice}



\vressort{3}



\begin{exercice}{Composition des atomes}\\
En vous aidant du tableau p�riodique des �l�ments,
compl�ter le tableau suivant :

\medskip

\noindent
%\begin{tabularx}{\textwidth}{|>{\centering}X|>{\centering}X|>{\centering}X|>{\centering}X|>{\centering}X|}
% \begin{tabularx}{\linewidth}{|X|X|X|X|X|}
% \hline
% \emph{nom}       & \emph{symbole}  & \emph{protons} & \emph{neutrons}
% & \emph{nucl�ons} \tbnl
% carbone   & \noyau{C}{6}{14}   &         &          & \rule[-0.5cm]{0cm}{1cm}         \tbnl
% fluor     & \noyau{F}{9}{19}   &         &          & \rule[-0.5cm]{0cm}{1cm}         \tbnl
% sodium    & \noyau{Na}{11}{23} &         &          & \rule[-0.5cm]{0cm}{1cm}         \tbnl
% oxyg�ne   & \noyau{O}{8}{16}   &         &          & \rule[-0.5cm]{0cm}{1cm}         \tbnl
% hydrog�ne &          &         & 0        & \rule[-0.5cm]{0cm}{1cm}         \tbnl
%           & \noyau{Cl}{17}{35} &         &          &  \rule[-0.5cm]{0cm}{1cm}        \tbnl
%           &          & 8       & \rule[-0.5cm]{0cm}{1cm}         & 16       \tbnl
% \end{tabularx}



\begin{tabularx}{\linewidth}{|>{\mystrut}X|X|X|X|X|}
\hline
% multicolumn pour faire dispara�tre le \mystrut
\multicolumn{1}{|X|}{\emph{nom}} & \emph{symbole}  &
\emph{protons} & \emph{neutrons} & \emph{nucl�ons} \tbnl
carbone   & \noyau{C}{6}{14}   &   &   &    \tbnl
fluor     & \noyau{F}{9}{19}   &   &   &    \tbnl
sodium    & \noyau{Na}{11}{23} &   &   &    \tbnl
oxyg�ne   & \noyau{O}{8}{16}   &   &   &    \tbnl
hydrog�ne &                    &   & 0 &    \tbnl
          & \noyau{Cl}{17}{35} &   &   &    \tbnl
          &                    & 8 &   & 16 \tbnl
\end{tabularx}


\end{exercice}


\vressort{3}


\begin{exercice}{Masse d'un atome de carbone 12}\\
Soit le carbone $12$ not� \noyau{C}{6}{12}.
\begin{enumerate}
\item L'�l�ment carbone peut-il avoir $5$ protons ? Pourquoi ?
\item Calculer la masse du noyau d'un atome de carbone $12$
sachant que la masse d'un nucl�on est $m_n = 1,67.10^{-27}~kg$
\item Calculer la masse des �lectrons de l'atome de carbone 12
sachant que la masse d'un �lectron vaut $m_e = 9,1.10^{-31}~kg$
\item Comparer la masse des �lectrons de l'atome � la masse du noyau.
Que concluez-vous ?
\item En d�duire, sans nouveau calcul, la masse de l'atome de carbone
  $12$.
\end{enumerate}
\end{exercice}

\newpage

\vressort{1}

\begin{exercice}{Couches �lectroniques}\\
Dans l'�tat le plus stable de l'atome, appel� �tat fondamental,
les �lectrons occupent successivement les couches,
en commen�ant par celles qui sont les plus proches du noyau : 
d'abord $K$ puis $L$ puis $M$.

Lorsqu'une couche est pleine, ou encore satur�e, on passe � la suivante.

La derni�re couche occup�e est appel�e couche externe.\\
Toutes les autres sont appel�es couches internes.

\medskip

\noindent
%\begin{tabularx}{\textwidth}{|>{\centering}X|>{\centering}X|>{\centering}X|>{\centering}X|}
\begin{tabularx}{\textwidth}{|>{\mystrut}X|X|X|X|}
\hline
\multicolumn{1}{|X|}{\emph{Symbole de la couche}}       & $K$ & $L$ & $M$  \tbnl
\emph{Nombre maximal d'�lectrons} & $2$ & $8$ & $18$ \tbnl
\end{tabularx}

\medskip

Ainsi, par exemple, l'atome de chlore ($Z=17$) a la configuration �lectronique :
$(K)^2(L)^8(M)^{7}$.

\begin{enumerate}
\item Indiquez le nombre d'�lectrons et donnez la configuration des atomes suivants :
  \begin{enumerate}
  \item \noyau{H}{1}{}
  \item \noyau{O}{8}{}
  \item \noyau{C}{6}{}
  \item \noyau{Ne}{10}{}
  \end{enumerate}
\item Parmi les ions ci-desssous, pr�cisez s'il s'agit d'anions ou de cations.
Indiquez le nombre d'�lectrons et donnez la configuration �lectronique
des ions suivants :
  \begin{enumerate}
  \item $Be^{2+}$ ($Z=4$)
  \item $Al^{3+}$ ($Z=13$)
  \item $O^{2-}$ ($Z=8$)
  \item $F^{-}$ ($Z=9$)
  \end{enumerate}
\end{enumerate}

\end{exercice}


\vressort{5}


\begin{exercice}{Charge d'un atome de Zinc}%\\
\begin{enumerate}
\item Combien de protons l'atome de zinc \noyau{Zn}{30}{65} contient-il ?
\item Combien d'�lectrons comporte-t-il ?
\item Calculer la charge totale des protons
sachant qu'un proton a pour charge $e = 1,6.10^{-19}~C$.
\item Calculer la charge totale des �lectrons
sachant qu'un �lectron a pour charge $-e = -1,6.10^{-19}~C$.
\item En d�duire la charge de l'atome de Zinc.
\item Ce r�sultat est-il identique pour tous les atomes ?
\item A l'issue d'une r�action dite d'oxydation, un atome de zinc $Zn$
  se transforme en un ion $Zn^{2+}$.
  \begin{enumerate}
  \item Donnez l'�quation de cette r�action
  (en faisant intervenir un ou plusieurs �lectrons not�s $e^-$).
  \item Indiquez la charge (en coulomb $C$) de cet ion.
  \end{enumerate}
\end{enumerate}
\end{exercice}

\vressort{3}    % Mettre les fig
\ds{Devoir Surveill�}{
%
}

\nomprenomclasse

\setcounter{numexercice}{0}

%\renewcommand{\tabularx}[1]{>{\centering}m{#1}} 

%\newcommand{\tabularxc}[1]{\tabularx{>{\centering}m{#1}}}

\vressort{3}

\begin{exercice}{Connaissance sur l'atome}%\\
\begin{enumerate}
\item De quoi est compos� un atome ?
\item Que signifie les lettres $A$, $Z$ et $X$ dans la repr�sentation \noyau{X}{Z}{A} ?
\item Comment trouve-t-on le nombre de neutrons d'un atome de l'�l�ment pr�c�dent.
\item Si un atome a $5$ protons, combien-a-t-il d'�lectrons ? Pourquoi ?
\item Qu'est-ce qui caract�rise un �l�ment chimique ?
\item Qu'est-ce qu'un isotope ?
\end{enumerate}
\end{exercice}



\vressort{3}



\begin{exercice}{Composition des atomes}\\
En vous aidant du tableau p�riodique des �l�ments,
compl�ter le tableau suivant :

\medskip

\noindent
%\begin{tabularx}{\textwidth}{|>{\centering}X|>{\centering}X|>{\centering}X|>{\centering}X|>{\centering}X|}
% \begin{tabularx}{\linewidth}{|X|X|X|X|X|}
% \hline
% \emph{nom}       & \emph{symbole}  & \emph{protons} & \emph{neutrons}
% & \emph{nucl�ons} \tbnl
% carbone   & \noyau{C}{6}{14}   &         &          & \rule[-0.5cm]{0cm}{1cm}         \tbnl
% fluor     & \noyau{F}{9}{19}   &         &          & \rule[-0.5cm]{0cm}{1cm}         \tbnl
% sodium    & \noyau{Na}{11}{23} &         &          & \rule[-0.5cm]{0cm}{1cm}         \tbnl
% oxyg�ne   & \noyau{O}{8}{16}   &         &          & \rule[-0.5cm]{0cm}{1cm}         \tbnl
% hydrog�ne &          &         & 0        & \rule[-0.5cm]{0cm}{1cm}         \tbnl
%           & \noyau{Cl}{17}{35} &         &          &  \rule[-0.5cm]{0cm}{1cm}        \tbnl
%           &          & 8       & \rule[-0.5cm]{0cm}{1cm}         & 16       \tbnl
% \end{tabularx}



\begin{tabularx}{\linewidth}{|>{\mystrut}X|X|X|X|X|}
\hline
% multicolumn pour faire dispara�tre le \mystrut
\multicolumn{1}{|X|}{\emph{nom}} & \emph{symbole}  &
\emph{protons} & \emph{neutrons} & \emph{nucl�ons} \tbnl
carbone   & \noyau{C}{6}{14}   &   &   &    \tbnl
fluor     & \noyau{F}{9}{19}   &   &   &    \tbnl
sodium    & \noyau{Na}{11}{23} &   &   &    \tbnl
oxyg�ne   & \noyau{O}{8}{16}   &   &   &    \tbnl
hydrog�ne &                    &   & 0 &    \tbnl
          & \noyau{Cl}{17}{35} &   &   &    \tbnl
          &                    & 8 &   & 16 \tbnl
\end{tabularx}


\end{exercice}


\vressort{3}


\begin{exercice}{Masse d'un atome de carbone 12}\\
Soit le carbone $12$ not� \noyau{C}{6}{12}.
\begin{enumerate}
\item L'�l�ment carbone peut-il avoir $5$ protons ? Pourquoi ?
\item Calculer la masse du noyau d'un atome de carbone $12$
sachant que la masse d'un nucl�on est $m_n = 1,67.10^{-27}~kg$
\item Calculer la masse des �lectrons de l'atome de carbone 12
sachant que la masse d'un �lectron vaut $m_e = 9,1.10^{-31}~kg$
\item Comparer la masse des �lectrons de l'atome � la masse du noyau.
Que concluez-vous ?
\item En d�duire, sans nouveau calcul, la masse de l'atome de carbone
  $12$.
\end{enumerate}
\end{exercice}

\newpage

\vressort{1}

\begin{exercice}{Couches �lectroniques}\\
Dans l'�tat le plus stable de l'atome, appel� �tat fondamental,
les �lectrons occupent successivement les couches,
en commen�ant par celles qui sont les plus proches du noyau : 
d'abord $K$ puis $L$ puis $M$.

Lorsqu'une couche est pleine, ou encore satur�e, on passe � la suivante.

La derni�re couche occup�e est appel�e couche externe.\\
Toutes les autres sont appel�es couches internes.

\medskip

\noindent
%\begin{tabularx}{\textwidth}{|>{\centering}X|>{\centering}X|>{\centering}X|>{\centering}X|}
\begin{tabularx}{\textwidth}{|>{\mystrut}X|X|X|X|}
\hline
\multicolumn{1}{|X|}{\emph{Symbole de la couche}}       & $K$ & $L$ & $M$  \tbnl
\emph{Nombre maximal d'�lectrons} & $2$ & $8$ & $18$ \tbnl
\end{tabularx}

\medskip

Ainsi, par exemple, l'atome de chlore ($Z=17$) a la configuration �lectronique :
$(K)^2(L)^8(M)^{7}$.

\begin{enumerate}
\item Indiquez le nombre d'�lectrons et donnez la configuration des atomes suivants :
  \begin{enumerate}
  \item \noyau{H}{1}{}
  \item \noyau{O}{8}{}
  \item \noyau{C}{6}{}
  \item \noyau{Ne}{10}{}
  \end{enumerate}
\item Parmi les ions ci-desssous, pr�cisez s'il s'agit d'anions ou de cations.
Indiquez le nombre d'�lectrons et donnez la configuration �lectronique
des ions suivants :
  \begin{enumerate}
  \item $Be^{2+}$ ($Z=4$)
  \item $Al^{3+}$ ($Z=13$)
  \item $O^{2-}$ ($Z=8$)
  \item $F^{-}$ ($Z=9$)
  \end{enumerate}
\end{enumerate}

\end{exercice}


\vressort{5}


\begin{exercice}{Charge d'un atome de Zinc}%\\
\begin{enumerate}
\item Combien de protons l'atome de zinc \noyau{Zn}{30}{65} contient-il ?
\item Combien d'�lectrons comporte-t-il ?
\item Calculer la charge totale des protons
sachant qu'un proton a pour charge $e = 1,6.10^{-19}~C$.
\item Calculer la charge totale des �lectrons
sachant qu'un �lectron a pour charge $-e = -1,6.10^{-19}~C$.
\item En d�duire la charge de l'atome de Zinc.
\item Ce r�sultat est-il identique pour tous les atomes ?
\item A l'issue d'une r�action dite d'oxydation, un atome de zinc $Zn$
  se transforme en un ion $Zn^{2+}$.
  \begin{enumerate}
  \item Donnez l'�quation de cette r�action
  (en faisant intervenir un ou plusieurs �lectrons not�s $e^-$).
  \item Indiquez la charge (en coulomb $C$) de cet ion.
  \end{enumerate}
\end{enumerate}
\end{exercice}

\vressort{3} % Mettre les fig

\ds{Devoir Surveill�}{
%
}

\nomprenomclasse

\setcounter{numexercice}{0}

%\renewcommand{\tabularx}[1]{>{\centering}m{#1}} 

%\newcommand{\tabularxc}[1]{\tabularx{>{\centering}m{#1}}}

\vressort{3}

\begin{exercice}{Connaissance sur l'atome}%\\
\begin{enumerate}
\item De quoi est compos� un atome ?
\item Que signifie les lettres $A$, $Z$ et $X$ dans la repr�sentation \noyau{X}{Z}{A} ?
\item Comment trouve-t-on le nombre de neutrons d'un atome de l'�l�ment pr�c�dent.
\item Si un atome a $5$ protons, combien-a-t-il d'�lectrons ? Pourquoi ?
\item Qu'est-ce qui caract�rise un �l�ment chimique ?
\item Qu'est-ce qu'un isotope ?
\end{enumerate}
\end{exercice}



\vressort{3}



\begin{exercice}{Composition des atomes}\\
En vous aidant du tableau p�riodique des �l�ments,
compl�ter le tableau suivant :

\medskip

\noindent
%\begin{tabularx}{\textwidth}{|>{\centering}X|>{\centering}X|>{\centering}X|>{\centering}X|>{\centering}X|}
% \begin{tabularx}{\linewidth}{|X|X|X|X|X|}
% \hline
% \emph{nom}       & \emph{symbole}  & \emph{protons} & \emph{neutrons}
% & \emph{nucl�ons} \tbnl
% carbone   & \noyau{C}{6}{14}   &         &          & \rule[-0.5cm]{0cm}{1cm}         \tbnl
% fluor     & \noyau{F}{9}{19}   &         &          & \rule[-0.5cm]{0cm}{1cm}         \tbnl
% sodium    & \noyau{Na}{11}{23} &         &          & \rule[-0.5cm]{0cm}{1cm}         \tbnl
% oxyg�ne   & \noyau{O}{8}{16}   &         &          & \rule[-0.5cm]{0cm}{1cm}         \tbnl
% hydrog�ne &          &         & 0        & \rule[-0.5cm]{0cm}{1cm}         \tbnl
%           & \noyau{Cl}{17}{35} &         &          &  \rule[-0.5cm]{0cm}{1cm}        \tbnl
%           &          & 8       & \rule[-0.5cm]{0cm}{1cm}         & 16       \tbnl
% \end{tabularx}



\begin{tabularx}{\linewidth}{|>{\mystrut}X|X|X|X|X|}
\hline
% multicolumn pour faire dispara�tre le \mystrut
\multicolumn{1}{|X|}{\emph{nom}} & \emph{symbole}  &
\emph{protons} & \emph{neutrons} & \emph{nucl�ons} \tbnl
carbone   & \noyau{C}{6}{14}   &   &   &    \tbnl
fluor     & \noyau{F}{9}{19}   &   &   &    \tbnl
sodium    & \noyau{Na}{11}{23} &   &   &    \tbnl
oxyg�ne   & \noyau{O}{8}{16}   &   &   &    \tbnl
hydrog�ne &                    &   & 0 &    \tbnl
          & \noyau{Cl}{17}{35} &   &   &    \tbnl
          &                    & 8 &   & 16 \tbnl
\end{tabularx}


\end{exercice}


\vressort{3}


\begin{exercice}{Masse d'un atome de carbone 12}\\
Soit le carbone $12$ not� \noyau{C}{6}{12}.
\begin{enumerate}
\item L'�l�ment carbone peut-il avoir $5$ protons ? Pourquoi ?
\item Calculer la masse du noyau d'un atome de carbone $12$
sachant que la masse d'un nucl�on est $m_n = 1,67.10^{-27}~kg$
\item Calculer la masse des �lectrons de l'atome de carbone 12
sachant que la masse d'un �lectron vaut $m_e = 9,1.10^{-31}~kg$
\item Comparer la masse des �lectrons de l'atome � la masse du noyau.
Que concluez-vous ?
\item En d�duire, sans nouveau calcul, la masse de l'atome de carbone
  $12$.
\end{enumerate}
\end{exercice}

\newpage

\vressort{1}

\begin{exercice}{Couches �lectroniques}\\
Dans l'�tat le plus stable de l'atome, appel� �tat fondamental,
les �lectrons occupent successivement les couches,
en commen�ant par celles qui sont les plus proches du noyau : 
d'abord $K$ puis $L$ puis $M$.

Lorsqu'une couche est pleine, ou encore satur�e, on passe � la suivante.

La derni�re couche occup�e est appel�e couche externe.\\
Toutes les autres sont appel�es couches internes.

\medskip

\noindent
%\begin{tabularx}{\textwidth}{|>{\centering}X|>{\centering}X|>{\centering}X|>{\centering}X|}
\begin{tabularx}{\textwidth}{|>{\mystrut}X|X|X|X|}
\hline
\multicolumn{1}{|X|}{\emph{Symbole de la couche}}       & $K$ & $L$ & $M$  \tbnl
\emph{Nombre maximal d'�lectrons} & $2$ & $8$ & $18$ \tbnl
\end{tabularx}

\medskip

Ainsi, par exemple, l'atome de chlore ($Z=17$) a la configuration �lectronique :
$(K)^2(L)^8(M)^{7}$.

\begin{enumerate}
\item Indiquez le nombre d'�lectrons et donnez la configuration des atomes suivants :
  \begin{enumerate}
  \item \noyau{H}{1}{}
  \item \noyau{O}{8}{}
  \item \noyau{C}{6}{}
  \item \noyau{Ne}{10}{}
  \end{enumerate}
\item Parmi les ions ci-desssous, pr�cisez s'il s'agit d'anions ou de cations.
Indiquez le nombre d'�lectrons et donnez la configuration �lectronique
des ions suivants :
  \begin{enumerate}
  \item $Be^{2+}$ ($Z=4$)
  \item $Al^{3+}$ ($Z=13$)
  \item $O^{2-}$ ($Z=8$)
  \item $F^{-}$ ($Z=9$)
  \end{enumerate}
\end{enumerate}

\end{exercice}


\vressort{5}


\begin{exercice}{Charge d'un atome de Zinc}%\\
\begin{enumerate}
\item Combien de protons l'atome de zinc \noyau{Zn}{30}{65} contient-il ?
\item Combien d'�lectrons comporte-t-il ?
\item Calculer la charge totale des protons
sachant qu'un proton a pour charge $e = 1,6.10^{-19}~C$.
\item Calculer la charge totale des �lectrons
sachant qu'un �lectron a pour charge $-e = -1,6.10^{-19}~C$.
\item En d�duire la charge de l'atome de Zinc.
\item Ce r�sultat est-il identique pour tous les atomes ?
\item A l'issue d'une r�action dite d'oxydation, un atome de zinc $Zn$
  se transforme en un ion $Zn^{2+}$.
  \begin{enumerate}
  \item Donnez l'�quation de cette r�action
  (en faisant intervenir un ou plusieurs �lectrons not�s $e^-$).
  \item Indiquez la charge (en coulomb $C$) de cet ion.
  \end{enumerate}
\end{enumerate}
\end{exercice}

\vressort{3}



\chapitre{Vitesses et mouvements}
\ds{Devoir Surveill�}{
%
}

\nomprenomclasse

\setcounter{numexercice}{0}

%\renewcommand{\tabularx}[1]{>{\centering}m{#1}} 

%\newcommand{\tabularxc}[1]{\tabularx{>{\centering}m{#1}}}

\vressort{3}

\begin{exercice}{Connaissance sur l'atome}%\\
\begin{enumerate}
\item De quoi est compos� un atome ?
\item Que signifie les lettres $A$, $Z$ et $X$ dans la repr�sentation \noyau{X}{Z}{A} ?
\item Comment trouve-t-on le nombre de neutrons d'un atome de l'�l�ment pr�c�dent.
\item Si un atome a $5$ protons, combien-a-t-il d'�lectrons ? Pourquoi ?
\item Qu'est-ce qui caract�rise un �l�ment chimique ?
\item Qu'est-ce qu'un isotope ?
\end{enumerate}
\end{exercice}



\vressort{3}



\begin{exercice}{Composition des atomes}\\
En vous aidant du tableau p�riodique des �l�ments,
compl�ter le tableau suivant :

\medskip

\noindent
%\begin{tabularx}{\textwidth}{|>{\centering}X|>{\centering}X|>{\centering}X|>{\centering}X|>{\centering}X|}
% \begin{tabularx}{\linewidth}{|X|X|X|X|X|}
% \hline
% \emph{nom}       & \emph{symbole}  & \emph{protons} & \emph{neutrons}
% & \emph{nucl�ons} \tbnl
% carbone   & \noyau{C}{6}{14}   &         &          & \rule[-0.5cm]{0cm}{1cm}         \tbnl
% fluor     & \noyau{F}{9}{19}   &         &          & \rule[-0.5cm]{0cm}{1cm}         \tbnl
% sodium    & \noyau{Na}{11}{23} &         &          & \rule[-0.5cm]{0cm}{1cm}         \tbnl
% oxyg�ne   & \noyau{O}{8}{16}   &         &          & \rule[-0.5cm]{0cm}{1cm}         \tbnl
% hydrog�ne &          &         & 0        & \rule[-0.5cm]{0cm}{1cm}         \tbnl
%           & \noyau{Cl}{17}{35} &         &          &  \rule[-0.5cm]{0cm}{1cm}        \tbnl
%           &          & 8       & \rule[-0.5cm]{0cm}{1cm}         & 16       \tbnl
% \end{tabularx}



\begin{tabularx}{\linewidth}{|>{\mystrut}X|X|X|X|X|}
\hline
% multicolumn pour faire dispara�tre le \mystrut
\multicolumn{1}{|X|}{\emph{nom}} & \emph{symbole}  &
\emph{protons} & \emph{neutrons} & \emph{nucl�ons} \tbnl
carbone   & \noyau{C}{6}{14}   &   &   &    \tbnl
fluor     & \noyau{F}{9}{19}   &   &   &    \tbnl
sodium    & \noyau{Na}{11}{23} &   &   &    \tbnl
oxyg�ne   & \noyau{O}{8}{16}   &   &   &    \tbnl
hydrog�ne &                    &   & 0 &    \tbnl
          & \noyau{Cl}{17}{35} &   &   &    \tbnl
          &                    & 8 &   & 16 \tbnl
\end{tabularx}


\end{exercice}


\vressort{3}


\begin{exercice}{Masse d'un atome de carbone 12}\\
Soit le carbone $12$ not� \noyau{C}{6}{12}.
\begin{enumerate}
\item L'�l�ment carbone peut-il avoir $5$ protons ? Pourquoi ?
\item Calculer la masse du noyau d'un atome de carbone $12$
sachant que la masse d'un nucl�on est $m_n = 1,67.10^{-27}~kg$
\item Calculer la masse des �lectrons de l'atome de carbone 12
sachant que la masse d'un �lectron vaut $m_e = 9,1.10^{-31}~kg$
\item Comparer la masse des �lectrons de l'atome � la masse du noyau.
Que concluez-vous ?
\item En d�duire, sans nouveau calcul, la masse de l'atome de carbone
  $12$.
\end{enumerate}
\end{exercice}

\newpage

\vressort{1}

\begin{exercice}{Couches �lectroniques}\\
Dans l'�tat le plus stable de l'atome, appel� �tat fondamental,
les �lectrons occupent successivement les couches,
en commen�ant par celles qui sont les plus proches du noyau : 
d'abord $K$ puis $L$ puis $M$.

Lorsqu'une couche est pleine, ou encore satur�e, on passe � la suivante.

La derni�re couche occup�e est appel�e couche externe.\\
Toutes les autres sont appel�es couches internes.

\medskip

\noindent
%\begin{tabularx}{\textwidth}{|>{\centering}X|>{\centering}X|>{\centering}X|>{\centering}X|}
\begin{tabularx}{\textwidth}{|>{\mystrut}X|X|X|X|}
\hline
\multicolumn{1}{|X|}{\emph{Symbole de la couche}}       & $K$ & $L$ & $M$  \tbnl
\emph{Nombre maximal d'�lectrons} & $2$ & $8$ & $18$ \tbnl
\end{tabularx}

\medskip

Ainsi, par exemple, l'atome de chlore ($Z=17$) a la configuration �lectronique :
$(K)^2(L)^8(M)^{7}$.

\begin{enumerate}
\item Indiquez le nombre d'�lectrons et donnez la configuration des atomes suivants :
  \begin{enumerate}
  \item \noyau{H}{1}{}
  \item \noyau{O}{8}{}
  \item \noyau{C}{6}{}
  \item \noyau{Ne}{10}{}
  \end{enumerate}
\item Parmi les ions ci-desssous, pr�cisez s'il s'agit d'anions ou de cations.
Indiquez le nombre d'�lectrons et donnez la configuration �lectronique
des ions suivants :
  \begin{enumerate}
  \item $Be^{2+}$ ($Z=4$)
  \item $Al^{3+}$ ($Z=13$)
  \item $O^{2-}$ ($Z=8$)
  \item $F^{-}$ ($Z=9$)
  \end{enumerate}
\end{enumerate}

\end{exercice}


\vressort{5}


\begin{exercice}{Charge d'un atome de Zinc}%\\
\begin{enumerate}
\item Combien de protons l'atome de zinc \noyau{Zn}{30}{65} contient-il ?
\item Combien d'�lectrons comporte-t-il ?
\item Calculer la charge totale des protons
sachant qu'un proton a pour charge $e = 1,6.10^{-19}~C$.
\item Calculer la charge totale des �lectrons
sachant qu'un �lectron a pour charge $-e = -1,6.10^{-19}~C$.
\item En d�duire la charge de l'atome de Zinc.
\item Ce r�sultat est-il identique pour tous les atomes ?
\item A l'issue d'une r�action dite d'oxydation, un atome de zinc $Zn$
  se transforme en un ion $Zn^{2+}$.
  \begin{enumerate}
  \item Donnez l'�quation de cette r�action
  (en faisant intervenir un ou plusieurs �lectrons not�s $e^-$).
  \item Indiquez la charge (en coulomb $C$) de cet ion.
  \end{enumerate}
\end{enumerate}
\end{exercice}

\vressort{3} % Mettre les fig
\ds{Devoir Surveill�}{
%
}

\nomprenomclasse

\setcounter{numexercice}{0}

%\renewcommand{\tabularx}[1]{>{\centering}m{#1}} 

%\newcommand{\tabularxc}[1]{\tabularx{>{\centering}m{#1}}}

\vressort{3}

\begin{exercice}{Connaissance sur l'atome}%\\
\begin{enumerate}
\item De quoi est compos� un atome ?
\item Que signifie les lettres $A$, $Z$ et $X$ dans la repr�sentation \noyau{X}{Z}{A} ?
\item Comment trouve-t-on le nombre de neutrons d'un atome de l'�l�ment pr�c�dent.
\item Si un atome a $5$ protons, combien-a-t-il d'�lectrons ? Pourquoi ?
\item Qu'est-ce qui caract�rise un �l�ment chimique ?
\item Qu'est-ce qu'un isotope ?
\end{enumerate}
\end{exercice}



\vressort{3}



\begin{exercice}{Composition des atomes}\\
En vous aidant du tableau p�riodique des �l�ments,
compl�ter le tableau suivant :

\medskip

\noindent
%\begin{tabularx}{\textwidth}{|>{\centering}X|>{\centering}X|>{\centering}X|>{\centering}X|>{\centering}X|}
% \begin{tabularx}{\linewidth}{|X|X|X|X|X|}
% \hline
% \emph{nom}       & \emph{symbole}  & \emph{protons} & \emph{neutrons}
% & \emph{nucl�ons} \tbnl
% carbone   & \noyau{C}{6}{14}   &         &          & \rule[-0.5cm]{0cm}{1cm}         \tbnl
% fluor     & \noyau{F}{9}{19}   &         &          & \rule[-0.5cm]{0cm}{1cm}         \tbnl
% sodium    & \noyau{Na}{11}{23} &         &          & \rule[-0.5cm]{0cm}{1cm}         \tbnl
% oxyg�ne   & \noyau{O}{8}{16}   &         &          & \rule[-0.5cm]{0cm}{1cm}         \tbnl
% hydrog�ne &          &         & 0        & \rule[-0.5cm]{0cm}{1cm}         \tbnl
%           & \noyau{Cl}{17}{35} &         &          &  \rule[-0.5cm]{0cm}{1cm}        \tbnl
%           &          & 8       & \rule[-0.5cm]{0cm}{1cm}         & 16       \tbnl
% \end{tabularx}



\begin{tabularx}{\linewidth}{|>{\mystrut}X|X|X|X|X|}
\hline
% multicolumn pour faire dispara�tre le \mystrut
\multicolumn{1}{|X|}{\emph{nom}} & \emph{symbole}  &
\emph{protons} & \emph{neutrons} & \emph{nucl�ons} \tbnl
carbone   & \noyau{C}{6}{14}   &   &   &    \tbnl
fluor     & \noyau{F}{9}{19}   &   &   &    \tbnl
sodium    & \noyau{Na}{11}{23} &   &   &    \tbnl
oxyg�ne   & \noyau{O}{8}{16}   &   &   &    \tbnl
hydrog�ne &                    &   & 0 &    \tbnl
          & \noyau{Cl}{17}{35} &   &   &    \tbnl
          &                    & 8 &   & 16 \tbnl
\end{tabularx}


\end{exercice}


\vressort{3}


\begin{exercice}{Masse d'un atome de carbone 12}\\
Soit le carbone $12$ not� \noyau{C}{6}{12}.
\begin{enumerate}
\item L'�l�ment carbone peut-il avoir $5$ protons ? Pourquoi ?
\item Calculer la masse du noyau d'un atome de carbone $12$
sachant que la masse d'un nucl�on est $m_n = 1,67.10^{-27}~kg$
\item Calculer la masse des �lectrons de l'atome de carbone 12
sachant que la masse d'un �lectron vaut $m_e = 9,1.10^{-31}~kg$
\item Comparer la masse des �lectrons de l'atome � la masse du noyau.
Que concluez-vous ?
\item En d�duire, sans nouveau calcul, la masse de l'atome de carbone
  $12$.
\end{enumerate}
\end{exercice}

\newpage

\vressort{1}

\begin{exercice}{Couches �lectroniques}\\
Dans l'�tat le plus stable de l'atome, appel� �tat fondamental,
les �lectrons occupent successivement les couches,
en commen�ant par celles qui sont les plus proches du noyau : 
d'abord $K$ puis $L$ puis $M$.

Lorsqu'une couche est pleine, ou encore satur�e, on passe � la suivante.

La derni�re couche occup�e est appel�e couche externe.\\
Toutes les autres sont appel�es couches internes.

\medskip

\noindent
%\begin{tabularx}{\textwidth}{|>{\centering}X|>{\centering}X|>{\centering}X|>{\centering}X|}
\begin{tabularx}{\textwidth}{|>{\mystrut}X|X|X|X|}
\hline
\multicolumn{1}{|X|}{\emph{Symbole de la couche}}       & $K$ & $L$ & $M$  \tbnl
\emph{Nombre maximal d'�lectrons} & $2$ & $8$ & $18$ \tbnl
\end{tabularx}

\medskip

Ainsi, par exemple, l'atome de chlore ($Z=17$) a la configuration �lectronique :
$(K)^2(L)^8(M)^{7}$.

\begin{enumerate}
\item Indiquez le nombre d'�lectrons et donnez la configuration des atomes suivants :
  \begin{enumerate}
  \item \noyau{H}{1}{}
  \item \noyau{O}{8}{}
  \item \noyau{C}{6}{}
  \item \noyau{Ne}{10}{}
  \end{enumerate}
\item Parmi les ions ci-desssous, pr�cisez s'il s'agit d'anions ou de cations.
Indiquez le nombre d'�lectrons et donnez la configuration �lectronique
des ions suivants :
  \begin{enumerate}
  \item $Be^{2+}$ ($Z=4$)
  \item $Al^{3+}$ ($Z=13$)
  \item $O^{2-}$ ($Z=8$)
  \item $F^{-}$ ($Z=9$)
  \end{enumerate}
\end{enumerate}

\end{exercice}


\vressort{5}


\begin{exercice}{Charge d'un atome de Zinc}%\\
\begin{enumerate}
\item Combien de protons l'atome de zinc \noyau{Zn}{30}{65} contient-il ?
\item Combien d'�lectrons comporte-t-il ?
\item Calculer la charge totale des protons
sachant qu'un proton a pour charge $e = 1,6.10^{-19}~C$.
\item Calculer la charge totale des �lectrons
sachant qu'un �lectron a pour charge $-e = -1,6.10^{-19}~C$.
\item En d�duire la charge de l'atome de Zinc.
\item Ce r�sultat est-il identique pour tous les atomes ?
\item A l'issue d'une r�action dite d'oxydation, un atome de zinc $Zn$
  se transforme en un ion $Zn^{2+}$.
  \begin{enumerate}
  \item Donnez l'�quation de cette r�action
  (en faisant intervenir un ou plusieurs �lectrons not�s $e^-$).
  \item Indiquez la charge (en coulomb $C$) de cet ion.
  \end{enumerate}
\end{enumerate}
\end{exercice}

\vressort{3}
\ds{Devoir Surveill�}{
%
}

\nomprenomclasse

\setcounter{numexercice}{0}

%\renewcommand{\tabularx}[1]{>{\centering}m{#1}} 

%\newcommand{\tabularxc}[1]{\tabularx{>{\centering}m{#1}}}

\vressort{3}

\begin{exercice}{Connaissance sur l'atome}%\\
\begin{enumerate}
\item De quoi est compos� un atome ?
\item Que signifie les lettres $A$, $Z$ et $X$ dans la repr�sentation \noyau{X}{Z}{A} ?
\item Comment trouve-t-on le nombre de neutrons d'un atome de l'�l�ment pr�c�dent.
\item Si un atome a $5$ protons, combien-a-t-il d'�lectrons ? Pourquoi ?
\item Qu'est-ce qui caract�rise un �l�ment chimique ?
\item Qu'est-ce qu'un isotope ?
\end{enumerate}
\end{exercice}



\vressort{3}



\begin{exercice}{Composition des atomes}\\
En vous aidant du tableau p�riodique des �l�ments,
compl�ter le tableau suivant :

\medskip

\noindent
%\begin{tabularx}{\textwidth}{|>{\centering}X|>{\centering}X|>{\centering}X|>{\centering}X|>{\centering}X|}
% \begin{tabularx}{\linewidth}{|X|X|X|X|X|}
% \hline
% \emph{nom}       & \emph{symbole}  & \emph{protons} & \emph{neutrons}
% & \emph{nucl�ons} \tbnl
% carbone   & \noyau{C}{6}{14}   &         &          & \rule[-0.5cm]{0cm}{1cm}         \tbnl
% fluor     & \noyau{F}{9}{19}   &         &          & \rule[-0.5cm]{0cm}{1cm}         \tbnl
% sodium    & \noyau{Na}{11}{23} &         &          & \rule[-0.5cm]{0cm}{1cm}         \tbnl
% oxyg�ne   & \noyau{O}{8}{16}   &         &          & \rule[-0.5cm]{0cm}{1cm}         \tbnl
% hydrog�ne &          &         & 0        & \rule[-0.5cm]{0cm}{1cm}         \tbnl
%           & \noyau{Cl}{17}{35} &         &          &  \rule[-0.5cm]{0cm}{1cm}        \tbnl
%           &          & 8       & \rule[-0.5cm]{0cm}{1cm}         & 16       \tbnl
% \end{tabularx}



\begin{tabularx}{\linewidth}{|>{\mystrut}X|X|X|X|X|}
\hline
% multicolumn pour faire dispara�tre le \mystrut
\multicolumn{1}{|X|}{\emph{nom}} & \emph{symbole}  &
\emph{protons} & \emph{neutrons} & \emph{nucl�ons} \tbnl
carbone   & \noyau{C}{6}{14}   &   &   &    \tbnl
fluor     & \noyau{F}{9}{19}   &   &   &    \tbnl
sodium    & \noyau{Na}{11}{23} &   &   &    \tbnl
oxyg�ne   & \noyau{O}{8}{16}   &   &   &    \tbnl
hydrog�ne &                    &   & 0 &    \tbnl
          & \noyau{Cl}{17}{35} &   &   &    \tbnl
          &                    & 8 &   & 16 \tbnl
\end{tabularx}


\end{exercice}


\vressort{3}


\begin{exercice}{Masse d'un atome de carbone 12}\\
Soit le carbone $12$ not� \noyau{C}{6}{12}.
\begin{enumerate}
\item L'�l�ment carbone peut-il avoir $5$ protons ? Pourquoi ?
\item Calculer la masse du noyau d'un atome de carbone $12$
sachant que la masse d'un nucl�on est $m_n = 1,67.10^{-27}~kg$
\item Calculer la masse des �lectrons de l'atome de carbone 12
sachant que la masse d'un �lectron vaut $m_e = 9,1.10^{-31}~kg$
\item Comparer la masse des �lectrons de l'atome � la masse du noyau.
Que concluez-vous ?
\item En d�duire, sans nouveau calcul, la masse de l'atome de carbone
  $12$.
\end{enumerate}
\end{exercice}

\newpage

\vressort{1}

\begin{exercice}{Couches �lectroniques}\\
Dans l'�tat le plus stable de l'atome, appel� �tat fondamental,
les �lectrons occupent successivement les couches,
en commen�ant par celles qui sont les plus proches du noyau : 
d'abord $K$ puis $L$ puis $M$.

Lorsqu'une couche est pleine, ou encore satur�e, on passe � la suivante.

La derni�re couche occup�e est appel�e couche externe.\\
Toutes les autres sont appel�es couches internes.

\medskip

\noindent
%\begin{tabularx}{\textwidth}{|>{\centering}X|>{\centering}X|>{\centering}X|>{\centering}X|}
\begin{tabularx}{\textwidth}{|>{\mystrut}X|X|X|X|}
\hline
\multicolumn{1}{|X|}{\emph{Symbole de la couche}}       & $K$ & $L$ & $M$  \tbnl
\emph{Nombre maximal d'�lectrons} & $2$ & $8$ & $18$ \tbnl
\end{tabularx}

\medskip

Ainsi, par exemple, l'atome de chlore ($Z=17$) a la configuration �lectronique :
$(K)^2(L)^8(M)^{7}$.

\begin{enumerate}
\item Indiquez le nombre d'�lectrons et donnez la configuration des atomes suivants :
  \begin{enumerate}
  \item \noyau{H}{1}{}
  \item \noyau{O}{8}{}
  \item \noyau{C}{6}{}
  \item \noyau{Ne}{10}{}
  \end{enumerate}
\item Parmi les ions ci-desssous, pr�cisez s'il s'agit d'anions ou de cations.
Indiquez le nombre d'�lectrons et donnez la configuration �lectronique
des ions suivants :
  \begin{enumerate}
  \item $Be^{2+}$ ($Z=4$)
  \item $Al^{3+}$ ($Z=13$)
  \item $O^{2-}$ ($Z=8$)
  \item $F^{-}$ ($Z=9$)
  \end{enumerate}
\end{enumerate}

\end{exercice}


\vressort{5}


\begin{exercice}{Charge d'un atome de Zinc}%\\
\begin{enumerate}
\item Combien de protons l'atome de zinc \noyau{Zn}{30}{65} contient-il ?
\item Combien d'�lectrons comporte-t-il ?
\item Calculer la charge totale des protons
sachant qu'un proton a pour charge $e = 1,6.10^{-19}~C$.
\item Calculer la charge totale des �lectrons
sachant qu'un �lectron a pour charge $-e = -1,6.10^{-19}~C$.
\item En d�duire la charge de l'atome de Zinc.
\item Ce r�sultat est-il identique pour tous les atomes ?
\item A l'issue d'une r�action dite d'oxydation, un atome de zinc $Zn$
  se transforme en un ion $Zn^{2+}$.
  \begin{enumerate}
  \item Donnez l'�quation de cette r�action
  (en faisant intervenir un ou plusieurs �lectrons not�s $e^-$).
  \item Indiquez la charge (en coulomb $C$) de cet ion.
  \end{enumerate}
\end{enumerate}
\end{exercice}

\vressort{3}
\ds{Devoir Surveill�}{
%
}

\nomprenomclasse

\setcounter{numexercice}{0}

%\renewcommand{\tabularx}[1]{>{\centering}m{#1}} 

%\newcommand{\tabularxc}[1]{\tabularx{>{\centering}m{#1}}}

\vressort{3}

\begin{exercice}{Connaissance sur l'atome}%\\
\begin{enumerate}
\item De quoi est compos� un atome ?
\item Que signifie les lettres $A$, $Z$ et $X$ dans la repr�sentation \noyau{X}{Z}{A} ?
\item Comment trouve-t-on le nombre de neutrons d'un atome de l'�l�ment pr�c�dent.
\item Si un atome a $5$ protons, combien-a-t-il d'�lectrons ? Pourquoi ?
\item Qu'est-ce qui caract�rise un �l�ment chimique ?
\item Qu'est-ce qu'un isotope ?
\end{enumerate}
\end{exercice}



\vressort{3}



\begin{exercice}{Composition des atomes}\\
En vous aidant du tableau p�riodique des �l�ments,
compl�ter le tableau suivant :

\medskip

\noindent
%\begin{tabularx}{\textwidth}{|>{\centering}X|>{\centering}X|>{\centering}X|>{\centering}X|>{\centering}X|}
% \begin{tabularx}{\linewidth}{|X|X|X|X|X|}
% \hline
% \emph{nom}       & \emph{symbole}  & \emph{protons} & \emph{neutrons}
% & \emph{nucl�ons} \tbnl
% carbone   & \noyau{C}{6}{14}   &         &          & \rule[-0.5cm]{0cm}{1cm}         \tbnl
% fluor     & \noyau{F}{9}{19}   &         &          & \rule[-0.5cm]{0cm}{1cm}         \tbnl
% sodium    & \noyau{Na}{11}{23} &         &          & \rule[-0.5cm]{0cm}{1cm}         \tbnl
% oxyg�ne   & \noyau{O}{8}{16}   &         &          & \rule[-0.5cm]{0cm}{1cm}         \tbnl
% hydrog�ne &          &         & 0        & \rule[-0.5cm]{0cm}{1cm}         \tbnl
%           & \noyau{Cl}{17}{35} &         &          &  \rule[-0.5cm]{0cm}{1cm}        \tbnl
%           &          & 8       & \rule[-0.5cm]{0cm}{1cm}         & 16       \tbnl
% \end{tabularx}



\begin{tabularx}{\linewidth}{|>{\mystrut}X|X|X|X|X|}
\hline
% multicolumn pour faire dispara�tre le \mystrut
\multicolumn{1}{|X|}{\emph{nom}} & \emph{symbole}  &
\emph{protons} & \emph{neutrons} & \emph{nucl�ons} \tbnl
carbone   & \noyau{C}{6}{14}   &   &   &    \tbnl
fluor     & \noyau{F}{9}{19}   &   &   &    \tbnl
sodium    & \noyau{Na}{11}{23} &   &   &    \tbnl
oxyg�ne   & \noyau{O}{8}{16}   &   &   &    \tbnl
hydrog�ne &                    &   & 0 &    \tbnl
          & \noyau{Cl}{17}{35} &   &   &    \tbnl
          &                    & 8 &   & 16 \tbnl
\end{tabularx}


\end{exercice}


\vressort{3}


\begin{exercice}{Masse d'un atome de carbone 12}\\
Soit le carbone $12$ not� \noyau{C}{6}{12}.
\begin{enumerate}
\item L'�l�ment carbone peut-il avoir $5$ protons ? Pourquoi ?
\item Calculer la masse du noyau d'un atome de carbone $12$
sachant que la masse d'un nucl�on est $m_n = 1,67.10^{-27}~kg$
\item Calculer la masse des �lectrons de l'atome de carbone 12
sachant que la masse d'un �lectron vaut $m_e = 9,1.10^{-31}~kg$
\item Comparer la masse des �lectrons de l'atome � la masse du noyau.
Que concluez-vous ?
\item En d�duire, sans nouveau calcul, la masse de l'atome de carbone
  $12$.
\end{enumerate}
\end{exercice}

\newpage

\vressort{1}

\begin{exercice}{Couches �lectroniques}\\
Dans l'�tat le plus stable de l'atome, appel� �tat fondamental,
les �lectrons occupent successivement les couches,
en commen�ant par celles qui sont les plus proches du noyau : 
d'abord $K$ puis $L$ puis $M$.

Lorsqu'une couche est pleine, ou encore satur�e, on passe � la suivante.

La derni�re couche occup�e est appel�e couche externe.\\
Toutes les autres sont appel�es couches internes.

\medskip

\noindent
%\begin{tabularx}{\textwidth}{|>{\centering}X|>{\centering}X|>{\centering}X|>{\centering}X|}
\begin{tabularx}{\textwidth}{|>{\mystrut}X|X|X|X|}
\hline
\multicolumn{1}{|X|}{\emph{Symbole de la couche}}       & $K$ & $L$ & $M$  \tbnl
\emph{Nombre maximal d'�lectrons} & $2$ & $8$ & $18$ \tbnl
\end{tabularx}

\medskip

Ainsi, par exemple, l'atome de chlore ($Z=17$) a la configuration �lectronique :
$(K)^2(L)^8(M)^{7}$.

\begin{enumerate}
\item Indiquez le nombre d'�lectrons et donnez la configuration des atomes suivants :
  \begin{enumerate}
  \item \noyau{H}{1}{}
  \item \noyau{O}{8}{}
  \item \noyau{C}{6}{}
  \item \noyau{Ne}{10}{}
  \end{enumerate}
\item Parmi les ions ci-desssous, pr�cisez s'il s'agit d'anions ou de cations.
Indiquez le nombre d'�lectrons et donnez la configuration �lectronique
des ions suivants :
  \begin{enumerate}
  \item $Be^{2+}$ ($Z=4$)
  \item $Al^{3+}$ ($Z=13$)
  \item $O^{2-}$ ($Z=8$)
  \item $F^{-}$ ($Z=9$)
  \end{enumerate}
\end{enumerate}

\end{exercice}


\vressort{5}


\begin{exercice}{Charge d'un atome de Zinc}%\\
\begin{enumerate}
\item Combien de protons l'atome de zinc \noyau{Zn}{30}{65} contient-il ?
\item Combien d'�lectrons comporte-t-il ?
\item Calculer la charge totale des protons
sachant qu'un proton a pour charge $e = 1,6.10^{-19}~C$.
\item Calculer la charge totale des �lectrons
sachant qu'un �lectron a pour charge $-e = -1,6.10^{-19}~C$.
\item En d�duire la charge de l'atome de Zinc.
\item Ce r�sultat est-il identique pour tous les atomes ?
\item A l'issue d'une r�action dite d'oxydation, un atome de zinc $Zn$
  se transforme en un ion $Zn^{2+}$.
  \begin{enumerate}
  \item Donnez l'�quation de cette r�action
  (en faisant intervenir un ou plusieurs �lectrons not�s $e^-$).
  \item Indiquez la charge (en coulomb $C$) de cet ion.
  \end{enumerate}
\end{enumerate}
\end{exercice}

\vressort{3}
\ds{Devoir Surveill�}{
%
}

\nomprenomclasse

\setcounter{numexercice}{0}

%\renewcommand{\tabularx}[1]{>{\centering}m{#1}} 

%\newcommand{\tabularxc}[1]{\tabularx{>{\centering}m{#1}}}

\vressort{3}

\begin{exercice}{Connaissance sur l'atome}%\\
\begin{enumerate}
\item De quoi est compos� un atome ?
\item Que signifie les lettres $A$, $Z$ et $X$ dans la repr�sentation \noyau{X}{Z}{A} ?
\item Comment trouve-t-on le nombre de neutrons d'un atome de l'�l�ment pr�c�dent.
\item Si un atome a $5$ protons, combien-a-t-il d'�lectrons ? Pourquoi ?
\item Qu'est-ce qui caract�rise un �l�ment chimique ?
\item Qu'est-ce qu'un isotope ?
\end{enumerate}
\end{exercice}



\vressort{3}



\begin{exercice}{Composition des atomes}\\
En vous aidant du tableau p�riodique des �l�ments,
compl�ter le tableau suivant :

\medskip

\noindent
%\begin{tabularx}{\textwidth}{|>{\centering}X|>{\centering}X|>{\centering}X|>{\centering}X|>{\centering}X|}
% \begin{tabularx}{\linewidth}{|X|X|X|X|X|}
% \hline
% \emph{nom}       & \emph{symbole}  & \emph{protons} & \emph{neutrons}
% & \emph{nucl�ons} \tbnl
% carbone   & \noyau{C}{6}{14}   &         &          & \rule[-0.5cm]{0cm}{1cm}         \tbnl
% fluor     & \noyau{F}{9}{19}   &         &          & \rule[-0.5cm]{0cm}{1cm}         \tbnl
% sodium    & \noyau{Na}{11}{23} &         &          & \rule[-0.5cm]{0cm}{1cm}         \tbnl
% oxyg�ne   & \noyau{O}{8}{16}   &         &          & \rule[-0.5cm]{0cm}{1cm}         \tbnl
% hydrog�ne &          &         & 0        & \rule[-0.5cm]{0cm}{1cm}         \tbnl
%           & \noyau{Cl}{17}{35} &         &          &  \rule[-0.5cm]{0cm}{1cm}        \tbnl
%           &          & 8       & \rule[-0.5cm]{0cm}{1cm}         & 16       \tbnl
% \end{tabularx}



\begin{tabularx}{\linewidth}{|>{\mystrut}X|X|X|X|X|}
\hline
% multicolumn pour faire dispara�tre le \mystrut
\multicolumn{1}{|X|}{\emph{nom}} & \emph{symbole}  &
\emph{protons} & \emph{neutrons} & \emph{nucl�ons} \tbnl
carbone   & \noyau{C}{6}{14}   &   &   &    \tbnl
fluor     & \noyau{F}{9}{19}   &   &   &    \tbnl
sodium    & \noyau{Na}{11}{23} &   &   &    \tbnl
oxyg�ne   & \noyau{O}{8}{16}   &   &   &    \tbnl
hydrog�ne &                    &   & 0 &    \tbnl
          & \noyau{Cl}{17}{35} &   &   &    \tbnl
          &                    & 8 &   & 16 \tbnl
\end{tabularx}


\end{exercice}


\vressort{3}


\begin{exercice}{Masse d'un atome de carbone 12}\\
Soit le carbone $12$ not� \noyau{C}{6}{12}.
\begin{enumerate}
\item L'�l�ment carbone peut-il avoir $5$ protons ? Pourquoi ?
\item Calculer la masse du noyau d'un atome de carbone $12$
sachant que la masse d'un nucl�on est $m_n = 1,67.10^{-27}~kg$
\item Calculer la masse des �lectrons de l'atome de carbone 12
sachant que la masse d'un �lectron vaut $m_e = 9,1.10^{-31}~kg$
\item Comparer la masse des �lectrons de l'atome � la masse du noyau.
Que concluez-vous ?
\item En d�duire, sans nouveau calcul, la masse de l'atome de carbone
  $12$.
\end{enumerate}
\end{exercice}

\newpage

\vressort{1}

\begin{exercice}{Couches �lectroniques}\\
Dans l'�tat le plus stable de l'atome, appel� �tat fondamental,
les �lectrons occupent successivement les couches,
en commen�ant par celles qui sont les plus proches du noyau : 
d'abord $K$ puis $L$ puis $M$.

Lorsqu'une couche est pleine, ou encore satur�e, on passe � la suivante.

La derni�re couche occup�e est appel�e couche externe.\\
Toutes les autres sont appel�es couches internes.

\medskip

\noindent
%\begin{tabularx}{\textwidth}{|>{\centering}X|>{\centering}X|>{\centering}X|>{\centering}X|}
\begin{tabularx}{\textwidth}{|>{\mystrut}X|X|X|X|}
\hline
\multicolumn{1}{|X|}{\emph{Symbole de la couche}}       & $K$ & $L$ & $M$  \tbnl
\emph{Nombre maximal d'�lectrons} & $2$ & $8$ & $18$ \tbnl
\end{tabularx}

\medskip

Ainsi, par exemple, l'atome de chlore ($Z=17$) a la configuration �lectronique :
$(K)^2(L)^8(M)^{7}$.

\begin{enumerate}
\item Indiquez le nombre d'�lectrons et donnez la configuration des atomes suivants :
  \begin{enumerate}
  \item \noyau{H}{1}{}
  \item \noyau{O}{8}{}
  \item \noyau{C}{6}{}
  \item \noyau{Ne}{10}{}
  \end{enumerate}
\item Parmi les ions ci-desssous, pr�cisez s'il s'agit d'anions ou de cations.
Indiquez le nombre d'�lectrons et donnez la configuration �lectronique
des ions suivants :
  \begin{enumerate}
  \item $Be^{2+}$ ($Z=4$)
  \item $Al^{3+}$ ($Z=13$)
  \item $O^{2-}$ ($Z=8$)
  \item $F^{-}$ ($Z=9$)
  \end{enumerate}
\end{enumerate}

\end{exercice}


\vressort{5}


\begin{exercice}{Charge d'un atome de Zinc}%\\
\begin{enumerate}
\item Combien de protons l'atome de zinc \noyau{Zn}{30}{65} contient-il ?
\item Combien d'�lectrons comporte-t-il ?
\item Calculer la charge totale des protons
sachant qu'un proton a pour charge $e = 1,6.10^{-19}~C$.
\item Calculer la charge totale des �lectrons
sachant qu'un �lectron a pour charge $-e = -1,6.10^{-19}~C$.
\item En d�duire la charge de l'atome de Zinc.
\item Ce r�sultat est-il identique pour tous les atomes ?
\item A l'issue d'une r�action dite d'oxydation, un atome de zinc $Zn$
  se transforme en un ion $Zn^{2+}$.
  \begin{enumerate}
  \item Donnez l'�quation de cette r�action
  (en faisant intervenir un ou plusieurs �lectrons not�s $e^-$).
  \item Indiquez la charge (en coulomb $C$) de cet ion.
  \end{enumerate}
\end{enumerate}
\end{exercice}

\vressort{3}


\chapitre{Forces}
\ds{Devoir Surveill�}{
%
}

\nomprenomclasse

\setcounter{numexercice}{0}

%\renewcommand{\tabularx}[1]{>{\centering}m{#1}} 

%\newcommand{\tabularxc}[1]{\tabularx{>{\centering}m{#1}}}

\vressort{3}

\begin{exercice}{Connaissance sur l'atome}%\\
\begin{enumerate}
\item De quoi est compos� un atome ?
\item Que signifie les lettres $A$, $Z$ et $X$ dans la repr�sentation \noyau{X}{Z}{A} ?
\item Comment trouve-t-on le nombre de neutrons d'un atome de l'�l�ment pr�c�dent.
\item Si un atome a $5$ protons, combien-a-t-il d'�lectrons ? Pourquoi ?
\item Qu'est-ce qui caract�rise un �l�ment chimique ?
\item Qu'est-ce qu'un isotope ?
\end{enumerate}
\end{exercice}



\vressort{3}



\begin{exercice}{Composition des atomes}\\
En vous aidant du tableau p�riodique des �l�ments,
compl�ter le tableau suivant :

\medskip

\noindent
%\begin{tabularx}{\textwidth}{|>{\centering}X|>{\centering}X|>{\centering}X|>{\centering}X|>{\centering}X|}
% \begin{tabularx}{\linewidth}{|X|X|X|X|X|}
% \hline
% \emph{nom}       & \emph{symbole}  & \emph{protons} & \emph{neutrons}
% & \emph{nucl�ons} \tbnl
% carbone   & \noyau{C}{6}{14}   &         &          & \rule[-0.5cm]{0cm}{1cm}         \tbnl
% fluor     & \noyau{F}{9}{19}   &         &          & \rule[-0.5cm]{0cm}{1cm}         \tbnl
% sodium    & \noyau{Na}{11}{23} &         &          & \rule[-0.5cm]{0cm}{1cm}         \tbnl
% oxyg�ne   & \noyau{O}{8}{16}   &         &          & \rule[-0.5cm]{0cm}{1cm}         \tbnl
% hydrog�ne &          &         & 0        & \rule[-0.5cm]{0cm}{1cm}         \tbnl
%           & \noyau{Cl}{17}{35} &         &          &  \rule[-0.5cm]{0cm}{1cm}        \tbnl
%           &          & 8       & \rule[-0.5cm]{0cm}{1cm}         & 16       \tbnl
% \end{tabularx}



\begin{tabularx}{\linewidth}{|>{\mystrut}X|X|X|X|X|}
\hline
% multicolumn pour faire dispara�tre le \mystrut
\multicolumn{1}{|X|}{\emph{nom}} & \emph{symbole}  &
\emph{protons} & \emph{neutrons} & \emph{nucl�ons} \tbnl
carbone   & \noyau{C}{6}{14}   &   &   &    \tbnl
fluor     & \noyau{F}{9}{19}   &   &   &    \tbnl
sodium    & \noyau{Na}{11}{23} &   &   &    \tbnl
oxyg�ne   & \noyau{O}{8}{16}   &   &   &    \tbnl
hydrog�ne &                    &   & 0 &    \tbnl
          & \noyau{Cl}{17}{35} &   &   &    \tbnl
          &                    & 8 &   & 16 \tbnl
\end{tabularx}


\end{exercice}


\vressort{3}


\begin{exercice}{Masse d'un atome de carbone 12}\\
Soit le carbone $12$ not� \noyau{C}{6}{12}.
\begin{enumerate}
\item L'�l�ment carbone peut-il avoir $5$ protons ? Pourquoi ?
\item Calculer la masse du noyau d'un atome de carbone $12$
sachant que la masse d'un nucl�on est $m_n = 1,67.10^{-27}~kg$
\item Calculer la masse des �lectrons de l'atome de carbone 12
sachant que la masse d'un �lectron vaut $m_e = 9,1.10^{-31}~kg$
\item Comparer la masse des �lectrons de l'atome � la masse du noyau.
Que concluez-vous ?
\item En d�duire, sans nouveau calcul, la masse de l'atome de carbone
  $12$.
\end{enumerate}
\end{exercice}

\newpage

\vressort{1}

\begin{exercice}{Couches �lectroniques}\\
Dans l'�tat le plus stable de l'atome, appel� �tat fondamental,
les �lectrons occupent successivement les couches,
en commen�ant par celles qui sont les plus proches du noyau : 
d'abord $K$ puis $L$ puis $M$.

Lorsqu'une couche est pleine, ou encore satur�e, on passe � la suivante.

La derni�re couche occup�e est appel�e couche externe.\\
Toutes les autres sont appel�es couches internes.

\medskip

\noindent
%\begin{tabularx}{\textwidth}{|>{\centering}X|>{\centering}X|>{\centering}X|>{\centering}X|}
\begin{tabularx}{\textwidth}{|>{\mystrut}X|X|X|X|}
\hline
\multicolumn{1}{|X|}{\emph{Symbole de la couche}}       & $K$ & $L$ & $M$  \tbnl
\emph{Nombre maximal d'�lectrons} & $2$ & $8$ & $18$ \tbnl
\end{tabularx}

\medskip

Ainsi, par exemple, l'atome de chlore ($Z=17$) a la configuration �lectronique :
$(K)^2(L)^8(M)^{7}$.

\begin{enumerate}
\item Indiquez le nombre d'�lectrons et donnez la configuration des atomes suivants :
  \begin{enumerate}
  \item \noyau{H}{1}{}
  \item \noyau{O}{8}{}
  \item \noyau{C}{6}{}
  \item \noyau{Ne}{10}{}
  \end{enumerate}
\item Parmi les ions ci-desssous, pr�cisez s'il s'agit d'anions ou de cations.
Indiquez le nombre d'�lectrons et donnez la configuration �lectronique
des ions suivants :
  \begin{enumerate}
  \item $Be^{2+}$ ($Z=4$)
  \item $Al^{3+}$ ($Z=13$)
  \item $O^{2-}$ ($Z=8$)
  \item $F^{-}$ ($Z=9$)
  \end{enumerate}
\end{enumerate}

\end{exercice}


\vressort{5}


\begin{exercice}{Charge d'un atome de Zinc}%\\
\begin{enumerate}
\item Combien de protons l'atome de zinc \noyau{Zn}{30}{65} contient-il ?
\item Combien d'�lectrons comporte-t-il ?
\item Calculer la charge totale des protons
sachant qu'un proton a pour charge $e = 1,6.10^{-19}~C$.
\item Calculer la charge totale des �lectrons
sachant qu'un �lectron a pour charge $-e = -1,6.10^{-19}~C$.
\item En d�duire la charge de l'atome de Zinc.
\item Ce r�sultat est-il identique pour tous les atomes ?
\item A l'issue d'une r�action dite d'oxydation, un atome de zinc $Zn$
  se transforme en un ion $Zn^{2+}$.
  \begin{enumerate}
  \item Donnez l'�quation de cette r�action
  (en faisant intervenir un ou plusieurs �lectrons not�s $e^-$).
  \item Indiquez la charge (en coulomb $C$) de cet ion.
  \end{enumerate}
\end{enumerate}
\end{exercice}

\vressort{3} % Mettre les fig
\ds{Devoir Surveill�}{
%
}

\nomprenomclasse

\setcounter{numexercice}{0}

%\renewcommand{\tabularx}[1]{>{\centering}m{#1}} 

%\newcommand{\tabularxc}[1]{\tabularx{>{\centering}m{#1}}}

\vressort{3}

\begin{exercice}{Connaissance sur l'atome}%\\
\begin{enumerate}
\item De quoi est compos� un atome ?
\item Que signifie les lettres $A$, $Z$ et $X$ dans la repr�sentation \noyau{X}{Z}{A} ?
\item Comment trouve-t-on le nombre de neutrons d'un atome de l'�l�ment pr�c�dent.
\item Si un atome a $5$ protons, combien-a-t-il d'�lectrons ? Pourquoi ?
\item Qu'est-ce qui caract�rise un �l�ment chimique ?
\item Qu'est-ce qu'un isotope ?
\end{enumerate}
\end{exercice}



\vressort{3}



\begin{exercice}{Composition des atomes}\\
En vous aidant du tableau p�riodique des �l�ments,
compl�ter le tableau suivant :

\medskip

\noindent
%\begin{tabularx}{\textwidth}{|>{\centering}X|>{\centering}X|>{\centering}X|>{\centering}X|>{\centering}X|}
% \begin{tabularx}{\linewidth}{|X|X|X|X|X|}
% \hline
% \emph{nom}       & \emph{symbole}  & \emph{protons} & \emph{neutrons}
% & \emph{nucl�ons} \tbnl
% carbone   & \noyau{C}{6}{14}   &         &          & \rule[-0.5cm]{0cm}{1cm}         \tbnl
% fluor     & \noyau{F}{9}{19}   &         &          & \rule[-0.5cm]{0cm}{1cm}         \tbnl
% sodium    & \noyau{Na}{11}{23} &         &          & \rule[-0.5cm]{0cm}{1cm}         \tbnl
% oxyg�ne   & \noyau{O}{8}{16}   &         &          & \rule[-0.5cm]{0cm}{1cm}         \tbnl
% hydrog�ne &          &         & 0        & \rule[-0.5cm]{0cm}{1cm}         \tbnl
%           & \noyau{Cl}{17}{35} &         &          &  \rule[-0.5cm]{0cm}{1cm}        \tbnl
%           &          & 8       & \rule[-0.5cm]{0cm}{1cm}         & 16       \tbnl
% \end{tabularx}



\begin{tabularx}{\linewidth}{|>{\mystrut}X|X|X|X|X|}
\hline
% multicolumn pour faire dispara�tre le \mystrut
\multicolumn{1}{|X|}{\emph{nom}} & \emph{symbole}  &
\emph{protons} & \emph{neutrons} & \emph{nucl�ons} \tbnl
carbone   & \noyau{C}{6}{14}   &   &   &    \tbnl
fluor     & \noyau{F}{9}{19}   &   &   &    \tbnl
sodium    & \noyau{Na}{11}{23} &   &   &    \tbnl
oxyg�ne   & \noyau{O}{8}{16}   &   &   &    \tbnl
hydrog�ne &                    &   & 0 &    \tbnl
          & \noyau{Cl}{17}{35} &   &   &    \tbnl
          &                    & 8 &   & 16 \tbnl
\end{tabularx}


\end{exercice}


\vressort{3}


\begin{exercice}{Masse d'un atome de carbone 12}\\
Soit le carbone $12$ not� \noyau{C}{6}{12}.
\begin{enumerate}
\item L'�l�ment carbone peut-il avoir $5$ protons ? Pourquoi ?
\item Calculer la masse du noyau d'un atome de carbone $12$
sachant que la masse d'un nucl�on est $m_n = 1,67.10^{-27}~kg$
\item Calculer la masse des �lectrons de l'atome de carbone 12
sachant que la masse d'un �lectron vaut $m_e = 9,1.10^{-31}~kg$
\item Comparer la masse des �lectrons de l'atome � la masse du noyau.
Que concluez-vous ?
\item En d�duire, sans nouveau calcul, la masse de l'atome de carbone
  $12$.
\end{enumerate}
\end{exercice}

\newpage

\vressort{1}

\begin{exercice}{Couches �lectroniques}\\
Dans l'�tat le plus stable de l'atome, appel� �tat fondamental,
les �lectrons occupent successivement les couches,
en commen�ant par celles qui sont les plus proches du noyau : 
d'abord $K$ puis $L$ puis $M$.

Lorsqu'une couche est pleine, ou encore satur�e, on passe � la suivante.

La derni�re couche occup�e est appel�e couche externe.\\
Toutes les autres sont appel�es couches internes.

\medskip

\noindent
%\begin{tabularx}{\textwidth}{|>{\centering}X|>{\centering}X|>{\centering}X|>{\centering}X|}
\begin{tabularx}{\textwidth}{|>{\mystrut}X|X|X|X|}
\hline
\multicolumn{1}{|X|}{\emph{Symbole de la couche}}       & $K$ & $L$ & $M$  \tbnl
\emph{Nombre maximal d'�lectrons} & $2$ & $8$ & $18$ \tbnl
\end{tabularx}

\medskip

Ainsi, par exemple, l'atome de chlore ($Z=17$) a la configuration �lectronique :
$(K)^2(L)^8(M)^{7}$.

\begin{enumerate}
\item Indiquez le nombre d'�lectrons et donnez la configuration des atomes suivants :
  \begin{enumerate}
  \item \noyau{H}{1}{}
  \item \noyau{O}{8}{}
  \item \noyau{C}{6}{}
  \item \noyau{Ne}{10}{}
  \end{enumerate}
\item Parmi les ions ci-desssous, pr�cisez s'il s'agit d'anions ou de cations.
Indiquez le nombre d'�lectrons et donnez la configuration �lectronique
des ions suivants :
  \begin{enumerate}
  \item $Be^{2+}$ ($Z=4$)
  \item $Al^{3+}$ ($Z=13$)
  \item $O^{2-}$ ($Z=8$)
  \item $F^{-}$ ($Z=9$)
  \end{enumerate}
\end{enumerate}

\end{exercice}


\vressort{5}


\begin{exercice}{Charge d'un atome de Zinc}%\\
\begin{enumerate}
\item Combien de protons l'atome de zinc \noyau{Zn}{30}{65} contient-il ?
\item Combien d'�lectrons comporte-t-il ?
\item Calculer la charge totale des protons
sachant qu'un proton a pour charge $e = 1,6.10^{-19}~C$.
\item Calculer la charge totale des �lectrons
sachant qu'un �lectron a pour charge $-e = -1,6.10^{-19}~C$.
\item En d�duire la charge de l'atome de Zinc.
\item Ce r�sultat est-il identique pour tous les atomes ?
\item A l'issue d'une r�action dite d'oxydation, un atome de zinc $Zn$
  se transforme en un ion $Zn^{2+}$.
  \begin{enumerate}
  \item Donnez l'�quation de cette r�action
  (en faisant intervenir un ou plusieurs �lectrons not�s $e^-$).
  \item Indiquez la charge (en coulomb $C$) de cet ion.
  \end{enumerate}
\end{enumerate}
\end{exercice}

\vressort{3}
\ds{Devoir Surveill�}{
%
}

\nomprenomclasse

\setcounter{numexercice}{0}

%\renewcommand{\tabularx}[1]{>{\centering}m{#1}} 

%\newcommand{\tabularxc}[1]{\tabularx{>{\centering}m{#1}}}

\vressort{3}

\begin{exercice}{Connaissance sur l'atome}%\\
\begin{enumerate}
\item De quoi est compos� un atome ?
\item Que signifie les lettres $A$, $Z$ et $X$ dans la repr�sentation \noyau{X}{Z}{A} ?
\item Comment trouve-t-on le nombre de neutrons d'un atome de l'�l�ment pr�c�dent.
\item Si un atome a $5$ protons, combien-a-t-il d'�lectrons ? Pourquoi ?
\item Qu'est-ce qui caract�rise un �l�ment chimique ?
\item Qu'est-ce qu'un isotope ?
\end{enumerate}
\end{exercice}



\vressort{3}



\begin{exercice}{Composition des atomes}\\
En vous aidant du tableau p�riodique des �l�ments,
compl�ter le tableau suivant :

\medskip

\noindent
%\begin{tabularx}{\textwidth}{|>{\centering}X|>{\centering}X|>{\centering}X|>{\centering}X|>{\centering}X|}
% \begin{tabularx}{\linewidth}{|X|X|X|X|X|}
% \hline
% \emph{nom}       & \emph{symbole}  & \emph{protons} & \emph{neutrons}
% & \emph{nucl�ons} \tbnl
% carbone   & \noyau{C}{6}{14}   &         &          & \rule[-0.5cm]{0cm}{1cm}         \tbnl
% fluor     & \noyau{F}{9}{19}   &         &          & \rule[-0.5cm]{0cm}{1cm}         \tbnl
% sodium    & \noyau{Na}{11}{23} &         &          & \rule[-0.5cm]{0cm}{1cm}         \tbnl
% oxyg�ne   & \noyau{O}{8}{16}   &         &          & \rule[-0.5cm]{0cm}{1cm}         \tbnl
% hydrog�ne &          &         & 0        & \rule[-0.5cm]{0cm}{1cm}         \tbnl
%           & \noyau{Cl}{17}{35} &         &          &  \rule[-0.5cm]{0cm}{1cm}        \tbnl
%           &          & 8       & \rule[-0.5cm]{0cm}{1cm}         & 16       \tbnl
% \end{tabularx}



\begin{tabularx}{\linewidth}{|>{\mystrut}X|X|X|X|X|}
\hline
% multicolumn pour faire dispara�tre le \mystrut
\multicolumn{1}{|X|}{\emph{nom}} & \emph{symbole}  &
\emph{protons} & \emph{neutrons} & \emph{nucl�ons} \tbnl
carbone   & \noyau{C}{6}{14}   &   &   &    \tbnl
fluor     & \noyau{F}{9}{19}   &   &   &    \tbnl
sodium    & \noyau{Na}{11}{23} &   &   &    \tbnl
oxyg�ne   & \noyau{O}{8}{16}   &   &   &    \tbnl
hydrog�ne &                    &   & 0 &    \tbnl
          & \noyau{Cl}{17}{35} &   &   &    \tbnl
          &                    & 8 &   & 16 \tbnl
\end{tabularx}


\end{exercice}


\vressort{3}


\begin{exercice}{Masse d'un atome de carbone 12}\\
Soit le carbone $12$ not� \noyau{C}{6}{12}.
\begin{enumerate}
\item L'�l�ment carbone peut-il avoir $5$ protons ? Pourquoi ?
\item Calculer la masse du noyau d'un atome de carbone $12$
sachant que la masse d'un nucl�on est $m_n = 1,67.10^{-27}~kg$
\item Calculer la masse des �lectrons de l'atome de carbone 12
sachant que la masse d'un �lectron vaut $m_e = 9,1.10^{-31}~kg$
\item Comparer la masse des �lectrons de l'atome � la masse du noyau.
Que concluez-vous ?
\item En d�duire, sans nouveau calcul, la masse de l'atome de carbone
  $12$.
\end{enumerate}
\end{exercice}

\newpage

\vressort{1}

\begin{exercice}{Couches �lectroniques}\\
Dans l'�tat le plus stable de l'atome, appel� �tat fondamental,
les �lectrons occupent successivement les couches,
en commen�ant par celles qui sont les plus proches du noyau : 
d'abord $K$ puis $L$ puis $M$.

Lorsqu'une couche est pleine, ou encore satur�e, on passe � la suivante.

La derni�re couche occup�e est appel�e couche externe.\\
Toutes les autres sont appel�es couches internes.

\medskip

\noindent
%\begin{tabularx}{\textwidth}{|>{\centering}X|>{\centering}X|>{\centering}X|>{\centering}X|}
\begin{tabularx}{\textwidth}{|>{\mystrut}X|X|X|X|}
\hline
\multicolumn{1}{|X|}{\emph{Symbole de la couche}}       & $K$ & $L$ & $M$  \tbnl
\emph{Nombre maximal d'�lectrons} & $2$ & $8$ & $18$ \tbnl
\end{tabularx}

\medskip

Ainsi, par exemple, l'atome de chlore ($Z=17$) a la configuration �lectronique :
$(K)^2(L)^8(M)^{7}$.

\begin{enumerate}
\item Indiquez le nombre d'�lectrons et donnez la configuration des atomes suivants :
  \begin{enumerate}
  \item \noyau{H}{1}{}
  \item \noyau{O}{8}{}
  \item \noyau{C}{6}{}
  \item \noyau{Ne}{10}{}
  \end{enumerate}
\item Parmi les ions ci-desssous, pr�cisez s'il s'agit d'anions ou de cations.
Indiquez le nombre d'�lectrons et donnez la configuration �lectronique
des ions suivants :
  \begin{enumerate}
  \item $Be^{2+}$ ($Z=4$)
  \item $Al^{3+}$ ($Z=13$)
  \item $O^{2-}$ ($Z=8$)
  \item $F^{-}$ ($Z=9$)
  \end{enumerate}
\end{enumerate}

\end{exercice}


\vressort{5}


\begin{exercice}{Charge d'un atome de Zinc}%\\
\begin{enumerate}
\item Combien de protons l'atome de zinc \noyau{Zn}{30}{65} contient-il ?
\item Combien d'�lectrons comporte-t-il ?
\item Calculer la charge totale des protons
sachant qu'un proton a pour charge $e = 1,6.10^{-19}~C$.
\item Calculer la charge totale des �lectrons
sachant qu'un �lectron a pour charge $-e = -1,6.10^{-19}~C$.
\item En d�duire la charge de l'atome de Zinc.
\item Ce r�sultat est-il identique pour tous les atomes ?
\item A l'issue d'une r�action dite d'oxydation, un atome de zinc $Zn$
  se transforme en un ion $Zn^{2+}$.
  \begin{enumerate}
  \item Donnez l'�quation de cette r�action
  (en faisant intervenir un ou plusieurs �lectrons not�s $e^-$).
  \item Indiquez la charge (en coulomb $C$) de cet ion.
  \end{enumerate}
\end{enumerate}
\end{exercice}

\vressort{3}
\ds{Devoir Surveill�}{
%
}

\nomprenomclasse

\setcounter{numexercice}{0}

%\renewcommand{\tabularx}[1]{>{\centering}m{#1}} 

%\newcommand{\tabularxc}[1]{\tabularx{>{\centering}m{#1}}}

\vressort{3}

\begin{exercice}{Connaissance sur l'atome}%\\
\begin{enumerate}
\item De quoi est compos� un atome ?
\item Que signifie les lettres $A$, $Z$ et $X$ dans la repr�sentation \noyau{X}{Z}{A} ?
\item Comment trouve-t-on le nombre de neutrons d'un atome de l'�l�ment pr�c�dent.
\item Si un atome a $5$ protons, combien-a-t-il d'�lectrons ? Pourquoi ?
\item Qu'est-ce qui caract�rise un �l�ment chimique ?
\item Qu'est-ce qu'un isotope ?
\end{enumerate}
\end{exercice}



\vressort{3}



\begin{exercice}{Composition des atomes}\\
En vous aidant du tableau p�riodique des �l�ments,
compl�ter le tableau suivant :

\medskip

\noindent
%\begin{tabularx}{\textwidth}{|>{\centering}X|>{\centering}X|>{\centering}X|>{\centering}X|>{\centering}X|}
% \begin{tabularx}{\linewidth}{|X|X|X|X|X|}
% \hline
% \emph{nom}       & \emph{symbole}  & \emph{protons} & \emph{neutrons}
% & \emph{nucl�ons} \tbnl
% carbone   & \noyau{C}{6}{14}   &         &          & \rule[-0.5cm]{0cm}{1cm}         \tbnl
% fluor     & \noyau{F}{9}{19}   &         &          & \rule[-0.5cm]{0cm}{1cm}         \tbnl
% sodium    & \noyau{Na}{11}{23} &         &          & \rule[-0.5cm]{0cm}{1cm}         \tbnl
% oxyg�ne   & \noyau{O}{8}{16}   &         &          & \rule[-0.5cm]{0cm}{1cm}         \tbnl
% hydrog�ne &          &         & 0        & \rule[-0.5cm]{0cm}{1cm}         \tbnl
%           & \noyau{Cl}{17}{35} &         &          &  \rule[-0.5cm]{0cm}{1cm}        \tbnl
%           &          & 8       & \rule[-0.5cm]{0cm}{1cm}         & 16       \tbnl
% \end{tabularx}



\begin{tabularx}{\linewidth}{|>{\mystrut}X|X|X|X|X|}
\hline
% multicolumn pour faire dispara�tre le \mystrut
\multicolumn{1}{|X|}{\emph{nom}} & \emph{symbole}  &
\emph{protons} & \emph{neutrons} & \emph{nucl�ons} \tbnl
carbone   & \noyau{C}{6}{14}   &   &   &    \tbnl
fluor     & \noyau{F}{9}{19}   &   &   &    \tbnl
sodium    & \noyau{Na}{11}{23} &   &   &    \tbnl
oxyg�ne   & \noyau{O}{8}{16}   &   &   &    \tbnl
hydrog�ne &                    &   & 0 &    \tbnl
          & \noyau{Cl}{17}{35} &   &   &    \tbnl
          &                    & 8 &   & 16 \tbnl
\end{tabularx}


\end{exercice}


\vressort{3}


\begin{exercice}{Masse d'un atome de carbone 12}\\
Soit le carbone $12$ not� \noyau{C}{6}{12}.
\begin{enumerate}
\item L'�l�ment carbone peut-il avoir $5$ protons ? Pourquoi ?
\item Calculer la masse du noyau d'un atome de carbone $12$
sachant que la masse d'un nucl�on est $m_n = 1,67.10^{-27}~kg$
\item Calculer la masse des �lectrons de l'atome de carbone 12
sachant que la masse d'un �lectron vaut $m_e = 9,1.10^{-31}~kg$
\item Comparer la masse des �lectrons de l'atome � la masse du noyau.
Que concluez-vous ?
\item En d�duire, sans nouveau calcul, la masse de l'atome de carbone
  $12$.
\end{enumerate}
\end{exercice}

\newpage

\vressort{1}

\begin{exercice}{Couches �lectroniques}\\
Dans l'�tat le plus stable de l'atome, appel� �tat fondamental,
les �lectrons occupent successivement les couches,
en commen�ant par celles qui sont les plus proches du noyau : 
d'abord $K$ puis $L$ puis $M$.

Lorsqu'une couche est pleine, ou encore satur�e, on passe � la suivante.

La derni�re couche occup�e est appel�e couche externe.\\
Toutes les autres sont appel�es couches internes.

\medskip

\noindent
%\begin{tabularx}{\textwidth}{|>{\centering}X|>{\centering}X|>{\centering}X|>{\centering}X|}
\begin{tabularx}{\textwidth}{|>{\mystrut}X|X|X|X|}
\hline
\multicolumn{1}{|X|}{\emph{Symbole de la couche}}       & $K$ & $L$ & $M$  \tbnl
\emph{Nombre maximal d'�lectrons} & $2$ & $8$ & $18$ \tbnl
\end{tabularx}

\medskip

Ainsi, par exemple, l'atome de chlore ($Z=17$) a la configuration �lectronique :
$(K)^2(L)^8(M)^{7}$.

\begin{enumerate}
\item Indiquez le nombre d'�lectrons et donnez la configuration des atomes suivants :
  \begin{enumerate}
  \item \noyau{H}{1}{}
  \item \noyau{O}{8}{}
  \item \noyau{C}{6}{}
  \item \noyau{Ne}{10}{}
  \end{enumerate}
\item Parmi les ions ci-desssous, pr�cisez s'il s'agit d'anions ou de cations.
Indiquez le nombre d'�lectrons et donnez la configuration �lectronique
des ions suivants :
  \begin{enumerate}
  \item $Be^{2+}$ ($Z=4$)
  \item $Al^{3+}$ ($Z=13$)
  \item $O^{2-}$ ($Z=8$)
  \item $F^{-}$ ($Z=9$)
  \end{enumerate}
\end{enumerate}

\end{exercice}


\vressort{5}


\begin{exercice}{Charge d'un atome de Zinc}%\\
\begin{enumerate}
\item Combien de protons l'atome de zinc \noyau{Zn}{30}{65} contient-il ?
\item Combien d'�lectrons comporte-t-il ?
\item Calculer la charge totale des protons
sachant qu'un proton a pour charge $e = 1,6.10^{-19}~C$.
\item Calculer la charge totale des �lectrons
sachant qu'un �lectron a pour charge $-e = -1,6.10^{-19}~C$.
\item En d�duire la charge de l'atome de Zinc.
\item Ce r�sultat est-il identique pour tous les atomes ?
\item A l'issue d'une r�action dite d'oxydation, un atome de zinc $Zn$
  se transforme en un ion $Zn^{2+}$.
  \begin{enumerate}
  \item Donnez l'�quation de cette r�action
  (en faisant intervenir un ou plusieurs �lectrons not�s $e^-$).
  \item Indiquez la charge (en coulomb $C$) de cet ion.
  \end{enumerate}
\end{enumerate}
\end{exercice}

\vressort{3}


\chapitre{Lois de Newton}
\ds{Devoir Surveill�}{
%
}

\nomprenomclasse

\setcounter{numexercice}{0}

%\renewcommand{\tabularx}[1]{>{\centering}m{#1}} 

%\newcommand{\tabularxc}[1]{\tabularx{>{\centering}m{#1}}}

\vressort{3}

\begin{exercice}{Connaissance sur l'atome}%\\
\begin{enumerate}
\item De quoi est compos� un atome ?
\item Que signifie les lettres $A$, $Z$ et $X$ dans la repr�sentation \noyau{X}{Z}{A} ?
\item Comment trouve-t-on le nombre de neutrons d'un atome de l'�l�ment pr�c�dent.
\item Si un atome a $5$ protons, combien-a-t-il d'�lectrons ? Pourquoi ?
\item Qu'est-ce qui caract�rise un �l�ment chimique ?
\item Qu'est-ce qu'un isotope ?
\end{enumerate}
\end{exercice}



\vressort{3}



\begin{exercice}{Composition des atomes}\\
En vous aidant du tableau p�riodique des �l�ments,
compl�ter le tableau suivant :

\medskip

\noindent
%\begin{tabularx}{\textwidth}{|>{\centering}X|>{\centering}X|>{\centering}X|>{\centering}X|>{\centering}X|}
% \begin{tabularx}{\linewidth}{|X|X|X|X|X|}
% \hline
% \emph{nom}       & \emph{symbole}  & \emph{protons} & \emph{neutrons}
% & \emph{nucl�ons} \tbnl
% carbone   & \noyau{C}{6}{14}   &         &          & \rule[-0.5cm]{0cm}{1cm}         \tbnl
% fluor     & \noyau{F}{9}{19}   &         &          & \rule[-0.5cm]{0cm}{1cm}         \tbnl
% sodium    & \noyau{Na}{11}{23} &         &          & \rule[-0.5cm]{0cm}{1cm}         \tbnl
% oxyg�ne   & \noyau{O}{8}{16}   &         &          & \rule[-0.5cm]{0cm}{1cm}         \tbnl
% hydrog�ne &          &         & 0        & \rule[-0.5cm]{0cm}{1cm}         \tbnl
%           & \noyau{Cl}{17}{35} &         &          &  \rule[-0.5cm]{0cm}{1cm}        \tbnl
%           &          & 8       & \rule[-0.5cm]{0cm}{1cm}         & 16       \tbnl
% \end{tabularx}



\begin{tabularx}{\linewidth}{|>{\mystrut}X|X|X|X|X|}
\hline
% multicolumn pour faire dispara�tre le \mystrut
\multicolumn{1}{|X|}{\emph{nom}} & \emph{symbole}  &
\emph{protons} & \emph{neutrons} & \emph{nucl�ons} \tbnl
carbone   & \noyau{C}{6}{14}   &   &   &    \tbnl
fluor     & \noyau{F}{9}{19}   &   &   &    \tbnl
sodium    & \noyau{Na}{11}{23} &   &   &    \tbnl
oxyg�ne   & \noyau{O}{8}{16}   &   &   &    \tbnl
hydrog�ne &                    &   & 0 &    \tbnl
          & \noyau{Cl}{17}{35} &   &   &    \tbnl
          &                    & 8 &   & 16 \tbnl
\end{tabularx}


\end{exercice}


\vressort{3}


\begin{exercice}{Masse d'un atome de carbone 12}\\
Soit le carbone $12$ not� \noyau{C}{6}{12}.
\begin{enumerate}
\item L'�l�ment carbone peut-il avoir $5$ protons ? Pourquoi ?
\item Calculer la masse du noyau d'un atome de carbone $12$
sachant que la masse d'un nucl�on est $m_n = 1,67.10^{-27}~kg$
\item Calculer la masse des �lectrons de l'atome de carbone 12
sachant que la masse d'un �lectron vaut $m_e = 9,1.10^{-31}~kg$
\item Comparer la masse des �lectrons de l'atome � la masse du noyau.
Que concluez-vous ?
\item En d�duire, sans nouveau calcul, la masse de l'atome de carbone
  $12$.
\end{enumerate}
\end{exercice}

\newpage

\vressort{1}

\begin{exercice}{Couches �lectroniques}\\
Dans l'�tat le plus stable de l'atome, appel� �tat fondamental,
les �lectrons occupent successivement les couches,
en commen�ant par celles qui sont les plus proches du noyau : 
d'abord $K$ puis $L$ puis $M$.

Lorsqu'une couche est pleine, ou encore satur�e, on passe � la suivante.

La derni�re couche occup�e est appel�e couche externe.\\
Toutes les autres sont appel�es couches internes.

\medskip

\noindent
%\begin{tabularx}{\textwidth}{|>{\centering}X|>{\centering}X|>{\centering}X|>{\centering}X|}
\begin{tabularx}{\textwidth}{|>{\mystrut}X|X|X|X|}
\hline
\multicolumn{1}{|X|}{\emph{Symbole de la couche}}       & $K$ & $L$ & $M$  \tbnl
\emph{Nombre maximal d'�lectrons} & $2$ & $8$ & $18$ \tbnl
\end{tabularx}

\medskip

Ainsi, par exemple, l'atome de chlore ($Z=17$) a la configuration �lectronique :
$(K)^2(L)^8(M)^{7}$.

\begin{enumerate}
\item Indiquez le nombre d'�lectrons et donnez la configuration des atomes suivants :
  \begin{enumerate}
  \item \noyau{H}{1}{}
  \item \noyau{O}{8}{}
  \item \noyau{C}{6}{}
  \item \noyau{Ne}{10}{}
  \end{enumerate}
\item Parmi les ions ci-desssous, pr�cisez s'il s'agit d'anions ou de cations.
Indiquez le nombre d'�lectrons et donnez la configuration �lectronique
des ions suivants :
  \begin{enumerate}
  \item $Be^{2+}$ ($Z=4$)
  \item $Al^{3+}$ ($Z=13$)
  \item $O^{2-}$ ($Z=8$)
  \item $F^{-}$ ($Z=9$)
  \end{enumerate}
\end{enumerate}

\end{exercice}


\vressort{5}


\begin{exercice}{Charge d'un atome de Zinc}%\\
\begin{enumerate}
\item Combien de protons l'atome de zinc \noyau{Zn}{30}{65} contient-il ?
\item Combien d'�lectrons comporte-t-il ?
\item Calculer la charge totale des protons
sachant qu'un proton a pour charge $e = 1,6.10^{-19}~C$.
\item Calculer la charge totale des �lectrons
sachant qu'un �lectron a pour charge $-e = -1,6.10^{-19}~C$.
\item En d�duire la charge de l'atome de Zinc.
\item Ce r�sultat est-il identique pour tous les atomes ?
\item A l'issue d'une r�action dite d'oxydation, un atome de zinc $Zn$
  se transforme en un ion $Zn^{2+}$.
  \begin{enumerate}
  \item Donnez l'�quation de cette r�action
  (en faisant intervenir un ou plusieurs �lectrons not�s $e^-$).
  \item Indiquez la charge (en coulomb $C$) de cet ion.
  \end{enumerate}
\end{enumerate}
\end{exercice}

\vressort{3}  % Mettre les fig
\ds{Devoir Surveill�}{
%
}

\nomprenomclasse

\setcounter{numexercice}{0}

%\renewcommand{\tabularx}[1]{>{\centering}m{#1}} 

%\newcommand{\tabularxc}[1]{\tabularx{>{\centering}m{#1}}}

\vressort{3}

\begin{exercice}{Connaissance sur l'atome}%\\
\begin{enumerate}
\item De quoi est compos� un atome ?
\item Que signifie les lettres $A$, $Z$ et $X$ dans la repr�sentation \noyau{X}{Z}{A} ?
\item Comment trouve-t-on le nombre de neutrons d'un atome de l'�l�ment pr�c�dent.
\item Si un atome a $5$ protons, combien-a-t-il d'�lectrons ? Pourquoi ?
\item Qu'est-ce qui caract�rise un �l�ment chimique ?
\item Qu'est-ce qu'un isotope ?
\end{enumerate}
\end{exercice}



\vressort{3}



\begin{exercice}{Composition des atomes}\\
En vous aidant du tableau p�riodique des �l�ments,
compl�ter le tableau suivant :

\medskip

\noindent
%\begin{tabularx}{\textwidth}{|>{\centering}X|>{\centering}X|>{\centering}X|>{\centering}X|>{\centering}X|}
% \begin{tabularx}{\linewidth}{|X|X|X|X|X|}
% \hline
% \emph{nom}       & \emph{symbole}  & \emph{protons} & \emph{neutrons}
% & \emph{nucl�ons} \tbnl
% carbone   & \noyau{C}{6}{14}   &         &          & \rule[-0.5cm]{0cm}{1cm}         \tbnl
% fluor     & \noyau{F}{9}{19}   &         &          & \rule[-0.5cm]{0cm}{1cm}         \tbnl
% sodium    & \noyau{Na}{11}{23} &         &          & \rule[-0.5cm]{0cm}{1cm}         \tbnl
% oxyg�ne   & \noyau{O}{8}{16}   &         &          & \rule[-0.5cm]{0cm}{1cm}         \tbnl
% hydrog�ne &          &         & 0        & \rule[-0.5cm]{0cm}{1cm}         \tbnl
%           & \noyau{Cl}{17}{35} &         &          &  \rule[-0.5cm]{0cm}{1cm}        \tbnl
%           &          & 8       & \rule[-0.5cm]{0cm}{1cm}         & 16       \tbnl
% \end{tabularx}



\begin{tabularx}{\linewidth}{|>{\mystrut}X|X|X|X|X|}
\hline
% multicolumn pour faire dispara�tre le \mystrut
\multicolumn{1}{|X|}{\emph{nom}} & \emph{symbole}  &
\emph{protons} & \emph{neutrons} & \emph{nucl�ons} \tbnl
carbone   & \noyau{C}{6}{14}   &   &   &    \tbnl
fluor     & \noyau{F}{9}{19}   &   &   &    \tbnl
sodium    & \noyau{Na}{11}{23} &   &   &    \tbnl
oxyg�ne   & \noyau{O}{8}{16}   &   &   &    \tbnl
hydrog�ne &                    &   & 0 &    \tbnl
          & \noyau{Cl}{17}{35} &   &   &    \tbnl
          &                    & 8 &   & 16 \tbnl
\end{tabularx}


\end{exercice}


\vressort{3}


\begin{exercice}{Masse d'un atome de carbone 12}\\
Soit le carbone $12$ not� \noyau{C}{6}{12}.
\begin{enumerate}
\item L'�l�ment carbone peut-il avoir $5$ protons ? Pourquoi ?
\item Calculer la masse du noyau d'un atome de carbone $12$
sachant que la masse d'un nucl�on est $m_n = 1,67.10^{-27}~kg$
\item Calculer la masse des �lectrons de l'atome de carbone 12
sachant que la masse d'un �lectron vaut $m_e = 9,1.10^{-31}~kg$
\item Comparer la masse des �lectrons de l'atome � la masse du noyau.
Que concluez-vous ?
\item En d�duire, sans nouveau calcul, la masse de l'atome de carbone
  $12$.
\end{enumerate}
\end{exercice}

\newpage

\vressort{1}

\begin{exercice}{Couches �lectroniques}\\
Dans l'�tat le plus stable de l'atome, appel� �tat fondamental,
les �lectrons occupent successivement les couches,
en commen�ant par celles qui sont les plus proches du noyau : 
d'abord $K$ puis $L$ puis $M$.

Lorsqu'une couche est pleine, ou encore satur�e, on passe � la suivante.

La derni�re couche occup�e est appel�e couche externe.\\
Toutes les autres sont appel�es couches internes.

\medskip

\noindent
%\begin{tabularx}{\textwidth}{|>{\centering}X|>{\centering}X|>{\centering}X|>{\centering}X|}
\begin{tabularx}{\textwidth}{|>{\mystrut}X|X|X|X|}
\hline
\multicolumn{1}{|X|}{\emph{Symbole de la couche}}       & $K$ & $L$ & $M$  \tbnl
\emph{Nombre maximal d'�lectrons} & $2$ & $8$ & $18$ \tbnl
\end{tabularx}

\medskip

Ainsi, par exemple, l'atome de chlore ($Z=17$) a la configuration �lectronique :
$(K)^2(L)^8(M)^{7}$.

\begin{enumerate}
\item Indiquez le nombre d'�lectrons et donnez la configuration des atomes suivants :
  \begin{enumerate}
  \item \noyau{H}{1}{}
  \item \noyau{O}{8}{}
  \item \noyau{C}{6}{}
  \item \noyau{Ne}{10}{}
  \end{enumerate}
\item Parmi les ions ci-desssous, pr�cisez s'il s'agit d'anions ou de cations.
Indiquez le nombre d'�lectrons et donnez la configuration �lectronique
des ions suivants :
  \begin{enumerate}
  \item $Be^{2+}$ ($Z=4$)
  \item $Al^{3+}$ ($Z=13$)
  \item $O^{2-}$ ($Z=8$)
  \item $F^{-}$ ($Z=9$)
  \end{enumerate}
\end{enumerate}

\end{exercice}


\vressort{5}


\begin{exercice}{Charge d'un atome de Zinc}%\\
\begin{enumerate}
\item Combien de protons l'atome de zinc \noyau{Zn}{30}{65} contient-il ?
\item Combien d'�lectrons comporte-t-il ?
\item Calculer la charge totale des protons
sachant qu'un proton a pour charge $e = 1,6.10^{-19}~C$.
\item Calculer la charge totale des �lectrons
sachant qu'un �lectron a pour charge $-e = -1,6.10^{-19}~C$.
\item En d�duire la charge de l'atome de Zinc.
\item Ce r�sultat est-il identique pour tous les atomes ?
\item A l'issue d'une r�action dite d'oxydation, un atome de zinc $Zn$
  se transforme en un ion $Zn^{2+}$.
  \begin{enumerate}
  \item Donnez l'�quation de cette r�action
  (en faisant intervenir un ou plusieurs �lectrons not�s $e^-$).
  \item Indiquez la charge (en coulomb $C$) de cet ion.
  \end{enumerate}
\end{enumerate}
\end{exercice}

\vressort{3}
\ds{Devoir Surveill�}{
%
}

\nomprenomclasse

\setcounter{numexercice}{0}

%\renewcommand{\tabularx}[1]{>{\centering}m{#1}} 

%\newcommand{\tabularxc}[1]{\tabularx{>{\centering}m{#1}}}

\vressort{3}

\begin{exercice}{Connaissance sur l'atome}%\\
\begin{enumerate}
\item De quoi est compos� un atome ?
\item Que signifie les lettres $A$, $Z$ et $X$ dans la repr�sentation \noyau{X}{Z}{A} ?
\item Comment trouve-t-on le nombre de neutrons d'un atome de l'�l�ment pr�c�dent.
\item Si un atome a $5$ protons, combien-a-t-il d'�lectrons ? Pourquoi ?
\item Qu'est-ce qui caract�rise un �l�ment chimique ?
\item Qu'est-ce qu'un isotope ?
\end{enumerate}
\end{exercice}



\vressort{3}



\begin{exercice}{Composition des atomes}\\
En vous aidant du tableau p�riodique des �l�ments,
compl�ter le tableau suivant :

\medskip

\noindent
%\begin{tabularx}{\textwidth}{|>{\centering}X|>{\centering}X|>{\centering}X|>{\centering}X|>{\centering}X|}
% \begin{tabularx}{\linewidth}{|X|X|X|X|X|}
% \hline
% \emph{nom}       & \emph{symbole}  & \emph{protons} & \emph{neutrons}
% & \emph{nucl�ons} \tbnl
% carbone   & \noyau{C}{6}{14}   &         &          & \rule[-0.5cm]{0cm}{1cm}         \tbnl
% fluor     & \noyau{F}{9}{19}   &         &          & \rule[-0.5cm]{0cm}{1cm}         \tbnl
% sodium    & \noyau{Na}{11}{23} &         &          & \rule[-0.5cm]{0cm}{1cm}         \tbnl
% oxyg�ne   & \noyau{O}{8}{16}   &         &          & \rule[-0.5cm]{0cm}{1cm}         \tbnl
% hydrog�ne &          &         & 0        & \rule[-0.5cm]{0cm}{1cm}         \tbnl
%           & \noyau{Cl}{17}{35} &         &          &  \rule[-0.5cm]{0cm}{1cm}        \tbnl
%           &          & 8       & \rule[-0.5cm]{0cm}{1cm}         & 16       \tbnl
% \end{tabularx}



\begin{tabularx}{\linewidth}{|>{\mystrut}X|X|X|X|X|}
\hline
% multicolumn pour faire dispara�tre le \mystrut
\multicolumn{1}{|X|}{\emph{nom}} & \emph{symbole}  &
\emph{protons} & \emph{neutrons} & \emph{nucl�ons} \tbnl
carbone   & \noyau{C}{6}{14}   &   &   &    \tbnl
fluor     & \noyau{F}{9}{19}   &   &   &    \tbnl
sodium    & \noyau{Na}{11}{23} &   &   &    \tbnl
oxyg�ne   & \noyau{O}{8}{16}   &   &   &    \tbnl
hydrog�ne &                    &   & 0 &    \tbnl
          & \noyau{Cl}{17}{35} &   &   &    \tbnl
          &                    & 8 &   & 16 \tbnl
\end{tabularx}


\end{exercice}


\vressort{3}


\begin{exercice}{Masse d'un atome de carbone 12}\\
Soit le carbone $12$ not� \noyau{C}{6}{12}.
\begin{enumerate}
\item L'�l�ment carbone peut-il avoir $5$ protons ? Pourquoi ?
\item Calculer la masse du noyau d'un atome de carbone $12$
sachant que la masse d'un nucl�on est $m_n = 1,67.10^{-27}~kg$
\item Calculer la masse des �lectrons de l'atome de carbone 12
sachant que la masse d'un �lectron vaut $m_e = 9,1.10^{-31}~kg$
\item Comparer la masse des �lectrons de l'atome � la masse du noyau.
Que concluez-vous ?
\item En d�duire, sans nouveau calcul, la masse de l'atome de carbone
  $12$.
\end{enumerate}
\end{exercice}

\newpage

\vressort{1}

\begin{exercice}{Couches �lectroniques}\\
Dans l'�tat le plus stable de l'atome, appel� �tat fondamental,
les �lectrons occupent successivement les couches,
en commen�ant par celles qui sont les plus proches du noyau : 
d'abord $K$ puis $L$ puis $M$.

Lorsqu'une couche est pleine, ou encore satur�e, on passe � la suivante.

La derni�re couche occup�e est appel�e couche externe.\\
Toutes les autres sont appel�es couches internes.

\medskip

\noindent
%\begin{tabularx}{\textwidth}{|>{\centering}X|>{\centering}X|>{\centering}X|>{\centering}X|}
\begin{tabularx}{\textwidth}{|>{\mystrut}X|X|X|X|}
\hline
\multicolumn{1}{|X|}{\emph{Symbole de la couche}}       & $K$ & $L$ & $M$  \tbnl
\emph{Nombre maximal d'�lectrons} & $2$ & $8$ & $18$ \tbnl
\end{tabularx}

\medskip

Ainsi, par exemple, l'atome de chlore ($Z=17$) a la configuration �lectronique :
$(K)^2(L)^8(M)^{7}$.

\begin{enumerate}
\item Indiquez le nombre d'�lectrons et donnez la configuration des atomes suivants :
  \begin{enumerate}
  \item \noyau{H}{1}{}
  \item \noyau{O}{8}{}
  \item \noyau{C}{6}{}
  \item \noyau{Ne}{10}{}
  \end{enumerate}
\item Parmi les ions ci-desssous, pr�cisez s'il s'agit d'anions ou de cations.
Indiquez le nombre d'�lectrons et donnez la configuration �lectronique
des ions suivants :
  \begin{enumerate}
  \item $Be^{2+}$ ($Z=4$)
  \item $Al^{3+}$ ($Z=13$)
  \item $O^{2-}$ ($Z=8$)
  \item $F^{-}$ ($Z=9$)
  \end{enumerate}
\end{enumerate}

\end{exercice}


\vressort{5}


\begin{exercice}{Charge d'un atome de Zinc}%\\
\begin{enumerate}
\item Combien de protons l'atome de zinc \noyau{Zn}{30}{65} contient-il ?
\item Combien d'�lectrons comporte-t-il ?
\item Calculer la charge totale des protons
sachant qu'un proton a pour charge $e = 1,6.10^{-19}~C$.
\item Calculer la charge totale des �lectrons
sachant qu'un �lectron a pour charge $-e = -1,6.10^{-19}~C$.
\item En d�duire la charge de l'atome de Zinc.
\item Ce r�sultat est-il identique pour tous les atomes ?
\item A l'issue d'une r�action dite d'oxydation, un atome de zinc $Zn$
  se transforme en un ion $Zn^{2+}$.
  \begin{enumerate}
  \item Donnez l'�quation de cette r�action
  (en faisant intervenir un ou plusieurs �lectrons not�s $e^-$).
  \item Indiquez la charge (en coulomb $C$) de cet ion.
  \end{enumerate}
\end{enumerate}
\end{exercice}

\vressort{3}


\chapitre{Travail d'une force}
\ds{Devoir Surveill�}{
%
}

\nomprenomclasse

\setcounter{numexercice}{0}

%\renewcommand{\tabularx}[1]{>{\centering}m{#1}} 

%\newcommand{\tabularxc}[1]{\tabularx{>{\centering}m{#1}}}

\vressort{3}

\begin{exercice}{Connaissance sur l'atome}%\\
\begin{enumerate}
\item De quoi est compos� un atome ?
\item Que signifie les lettres $A$, $Z$ et $X$ dans la repr�sentation \noyau{X}{Z}{A} ?
\item Comment trouve-t-on le nombre de neutrons d'un atome de l'�l�ment pr�c�dent.
\item Si un atome a $5$ protons, combien-a-t-il d'�lectrons ? Pourquoi ?
\item Qu'est-ce qui caract�rise un �l�ment chimique ?
\item Qu'est-ce qu'un isotope ?
\end{enumerate}
\end{exercice}



\vressort{3}



\begin{exercice}{Composition des atomes}\\
En vous aidant du tableau p�riodique des �l�ments,
compl�ter le tableau suivant :

\medskip

\noindent
%\begin{tabularx}{\textwidth}{|>{\centering}X|>{\centering}X|>{\centering}X|>{\centering}X|>{\centering}X|}
% \begin{tabularx}{\linewidth}{|X|X|X|X|X|}
% \hline
% \emph{nom}       & \emph{symbole}  & \emph{protons} & \emph{neutrons}
% & \emph{nucl�ons} \tbnl
% carbone   & \noyau{C}{6}{14}   &         &          & \rule[-0.5cm]{0cm}{1cm}         \tbnl
% fluor     & \noyau{F}{9}{19}   &         &          & \rule[-0.5cm]{0cm}{1cm}         \tbnl
% sodium    & \noyau{Na}{11}{23} &         &          & \rule[-0.5cm]{0cm}{1cm}         \tbnl
% oxyg�ne   & \noyau{O}{8}{16}   &         &          & \rule[-0.5cm]{0cm}{1cm}         \tbnl
% hydrog�ne &          &         & 0        & \rule[-0.5cm]{0cm}{1cm}         \tbnl
%           & \noyau{Cl}{17}{35} &         &          &  \rule[-0.5cm]{0cm}{1cm}        \tbnl
%           &          & 8       & \rule[-0.5cm]{0cm}{1cm}         & 16       \tbnl
% \end{tabularx}



\begin{tabularx}{\linewidth}{|>{\mystrut}X|X|X|X|X|}
\hline
% multicolumn pour faire dispara�tre le \mystrut
\multicolumn{1}{|X|}{\emph{nom}} & \emph{symbole}  &
\emph{protons} & \emph{neutrons} & \emph{nucl�ons} \tbnl
carbone   & \noyau{C}{6}{14}   &   &   &    \tbnl
fluor     & \noyau{F}{9}{19}   &   &   &    \tbnl
sodium    & \noyau{Na}{11}{23} &   &   &    \tbnl
oxyg�ne   & \noyau{O}{8}{16}   &   &   &    \tbnl
hydrog�ne &                    &   & 0 &    \tbnl
          & \noyau{Cl}{17}{35} &   &   &    \tbnl
          &                    & 8 &   & 16 \tbnl
\end{tabularx}


\end{exercice}


\vressort{3}


\begin{exercice}{Masse d'un atome de carbone 12}\\
Soit le carbone $12$ not� \noyau{C}{6}{12}.
\begin{enumerate}
\item L'�l�ment carbone peut-il avoir $5$ protons ? Pourquoi ?
\item Calculer la masse du noyau d'un atome de carbone $12$
sachant que la masse d'un nucl�on est $m_n = 1,67.10^{-27}~kg$
\item Calculer la masse des �lectrons de l'atome de carbone 12
sachant que la masse d'un �lectron vaut $m_e = 9,1.10^{-31}~kg$
\item Comparer la masse des �lectrons de l'atome � la masse du noyau.
Que concluez-vous ?
\item En d�duire, sans nouveau calcul, la masse de l'atome de carbone
  $12$.
\end{enumerate}
\end{exercice}

\newpage

\vressort{1}

\begin{exercice}{Couches �lectroniques}\\
Dans l'�tat le plus stable de l'atome, appel� �tat fondamental,
les �lectrons occupent successivement les couches,
en commen�ant par celles qui sont les plus proches du noyau : 
d'abord $K$ puis $L$ puis $M$.

Lorsqu'une couche est pleine, ou encore satur�e, on passe � la suivante.

La derni�re couche occup�e est appel�e couche externe.\\
Toutes les autres sont appel�es couches internes.

\medskip

\noindent
%\begin{tabularx}{\textwidth}{|>{\centering}X|>{\centering}X|>{\centering}X|>{\centering}X|}
\begin{tabularx}{\textwidth}{|>{\mystrut}X|X|X|X|}
\hline
\multicolumn{1}{|X|}{\emph{Symbole de la couche}}       & $K$ & $L$ & $M$  \tbnl
\emph{Nombre maximal d'�lectrons} & $2$ & $8$ & $18$ \tbnl
\end{tabularx}

\medskip

Ainsi, par exemple, l'atome de chlore ($Z=17$) a la configuration �lectronique :
$(K)^2(L)^8(M)^{7}$.

\begin{enumerate}
\item Indiquez le nombre d'�lectrons et donnez la configuration des atomes suivants :
  \begin{enumerate}
  \item \noyau{H}{1}{}
  \item \noyau{O}{8}{}
  \item \noyau{C}{6}{}
  \item \noyau{Ne}{10}{}
  \end{enumerate}
\item Parmi les ions ci-desssous, pr�cisez s'il s'agit d'anions ou de cations.
Indiquez le nombre d'�lectrons et donnez la configuration �lectronique
des ions suivants :
  \begin{enumerate}
  \item $Be^{2+}$ ($Z=4$)
  \item $Al^{3+}$ ($Z=13$)
  \item $O^{2-}$ ($Z=8$)
  \item $F^{-}$ ($Z=9$)
  \end{enumerate}
\end{enumerate}

\end{exercice}


\vressort{5}


\begin{exercice}{Charge d'un atome de Zinc}%\\
\begin{enumerate}
\item Combien de protons l'atome de zinc \noyau{Zn}{30}{65} contient-il ?
\item Combien d'�lectrons comporte-t-il ?
\item Calculer la charge totale des protons
sachant qu'un proton a pour charge $e = 1,6.10^{-19}~C$.
\item Calculer la charge totale des �lectrons
sachant qu'un �lectron a pour charge $-e = -1,6.10^{-19}~C$.
\item En d�duire la charge de l'atome de Zinc.
\item Ce r�sultat est-il identique pour tous les atomes ?
\item A l'issue d'une r�action dite d'oxydation, un atome de zinc $Zn$
  se transforme en un ion $Zn^{2+}$.
  \begin{enumerate}
  \item Donnez l'�quation de cette r�action
  (en faisant intervenir un ou plusieurs �lectrons not�s $e^-$).
  \item Indiquez la charge (en coulomb $C$) de cet ion.
  \end{enumerate}
\end{enumerate}
\end{exercice}

\vressort{3}  % Mettre les fig


\chapitre{\'Energie cin�tique}
\ds{Devoir Surveill�}{
%
}

\nomprenomclasse

\setcounter{numexercice}{0}

%\renewcommand{\tabularx}[1]{>{\centering}m{#1}} 

%\newcommand{\tabularxc}[1]{\tabularx{>{\centering}m{#1}}}

\vressort{3}

\begin{exercice}{Connaissance sur l'atome}%\\
\begin{enumerate}
\item De quoi est compos� un atome ?
\item Que signifie les lettres $A$, $Z$ et $X$ dans la repr�sentation \noyau{X}{Z}{A} ?
\item Comment trouve-t-on le nombre de neutrons d'un atome de l'�l�ment pr�c�dent.
\item Si un atome a $5$ protons, combien-a-t-il d'�lectrons ? Pourquoi ?
\item Qu'est-ce qui caract�rise un �l�ment chimique ?
\item Qu'est-ce qu'un isotope ?
\end{enumerate}
\end{exercice}



\vressort{3}



\begin{exercice}{Composition des atomes}\\
En vous aidant du tableau p�riodique des �l�ments,
compl�ter le tableau suivant :

\medskip

\noindent
%\begin{tabularx}{\textwidth}{|>{\centering}X|>{\centering}X|>{\centering}X|>{\centering}X|>{\centering}X|}
% \begin{tabularx}{\linewidth}{|X|X|X|X|X|}
% \hline
% \emph{nom}       & \emph{symbole}  & \emph{protons} & \emph{neutrons}
% & \emph{nucl�ons} \tbnl
% carbone   & \noyau{C}{6}{14}   &         &          & \rule[-0.5cm]{0cm}{1cm}         \tbnl
% fluor     & \noyau{F}{9}{19}   &         &          & \rule[-0.5cm]{0cm}{1cm}         \tbnl
% sodium    & \noyau{Na}{11}{23} &         &          & \rule[-0.5cm]{0cm}{1cm}         \tbnl
% oxyg�ne   & \noyau{O}{8}{16}   &         &          & \rule[-0.5cm]{0cm}{1cm}         \tbnl
% hydrog�ne &          &         & 0        & \rule[-0.5cm]{0cm}{1cm}         \tbnl
%           & \noyau{Cl}{17}{35} &         &          &  \rule[-0.5cm]{0cm}{1cm}        \tbnl
%           &          & 8       & \rule[-0.5cm]{0cm}{1cm}         & 16       \tbnl
% \end{tabularx}



\begin{tabularx}{\linewidth}{|>{\mystrut}X|X|X|X|X|}
\hline
% multicolumn pour faire dispara�tre le \mystrut
\multicolumn{1}{|X|}{\emph{nom}} & \emph{symbole}  &
\emph{protons} & \emph{neutrons} & \emph{nucl�ons} \tbnl
carbone   & \noyau{C}{6}{14}   &   &   &    \tbnl
fluor     & \noyau{F}{9}{19}   &   &   &    \tbnl
sodium    & \noyau{Na}{11}{23} &   &   &    \tbnl
oxyg�ne   & \noyau{O}{8}{16}   &   &   &    \tbnl
hydrog�ne &                    &   & 0 &    \tbnl
          & \noyau{Cl}{17}{35} &   &   &    \tbnl
          &                    & 8 &   & 16 \tbnl
\end{tabularx}


\end{exercice}


\vressort{3}


\begin{exercice}{Masse d'un atome de carbone 12}\\
Soit le carbone $12$ not� \noyau{C}{6}{12}.
\begin{enumerate}
\item L'�l�ment carbone peut-il avoir $5$ protons ? Pourquoi ?
\item Calculer la masse du noyau d'un atome de carbone $12$
sachant que la masse d'un nucl�on est $m_n = 1,67.10^{-27}~kg$
\item Calculer la masse des �lectrons de l'atome de carbone 12
sachant que la masse d'un �lectron vaut $m_e = 9,1.10^{-31}~kg$
\item Comparer la masse des �lectrons de l'atome � la masse du noyau.
Que concluez-vous ?
\item En d�duire, sans nouveau calcul, la masse de l'atome de carbone
  $12$.
\end{enumerate}
\end{exercice}

\newpage

\vressort{1}

\begin{exercice}{Couches �lectroniques}\\
Dans l'�tat le plus stable de l'atome, appel� �tat fondamental,
les �lectrons occupent successivement les couches,
en commen�ant par celles qui sont les plus proches du noyau : 
d'abord $K$ puis $L$ puis $M$.

Lorsqu'une couche est pleine, ou encore satur�e, on passe � la suivante.

La derni�re couche occup�e est appel�e couche externe.\\
Toutes les autres sont appel�es couches internes.

\medskip

\noindent
%\begin{tabularx}{\textwidth}{|>{\centering}X|>{\centering}X|>{\centering}X|>{\centering}X|}
\begin{tabularx}{\textwidth}{|>{\mystrut}X|X|X|X|}
\hline
\multicolumn{1}{|X|}{\emph{Symbole de la couche}}       & $K$ & $L$ & $M$  \tbnl
\emph{Nombre maximal d'�lectrons} & $2$ & $8$ & $18$ \tbnl
\end{tabularx}

\medskip

Ainsi, par exemple, l'atome de chlore ($Z=17$) a la configuration �lectronique :
$(K)^2(L)^8(M)^{7}$.

\begin{enumerate}
\item Indiquez le nombre d'�lectrons et donnez la configuration des atomes suivants :
  \begin{enumerate}
  \item \noyau{H}{1}{}
  \item \noyau{O}{8}{}
  \item \noyau{C}{6}{}
  \item \noyau{Ne}{10}{}
  \end{enumerate}
\item Parmi les ions ci-desssous, pr�cisez s'il s'agit d'anions ou de cations.
Indiquez le nombre d'�lectrons et donnez la configuration �lectronique
des ions suivants :
  \begin{enumerate}
  \item $Be^{2+}$ ($Z=4$)
  \item $Al^{3+}$ ($Z=13$)
  \item $O^{2-}$ ($Z=8$)
  \item $F^{-}$ ($Z=9$)
  \end{enumerate}
\end{enumerate}

\end{exercice}


\vressort{5}


\begin{exercice}{Charge d'un atome de Zinc}%\\
\begin{enumerate}
\item Combien de protons l'atome de zinc \noyau{Zn}{30}{65} contient-il ?
\item Combien d'�lectrons comporte-t-il ?
\item Calculer la charge totale des protons
sachant qu'un proton a pour charge $e = 1,6.10^{-19}~C$.
\item Calculer la charge totale des �lectrons
sachant qu'un �lectron a pour charge $-e = -1,6.10^{-19}~C$.
\item En d�duire la charge de l'atome de Zinc.
\item Ce r�sultat est-il identique pour tous les atomes ?
\item A l'issue d'une r�action dite d'oxydation, un atome de zinc $Zn$
  se transforme en un ion $Zn^{2+}$.
  \begin{enumerate}
  \item Donnez l'�quation de cette r�action
  (en faisant intervenir un ou plusieurs �lectrons not�s $e^-$).
  \item Indiquez la charge (en coulomb $C$) de cet ion.
  \end{enumerate}
\end{enumerate}
\end{exercice}

\vressort{3}  % Mettre les fig
\ds{Devoir Surveill�}{
%
}

\nomprenomclasse

\setcounter{numexercice}{0}

%\renewcommand{\tabularx}[1]{>{\centering}m{#1}} 

%\newcommand{\tabularxc}[1]{\tabularx{>{\centering}m{#1}}}

\vressort{3}

\begin{exercice}{Connaissance sur l'atome}%\\
\begin{enumerate}
\item De quoi est compos� un atome ?
\item Que signifie les lettres $A$, $Z$ et $X$ dans la repr�sentation \noyau{X}{Z}{A} ?
\item Comment trouve-t-on le nombre de neutrons d'un atome de l'�l�ment pr�c�dent.
\item Si un atome a $5$ protons, combien-a-t-il d'�lectrons ? Pourquoi ?
\item Qu'est-ce qui caract�rise un �l�ment chimique ?
\item Qu'est-ce qu'un isotope ?
\end{enumerate}
\end{exercice}



\vressort{3}



\begin{exercice}{Composition des atomes}\\
En vous aidant du tableau p�riodique des �l�ments,
compl�ter le tableau suivant :

\medskip

\noindent
%\begin{tabularx}{\textwidth}{|>{\centering}X|>{\centering}X|>{\centering}X|>{\centering}X|>{\centering}X|}
% \begin{tabularx}{\linewidth}{|X|X|X|X|X|}
% \hline
% \emph{nom}       & \emph{symbole}  & \emph{protons} & \emph{neutrons}
% & \emph{nucl�ons} \tbnl
% carbone   & \noyau{C}{6}{14}   &         &          & \rule[-0.5cm]{0cm}{1cm}         \tbnl
% fluor     & \noyau{F}{9}{19}   &         &          & \rule[-0.5cm]{0cm}{1cm}         \tbnl
% sodium    & \noyau{Na}{11}{23} &         &          & \rule[-0.5cm]{0cm}{1cm}         \tbnl
% oxyg�ne   & \noyau{O}{8}{16}   &         &          & \rule[-0.5cm]{0cm}{1cm}         \tbnl
% hydrog�ne &          &         & 0        & \rule[-0.5cm]{0cm}{1cm}         \tbnl
%           & \noyau{Cl}{17}{35} &         &          &  \rule[-0.5cm]{0cm}{1cm}        \tbnl
%           &          & 8       & \rule[-0.5cm]{0cm}{1cm}         & 16       \tbnl
% \end{tabularx}



\begin{tabularx}{\linewidth}{|>{\mystrut}X|X|X|X|X|}
\hline
% multicolumn pour faire dispara�tre le \mystrut
\multicolumn{1}{|X|}{\emph{nom}} & \emph{symbole}  &
\emph{protons} & \emph{neutrons} & \emph{nucl�ons} \tbnl
carbone   & \noyau{C}{6}{14}   &   &   &    \tbnl
fluor     & \noyau{F}{9}{19}   &   &   &    \tbnl
sodium    & \noyau{Na}{11}{23} &   &   &    \tbnl
oxyg�ne   & \noyau{O}{8}{16}   &   &   &    \tbnl
hydrog�ne &                    &   & 0 &    \tbnl
          & \noyau{Cl}{17}{35} &   &   &    \tbnl
          &                    & 8 &   & 16 \tbnl
\end{tabularx}


\end{exercice}


\vressort{3}


\begin{exercice}{Masse d'un atome de carbone 12}\\
Soit le carbone $12$ not� \noyau{C}{6}{12}.
\begin{enumerate}
\item L'�l�ment carbone peut-il avoir $5$ protons ? Pourquoi ?
\item Calculer la masse du noyau d'un atome de carbone $12$
sachant que la masse d'un nucl�on est $m_n = 1,67.10^{-27}~kg$
\item Calculer la masse des �lectrons de l'atome de carbone 12
sachant que la masse d'un �lectron vaut $m_e = 9,1.10^{-31}~kg$
\item Comparer la masse des �lectrons de l'atome � la masse du noyau.
Que concluez-vous ?
\item En d�duire, sans nouveau calcul, la masse de l'atome de carbone
  $12$.
\end{enumerate}
\end{exercice}

\newpage

\vressort{1}

\begin{exercice}{Couches �lectroniques}\\
Dans l'�tat le plus stable de l'atome, appel� �tat fondamental,
les �lectrons occupent successivement les couches,
en commen�ant par celles qui sont les plus proches du noyau : 
d'abord $K$ puis $L$ puis $M$.

Lorsqu'une couche est pleine, ou encore satur�e, on passe � la suivante.

La derni�re couche occup�e est appel�e couche externe.\\
Toutes les autres sont appel�es couches internes.

\medskip

\noindent
%\begin{tabularx}{\textwidth}{|>{\centering}X|>{\centering}X|>{\centering}X|>{\centering}X|}
\begin{tabularx}{\textwidth}{|>{\mystrut}X|X|X|X|}
\hline
\multicolumn{1}{|X|}{\emph{Symbole de la couche}}       & $K$ & $L$ & $M$  \tbnl
\emph{Nombre maximal d'�lectrons} & $2$ & $8$ & $18$ \tbnl
\end{tabularx}

\medskip

Ainsi, par exemple, l'atome de chlore ($Z=17$) a la configuration �lectronique :
$(K)^2(L)^8(M)^{7}$.

\begin{enumerate}
\item Indiquez le nombre d'�lectrons et donnez la configuration des atomes suivants :
  \begin{enumerate}
  \item \noyau{H}{1}{}
  \item \noyau{O}{8}{}
  \item \noyau{C}{6}{}
  \item \noyau{Ne}{10}{}
  \end{enumerate}
\item Parmi les ions ci-desssous, pr�cisez s'il s'agit d'anions ou de cations.
Indiquez le nombre d'�lectrons et donnez la configuration �lectronique
des ions suivants :
  \begin{enumerate}
  \item $Be^{2+}$ ($Z=4$)
  \item $Al^{3+}$ ($Z=13$)
  \item $O^{2-}$ ($Z=8$)
  \item $F^{-}$ ($Z=9$)
  \end{enumerate}
\end{enumerate}

\end{exercice}


\vressort{5}


\begin{exercice}{Charge d'un atome de Zinc}%\\
\begin{enumerate}
\item Combien de protons l'atome de zinc \noyau{Zn}{30}{65} contient-il ?
\item Combien d'�lectrons comporte-t-il ?
\item Calculer la charge totale des protons
sachant qu'un proton a pour charge $e = 1,6.10^{-19}~C$.
\item Calculer la charge totale des �lectrons
sachant qu'un �lectron a pour charge $-e = -1,6.10^{-19}~C$.
\item En d�duire la charge de l'atome de Zinc.
\item Ce r�sultat est-il identique pour tous les atomes ?
\item A l'issue d'une r�action dite d'oxydation, un atome de zinc $Zn$
  se transforme en un ion $Zn^{2+}$.
  \begin{enumerate}
  \item Donnez l'�quation de cette r�action
  (en faisant intervenir un ou plusieurs �lectrons not�s $e^-$).
  \item Indiquez la charge (en coulomb $C$) de cet ion.
  \end{enumerate}
\end{enumerate}
\end{exercice}

\vressort{3}


\chapitre{\'Energie m�canique}
\ds{Devoir Surveill�}{
%
}

\nomprenomclasse

\setcounter{numexercice}{0}

%\renewcommand{\tabularx}[1]{>{\centering}m{#1}} 

%\newcommand{\tabularxc}[1]{\tabularx{>{\centering}m{#1}}}

\vressort{3}

\begin{exercice}{Connaissance sur l'atome}%\\
\begin{enumerate}
\item De quoi est compos� un atome ?
\item Que signifie les lettres $A$, $Z$ et $X$ dans la repr�sentation \noyau{X}{Z}{A} ?
\item Comment trouve-t-on le nombre de neutrons d'un atome de l'�l�ment pr�c�dent.
\item Si un atome a $5$ protons, combien-a-t-il d'�lectrons ? Pourquoi ?
\item Qu'est-ce qui caract�rise un �l�ment chimique ?
\item Qu'est-ce qu'un isotope ?
\end{enumerate}
\end{exercice}



\vressort{3}



\begin{exercice}{Composition des atomes}\\
En vous aidant du tableau p�riodique des �l�ments,
compl�ter le tableau suivant :

\medskip

\noindent
%\begin{tabularx}{\textwidth}{|>{\centering}X|>{\centering}X|>{\centering}X|>{\centering}X|>{\centering}X|}
% \begin{tabularx}{\linewidth}{|X|X|X|X|X|}
% \hline
% \emph{nom}       & \emph{symbole}  & \emph{protons} & \emph{neutrons}
% & \emph{nucl�ons} \tbnl
% carbone   & \noyau{C}{6}{14}   &         &          & \rule[-0.5cm]{0cm}{1cm}         \tbnl
% fluor     & \noyau{F}{9}{19}   &         &          & \rule[-0.5cm]{0cm}{1cm}         \tbnl
% sodium    & \noyau{Na}{11}{23} &         &          & \rule[-0.5cm]{0cm}{1cm}         \tbnl
% oxyg�ne   & \noyau{O}{8}{16}   &         &          & \rule[-0.5cm]{0cm}{1cm}         \tbnl
% hydrog�ne &          &         & 0        & \rule[-0.5cm]{0cm}{1cm}         \tbnl
%           & \noyau{Cl}{17}{35} &         &          &  \rule[-0.5cm]{0cm}{1cm}        \tbnl
%           &          & 8       & \rule[-0.5cm]{0cm}{1cm}         & 16       \tbnl
% \end{tabularx}



\begin{tabularx}{\linewidth}{|>{\mystrut}X|X|X|X|X|}
\hline
% multicolumn pour faire dispara�tre le \mystrut
\multicolumn{1}{|X|}{\emph{nom}} & \emph{symbole}  &
\emph{protons} & \emph{neutrons} & \emph{nucl�ons} \tbnl
carbone   & \noyau{C}{6}{14}   &   &   &    \tbnl
fluor     & \noyau{F}{9}{19}   &   &   &    \tbnl
sodium    & \noyau{Na}{11}{23} &   &   &    \tbnl
oxyg�ne   & \noyau{O}{8}{16}   &   &   &    \tbnl
hydrog�ne &                    &   & 0 &    \tbnl
          & \noyau{Cl}{17}{35} &   &   &    \tbnl
          &                    & 8 &   & 16 \tbnl
\end{tabularx}


\end{exercice}


\vressort{3}


\begin{exercice}{Masse d'un atome de carbone 12}\\
Soit le carbone $12$ not� \noyau{C}{6}{12}.
\begin{enumerate}
\item L'�l�ment carbone peut-il avoir $5$ protons ? Pourquoi ?
\item Calculer la masse du noyau d'un atome de carbone $12$
sachant que la masse d'un nucl�on est $m_n = 1,67.10^{-27}~kg$
\item Calculer la masse des �lectrons de l'atome de carbone 12
sachant que la masse d'un �lectron vaut $m_e = 9,1.10^{-31}~kg$
\item Comparer la masse des �lectrons de l'atome � la masse du noyau.
Que concluez-vous ?
\item En d�duire, sans nouveau calcul, la masse de l'atome de carbone
  $12$.
\end{enumerate}
\end{exercice}

\newpage

\vressort{1}

\begin{exercice}{Couches �lectroniques}\\
Dans l'�tat le plus stable de l'atome, appel� �tat fondamental,
les �lectrons occupent successivement les couches,
en commen�ant par celles qui sont les plus proches du noyau : 
d'abord $K$ puis $L$ puis $M$.

Lorsqu'une couche est pleine, ou encore satur�e, on passe � la suivante.

La derni�re couche occup�e est appel�e couche externe.\\
Toutes les autres sont appel�es couches internes.

\medskip

\noindent
%\begin{tabularx}{\textwidth}{|>{\centering}X|>{\centering}X|>{\centering}X|>{\centering}X|}
\begin{tabularx}{\textwidth}{|>{\mystrut}X|X|X|X|}
\hline
\multicolumn{1}{|X|}{\emph{Symbole de la couche}}       & $K$ & $L$ & $M$  \tbnl
\emph{Nombre maximal d'�lectrons} & $2$ & $8$ & $18$ \tbnl
\end{tabularx}

\medskip

Ainsi, par exemple, l'atome de chlore ($Z=17$) a la configuration �lectronique :
$(K)^2(L)^8(M)^{7}$.

\begin{enumerate}
\item Indiquez le nombre d'�lectrons et donnez la configuration des atomes suivants :
  \begin{enumerate}
  \item \noyau{H}{1}{}
  \item \noyau{O}{8}{}
  \item \noyau{C}{6}{}
  \item \noyau{Ne}{10}{}
  \end{enumerate}
\item Parmi les ions ci-desssous, pr�cisez s'il s'agit d'anions ou de cations.
Indiquez le nombre d'�lectrons et donnez la configuration �lectronique
des ions suivants :
  \begin{enumerate}
  \item $Be^{2+}$ ($Z=4$)
  \item $Al^{3+}$ ($Z=13$)
  \item $O^{2-}$ ($Z=8$)
  \item $F^{-}$ ($Z=9$)
  \end{enumerate}
\end{enumerate}

\end{exercice}


\vressort{5}


\begin{exercice}{Charge d'un atome de Zinc}%\\
\begin{enumerate}
\item Combien de protons l'atome de zinc \noyau{Zn}{30}{65} contient-il ?
\item Combien d'�lectrons comporte-t-il ?
\item Calculer la charge totale des protons
sachant qu'un proton a pour charge $e = 1,6.10^{-19}~C$.
\item Calculer la charge totale des �lectrons
sachant qu'un �lectron a pour charge $-e = -1,6.10^{-19}~C$.
\item En d�duire la charge de l'atome de Zinc.
\item Ce r�sultat est-il identique pour tous les atomes ?
\item A l'issue d'une r�action dite d'oxydation, un atome de zinc $Zn$
  se transforme en un ion $Zn^{2+}$.
  \begin{enumerate}
  \item Donnez l'�quation de cette r�action
  (en faisant intervenir un ou plusieurs �lectrons not�s $e^-$).
  \item Indiquez la charge (en coulomb $C$) de cet ion.
  \end{enumerate}
\end{enumerate}
\end{exercice}

\vressort{3}  % Mettre les fig
\ds{Devoir Surveill�}{
%
}

\nomprenomclasse

\setcounter{numexercice}{0}

%\renewcommand{\tabularx}[1]{>{\centering}m{#1}} 

%\newcommand{\tabularxc}[1]{\tabularx{>{\centering}m{#1}}}

\vressort{3}

\begin{exercice}{Connaissance sur l'atome}%\\
\begin{enumerate}
\item De quoi est compos� un atome ?
\item Que signifie les lettres $A$, $Z$ et $X$ dans la repr�sentation \noyau{X}{Z}{A} ?
\item Comment trouve-t-on le nombre de neutrons d'un atome de l'�l�ment pr�c�dent.
\item Si un atome a $5$ protons, combien-a-t-il d'�lectrons ? Pourquoi ?
\item Qu'est-ce qui caract�rise un �l�ment chimique ?
\item Qu'est-ce qu'un isotope ?
\end{enumerate}
\end{exercice}



\vressort{3}



\begin{exercice}{Composition des atomes}\\
En vous aidant du tableau p�riodique des �l�ments,
compl�ter le tableau suivant :

\medskip

\noindent
%\begin{tabularx}{\textwidth}{|>{\centering}X|>{\centering}X|>{\centering}X|>{\centering}X|>{\centering}X|}
% \begin{tabularx}{\linewidth}{|X|X|X|X|X|}
% \hline
% \emph{nom}       & \emph{symbole}  & \emph{protons} & \emph{neutrons}
% & \emph{nucl�ons} \tbnl
% carbone   & \noyau{C}{6}{14}   &         &          & \rule[-0.5cm]{0cm}{1cm}         \tbnl
% fluor     & \noyau{F}{9}{19}   &         &          & \rule[-0.5cm]{0cm}{1cm}         \tbnl
% sodium    & \noyau{Na}{11}{23} &         &          & \rule[-0.5cm]{0cm}{1cm}         \tbnl
% oxyg�ne   & \noyau{O}{8}{16}   &         &          & \rule[-0.5cm]{0cm}{1cm}         \tbnl
% hydrog�ne &          &         & 0        & \rule[-0.5cm]{0cm}{1cm}         \tbnl
%           & \noyau{Cl}{17}{35} &         &          &  \rule[-0.5cm]{0cm}{1cm}        \tbnl
%           &          & 8       & \rule[-0.5cm]{0cm}{1cm}         & 16       \tbnl
% \end{tabularx}



\begin{tabularx}{\linewidth}{|>{\mystrut}X|X|X|X|X|}
\hline
% multicolumn pour faire dispara�tre le \mystrut
\multicolumn{1}{|X|}{\emph{nom}} & \emph{symbole}  &
\emph{protons} & \emph{neutrons} & \emph{nucl�ons} \tbnl
carbone   & \noyau{C}{6}{14}   &   &   &    \tbnl
fluor     & \noyau{F}{9}{19}   &   &   &    \tbnl
sodium    & \noyau{Na}{11}{23} &   &   &    \tbnl
oxyg�ne   & \noyau{O}{8}{16}   &   &   &    \tbnl
hydrog�ne &                    &   & 0 &    \tbnl
          & \noyau{Cl}{17}{35} &   &   &    \tbnl
          &                    & 8 &   & 16 \tbnl
\end{tabularx}


\end{exercice}


\vressort{3}


\begin{exercice}{Masse d'un atome de carbone 12}\\
Soit le carbone $12$ not� \noyau{C}{6}{12}.
\begin{enumerate}
\item L'�l�ment carbone peut-il avoir $5$ protons ? Pourquoi ?
\item Calculer la masse du noyau d'un atome de carbone $12$
sachant que la masse d'un nucl�on est $m_n = 1,67.10^{-27}~kg$
\item Calculer la masse des �lectrons de l'atome de carbone 12
sachant que la masse d'un �lectron vaut $m_e = 9,1.10^{-31}~kg$
\item Comparer la masse des �lectrons de l'atome � la masse du noyau.
Que concluez-vous ?
\item En d�duire, sans nouveau calcul, la masse de l'atome de carbone
  $12$.
\end{enumerate}
\end{exercice}

\newpage

\vressort{1}

\begin{exercice}{Couches �lectroniques}\\
Dans l'�tat le plus stable de l'atome, appel� �tat fondamental,
les �lectrons occupent successivement les couches,
en commen�ant par celles qui sont les plus proches du noyau : 
d'abord $K$ puis $L$ puis $M$.

Lorsqu'une couche est pleine, ou encore satur�e, on passe � la suivante.

La derni�re couche occup�e est appel�e couche externe.\\
Toutes les autres sont appel�es couches internes.

\medskip

\noindent
%\begin{tabularx}{\textwidth}{|>{\centering}X|>{\centering}X|>{\centering}X|>{\centering}X|}
\begin{tabularx}{\textwidth}{|>{\mystrut}X|X|X|X|}
\hline
\multicolumn{1}{|X|}{\emph{Symbole de la couche}}       & $K$ & $L$ & $M$  \tbnl
\emph{Nombre maximal d'�lectrons} & $2$ & $8$ & $18$ \tbnl
\end{tabularx}

\medskip

Ainsi, par exemple, l'atome de chlore ($Z=17$) a la configuration �lectronique :
$(K)^2(L)^8(M)^{7}$.

\begin{enumerate}
\item Indiquez le nombre d'�lectrons et donnez la configuration des atomes suivants :
  \begin{enumerate}
  \item \noyau{H}{1}{}
  \item \noyau{O}{8}{}
  \item \noyau{C}{6}{}
  \item \noyau{Ne}{10}{}
  \end{enumerate}
\item Parmi les ions ci-desssous, pr�cisez s'il s'agit d'anions ou de cations.
Indiquez le nombre d'�lectrons et donnez la configuration �lectronique
des ions suivants :
  \begin{enumerate}
  \item $Be^{2+}$ ($Z=4$)
  \item $Al^{3+}$ ($Z=13$)
  \item $O^{2-}$ ($Z=8$)
  \item $F^{-}$ ($Z=9$)
  \end{enumerate}
\end{enumerate}

\end{exercice}


\vressort{5}


\begin{exercice}{Charge d'un atome de Zinc}%\\
\begin{enumerate}
\item Combien de protons l'atome de zinc \noyau{Zn}{30}{65} contient-il ?
\item Combien d'�lectrons comporte-t-il ?
\item Calculer la charge totale des protons
sachant qu'un proton a pour charge $e = 1,6.10^{-19}~C$.
\item Calculer la charge totale des �lectrons
sachant qu'un �lectron a pour charge $-e = -1,6.10^{-19}~C$.
\item En d�duire la charge de l'atome de Zinc.
\item Ce r�sultat est-il identique pour tous les atomes ?
\item A l'issue d'une r�action dite d'oxydation, un atome de zinc $Zn$
  se transforme en un ion $Zn^{2+}$.
  \begin{enumerate}
  \item Donnez l'�quation de cette r�action
  (en faisant intervenir un ou plusieurs �lectrons not�s $e^-$).
  \item Indiquez la charge (en coulomb $C$) de cet ion.
  \end{enumerate}
\end{enumerate}
\end{exercice}

\vressort{3}


\chapitre{Transferts thermiques} %Calorim�trie}
\ds{Devoir Surveill�}{
%
}

\nomprenomclasse

\setcounter{numexercice}{0}

%\renewcommand{\tabularx}[1]{>{\centering}m{#1}} 

%\newcommand{\tabularxc}[1]{\tabularx{>{\centering}m{#1}}}

\vressort{3}

\begin{exercice}{Connaissance sur l'atome}%\\
\begin{enumerate}
\item De quoi est compos� un atome ?
\item Que signifie les lettres $A$, $Z$ et $X$ dans la repr�sentation \noyau{X}{Z}{A} ?
\item Comment trouve-t-on le nombre de neutrons d'un atome de l'�l�ment pr�c�dent.
\item Si un atome a $5$ protons, combien-a-t-il d'�lectrons ? Pourquoi ?
\item Qu'est-ce qui caract�rise un �l�ment chimique ?
\item Qu'est-ce qu'un isotope ?
\end{enumerate}
\end{exercice}



\vressort{3}



\begin{exercice}{Composition des atomes}\\
En vous aidant du tableau p�riodique des �l�ments,
compl�ter le tableau suivant :

\medskip

\noindent
%\begin{tabularx}{\textwidth}{|>{\centering}X|>{\centering}X|>{\centering}X|>{\centering}X|>{\centering}X|}
% \begin{tabularx}{\linewidth}{|X|X|X|X|X|}
% \hline
% \emph{nom}       & \emph{symbole}  & \emph{protons} & \emph{neutrons}
% & \emph{nucl�ons} \tbnl
% carbone   & \noyau{C}{6}{14}   &         &          & \rule[-0.5cm]{0cm}{1cm}         \tbnl
% fluor     & \noyau{F}{9}{19}   &         &          & \rule[-0.5cm]{0cm}{1cm}         \tbnl
% sodium    & \noyau{Na}{11}{23} &         &          & \rule[-0.5cm]{0cm}{1cm}         \tbnl
% oxyg�ne   & \noyau{O}{8}{16}   &         &          & \rule[-0.5cm]{0cm}{1cm}         \tbnl
% hydrog�ne &          &         & 0        & \rule[-0.5cm]{0cm}{1cm}         \tbnl
%           & \noyau{Cl}{17}{35} &         &          &  \rule[-0.5cm]{0cm}{1cm}        \tbnl
%           &          & 8       & \rule[-0.5cm]{0cm}{1cm}         & 16       \tbnl
% \end{tabularx}



\begin{tabularx}{\linewidth}{|>{\mystrut}X|X|X|X|X|}
\hline
% multicolumn pour faire dispara�tre le \mystrut
\multicolumn{1}{|X|}{\emph{nom}} & \emph{symbole}  &
\emph{protons} & \emph{neutrons} & \emph{nucl�ons} \tbnl
carbone   & \noyau{C}{6}{14}   &   &   &    \tbnl
fluor     & \noyau{F}{9}{19}   &   &   &    \tbnl
sodium    & \noyau{Na}{11}{23} &   &   &    \tbnl
oxyg�ne   & \noyau{O}{8}{16}   &   &   &    \tbnl
hydrog�ne &                    &   & 0 &    \tbnl
          & \noyau{Cl}{17}{35} &   &   &    \tbnl
          &                    & 8 &   & 16 \tbnl
\end{tabularx}


\end{exercice}


\vressort{3}


\begin{exercice}{Masse d'un atome de carbone 12}\\
Soit le carbone $12$ not� \noyau{C}{6}{12}.
\begin{enumerate}
\item L'�l�ment carbone peut-il avoir $5$ protons ? Pourquoi ?
\item Calculer la masse du noyau d'un atome de carbone $12$
sachant que la masse d'un nucl�on est $m_n = 1,67.10^{-27}~kg$
\item Calculer la masse des �lectrons de l'atome de carbone 12
sachant que la masse d'un �lectron vaut $m_e = 9,1.10^{-31}~kg$
\item Comparer la masse des �lectrons de l'atome � la masse du noyau.
Que concluez-vous ?
\item En d�duire, sans nouveau calcul, la masse de l'atome de carbone
  $12$.
\end{enumerate}
\end{exercice}

\newpage

\vressort{1}

\begin{exercice}{Couches �lectroniques}\\
Dans l'�tat le plus stable de l'atome, appel� �tat fondamental,
les �lectrons occupent successivement les couches,
en commen�ant par celles qui sont les plus proches du noyau : 
d'abord $K$ puis $L$ puis $M$.

Lorsqu'une couche est pleine, ou encore satur�e, on passe � la suivante.

La derni�re couche occup�e est appel�e couche externe.\\
Toutes les autres sont appel�es couches internes.

\medskip

\noindent
%\begin{tabularx}{\textwidth}{|>{\centering}X|>{\centering}X|>{\centering}X|>{\centering}X|}
\begin{tabularx}{\textwidth}{|>{\mystrut}X|X|X|X|}
\hline
\multicolumn{1}{|X|}{\emph{Symbole de la couche}}       & $K$ & $L$ & $M$  \tbnl
\emph{Nombre maximal d'�lectrons} & $2$ & $8$ & $18$ \tbnl
\end{tabularx}

\medskip

Ainsi, par exemple, l'atome de chlore ($Z=17$) a la configuration �lectronique :
$(K)^2(L)^8(M)^{7}$.

\begin{enumerate}
\item Indiquez le nombre d'�lectrons et donnez la configuration des atomes suivants :
  \begin{enumerate}
  \item \noyau{H}{1}{}
  \item \noyau{O}{8}{}
  \item \noyau{C}{6}{}
  \item \noyau{Ne}{10}{}
  \end{enumerate}
\item Parmi les ions ci-desssous, pr�cisez s'il s'agit d'anions ou de cations.
Indiquez le nombre d'�lectrons et donnez la configuration �lectronique
des ions suivants :
  \begin{enumerate}
  \item $Be^{2+}$ ($Z=4$)
  \item $Al^{3+}$ ($Z=13$)
  \item $O^{2-}$ ($Z=8$)
  \item $F^{-}$ ($Z=9$)
  \end{enumerate}
\end{enumerate}

\end{exercice}


\vressort{5}


\begin{exercice}{Charge d'un atome de Zinc}%\\
\begin{enumerate}
\item Combien de protons l'atome de zinc \noyau{Zn}{30}{65} contient-il ?
\item Combien d'�lectrons comporte-t-il ?
\item Calculer la charge totale des protons
sachant qu'un proton a pour charge $e = 1,6.10^{-19}~C$.
\item Calculer la charge totale des �lectrons
sachant qu'un �lectron a pour charge $-e = -1,6.10^{-19}~C$.
\item En d�duire la charge de l'atome de Zinc.
\item Ce r�sultat est-il identique pour tous les atomes ?
\item A l'issue d'une r�action dite d'oxydation, un atome de zinc $Zn$
  se transforme en un ion $Zn^{2+}$.
  \begin{enumerate}
  \item Donnez l'�quation de cette r�action
  (en faisant intervenir un ou plusieurs �lectrons not�s $e^-$).
  \item Indiquez la charge (en coulomb $C$) de cet ion.
  \end{enumerate}
\end{enumerate}
\end{exercice}

\vressort{3}
\ds{Devoir Surveill�}{
%
}

\nomprenomclasse

\setcounter{numexercice}{0}

%\renewcommand{\tabularx}[1]{>{\centering}m{#1}} 

%\newcommand{\tabularxc}[1]{\tabularx{>{\centering}m{#1}}}

\vressort{3}

\begin{exercice}{Connaissance sur l'atome}%\\
\begin{enumerate}
\item De quoi est compos� un atome ?
\item Que signifie les lettres $A$, $Z$ et $X$ dans la repr�sentation \noyau{X}{Z}{A} ?
\item Comment trouve-t-on le nombre de neutrons d'un atome de l'�l�ment pr�c�dent.
\item Si un atome a $5$ protons, combien-a-t-il d'�lectrons ? Pourquoi ?
\item Qu'est-ce qui caract�rise un �l�ment chimique ?
\item Qu'est-ce qu'un isotope ?
\end{enumerate}
\end{exercice}



\vressort{3}



\begin{exercice}{Composition des atomes}\\
En vous aidant du tableau p�riodique des �l�ments,
compl�ter le tableau suivant :

\medskip

\noindent
%\begin{tabularx}{\textwidth}{|>{\centering}X|>{\centering}X|>{\centering}X|>{\centering}X|>{\centering}X|}
% \begin{tabularx}{\linewidth}{|X|X|X|X|X|}
% \hline
% \emph{nom}       & \emph{symbole}  & \emph{protons} & \emph{neutrons}
% & \emph{nucl�ons} \tbnl
% carbone   & \noyau{C}{6}{14}   &         &          & \rule[-0.5cm]{0cm}{1cm}         \tbnl
% fluor     & \noyau{F}{9}{19}   &         &          & \rule[-0.5cm]{0cm}{1cm}         \tbnl
% sodium    & \noyau{Na}{11}{23} &         &          & \rule[-0.5cm]{0cm}{1cm}         \tbnl
% oxyg�ne   & \noyau{O}{8}{16}   &         &          & \rule[-0.5cm]{0cm}{1cm}         \tbnl
% hydrog�ne &          &         & 0        & \rule[-0.5cm]{0cm}{1cm}         \tbnl
%           & \noyau{Cl}{17}{35} &         &          &  \rule[-0.5cm]{0cm}{1cm}        \tbnl
%           &          & 8       & \rule[-0.5cm]{0cm}{1cm}         & 16       \tbnl
% \end{tabularx}



\begin{tabularx}{\linewidth}{|>{\mystrut}X|X|X|X|X|}
\hline
% multicolumn pour faire dispara�tre le \mystrut
\multicolumn{1}{|X|}{\emph{nom}} & \emph{symbole}  &
\emph{protons} & \emph{neutrons} & \emph{nucl�ons} \tbnl
carbone   & \noyau{C}{6}{14}   &   &   &    \tbnl
fluor     & \noyau{F}{9}{19}   &   &   &    \tbnl
sodium    & \noyau{Na}{11}{23} &   &   &    \tbnl
oxyg�ne   & \noyau{O}{8}{16}   &   &   &    \tbnl
hydrog�ne &                    &   & 0 &    \tbnl
          & \noyau{Cl}{17}{35} &   &   &    \tbnl
          &                    & 8 &   & 16 \tbnl
\end{tabularx}


\end{exercice}


\vressort{3}


\begin{exercice}{Masse d'un atome de carbone 12}\\
Soit le carbone $12$ not� \noyau{C}{6}{12}.
\begin{enumerate}
\item L'�l�ment carbone peut-il avoir $5$ protons ? Pourquoi ?
\item Calculer la masse du noyau d'un atome de carbone $12$
sachant que la masse d'un nucl�on est $m_n = 1,67.10^{-27}~kg$
\item Calculer la masse des �lectrons de l'atome de carbone 12
sachant que la masse d'un �lectron vaut $m_e = 9,1.10^{-31}~kg$
\item Comparer la masse des �lectrons de l'atome � la masse du noyau.
Que concluez-vous ?
\item En d�duire, sans nouveau calcul, la masse de l'atome de carbone
  $12$.
\end{enumerate}
\end{exercice}

\newpage

\vressort{1}

\begin{exercice}{Couches �lectroniques}\\
Dans l'�tat le plus stable de l'atome, appel� �tat fondamental,
les �lectrons occupent successivement les couches,
en commen�ant par celles qui sont les plus proches du noyau : 
d'abord $K$ puis $L$ puis $M$.

Lorsqu'une couche est pleine, ou encore satur�e, on passe � la suivante.

La derni�re couche occup�e est appel�e couche externe.\\
Toutes les autres sont appel�es couches internes.

\medskip

\noindent
%\begin{tabularx}{\textwidth}{|>{\centering}X|>{\centering}X|>{\centering}X|>{\centering}X|}
\begin{tabularx}{\textwidth}{|>{\mystrut}X|X|X|X|}
\hline
\multicolumn{1}{|X|}{\emph{Symbole de la couche}}       & $K$ & $L$ & $M$  \tbnl
\emph{Nombre maximal d'�lectrons} & $2$ & $8$ & $18$ \tbnl
\end{tabularx}

\medskip

Ainsi, par exemple, l'atome de chlore ($Z=17$) a la configuration �lectronique :
$(K)^2(L)^8(M)^{7}$.

\begin{enumerate}
\item Indiquez le nombre d'�lectrons et donnez la configuration des atomes suivants :
  \begin{enumerate}
  \item \noyau{H}{1}{}
  \item \noyau{O}{8}{}
  \item \noyau{C}{6}{}
  \item \noyau{Ne}{10}{}
  \end{enumerate}
\item Parmi les ions ci-desssous, pr�cisez s'il s'agit d'anions ou de cations.
Indiquez le nombre d'�lectrons et donnez la configuration �lectronique
des ions suivants :
  \begin{enumerate}
  \item $Be^{2+}$ ($Z=4$)
  \item $Al^{3+}$ ($Z=13$)
  \item $O^{2-}$ ($Z=8$)
  \item $F^{-}$ ($Z=9$)
  \end{enumerate}
\end{enumerate}

\end{exercice}


\vressort{5}


\begin{exercice}{Charge d'un atome de Zinc}%\\
\begin{enumerate}
\item Combien de protons l'atome de zinc \noyau{Zn}{30}{65} contient-il ?
\item Combien d'�lectrons comporte-t-il ?
\item Calculer la charge totale des protons
sachant qu'un proton a pour charge $e = 1,6.10^{-19}~C$.
\item Calculer la charge totale des �lectrons
sachant qu'un �lectron a pour charge $-e = -1,6.10^{-19}~C$.
\item En d�duire la charge de l'atome de Zinc.
\item Ce r�sultat est-il identique pour tous les atomes ?
\item A l'issue d'une r�action dite d'oxydation, un atome de zinc $Zn$
  se transforme en un ion $Zn^{2+}$.
  \begin{enumerate}
  \item Donnez l'�quation de cette r�action
  (en faisant intervenir un ou plusieurs �lectrons not�s $e^-$).
  \item Indiquez la charge (en coulomb $C$) de cet ion.
  \end{enumerate}
\end{enumerate}
\end{exercice}

\vressort{3}
\ds{Devoir Surveill�}{
%
}

\nomprenomclasse

\setcounter{numexercice}{0}

%\renewcommand{\tabularx}[1]{>{\centering}m{#1}} 

%\newcommand{\tabularxc}[1]{\tabularx{>{\centering}m{#1}}}

\vressort{3}

\begin{exercice}{Connaissance sur l'atome}%\\
\begin{enumerate}
\item De quoi est compos� un atome ?
\item Que signifie les lettres $A$, $Z$ et $X$ dans la repr�sentation \noyau{X}{Z}{A} ?
\item Comment trouve-t-on le nombre de neutrons d'un atome de l'�l�ment pr�c�dent.
\item Si un atome a $5$ protons, combien-a-t-il d'�lectrons ? Pourquoi ?
\item Qu'est-ce qui caract�rise un �l�ment chimique ?
\item Qu'est-ce qu'un isotope ?
\end{enumerate}
\end{exercice}



\vressort{3}



\begin{exercice}{Composition des atomes}\\
En vous aidant du tableau p�riodique des �l�ments,
compl�ter le tableau suivant :

\medskip

\noindent
%\begin{tabularx}{\textwidth}{|>{\centering}X|>{\centering}X|>{\centering}X|>{\centering}X|>{\centering}X|}
% \begin{tabularx}{\linewidth}{|X|X|X|X|X|}
% \hline
% \emph{nom}       & \emph{symbole}  & \emph{protons} & \emph{neutrons}
% & \emph{nucl�ons} \tbnl
% carbone   & \noyau{C}{6}{14}   &         &          & \rule[-0.5cm]{0cm}{1cm}         \tbnl
% fluor     & \noyau{F}{9}{19}   &         &          & \rule[-0.5cm]{0cm}{1cm}         \tbnl
% sodium    & \noyau{Na}{11}{23} &         &          & \rule[-0.5cm]{0cm}{1cm}         \tbnl
% oxyg�ne   & \noyau{O}{8}{16}   &         &          & \rule[-0.5cm]{0cm}{1cm}         \tbnl
% hydrog�ne &          &         & 0        & \rule[-0.5cm]{0cm}{1cm}         \tbnl
%           & \noyau{Cl}{17}{35} &         &          &  \rule[-0.5cm]{0cm}{1cm}        \tbnl
%           &          & 8       & \rule[-0.5cm]{0cm}{1cm}         & 16       \tbnl
% \end{tabularx}



\begin{tabularx}{\linewidth}{|>{\mystrut}X|X|X|X|X|}
\hline
% multicolumn pour faire dispara�tre le \mystrut
\multicolumn{1}{|X|}{\emph{nom}} & \emph{symbole}  &
\emph{protons} & \emph{neutrons} & \emph{nucl�ons} \tbnl
carbone   & \noyau{C}{6}{14}   &   &   &    \tbnl
fluor     & \noyau{F}{9}{19}   &   &   &    \tbnl
sodium    & \noyau{Na}{11}{23} &   &   &    \tbnl
oxyg�ne   & \noyau{O}{8}{16}   &   &   &    \tbnl
hydrog�ne &                    &   & 0 &    \tbnl
          & \noyau{Cl}{17}{35} &   &   &    \tbnl
          &                    & 8 &   & 16 \tbnl
\end{tabularx}


\end{exercice}


\vressort{3}


\begin{exercice}{Masse d'un atome de carbone 12}\\
Soit le carbone $12$ not� \noyau{C}{6}{12}.
\begin{enumerate}
\item L'�l�ment carbone peut-il avoir $5$ protons ? Pourquoi ?
\item Calculer la masse du noyau d'un atome de carbone $12$
sachant que la masse d'un nucl�on est $m_n = 1,67.10^{-27}~kg$
\item Calculer la masse des �lectrons de l'atome de carbone 12
sachant que la masse d'un �lectron vaut $m_e = 9,1.10^{-31}~kg$
\item Comparer la masse des �lectrons de l'atome � la masse du noyau.
Que concluez-vous ?
\item En d�duire, sans nouveau calcul, la masse de l'atome de carbone
  $12$.
\end{enumerate}
\end{exercice}

\newpage

\vressort{1}

\begin{exercice}{Couches �lectroniques}\\
Dans l'�tat le plus stable de l'atome, appel� �tat fondamental,
les �lectrons occupent successivement les couches,
en commen�ant par celles qui sont les plus proches du noyau : 
d'abord $K$ puis $L$ puis $M$.

Lorsqu'une couche est pleine, ou encore satur�e, on passe � la suivante.

La derni�re couche occup�e est appel�e couche externe.\\
Toutes les autres sont appel�es couches internes.

\medskip

\noindent
%\begin{tabularx}{\textwidth}{|>{\centering}X|>{\centering}X|>{\centering}X|>{\centering}X|}
\begin{tabularx}{\textwidth}{|>{\mystrut}X|X|X|X|}
\hline
\multicolumn{1}{|X|}{\emph{Symbole de la couche}}       & $K$ & $L$ & $M$  \tbnl
\emph{Nombre maximal d'�lectrons} & $2$ & $8$ & $18$ \tbnl
\end{tabularx}

\medskip

Ainsi, par exemple, l'atome de chlore ($Z=17$) a la configuration �lectronique :
$(K)^2(L)^8(M)^{7}$.

\begin{enumerate}
\item Indiquez le nombre d'�lectrons et donnez la configuration des atomes suivants :
  \begin{enumerate}
  \item \noyau{H}{1}{}
  \item \noyau{O}{8}{}
  \item \noyau{C}{6}{}
  \item \noyau{Ne}{10}{}
  \end{enumerate}
\item Parmi les ions ci-desssous, pr�cisez s'il s'agit d'anions ou de cations.
Indiquez le nombre d'�lectrons et donnez la configuration �lectronique
des ions suivants :
  \begin{enumerate}
  \item $Be^{2+}$ ($Z=4$)
  \item $Al^{3+}$ ($Z=13$)
  \item $O^{2-}$ ($Z=8$)
  \item $F^{-}$ ($Z=9$)
  \end{enumerate}
\end{enumerate}

\end{exercice}


\vressort{5}


\begin{exercice}{Charge d'un atome de Zinc}%\\
\begin{enumerate}
\item Combien de protons l'atome de zinc \noyau{Zn}{30}{65} contient-il ?
\item Combien d'�lectrons comporte-t-il ?
\item Calculer la charge totale des protons
sachant qu'un proton a pour charge $e = 1,6.10^{-19}~C$.
\item Calculer la charge totale des �lectrons
sachant qu'un �lectron a pour charge $-e = -1,6.10^{-19}~C$.
\item En d�duire la charge de l'atome de Zinc.
\item Ce r�sultat est-il identique pour tous les atomes ?
\item A l'issue d'une r�action dite d'oxydation, un atome de zinc $Zn$
  se transforme en un ion $Zn^{2+}$.
  \begin{enumerate}
  \item Donnez l'�quation de cette r�action
  (en faisant intervenir un ou plusieurs �lectrons not�s $e^-$).
  \item Indiquez la charge (en coulomb $C$) de cet ion.
  \end{enumerate}
\end{enumerate}
\end{exercice}

\vressort{3}
\ds{Devoir Surveill�}{
%
}

\nomprenomclasse

\setcounter{numexercice}{0}

%\renewcommand{\tabularx}[1]{>{\centering}m{#1}} 

%\newcommand{\tabularxc}[1]{\tabularx{>{\centering}m{#1}}}

\vressort{3}

\begin{exercice}{Connaissance sur l'atome}%\\
\begin{enumerate}
\item De quoi est compos� un atome ?
\item Que signifie les lettres $A$, $Z$ et $X$ dans la repr�sentation \noyau{X}{Z}{A} ?
\item Comment trouve-t-on le nombre de neutrons d'un atome de l'�l�ment pr�c�dent.
\item Si un atome a $5$ protons, combien-a-t-il d'�lectrons ? Pourquoi ?
\item Qu'est-ce qui caract�rise un �l�ment chimique ?
\item Qu'est-ce qu'un isotope ?
\end{enumerate}
\end{exercice}



\vressort{3}



\begin{exercice}{Composition des atomes}\\
En vous aidant du tableau p�riodique des �l�ments,
compl�ter le tableau suivant :

\medskip

\noindent
%\begin{tabularx}{\textwidth}{|>{\centering}X|>{\centering}X|>{\centering}X|>{\centering}X|>{\centering}X|}
% \begin{tabularx}{\linewidth}{|X|X|X|X|X|}
% \hline
% \emph{nom}       & \emph{symbole}  & \emph{protons} & \emph{neutrons}
% & \emph{nucl�ons} \tbnl
% carbone   & \noyau{C}{6}{14}   &         &          & \rule[-0.5cm]{0cm}{1cm}         \tbnl
% fluor     & \noyau{F}{9}{19}   &         &          & \rule[-0.5cm]{0cm}{1cm}         \tbnl
% sodium    & \noyau{Na}{11}{23} &         &          & \rule[-0.5cm]{0cm}{1cm}         \tbnl
% oxyg�ne   & \noyau{O}{8}{16}   &         &          & \rule[-0.5cm]{0cm}{1cm}         \tbnl
% hydrog�ne &          &         & 0        & \rule[-0.5cm]{0cm}{1cm}         \tbnl
%           & \noyau{Cl}{17}{35} &         &          &  \rule[-0.5cm]{0cm}{1cm}        \tbnl
%           &          & 8       & \rule[-0.5cm]{0cm}{1cm}         & 16       \tbnl
% \end{tabularx}



\begin{tabularx}{\linewidth}{|>{\mystrut}X|X|X|X|X|}
\hline
% multicolumn pour faire dispara�tre le \mystrut
\multicolumn{1}{|X|}{\emph{nom}} & \emph{symbole}  &
\emph{protons} & \emph{neutrons} & \emph{nucl�ons} \tbnl
carbone   & \noyau{C}{6}{14}   &   &   &    \tbnl
fluor     & \noyau{F}{9}{19}   &   &   &    \tbnl
sodium    & \noyau{Na}{11}{23} &   &   &    \tbnl
oxyg�ne   & \noyau{O}{8}{16}   &   &   &    \tbnl
hydrog�ne &                    &   & 0 &    \tbnl
          & \noyau{Cl}{17}{35} &   &   &    \tbnl
          &                    & 8 &   & 16 \tbnl
\end{tabularx}


\end{exercice}


\vressort{3}


\begin{exercice}{Masse d'un atome de carbone 12}\\
Soit le carbone $12$ not� \noyau{C}{6}{12}.
\begin{enumerate}
\item L'�l�ment carbone peut-il avoir $5$ protons ? Pourquoi ?
\item Calculer la masse du noyau d'un atome de carbone $12$
sachant que la masse d'un nucl�on est $m_n = 1,67.10^{-27}~kg$
\item Calculer la masse des �lectrons de l'atome de carbone 12
sachant que la masse d'un �lectron vaut $m_e = 9,1.10^{-31}~kg$
\item Comparer la masse des �lectrons de l'atome � la masse du noyau.
Que concluez-vous ?
\item En d�duire, sans nouveau calcul, la masse de l'atome de carbone
  $12$.
\end{enumerate}
\end{exercice}

\newpage

\vressort{1}

\begin{exercice}{Couches �lectroniques}\\
Dans l'�tat le plus stable de l'atome, appel� �tat fondamental,
les �lectrons occupent successivement les couches,
en commen�ant par celles qui sont les plus proches du noyau : 
d'abord $K$ puis $L$ puis $M$.

Lorsqu'une couche est pleine, ou encore satur�e, on passe � la suivante.

La derni�re couche occup�e est appel�e couche externe.\\
Toutes les autres sont appel�es couches internes.

\medskip

\noindent
%\begin{tabularx}{\textwidth}{|>{\centering}X|>{\centering}X|>{\centering}X|>{\centering}X|}
\begin{tabularx}{\textwidth}{|>{\mystrut}X|X|X|X|}
\hline
\multicolumn{1}{|X|}{\emph{Symbole de la couche}}       & $K$ & $L$ & $M$  \tbnl
\emph{Nombre maximal d'�lectrons} & $2$ & $8$ & $18$ \tbnl
\end{tabularx}

\medskip

Ainsi, par exemple, l'atome de chlore ($Z=17$) a la configuration �lectronique :
$(K)^2(L)^8(M)^{7}$.

\begin{enumerate}
\item Indiquez le nombre d'�lectrons et donnez la configuration des atomes suivants :
  \begin{enumerate}
  \item \noyau{H}{1}{}
  \item \noyau{O}{8}{}
  \item \noyau{C}{6}{}
  \item \noyau{Ne}{10}{}
  \end{enumerate}
\item Parmi les ions ci-desssous, pr�cisez s'il s'agit d'anions ou de cations.
Indiquez le nombre d'�lectrons et donnez la configuration �lectronique
des ions suivants :
  \begin{enumerate}
  \item $Be^{2+}$ ($Z=4$)
  \item $Al^{3+}$ ($Z=13$)
  \item $O^{2-}$ ($Z=8$)
  \item $F^{-}$ ($Z=9$)
  \end{enumerate}
\end{enumerate}

\end{exercice}


\vressort{5}


\begin{exercice}{Charge d'un atome de Zinc}%\\
\begin{enumerate}
\item Combien de protons l'atome de zinc \noyau{Zn}{30}{65} contient-il ?
\item Combien d'�lectrons comporte-t-il ?
\item Calculer la charge totale des protons
sachant qu'un proton a pour charge $e = 1,6.10^{-19}~C$.
\item Calculer la charge totale des �lectrons
sachant qu'un �lectron a pour charge $-e = -1,6.10^{-19}~C$.
\item En d�duire la charge de l'atome de Zinc.
\item Ce r�sultat est-il identique pour tous les atomes ?
\item A l'issue d'une r�action dite d'oxydation, un atome de zinc $Zn$
  se transforme en un ion $Zn^{2+}$.
  \begin{enumerate}
  \item Donnez l'�quation de cette r�action
  (en faisant intervenir un ou plusieurs �lectrons not�s $e^-$).
  \item Indiquez la charge (en coulomb $C$) de cet ion.
  \end{enumerate}
\end{enumerate}
\end{exercice}

\vressort{3}
\ds{Devoir Surveill�}{
%
}

\nomprenomclasse

\setcounter{numexercice}{0}

%\renewcommand{\tabularx}[1]{>{\centering}m{#1}} 

%\newcommand{\tabularxc}[1]{\tabularx{>{\centering}m{#1}}}

\vressort{3}

\begin{exercice}{Connaissance sur l'atome}%\\
\begin{enumerate}
\item De quoi est compos� un atome ?
\item Que signifie les lettres $A$, $Z$ et $X$ dans la repr�sentation \noyau{X}{Z}{A} ?
\item Comment trouve-t-on le nombre de neutrons d'un atome de l'�l�ment pr�c�dent.
\item Si un atome a $5$ protons, combien-a-t-il d'�lectrons ? Pourquoi ?
\item Qu'est-ce qui caract�rise un �l�ment chimique ?
\item Qu'est-ce qu'un isotope ?
\end{enumerate}
\end{exercice}



\vressort{3}



\begin{exercice}{Composition des atomes}\\
En vous aidant du tableau p�riodique des �l�ments,
compl�ter le tableau suivant :

\medskip

\noindent
%\begin{tabularx}{\textwidth}{|>{\centering}X|>{\centering}X|>{\centering}X|>{\centering}X|>{\centering}X|}
% \begin{tabularx}{\linewidth}{|X|X|X|X|X|}
% \hline
% \emph{nom}       & \emph{symbole}  & \emph{protons} & \emph{neutrons}
% & \emph{nucl�ons} \tbnl
% carbone   & \noyau{C}{6}{14}   &         &          & \rule[-0.5cm]{0cm}{1cm}         \tbnl
% fluor     & \noyau{F}{9}{19}   &         &          & \rule[-0.5cm]{0cm}{1cm}         \tbnl
% sodium    & \noyau{Na}{11}{23} &         &          & \rule[-0.5cm]{0cm}{1cm}         \tbnl
% oxyg�ne   & \noyau{O}{8}{16}   &         &          & \rule[-0.5cm]{0cm}{1cm}         \tbnl
% hydrog�ne &          &         & 0        & \rule[-0.5cm]{0cm}{1cm}         \tbnl
%           & \noyau{Cl}{17}{35} &         &          &  \rule[-0.5cm]{0cm}{1cm}        \tbnl
%           &          & 8       & \rule[-0.5cm]{0cm}{1cm}         & 16       \tbnl
% \end{tabularx}



\begin{tabularx}{\linewidth}{|>{\mystrut}X|X|X|X|X|}
\hline
% multicolumn pour faire dispara�tre le \mystrut
\multicolumn{1}{|X|}{\emph{nom}} & \emph{symbole}  &
\emph{protons} & \emph{neutrons} & \emph{nucl�ons} \tbnl
carbone   & \noyau{C}{6}{14}   &   &   &    \tbnl
fluor     & \noyau{F}{9}{19}   &   &   &    \tbnl
sodium    & \noyau{Na}{11}{23} &   &   &    \tbnl
oxyg�ne   & \noyau{O}{8}{16}   &   &   &    \tbnl
hydrog�ne &                    &   & 0 &    \tbnl
          & \noyau{Cl}{17}{35} &   &   &    \tbnl
          &                    & 8 &   & 16 \tbnl
\end{tabularx}


\end{exercice}


\vressort{3}


\begin{exercice}{Masse d'un atome de carbone 12}\\
Soit le carbone $12$ not� \noyau{C}{6}{12}.
\begin{enumerate}
\item L'�l�ment carbone peut-il avoir $5$ protons ? Pourquoi ?
\item Calculer la masse du noyau d'un atome de carbone $12$
sachant que la masse d'un nucl�on est $m_n = 1,67.10^{-27}~kg$
\item Calculer la masse des �lectrons de l'atome de carbone 12
sachant que la masse d'un �lectron vaut $m_e = 9,1.10^{-31}~kg$
\item Comparer la masse des �lectrons de l'atome � la masse du noyau.
Que concluez-vous ?
\item En d�duire, sans nouveau calcul, la masse de l'atome de carbone
  $12$.
\end{enumerate}
\end{exercice}

\newpage

\vressort{1}

\begin{exercice}{Couches �lectroniques}\\
Dans l'�tat le plus stable de l'atome, appel� �tat fondamental,
les �lectrons occupent successivement les couches,
en commen�ant par celles qui sont les plus proches du noyau : 
d'abord $K$ puis $L$ puis $M$.

Lorsqu'une couche est pleine, ou encore satur�e, on passe � la suivante.

La derni�re couche occup�e est appel�e couche externe.\\
Toutes les autres sont appel�es couches internes.

\medskip

\noindent
%\begin{tabularx}{\textwidth}{|>{\centering}X|>{\centering}X|>{\centering}X|>{\centering}X|}
\begin{tabularx}{\textwidth}{|>{\mystrut}X|X|X|X|}
\hline
\multicolumn{1}{|X|}{\emph{Symbole de la couche}}       & $K$ & $L$ & $M$  \tbnl
\emph{Nombre maximal d'�lectrons} & $2$ & $8$ & $18$ \tbnl
\end{tabularx}

\medskip

Ainsi, par exemple, l'atome de chlore ($Z=17$) a la configuration �lectronique :
$(K)^2(L)^8(M)^{7}$.

\begin{enumerate}
\item Indiquez le nombre d'�lectrons et donnez la configuration des atomes suivants :
  \begin{enumerate}
  \item \noyau{H}{1}{}
  \item \noyau{O}{8}{}
  \item \noyau{C}{6}{}
  \item \noyau{Ne}{10}{}
  \end{enumerate}
\item Parmi les ions ci-desssous, pr�cisez s'il s'agit d'anions ou de cations.
Indiquez le nombre d'�lectrons et donnez la configuration �lectronique
des ions suivants :
  \begin{enumerate}
  \item $Be^{2+}$ ($Z=4$)
  \item $Al^{3+}$ ($Z=13$)
  \item $O^{2-}$ ($Z=8$)
  \item $F^{-}$ ($Z=9$)
  \end{enumerate}
\end{enumerate}

\end{exercice}


\vressort{5}


\begin{exercice}{Charge d'un atome de Zinc}%\\
\begin{enumerate}
\item Combien de protons l'atome de zinc \noyau{Zn}{30}{65} contient-il ?
\item Combien d'�lectrons comporte-t-il ?
\item Calculer la charge totale des protons
sachant qu'un proton a pour charge $e = 1,6.10^{-19}~C$.
\item Calculer la charge totale des �lectrons
sachant qu'un �lectron a pour charge $-e = -1,6.10^{-19}~C$.
\item En d�duire la charge de l'atome de Zinc.
\item Ce r�sultat est-il identique pour tous les atomes ?
\item A l'issue d'une r�action dite d'oxydation, un atome de zinc $Zn$
  se transforme en un ion $Zn^{2+}$.
  \begin{enumerate}
  \item Donnez l'�quation de cette r�action
  (en faisant intervenir un ou plusieurs �lectrons not�s $e^-$).
  \item Indiquez la charge (en coulomb $C$) de cet ion.
  \end{enumerate}
\end{enumerate}
\end{exercice}

\vressort{3}


\chapitre{\'Electricit�}
% Cours : Rappels de coll�ge


\ds{Devoir Surveill�}{
%
}

\nomprenomclasse

\setcounter{numexercice}{0}

%\renewcommand{\tabularx}[1]{>{\centering}m{#1}} 

%\newcommand{\tabularxc}[1]{\tabularx{>{\centering}m{#1}}}

\vressort{3}

\begin{exercice}{Connaissance sur l'atome}%\\
\begin{enumerate}
\item De quoi est compos� un atome ?
\item Que signifie les lettres $A$, $Z$ et $X$ dans la repr�sentation \noyau{X}{Z}{A} ?
\item Comment trouve-t-on le nombre de neutrons d'un atome de l'�l�ment pr�c�dent.
\item Si un atome a $5$ protons, combien-a-t-il d'�lectrons ? Pourquoi ?
\item Qu'est-ce qui caract�rise un �l�ment chimique ?
\item Qu'est-ce qu'un isotope ?
\end{enumerate}
\end{exercice}



\vressort{3}



\begin{exercice}{Composition des atomes}\\
En vous aidant du tableau p�riodique des �l�ments,
compl�ter le tableau suivant :

\medskip

\noindent
%\begin{tabularx}{\textwidth}{|>{\centering}X|>{\centering}X|>{\centering}X|>{\centering}X|>{\centering}X|}
% \begin{tabularx}{\linewidth}{|X|X|X|X|X|}
% \hline
% \emph{nom}       & \emph{symbole}  & \emph{protons} & \emph{neutrons}
% & \emph{nucl�ons} \tbnl
% carbone   & \noyau{C}{6}{14}   &         &          & \rule[-0.5cm]{0cm}{1cm}         \tbnl
% fluor     & \noyau{F}{9}{19}   &         &          & \rule[-0.5cm]{0cm}{1cm}         \tbnl
% sodium    & \noyau{Na}{11}{23} &         &          & \rule[-0.5cm]{0cm}{1cm}         \tbnl
% oxyg�ne   & \noyau{O}{8}{16}   &         &          & \rule[-0.5cm]{0cm}{1cm}         \tbnl
% hydrog�ne &          &         & 0        & \rule[-0.5cm]{0cm}{1cm}         \tbnl
%           & \noyau{Cl}{17}{35} &         &          &  \rule[-0.5cm]{0cm}{1cm}        \tbnl
%           &          & 8       & \rule[-0.5cm]{0cm}{1cm}         & 16       \tbnl
% \end{tabularx}



\begin{tabularx}{\linewidth}{|>{\mystrut}X|X|X|X|X|}
\hline
% multicolumn pour faire dispara�tre le \mystrut
\multicolumn{1}{|X|}{\emph{nom}} & \emph{symbole}  &
\emph{protons} & \emph{neutrons} & \emph{nucl�ons} \tbnl
carbone   & \noyau{C}{6}{14}   &   &   &    \tbnl
fluor     & \noyau{F}{9}{19}   &   &   &    \tbnl
sodium    & \noyau{Na}{11}{23} &   &   &    \tbnl
oxyg�ne   & \noyau{O}{8}{16}   &   &   &    \tbnl
hydrog�ne &                    &   & 0 &    \tbnl
          & \noyau{Cl}{17}{35} &   &   &    \tbnl
          &                    & 8 &   & 16 \tbnl
\end{tabularx}


\end{exercice}


\vressort{3}


\begin{exercice}{Masse d'un atome de carbone 12}\\
Soit le carbone $12$ not� \noyau{C}{6}{12}.
\begin{enumerate}
\item L'�l�ment carbone peut-il avoir $5$ protons ? Pourquoi ?
\item Calculer la masse du noyau d'un atome de carbone $12$
sachant que la masse d'un nucl�on est $m_n = 1,67.10^{-27}~kg$
\item Calculer la masse des �lectrons de l'atome de carbone 12
sachant que la masse d'un �lectron vaut $m_e = 9,1.10^{-31}~kg$
\item Comparer la masse des �lectrons de l'atome � la masse du noyau.
Que concluez-vous ?
\item En d�duire, sans nouveau calcul, la masse de l'atome de carbone
  $12$.
\end{enumerate}
\end{exercice}

\newpage

\vressort{1}

\begin{exercice}{Couches �lectroniques}\\
Dans l'�tat le plus stable de l'atome, appel� �tat fondamental,
les �lectrons occupent successivement les couches,
en commen�ant par celles qui sont les plus proches du noyau : 
d'abord $K$ puis $L$ puis $M$.

Lorsqu'une couche est pleine, ou encore satur�e, on passe � la suivante.

La derni�re couche occup�e est appel�e couche externe.\\
Toutes les autres sont appel�es couches internes.

\medskip

\noindent
%\begin{tabularx}{\textwidth}{|>{\centering}X|>{\centering}X|>{\centering}X|>{\centering}X|}
\begin{tabularx}{\textwidth}{|>{\mystrut}X|X|X|X|}
\hline
\multicolumn{1}{|X|}{\emph{Symbole de la couche}}       & $K$ & $L$ & $M$  \tbnl
\emph{Nombre maximal d'�lectrons} & $2$ & $8$ & $18$ \tbnl
\end{tabularx}

\medskip

Ainsi, par exemple, l'atome de chlore ($Z=17$) a la configuration �lectronique :
$(K)^2(L)^8(M)^{7}$.

\begin{enumerate}
\item Indiquez le nombre d'�lectrons et donnez la configuration des atomes suivants :
  \begin{enumerate}
  \item \noyau{H}{1}{}
  \item \noyau{O}{8}{}
  \item \noyau{C}{6}{}
  \item \noyau{Ne}{10}{}
  \end{enumerate}
\item Parmi les ions ci-desssous, pr�cisez s'il s'agit d'anions ou de cations.
Indiquez le nombre d'�lectrons et donnez la configuration �lectronique
des ions suivants :
  \begin{enumerate}
  \item $Be^{2+}$ ($Z=4$)
  \item $Al^{3+}$ ($Z=13$)
  \item $O^{2-}$ ($Z=8$)
  \item $F^{-}$ ($Z=9$)
  \end{enumerate}
\end{enumerate}

\end{exercice}


\vressort{5}


\begin{exercice}{Charge d'un atome de Zinc}%\\
\begin{enumerate}
\item Combien de protons l'atome de zinc \noyau{Zn}{30}{65} contient-il ?
\item Combien d'�lectrons comporte-t-il ?
\item Calculer la charge totale des protons
sachant qu'un proton a pour charge $e = 1,6.10^{-19}~C$.
\item Calculer la charge totale des �lectrons
sachant qu'un �lectron a pour charge $-e = -1,6.10^{-19}~C$.
\item En d�duire la charge de l'atome de Zinc.
\item Ce r�sultat est-il identique pour tous les atomes ?
\item A l'issue d'une r�action dite d'oxydation, un atome de zinc $Zn$
  se transforme en un ion $Zn^{2+}$.
  \begin{enumerate}
  \item Donnez l'�quation de cette r�action
  (en faisant intervenir un ou plusieurs �lectrons not�s $e^-$).
  \item Indiquez la charge (en coulomb $C$) de cet ion.
  \end{enumerate}
\end{enumerate}
\end{exercice}

\vressort{3} % document fiche
    % m�thode montage
    % utilisation d'un multim�tre, ...


% TP Potentiel le long d'un circuit ?


\ds{Devoir Surveill�}{
%
}

\nomprenomclasse

\setcounter{numexercice}{0}

%\renewcommand{\tabularx}[1]{>{\centering}m{#1}} 

%\newcommand{\tabularxc}[1]{\tabularx{>{\centering}m{#1}}}

\vressort{3}

\begin{exercice}{Connaissance sur l'atome}%\\
\begin{enumerate}
\item De quoi est compos� un atome ?
\item Que signifie les lettres $A$, $Z$ et $X$ dans la repr�sentation \noyau{X}{Z}{A} ?
\item Comment trouve-t-on le nombre de neutrons d'un atome de l'�l�ment pr�c�dent.
\item Si un atome a $5$ protons, combien-a-t-il d'�lectrons ? Pourquoi ?
\item Qu'est-ce qui caract�rise un �l�ment chimique ?
\item Qu'est-ce qu'un isotope ?
\end{enumerate}
\end{exercice}



\vressort{3}



\begin{exercice}{Composition des atomes}\\
En vous aidant du tableau p�riodique des �l�ments,
compl�ter le tableau suivant :

\medskip

\noindent
%\begin{tabularx}{\textwidth}{|>{\centering}X|>{\centering}X|>{\centering}X|>{\centering}X|>{\centering}X|}
% \begin{tabularx}{\linewidth}{|X|X|X|X|X|}
% \hline
% \emph{nom}       & \emph{symbole}  & \emph{protons} & \emph{neutrons}
% & \emph{nucl�ons} \tbnl
% carbone   & \noyau{C}{6}{14}   &         &          & \rule[-0.5cm]{0cm}{1cm}         \tbnl
% fluor     & \noyau{F}{9}{19}   &         &          & \rule[-0.5cm]{0cm}{1cm}         \tbnl
% sodium    & \noyau{Na}{11}{23} &         &          & \rule[-0.5cm]{0cm}{1cm}         \tbnl
% oxyg�ne   & \noyau{O}{8}{16}   &         &          & \rule[-0.5cm]{0cm}{1cm}         \tbnl
% hydrog�ne &          &         & 0        & \rule[-0.5cm]{0cm}{1cm}         \tbnl
%           & \noyau{Cl}{17}{35} &         &          &  \rule[-0.5cm]{0cm}{1cm}        \tbnl
%           &          & 8       & \rule[-0.5cm]{0cm}{1cm}         & 16       \tbnl
% \end{tabularx}



\begin{tabularx}{\linewidth}{|>{\mystrut}X|X|X|X|X|}
\hline
% multicolumn pour faire dispara�tre le \mystrut
\multicolumn{1}{|X|}{\emph{nom}} & \emph{symbole}  &
\emph{protons} & \emph{neutrons} & \emph{nucl�ons} \tbnl
carbone   & \noyau{C}{6}{14}   &   &   &    \tbnl
fluor     & \noyau{F}{9}{19}   &   &   &    \tbnl
sodium    & \noyau{Na}{11}{23} &   &   &    \tbnl
oxyg�ne   & \noyau{O}{8}{16}   &   &   &    \tbnl
hydrog�ne &                    &   & 0 &    \tbnl
          & \noyau{Cl}{17}{35} &   &   &    \tbnl
          &                    & 8 &   & 16 \tbnl
\end{tabularx}


\end{exercice}


\vressort{3}


\begin{exercice}{Masse d'un atome de carbone 12}\\
Soit le carbone $12$ not� \noyau{C}{6}{12}.
\begin{enumerate}
\item L'�l�ment carbone peut-il avoir $5$ protons ? Pourquoi ?
\item Calculer la masse du noyau d'un atome de carbone $12$
sachant que la masse d'un nucl�on est $m_n = 1,67.10^{-27}~kg$
\item Calculer la masse des �lectrons de l'atome de carbone 12
sachant que la masse d'un �lectron vaut $m_e = 9,1.10^{-31}~kg$
\item Comparer la masse des �lectrons de l'atome � la masse du noyau.
Que concluez-vous ?
\item En d�duire, sans nouveau calcul, la masse de l'atome de carbone
  $12$.
\end{enumerate}
\end{exercice}

\newpage

\vressort{1}

\begin{exercice}{Couches �lectroniques}\\
Dans l'�tat le plus stable de l'atome, appel� �tat fondamental,
les �lectrons occupent successivement les couches,
en commen�ant par celles qui sont les plus proches du noyau : 
d'abord $K$ puis $L$ puis $M$.

Lorsqu'une couche est pleine, ou encore satur�e, on passe � la suivante.

La derni�re couche occup�e est appel�e couche externe.\\
Toutes les autres sont appel�es couches internes.

\medskip

\noindent
%\begin{tabularx}{\textwidth}{|>{\centering}X|>{\centering}X|>{\centering}X|>{\centering}X|}
\begin{tabularx}{\textwidth}{|>{\mystrut}X|X|X|X|}
\hline
\multicolumn{1}{|X|}{\emph{Symbole de la couche}}       & $K$ & $L$ & $M$  \tbnl
\emph{Nombre maximal d'�lectrons} & $2$ & $8$ & $18$ \tbnl
\end{tabularx}

\medskip

Ainsi, par exemple, l'atome de chlore ($Z=17$) a la configuration �lectronique :
$(K)^2(L)^8(M)^{7}$.

\begin{enumerate}
\item Indiquez le nombre d'�lectrons et donnez la configuration des atomes suivants :
  \begin{enumerate}
  \item \noyau{H}{1}{}
  \item \noyau{O}{8}{}
  \item \noyau{C}{6}{}
  \item \noyau{Ne}{10}{}
  \end{enumerate}
\item Parmi les ions ci-desssous, pr�cisez s'il s'agit d'anions ou de cations.
Indiquez le nombre d'�lectrons et donnez la configuration �lectronique
des ions suivants :
  \begin{enumerate}
  \item $Be^{2+}$ ($Z=4$)
  \item $Al^{3+}$ ($Z=13$)
  \item $O^{2-}$ ($Z=8$)
  \item $F^{-}$ ($Z=9$)
  \end{enumerate}
\end{enumerate}

\end{exercice}


\vressort{5}


\begin{exercice}{Charge d'un atome de Zinc}%\\
\begin{enumerate}
\item Combien de protons l'atome de zinc \noyau{Zn}{30}{65} contient-il ?
\item Combien d'�lectrons comporte-t-il ?
\item Calculer la charge totale des protons
sachant qu'un proton a pour charge $e = 1,6.10^{-19}~C$.
\item Calculer la charge totale des �lectrons
sachant qu'un �lectron a pour charge $-e = -1,6.10^{-19}~C$.
\item En d�duire la charge de l'atome de Zinc.
\item Ce r�sultat est-il identique pour tous les atomes ?
\item A l'issue d'une r�action dite d'oxydation, un atome de zinc $Zn$
  se transforme en un ion $Zn^{2+}$.
  \begin{enumerate}
  \item Donnez l'�quation de cette r�action
  (en faisant intervenir un ou plusieurs �lectrons not�s $e^-$).
  \item Indiquez la charge (en coulomb $C$) de cet ion.
  \end{enumerate}
\end{enumerate}
\end{exercice}

\vressort{3} % cours elec g�n�rateurs, r�cepteurs
\ds{Devoir Surveill�}{
%
}

\nomprenomclasse

\setcounter{numexercice}{0}

%\renewcommand{\tabularx}[1]{>{\centering}m{#1}} 

%\newcommand{\tabularxc}[1]{\tabularx{>{\centering}m{#1}}}

\vressort{3}

\begin{exercice}{Connaissance sur l'atome}%\\
\begin{enumerate}
\item De quoi est compos� un atome ?
\item Que signifie les lettres $A$, $Z$ et $X$ dans la repr�sentation \noyau{X}{Z}{A} ?
\item Comment trouve-t-on le nombre de neutrons d'un atome de l'�l�ment pr�c�dent.
\item Si un atome a $5$ protons, combien-a-t-il d'�lectrons ? Pourquoi ?
\item Qu'est-ce qui caract�rise un �l�ment chimique ?
\item Qu'est-ce qu'un isotope ?
\end{enumerate}
\end{exercice}



\vressort{3}



\begin{exercice}{Composition des atomes}\\
En vous aidant du tableau p�riodique des �l�ments,
compl�ter le tableau suivant :

\medskip

\noindent
%\begin{tabularx}{\textwidth}{|>{\centering}X|>{\centering}X|>{\centering}X|>{\centering}X|>{\centering}X|}
% \begin{tabularx}{\linewidth}{|X|X|X|X|X|}
% \hline
% \emph{nom}       & \emph{symbole}  & \emph{protons} & \emph{neutrons}
% & \emph{nucl�ons} \tbnl
% carbone   & \noyau{C}{6}{14}   &         &          & \rule[-0.5cm]{0cm}{1cm}         \tbnl
% fluor     & \noyau{F}{9}{19}   &         &          & \rule[-0.5cm]{0cm}{1cm}         \tbnl
% sodium    & \noyau{Na}{11}{23} &         &          & \rule[-0.5cm]{0cm}{1cm}         \tbnl
% oxyg�ne   & \noyau{O}{8}{16}   &         &          & \rule[-0.5cm]{0cm}{1cm}         \tbnl
% hydrog�ne &          &         & 0        & \rule[-0.5cm]{0cm}{1cm}         \tbnl
%           & \noyau{Cl}{17}{35} &         &          &  \rule[-0.5cm]{0cm}{1cm}        \tbnl
%           &          & 8       & \rule[-0.5cm]{0cm}{1cm}         & 16       \tbnl
% \end{tabularx}



\begin{tabularx}{\linewidth}{|>{\mystrut}X|X|X|X|X|}
\hline
% multicolumn pour faire dispara�tre le \mystrut
\multicolumn{1}{|X|}{\emph{nom}} & \emph{symbole}  &
\emph{protons} & \emph{neutrons} & \emph{nucl�ons} \tbnl
carbone   & \noyau{C}{6}{14}   &   &   &    \tbnl
fluor     & \noyau{F}{9}{19}   &   &   &    \tbnl
sodium    & \noyau{Na}{11}{23} &   &   &    \tbnl
oxyg�ne   & \noyau{O}{8}{16}   &   &   &    \tbnl
hydrog�ne &                    &   & 0 &    \tbnl
          & \noyau{Cl}{17}{35} &   &   &    \tbnl
          &                    & 8 &   & 16 \tbnl
\end{tabularx}


\end{exercice}


\vressort{3}


\begin{exercice}{Masse d'un atome de carbone 12}\\
Soit le carbone $12$ not� \noyau{C}{6}{12}.
\begin{enumerate}
\item L'�l�ment carbone peut-il avoir $5$ protons ? Pourquoi ?
\item Calculer la masse du noyau d'un atome de carbone $12$
sachant que la masse d'un nucl�on est $m_n = 1,67.10^{-27}~kg$
\item Calculer la masse des �lectrons de l'atome de carbone 12
sachant que la masse d'un �lectron vaut $m_e = 9,1.10^{-31}~kg$
\item Comparer la masse des �lectrons de l'atome � la masse du noyau.
Que concluez-vous ?
\item En d�duire, sans nouveau calcul, la masse de l'atome de carbone
  $12$.
\end{enumerate}
\end{exercice}

\newpage

\vressort{1}

\begin{exercice}{Couches �lectroniques}\\
Dans l'�tat le plus stable de l'atome, appel� �tat fondamental,
les �lectrons occupent successivement les couches,
en commen�ant par celles qui sont les plus proches du noyau : 
d'abord $K$ puis $L$ puis $M$.

Lorsqu'une couche est pleine, ou encore satur�e, on passe � la suivante.

La derni�re couche occup�e est appel�e couche externe.\\
Toutes les autres sont appel�es couches internes.

\medskip

\noindent
%\begin{tabularx}{\textwidth}{|>{\centering}X|>{\centering}X|>{\centering}X|>{\centering}X|}
\begin{tabularx}{\textwidth}{|>{\mystrut}X|X|X|X|}
\hline
\multicolumn{1}{|X|}{\emph{Symbole de la couche}}       & $K$ & $L$ & $M$  \tbnl
\emph{Nombre maximal d'�lectrons} & $2$ & $8$ & $18$ \tbnl
\end{tabularx}

\medskip

Ainsi, par exemple, l'atome de chlore ($Z=17$) a la configuration �lectronique :
$(K)^2(L)^8(M)^{7}$.

\begin{enumerate}
\item Indiquez le nombre d'�lectrons et donnez la configuration des atomes suivants :
  \begin{enumerate}
  \item \noyau{H}{1}{}
  \item \noyau{O}{8}{}
  \item \noyau{C}{6}{}
  \item \noyau{Ne}{10}{}
  \end{enumerate}
\item Parmi les ions ci-desssous, pr�cisez s'il s'agit d'anions ou de cations.
Indiquez le nombre d'�lectrons et donnez la configuration �lectronique
des ions suivants :
  \begin{enumerate}
  \item $Be^{2+}$ ($Z=4$)
  \item $Al^{3+}$ ($Z=13$)
  \item $O^{2-}$ ($Z=8$)
  \item $F^{-}$ ($Z=9$)
  \end{enumerate}
\end{enumerate}

\end{exercice}


\vressort{5}


\begin{exercice}{Charge d'un atome de Zinc}%\\
\begin{enumerate}
\item Combien de protons l'atome de zinc \noyau{Zn}{30}{65} contient-il ?
\item Combien d'�lectrons comporte-t-il ?
\item Calculer la charge totale des protons
sachant qu'un proton a pour charge $e = 1,6.10^{-19}~C$.
\item Calculer la charge totale des �lectrons
sachant qu'un �lectron a pour charge $-e = -1,6.10^{-19}~C$.
\item En d�duire la charge de l'atome de Zinc.
\item Ce r�sultat est-il identique pour tous les atomes ?
\item A l'issue d'une r�action dite d'oxydation, un atome de zinc $Zn$
  se transforme en un ion $Zn^{2+}$.
  \begin{enumerate}
  \item Donnez l'�quation de cette r�action
  (en faisant intervenir un ou plusieurs �lectrons not�s $e^-$).
  \item Indiquez la charge (en coulomb $C$) de cet ion.
  \end{enumerate}
\end{enumerate}
\end{exercice}

\vressort{3} % tp caract�ristique d'un g�n�rateur
\ds{Devoir Surveill�}{
%
}

\nomprenomclasse

\setcounter{numexercice}{0}

%\renewcommand{\tabularx}[1]{>{\centering}m{#1}} 

%\newcommand{\tabularxc}[1]{\tabularx{>{\centering}m{#1}}}

\vressort{3}

\begin{exercice}{Connaissance sur l'atome}%\\
\begin{enumerate}
\item De quoi est compos� un atome ?
\item Que signifie les lettres $A$, $Z$ et $X$ dans la repr�sentation \noyau{X}{Z}{A} ?
\item Comment trouve-t-on le nombre de neutrons d'un atome de l'�l�ment pr�c�dent.
\item Si un atome a $5$ protons, combien-a-t-il d'�lectrons ? Pourquoi ?
\item Qu'est-ce qui caract�rise un �l�ment chimique ?
\item Qu'est-ce qu'un isotope ?
\end{enumerate}
\end{exercice}



\vressort{3}



\begin{exercice}{Composition des atomes}\\
En vous aidant du tableau p�riodique des �l�ments,
compl�ter le tableau suivant :

\medskip

\noindent
%\begin{tabularx}{\textwidth}{|>{\centering}X|>{\centering}X|>{\centering}X|>{\centering}X|>{\centering}X|}
% \begin{tabularx}{\linewidth}{|X|X|X|X|X|}
% \hline
% \emph{nom}       & \emph{symbole}  & \emph{protons} & \emph{neutrons}
% & \emph{nucl�ons} \tbnl
% carbone   & \noyau{C}{6}{14}   &         &          & \rule[-0.5cm]{0cm}{1cm}         \tbnl
% fluor     & \noyau{F}{9}{19}   &         &          & \rule[-0.5cm]{0cm}{1cm}         \tbnl
% sodium    & \noyau{Na}{11}{23} &         &          & \rule[-0.5cm]{0cm}{1cm}         \tbnl
% oxyg�ne   & \noyau{O}{8}{16}   &         &          & \rule[-0.5cm]{0cm}{1cm}         \tbnl
% hydrog�ne &          &         & 0        & \rule[-0.5cm]{0cm}{1cm}         \tbnl
%           & \noyau{Cl}{17}{35} &         &          &  \rule[-0.5cm]{0cm}{1cm}        \tbnl
%           &          & 8       & \rule[-0.5cm]{0cm}{1cm}         & 16       \tbnl
% \end{tabularx}



\begin{tabularx}{\linewidth}{|>{\mystrut}X|X|X|X|X|}
\hline
% multicolumn pour faire dispara�tre le \mystrut
\multicolumn{1}{|X|}{\emph{nom}} & \emph{symbole}  &
\emph{protons} & \emph{neutrons} & \emph{nucl�ons} \tbnl
carbone   & \noyau{C}{6}{14}   &   &   &    \tbnl
fluor     & \noyau{F}{9}{19}   &   &   &    \tbnl
sodium    & \noyau{Na}{11}{23} &   &   &    \tbnl
oxyg�ne   & \noyau{O}{8}{16}   &   &   &    \tbnl
hydrog�ne &                    &   & 0 &    \tbnl
          & \noyau{Cl}{17}{35} &   &   &    \tbnl
          &                    & 8 &   & 16 \tbnl
\end{tabularx}


\end{exercice}


\vressort{3}


\begin{exercice}{Masse d'un atome de carbone 12}\\
Soit le carbone $12$ not� \noyau{C}{6}{12}.
\begin{enumerate}
\item L'�l�ment carbone peut-il avoir $5$ protons ? Pourquoi ?
\item Calculer la masse du noyau d'un atome de carbone $12$
sachant que la masse d'un nucl�on est $m_n = 1,67.10^{-27}~kg$
\item Calculer la masse des �lectrons de l'atome de carbone 12
sachant que la masse d'un �lectron vaut $m_e = 9,1.10^{-31}~kg$
\item Comparer la masse des �lectrons de l'atome � la masse du noyau.
Que concluez-vous ?
\item En d�duire, sans nouveau calcul, la masse de l'atome de carbone
  $12$.
\end{enumerate}
\end{exercice}

\newpage

\vressort{1}

\begin{exercice}{Couches �lectroniques}\\
Dans l'�tat le plus stable de l'atome, appel� �tat fondamental,
les �lectrons occupent successivement les couches,
en commen�ant par celles qui sont les plus proches du noyau : 
d'abord $K$ puis $L$ puis $M$.

Lorsqu'une couche est pleine, ou encore satur�e, on passe � la suivante.

La derni�re couche occup�e est appel�e couche externe.\\
Toutes les autres sont appel�es couches internes.

\medskip

\noindent
%\begin{tabularx}{\textwidth}{|>{\centering}X|>{\centering}X|>{\centering}X|>{\centering}X|}
\begin{tabularx}{\textwidth}{|>{\mystrut}X|X|X|X|}
\hline
\multicolumn{1}{|X|}{\emph{Symbole de la couche}}       & $K$ & $L$ & $M$  \tbnl
\emph{Nombre maximal d'�lectrons} & $2$ & $8$ & $18$ \tbnl
\end{tabularx}

\medskip

Ainsi, par exemple, l'atome de chlore ($Z=17$) a la configuration �lectronique :
$(K)^2(L)^8(M)^{7}$.

\begin{enumerate}
\item Indiquez le nombre d'�lectrons et donnez la configuration des atomes suivants :
  \begin{enumerate}
  \item \noyau{H}{1}{}
  \item \noyau{O}{8}{}
  \item \noyau{C}{6}{}
  \item \noyau{Ne}{10}{}
  \end{enumerate}
\item Parmi les ions ci-desssous, pr�cisez s'il s'agit d'anions ou de cations.
Indiquez le nombre d'�lectrons et donnez la configuration �lectronique
des ions suivants :
  \begin{enumerate}
  \item $Be^{2+}$ ($Z=4$)
  \item $Al^{3+}$ ($Z=13$)
  \item $O^{2-}$ ($Z=8$)
  \item $F^{-}$ ($Z=9$)
  \end{enumerate}
\end{enumerate}

\end{exercice}


\vressort{5}


\begin{exercice}{Charge d'un atome de Zinc}%\\
\begin{enumerate}
\item Combien de protons l'atome de zinc \noyau{Zn}{30}{65} contient-il ?
\item Combien d'�lectrons comporte-t-il ?
\item Calculer la charge totale des protons
sachant qu'un proton a pour charge $e = 1,6.10^{-19}~C$.
\item Calculer la charge totale des �lectrons
sachant qu'un �lectron a pour charge $-e = -1,6.10^{-19}~C$.
\item En d�duire la charge de l'atome de Zinc.
\item Ce r�sultat est-il identique pour tous les atomes ?
\item A l'issue d'une r�action dite d'oxydation, un atome de zinc $Zn$
  se transforme en un ion $Zn^{2+}$.
  \begin{enumerate}
  \item Donnez l'�quation de cette r�action
  (en faisant intervenir un ou plusieurs �lectrons not�s $e^-$).
  \item Indiquez la charge (en coulomb $C$) de cet ion.
  \end{enumerate}
\end{enumerate}
\end{exercice}

\vressort{3} % tp caract�ristique d'un
                                % r�cepteur (�lectrolyseur)

% Cours : Circuits �lectriques




\chapitre{Optique g�om�trique}
\ds{Devoir Surveill�}{
%
}

\nomprenomclasse

\setcounter{numexercice}{0}

%\renewcommand{\tabularx}[1]{>{\centering}m{#1}} 

%\newcommand{\tabularxc}[1]{\tabularx{>{\centering}m{#1}}}

\vressort{3}

\begin{exercice}{Connaissance sur l'atome}%\\
\begin{enumerate}
\item De quoi est compos� un atome ?
\item Que signifie les lettres $A$, $Z$ et $X$ dans la repr�sentation \noyau{X}{Z}{A} ?
\item Comment trouve-t-on le nombre de neutrons d'un atome de l'�l�ment pr�c�dent.
\item Si un atome a $5$ protons, combien-a-t-il d'�lectrons ? Pourquoi ?
\item Qu'est-ce qui caract�rise un �l�ment chimique ?
\item Qu'est-ce qu'un isotope ?
\end{enumerate}
\end{exercice}



\vressort{3}



\begin{exercice}{Composition des atomes}\\
En vous aidant du tableau p�riodique des �l�ments,
compl�ter le tableau suivant :

\medskip

\noindent
%\begin{tabularx}{\textwidth}{|>{\centering}X|>{\centering}X|>{\centering}X|>{\centering}X|>{\centering}X|}
% \begin{tabularx}{\linewidth}{|X|X|X|X|X|}
% \hline
% \emph{nom}       & \emph{symbole}  & \emph{protons} & \emph{neutrons}
% & \emph{nucl�ons} \tbnl
% carbone   & \noyau{C}{6}{14}   &         &          & \rule[-0.5cm]{0cm}{1cm}         \tbnl
% fluor     & \noyau{F}{9}{19}   &         &          & \rule[-0.5cm]{0cm}{1cm}         \tbnl
% sodium    & \noyau{Na}{11}{23} &         &          & \rule[-0.5cm]{0cm}{1cm}         \tbnl
% oxyg�ne   & \noyau{O}{8}{16}   &         &          & \rule[-0.5cm]{0cm}{1cm}         \tbnl
% hydrog�ne &          &         & 0        & \rule[-0.5cm]{0cm}{1cm}         \tbnl
%           & \noyau{Cl}{17}{35} &         &          &  \rule[-0.5cm]{0cm}{1cm}        \tbnl
%           &          & 8       & \rule[-0.5cm]{0cm}{1cm}         & 16       \tbnl
% \end{tabularx}



\begin{tabularx}{\linewidth}{|>{\mystrut}X|X|X|X|X|}
\hline
% multicolumn pour faire dispara�tre le \mystrut
\multicolumn{1}{|X|}{\emph{nom}} & \emph{symbole}  &
\emph{protons} & \emph{neutrons} & \emph{nucl�ons} \tbnl
carbone   & \noyau{C}{6}{14}   &   &   &    \tbnl
fluor     & \noyau{F}{9}{19}   &   &   &    \tbnl
sodium    & \noyau{Na}{11}{23} &   &   &    \tbnl
oxyg�ne   & \noyau{O}{8}{16}   &   &   &    \tbnl
hydrog�ne &                    &   & 0 &    \tbnl
          & \noyau{Cl}{17}{35} &   &   &    \tbnl
          &                    & 8 &   & 16 \tbnl
\end{tabularx}


\end{exercice}


\vressort{3}


\begin{exercice}{Masse d'un atome de carbone 12}\\
Soit le carbone $12$ not� \noyau{C}{6}{12}.
\begin{enumerate}
\item L'�l�ment carbone peut-il avoir $5$ protons ? Pourquoi ?
\item Calculer la masse du noyau d'un atome de carbone $12$
sachant que la masse d'un nucl�on est $m_n = 1,67.10^{-27}~kg$
\item Calculer la masse des �lectrons de l'atome de carbone 12
sachant que la masse d'un �lectron vaut $m_e = 9,1.10^{-31}~kg$
\item Comparer la masse des �lectrons de l'atome � la masse du noyau.
Que concluez-vous ?
\item En d�duire, sans nouveau calcul, la masse de l'atome de carbone
  $12$.
\end{enumerate}
\end{exercice}

\newpage

\vressort{1}

\begin{exercice}{Couches �lectroniques}\\
Dans l'�tat le plus stable de l'atome, appel� �tat fondamental,
les �lectrons occupent successivement les couches,
en commen�ant par celles qui sont les plus proches du noyau : 
d'abord $K$ puis $L$ puis $M$.

Lorsqu'une couche est pleine, ou encore satur�e, on passe � la suivante.

La derni�re couche occup�e est appel�e couche externe.\\
Toutes les autres sont appel�es couches internes.

\medskip

\noindent
%\begin{tabularx}{\textwidth}{|>{\centering}X|>{\centering}X|>{\centering}X|>{\centering}X|}
\begin{tabularx}{\textwidth}{|>{\mystrut}X|X|X|X|}
\hline
\multicolumn{1}{|X|}{\emph{Symbole de la couche}}       & $K$ & $L$ & $M$  \tbnl
\emph{Nombre maximal d'�lectrons} & $2$ & $8$ & $18$ \tbnl
\end{tabularx}

\medskip

Ainsi, par exemple, l'atome de chlore ($Z=17$) a la configuration �lectronique :
$(K)^2(L)^8(M)^{7}$.

\begin{enumerate}
\item Indiquez le nombre d'�lectrons et donnez la configuration des atomes suivants :
  \begin{enumerate}
  \item \noyau{H}{1}{}
  \item \noyau{O}{8}{}
  \item \noyau{C}{6}{}
  \item \noyau{Ne}{10}{}
  \end{enumerate}
\item Parmi les ions ci-desssous, pr�cisez s'il s'agit d'anions ou de cations.
Indiquez le nombre d'�lectrons et donnez la configuration �lectronique
des ions suivants :
  \begin{enumerate}
  \item $Be^{2+}$ ($Z=4$)
  \item $Al^{3+}$ ($Z=13$)
  \item $O^{2-}$ ($Z=8$)
  \item $F^{-}$ ($Z=9$)
  \end{enumerate}
\end{enumerate}

\end{exercice}


\vressort{5}


\begin{exercice}{Charge d'un atome de Zinc}%\\
\begin{enumerate}
\item Combien de protons l'atome de zinc \noyau{Zn}{30}{65} contient-il ?
\item Combien d'�lectrons comporte-t-il ?
\item Calculer la charge totale des protons
sachant qu'un proton a pour charge $e = 1,6.10^{-19}~C$.
\item Calculer la charge totale des �lectrons
sachant qu'un �lectron a pour charge $-e = -1,6.10^{-19}~C$.
\item En d�duire la charge de l'atome de Zinc.
\item Ce r�sultat est-il identique pour tous les atomes ?
\item A l'issue d'une r�action dite d'oxydation, un atome de zinc $Zn$
  se transforme en un ion $Zn^{2+}$.
  \begin{enumerate}
  \item Donnez l'�quation de cette r�action
  (en faisant intervenir un ou plusieurs �lectrons not�s $e^-$).
  \item Indiquez la charge (en coulomb $C$) de cet ion.
  \end{enumerate}
\end{enumerate}
\end{exercice}

\vressort{3} % conditions de visibilit�
\ds{Devoir Surveill�}{
%
}

\nomprenomclasse

\setcounter{numexercice}{0}

%\renewcommand{\tabularx}[1]{>{\centering}m{#1}} 

%\newcommand{\tabularxc}[1]{\tabularx{>{\centering}m{#1}}}

\vressort{3}

\begin{exercice}{Connaissance sur l'atome}%\\
\begin{enumerate}
\item De quoi est compos� un atome ?
\item Que signifie les lettres $A$, $Z$ et $X$ dans la repr�sentation \noyau{X}{Z}{A} ?
\item Comment trouve-t-on le nombre de neutrons d'un atome de l'�l�ment pr�c�dent.
\item Si un atome a $5$ protons, combien-a-t-il d'�lectrons ? Pourquoi ?
\item Qu'est-ce qui caract�rise un �l�ment chimique ?
\item Qu'est-ce qu'un isotope ?
\end{enumerate}
\end{exercice}



\vressort{3}



\begin{exercice}{Composition des atomes}\\
En vous aidant du tableau p�riodique des �l�ments,
compl�ter le tableau suivant :

\medskip

\noindent
%\begin{tabularx}{\textwidth}{|>{\centering}X|>{\centering}X|>{\centering}X|>{\centering}X|>{\centering}X|}
% \begin{tabularx}{\linewidth}{|X|X|X|X|X|}
% \hline
% \emph{nom}       & \emph{symbole}  & \emph{protons} & \emph{neutrons}
% & \emph{nucl�ons} \tbnl
% carbone   & \noyau{C}{6}{14}   &         &          & \rule[-0.5cm]{0cm}{1cm}         \tbnl
% fluor     & \noyau{F}{9}{19}   &         &          & \rule[-0.5cm]{0cm}{1cm}         \tbnl
% sodium    & \noyau{Na}{11}{23} &         &          & \rule[-0.5cm]{0cm}{1cm}         \tbnl
% oxyg�ne   & \noyau{O}{8}{16}   &         &          & \rule[-0.5cm]{0cm}{1cm}         \tbnl
% hydrog�ne &          &         & 0        & \rule[-0.5cm]{0cm}{1cm}         \tbnl
%           & \noyau{Cl}{17}{35} &         &          &  \rule[-0.5cm]{0cm}{1cm}        \tbnl
%           &          & 8       & \rule[-0.5cm]{0cm}{1cm}         & 16       \tbnl
% \end{tabularx}



\begin{tabularx}{\linewidth}{|>{\mystrut}X|X|X|X|X|}
\hline
% multicolumn pour faire dispara�tre le \mystrut
\multicolumn{1}{|X|}{\emph{nom}} & \emph{symbole}  &
\emph{protons} & \emph{neutrons} & \emph{nucl�ons} \tbnl
carbone   & \noyau{C}{6}{14}   &   &   &    \tbnl
fluor     & \noyau{F}{9}{19}   &   &   &    \tbnl
sodium    & \noyau{Na}{11}{23} &   &   &    \tbnl
oxyg�ne   & \noyau{O}{8}{16}   &   &   &    \tbnl
hydrog�ne &                    &   & 0 &    \tbnl
          & \noyau{Cl}{17}{35} &   &   &    \tbnl
          &                    & 8 &   & 16 \tbnl
\end{tabularx}


\end{exercice}


\vressort{3}


\begin{exercice}{Masse d'un atome de carbone 12}\\
Soit le carbone $12$ not� \noyau{C}{6}{12}.
\begin{enumerate}
\item L'�l�ment carbone peut-il avoir $5$ protons ? Pourquoi ?
\item Calculer la masse du noyau d'un atome de carbone $12$
sachant que la masse d'un nucl�on est $m_n = 1,67.10^{-27}~kg$
\item Calculer la masse des �lectrons de l'atome de carbone 12
sachant que la masse d'un �lectron vaut $m_e = 9,1.10^{-31}~kg$
\item Comparer la masse des �lectrons de l'atome � la masse du noyau.
Que concluez-vous ?
\item En d�duire, sans nouveau calcul, la masse de l'atome de carbone
  $12$.
\end{enumerate}
\end{exercice}

\newpage

\vressort{1}

\begin{exercice}{Couches �lectroniques}\\
Dans l'�tat le plus stable de l'atome, appel� �tat fondamental,
les �lectrons occupent successivement les couches,
en commen�ant par celles qui sont les plus proches du noyau : 
d'abord $K$ puis $L$ puis $M$.

Lorsqu'une couche est pleine, ou encore satur�e, on passe � la suivante.

La derni�re couche occup�e est appel�e couche externe.\\
Toutes les autres sont appel�es couches internes.

\medskip

\noindent
%\begin{tabularx}{\textwidth}{|>{\centering}X|>{\centering}X|>{\centering}X|>{\centering}X|}
\begin{tabularx}{\textwidth}{|>{\mystrut}X|X|X|X|}
\hline
\multicolumn{1}{|X|}{\emph{Symbole de la couche}}       & $K$ & $L$ & $M$  \tbnl
\emph{Nombre maximal d'�lectrons} & $2$ & $8$ & $18$ \tbnl
\end{tabularx}

\medskip

Ainsi, par exemple, l'atome de chlore ($Z=17$) a la configuration �lectronique :
$(K)^2(L)^8(M)^{7}$.

\begin{enumerate}
\item Indiquez le nombre d'�lectrons et donnez la configuration des atomes suivants :
  \begin{enumerate}
  \item \noyau{H}{1}{}
  \item \noyau{O}{8}{}
  \item \noyau{C}{6}{}
  \item \noyau{Ne}{10}{}
  \end{enumerate}
\item Parmi les ions ci-desssous, pr�cisez s'il s'agit d'anions ou de cations.
Indiquez le nombre d'�lectrons et donnez la configuration �lectronique
des ions suivants :
  \begin{enumerate}
  \item $Be^{2+}$ ($Z=4$)
  \item $Al^{3+}$ ($Z=13$)
  \item $O^{2-}$ ($Z=8$)
  \item $F^{-}$ ($Z=9$)
  \end{enumerate}
\end{enumerate}

\end{exercice}


\vressort{5}


\begin{exercice}{Charge d'un atome de Zinc}%\\
\begin{enumerate}
\item Combien de protons l'atome de zinc \noyau{Zn}{30}{65} contient-il ?
\item Combien d'�lectrons comporte-t-il ?
\item Calculer la charge totale des protons
sachant qu'un proton a pour charge $e = 1,6.10^{-19}~C$.
\item Calculer la charge totale des �lectrons
sachant qu'un �lectron a pour charge $-e = -1,6.10^{-19}~C$.
\item En d�duire la charge de l'atome de Zinc.
\item Ce r�sultat est-il identique pour tous les atomes ?
\item A l'issue d'une r�action dite d'oxydation, un atome de zinc $Zn$
  se transforme en un ion $Zn^{2+}$.
  \begin{enumerate}
  \item Donnez l'�quation de cette r�action
  (en faisant intervenir un ou plusieurs �lectrons not�s $e^-$).
  \item Indiquez la charge (en coulomb $C$) de cet ion.
  \end{enumerate}
\end{enumerate}
\end{exercice}

\vressort{3} % miroir plan
\ds{Devoir Surveill�}{
%
}

\nomprenomclasse

\setcounter{numexercice}{0}

%\renewcommand{\tabularx}[1]{>{\centering}m{#1}} 

%\newcommand{\tabularxc}[1]{\tabularx{>{\centering}m{#1}}}

\vressort{3}

\begin{exercice}{Connaissance sur l'atome}%\\
\begin{enumerate}
\item De quoi est compos� un atome ?
\item Que signifie les lettres $A$, $Z$ et $X$ dans la repr�sentation \noyau{X}{Z}{A} ?
\item Comment trouve-t-on le nombre de neutrons d'un atome de l'�l�ment pr�c�dent.
\item Si un atome a $5$ protons, combien-a-t-il d'�lectrons ? Pourquoi ?
\item Qu'est-ce qui caract�rise un �l�ment chimique ?
\item Qu'est-ce qu'un isotope ?
\end{enumerate}
\end{exercice}



\vressort{3}



\begin{exercice}{Composition des atomes}\\
En vous aidant du tableau p�riodique des �l�ments,
compl�ter le tableau suivant :

\medskip

\noindent
%\begin{tabularx}{\textwidth}{|>{\centering}X|>{\centering}X|>{\centering}X|>{\centering}X|>{\centering}X|}
% \begin{tabularx}{\linewidth}{|X|X|X|X|X|}
% \hline
% \emph{nom}       & \emph{symbole}  & \emph{protons} & \emph{neutrons}
% & \emph{nucl�ons} \tbnl
% carbone   & \noyau{C}{6}{14}   &         &          & \rule[-0.5cm]{0cm}{1cm}         \tbnl
% fluor     & \noyau{F}{9}{19}   &         &          & \rule[-0.5cm]{0cm}{1cm}         \tbnl
% sodium    & \noyau{Na}{11}{23} &         &          & \rule[-0.5cm]{0cm}{1cm}         \tbnl
% oxyg�ne   & \noyau{O}{8}{16}   &         &          & \rule[-0.5cm]{0cm}{1cm}         \tbnl
% hydrog�ne &          &         & 0        & \rule[-0.5cm]{0cm}{1cm}         \tbnl
%           & \noyau{Cl}{17}{35} &         &          &  \rule[-0.5cm]{0cm}{1cm}        \tbnl
%           &          & 8       & \rule[-0.5cm]{0cm}{1cm}         & 16       \tbnl
% \end{tabularx}



\begin{tabularx}{\linewidth}{|>{\mystrut}X|X|X|X|X|}
\hline
% multicolumn pour faire dispara�tre le \mystrut
\multicolumn{1}{|X|}{\emph{nom}} & \emph{symbole}  &
\emph{protons} & \emph{neutrons} & \emph{nucl�ons} \tbnl
carbone   & \noyau{C}{6}{14}   &   &   &    \tbnl
fluor     & \noyau{F}{9}{19}   &   &   &    \tbnl
sodium    & \noyau{Na}{11}{23} &   &   &    \tbnl
oxyg�ne   & \noyau{O}{8}{16}   &   &   &    \tbnl
hydrog�ne &                    &   & 0 &    \tbnl
          & \noyau{Cl}{17}{35} &   &   &    \tbnl
          &                    & 8 &   & 16 \tbnl
\end{tabularx}


\end{exercice}


\vressort{3}


\begin{exercice}{Masse d'un atome de carbone 12}\\
Soit le carbone $12$ not� \noyau{C}{6}{12}.
\begin{enumerate}
\item L'�l�ment carbone peut-il avoir $5$ protons ? Pourquoi ?
\item Calculer la masse du noyau d'un atome de carbone $12$
sachant que la masse d'un nucl�on est $m_n = 1,67.10^{-27}~kg$
\item Calculer la masse des �lectrons de l'atome de carbone 12
sachant que la masse d'un �lectron vaut $m_e = 9,1.10^{-31}~kg$
\item Comparer la masse des �lectrons de l'atome � la masse du noyau.
Que concluez-vous ?
\item En d�duire, sans nouveau calcul, la masse de l'atome de carbone
  $12$.
\end{enumerate}
\end{exercice}

\newpage

\vressort{1}

\begin{exercice}{Couches �lectroniques}\\
Dans l'�tat le plus stable de l'atome, appel� �tat fondamental,
les �lectrons occupent successivement les couches,
en commen�ant par celles qui sont les plus proches du noyau : 
d'abord $K$ puis $L$ puis $M$.

Lorsqu'une couche est pleine, ou encore satur�e, on passe � la suivante.

La derni�re couche occup�e est appel�e couche externe.\\
Toutes les autres sont appel�es couches internes.

\medskip

\noindent
%\begin{tabularx}{\textwidth}{|>{\centering}X|>{\centering}X|>{\centering}X|>{\centering}X|}
\begin{tabularx}{\textwidth}{|>{\mystrut}X|X|X|X|}
\hline
\multicolumn{1}{|X|}{\emph{Symbole de la couche}}       & $K$ & $L$ & $M$  \tbnl
\emph{Nombre maximal d'�lectrons} & $2$ & $8$ & $18$ \tbnl
\end{tabularx}

\medskip

Ainsi, par exemple, l'atome de chlore ($Z=17$) a la configuration �lectronique :
$(K)^2(L)^8(M)^{7}$.

\begin{enumerate}
\item Indiquez le nombre d'�lectrons et donnez la configuration des atomes suivants :
  \begin{enumerate}
  \item \noyau{H}{1}{}
  \item \noyau{O}{8}{}
  \item \noyau{C}{6}{}
  \item \noyau{Ne}{10}{}
  \end{enumerate}
\item Parmi les ions ci-desssous, pr�cisez s'il s'agit d'anions ou de cations.
Indiquez le nombre d'�lectrons et donnez la configuration �lectronique
des ions suivants :
  \begin{enumerate}
  \item $Be^{2+}$ ($Z=4$)
  \item $Al^{3+}$ ($Z=13$)
  \item $O^{2-}$ ($Z=8$)
  \item $F^{-}$ ($Z=9$)
  \end{enumerate}
\end{enumerate}

\end{exercice}


\vressort{5}


\begin{exercice}{Charge d'un atome de Zinc}%\\
\begin{enumerate}
\item Combien de protons l'atome de zinc \noyau{Zn}{30}{65} contient-il ?
\item Combien d'�lectrons comporte-t-il ?
\item Calculer la charge totale des protons
sachant qu'un proton a pour charge $e = 1,6.10^{-19}~C$.
\item Calculer la charge totale des �lectrons
sachant qu'un �lectron a pour charge $-e = -1,6.10^{-19}~C$.
\item En d�duire la charge de l'atome de Zinc.
\item Ce r�sultat est-il identique pour tous les atomes ?
\item A l'issue d'une r�action dite d'oxydation, un atome de zinc $Zn$
  se transforme en un ion $Zn^{2+}$.
  \begin{enumerate}
  \item Donnez l'�quation de cette r�action
  (en faisant intervenir un ou plusieurs �lectrons not�s $e^-$).
  \item Indiquez la charge (en coulomb $C$) de cet ion.
  \end{enumerate}
\end{enumerate}
\end{exercice}

\vressort{3} % lentilles minces
\ds{Devoir Surveill�}{
%
}

\nomprenomclasse

\setcounter{numexercice}{0}

%\renewcommand{\tabularx}[1]{>{\centering}m{#1}} 

%\newcommand{\tabularxc}[1]{\tabularx{>{\centering}m{#1}}}

\vressort{3}

\begin{exercice}{Connaissance sur l'atome}%\\
\begin{enumerate}
\item De quoi est compos� un atome ?
\item Que signifie les lettres $A$, $Z$ et $X$ dans la repr�sentation \noyau{X}{Z}{A} ?
\item Comment trouve-t-on le nombre de neutrons d'un atome de l'�l�ment pr�c�dent.
\item Si un atome a $5$ protons, combien-a-t-il d'�lectrons ? Pourquoi ?
\item Qu'est-ce qui caract�rise un �l�ment chimique ?
\item Qu'est-ce qu'un isotope ?
\end{enumerate}
\end{exercice}



\vressort{3}



\begin{exercice}{Composition des atomes}\\
En vous aidant du tableau p�riodique des �l�ments,
compl�ter le tableau suivant :

\medskip

\noindent
%\begin{tabularx}{\textwidth}{|>{\centering}X|>{\centering}X|>{\centering}X|>{\centering}X|>{\centering}X|}
% \begin{tabularx}{\linewidth}{|X|X|X|X|X|}
% \hline
% \emph{nom}       & \emph{symbole}  & \emph{protons} & \emph{neutrons}
% & \emph{nucl�ons} \tbnl
% carbone   & \noyau{C}{6}{14}   &         &          & \rule[-0.5cm]{0cm}{1cm}         \tbnl
% fluor     & \noyau{F}{9}{19}   &         &          & \rule[-0.5cm]{0cm}{1cm}         \tbnl
% sodium    & \noyau{Na}{11}{23} &         &          & \rule[-0.5cm]{0cm}{1cm}         \tbnl
% oxyg�ne   & \noyau{O}{8}{16}   &         &          & \rule[-0.5cm]{0cm}{1cm}         \tbnl
% hydrog�ne &          &         & 0        & \rule[-0.5cm]{0cm}{1cm}         \tbnl
%           & \noyau{Cl}{17}{35} &         &          &  \rule[-0.5cm]{0cm}{1cm}        \tbnl
%           &          & 8       & \rule[-0.5cm]{0cm}{1cm}         & 16       \tbnl
% \end{tabularx}



\begin{tabularx}{\linewidth}{|>{\mystrut}X|X|X|X|X|}
\hline
% multicolumn pour faire dispara�tre le \mystrut
\multicolumn{1}{|X|}{\emph{nom}} & \emph{symbole}  &
\emph{protons} & \emph{neutrons} & \emph{nucl�ons} \tbnl
carbone   & \noyau{C}{6}{14}   &   &   &    \tbnl
fluor     & \noyau{F}{9}{19}   &   &   &    \tbnl
sodium    & \noyau{Na}{11}{23} &   &   &    \tbnl
oxyg�ne   & \noyau{O}{8}{16}   &   &   &    \tbnl
hydrog�ne &                    &   & 0 &    \tbnl
          & \noyau{Cl}{17}{35} &   &   &    \tbnl
          &                    & 8 &   & 16 \tbnl
\end{tabularx}


\end{exercice}


\vressort{3}


\begin{exercice}{Masse d'un atome de carbone 12}\\
Soit le carbone $12$ not� \noyau{C}{6}{12}.
\begin{enumerate}
\item L'�l�ment carbone peut-il avoir $5$ protons ? Pourquoi ?
\item Calculer la masse du noyau d'un atome de carbone $12$
sachant que la masse d'un nucl�on est $m_n = 1,67.10^{-27}~kg$
\item Calculer la masse des �lectrons de l'atome de carbone 12
sachant que la masse d'un �lectron vaut $m_e = 9,1.10^{-31}~kg$
\item Comparer la masse des �lectrons de l'atome � la masse du noyau.
Que concluez-vous ?
\item En d�duire, sans nouveau calcul, la masse de l'atome de carbone
  $12$.
\end{enumerate}
\end{exercice}

\newpage

\vressort{1}

\begin{exercice}{Couches �lectroniques}\\
Dans l'�tat le plus stable de l'atome, appel� �tat fondamental,
les �lectrons occupent successivement les couches,
en commen�ant par celles qui sont les plus proches du noyau : 
d'abord $K$ puis $L$ puis $M$.

Lorsqu'une couche est pleine, ou encore satur�e, on passe � la suivante.

La derni�re couche occup�e est appel�e couche externe.\\
Toutes les autres sont appel�es couches internes.

\medskip

\noindent
%\begin{tabularx}{\textwidth}{|>{\centering}X|>{\centering}X|>{\centering}X|>{\centering}X|}
\begin{tabularx}{\textwidth}{|>{\mystrut}X|X|X|X|}
\hline
\multicolumn{1}{|X|}{\emph{Symbole de la couche}}       & $K$ & $L$ & $M$  \tbnl
\emph{Nombre maximal d'�lectrons} & $2$ & $8$ & $18$ \tbnl
\end{tabularx}

\medskip

Ainsi, par exemple, l'atome de chlore ($Z=17$) a la configuration �lectronique :
$(K)^2(L)^8(M)^{7}$.

\begin{enumerate}
\item Indiquez le nombre d'�lectrons et donnez la configuration des atomes suivants :
  \begin{enumerate}
  \item \noyau{H}{1}{}
  \item \noyau{O}{8}{}
  \item \noyau{C}{6}{}
  \item \noyau{Ne}{10}{}
  \end{enumerate}
\item Parmi les ions ci-desssous, pr�cisez s'il s'agit d'anions ou de cations.
Indiquez le nombre d'�lectrons et donnez la configuration �lectronique
des ions suivants :
  \begin{enumerate}
  \item $Be^{2+}$ ($Z=4$)
  \item $Al^{3+}$ ($Z=13$)
  \item $O^{2-}$ ($Z=8$)
  \item $F^{-}$ ($Z=9$)
  \end{enumerate}
\end{enumerate}

\end{exercice}


\vressort{5}


\begin{exercice}{Charge d'un atome de Zinc}%\\
\begin{enumerate}
\item Combien de protons l'atome de zinc \noyau{Zn}{30}{65} contient-il ?
\item Combien d'�lectrons comporte-t-il ?
\item Calculer la charge totale des protons
sachant qu'un proton a pour charge $e = 1,6.10^{-19}~C$.
\item Calculer la charge totale des �lectrons
sachant qu'un �lectron a pour charge $-e = -1,6.10^{-19}~C$.
\item En d�duire la charge de l'atome de Zinc.
\item Ce r�sultat est-il identique pour tous les atomes ?
\item A l'issue d'une r�action dite d'oxydation, un atome de zinc $Zn$
  se transforme en un ion $Zn^{2+}$.
  \begin{enumerate}
  \item Donnez l'�quation de cette r�action
  (en faisant intervenir un ou plusieurs �lectrons not�s $e^-$).
  \item Indiquez la charge (en coulomb $C$) de cet ion.
  \end{enumerate}
\end{enumerate}
\end{exercice}

\vressort{3}
\ds{Devoir Surveill�}{
%
}

\nomprenomclasse

\setcounter{numexercice}{0}

%\renewcommand{\tabularx}[1]{>{\centering}m{#1}} 

%\newcommand{\tabularxc}[1]{\tabularx{>{\centering}m{#1}}}

\vressort{3}

\begin{exercice}{Connaissance sur l'atome}%\\
\begin{enumerate}
\item De quoi est compos� un atome ?
\item Que signifie les lettres $A$, $Z$ et $X$ dans la repr�sentation \noyau{X}{Z}{A} ?
\item Comment trouve-t-on le nombre de neutrons d'un atome de l'�l�ment pr�c�dent.
\item Si un atome a $5$ protons, combien-a-t-il d'�lectrons ? Pourquoi ?
\item Qu'est-ce qui caract�rise un �l�ment chimique ?
\item Qu'est-ce qu'un isotope ?
\end{enumerate}
\end{exercice}



\vressort{3}



\begin{exercice}{Composition des atomes}\\
En vous aidant du tableau p�riodique des �l�ments,
compl�ter le tableau suivant :

\medskip

\noindent
%\begin{tabularx}{\textwidth}{|>{\centering}X|>{\centering}X|>{\centering}X|>{\centering}X|>{\centering}X|}
% \begin{tabularx}{\linewidth}{|X|X|X|X|X|}
% \hline
% \emph{nom}       & \emph{symbole}  & \emph{protons} & \emph{neutrons}
% & \emph{nucl�ons} \tbnl
% carbone   & \noyau{C}{6}{14}   &         &          & \rule[-0.5cm]{0cm}{1cm}         \tbnl
% fluor     & \noyau{F}{9}{19}   &         &          & \rule[-0.5cm]{0cm}{1cm}         \tbnl
% sodium    & \noyau{Na}{11}{23} &         &          & \rule[-0.5cm]{0cm}{1cm}         \tbnl
% oxyg�ne   & \noyau{O}{8}{16}   &         &          & \rule[-0.5cm]{0cm}{1cm}         \tbnl
% hydrog�ne &          &         & 0        & \rule[-0.5cm]{0cm}{1cm}         \tbnl
%           & \noyau{Cl}{17}{35} &         &          &  \rule[-0.5cm]{0cm}{1cm}        \tbnl
%           &          & 8       & \rule[-0.5cm]{0cm}{1cm}         & 16       \tbnl
% \end{tabularx}



\begin{tabularx}{\linewidth}{|>{\mystrut}X|X|X|X|X|}
\hline
% multicolumn pour faire dispara�tre le \mystrut
\multicolumn{1}{|X|}{\emph{nom}} & \emph{symbole}  &
\emph{protons} & \emph{neutrons} & \emph{nucl�ons} \tbnl
carbone   & \noyau{C}{6}{14}   &   &   &    \tbnl
fluor     & \noyau{F}{9}{19}   &   &   &    \tbnl
sodium    & \noyau{Na}{11}{23} &   &   &    \tbnl
oxyg�ne   & \noyau{O}{8}{16}   &   &   &    \tbnl
hydrog�ne &                    &   & 0 &    \tbnl
          & \noyau{Cl}{17}{35} &   &   &    \tbnl
          &                    & 8 &   & 16 \tbnl
\end{tabularx}


\end{exercice}


\vressort{3}


\begin{exercice}{Masse d'un atome de carbone 12}\\
Soit le carbone $12$ not� \noyau{C}{6}{12}.
\begin{enumerate}
\item L'�l�ment carbone peut-il avoir $5$ protons ? Pourquoi ?
\item Calculer la masse du noyau d'un atome de carbone $12$
sachant que la masse d'un nucl�on est $m_n = 1,67.10^{-27}~kg$
\item Calculer la masse des �lectrons de l'atome de carbone 12
sachant que la masse d'un �lectron vaut $m_e = 9,1.10^{-31}~kg$
\item Comparer la masse des �lectrons de l'atome � la masse du noyau.
Que concluez-vous ?
\item En d�duire, sans nouveau calcul, la masse de l'atome de carbone
  $12$.
\end{enumerate}
\end{exercice}

\newpage

\vressort{1}

\begin{exercice}{Couches �lectroniques}\\
Dans l'�tat le plus stable de l'atome, appel� �tat fondamental,
les �lectrons occupent successivement les couches,
en commen�ant par celles qui sont les plus proches du noyau : 
d'abord $K$ puis $L$ puis $M$.

Lorsqu'une couche est pleine, ou encore satur�e, on passe � la suivante.

La derni�re couche occup�e est appel�e couche externe.\\
Toutes les autres sont appel�es couches internes.

\medskip

\noindent
%\begin{tabularx}{\textwidth}{|>{\centering}X|>{\centering}X|>{\centering}X|>{\centering}X|}
\begin{tabularx}{\textwidth}{|>{\mystrut}X|X|X|X|}
\hline
\multicolumn{1}{|X|}{\emph{Symbole de la couche}}       & $K$ & $L$ & $M$  \tbnl
\emph{Nombre maximal d'�lectrons} & $2$ & $8$ & $18$ \tbnl
\end{tabularx}

\medskip

Ainsi, par exemple, l'atome de chlore ($Z=17$) a la configuration �lectronique :
$(K)^2(L)^8(M)^{7}$.

\begin{enumerate}
\item Indiquez le nombre d'�lectrons et donnez la configuration des atomes suivants :
  \begin{enumerate}
  \item \noyau{H}{1}{}
  \item \noyau{O}{8}{}
  \item \noyau{C}{6}{}
  \item \noyau{Ne}{10}{}
  \end{enumerate}
\item Parmi les ions ci-desssous, pr�cisez s'il s'agit d'anions ou de cations.
Indiquez le nombre d'�lectrons et donnez la configuration �lectronique
des ions suivants :
  \begin{enumerate}
  \item $Be^{2+}$ ($Z=4$)
  \item $Al^{3+}$ ($Z=13$)
  \item $O^{2-}$ ($Z=8$)
  \item $F^{-}$ ($Z=9$)
  \end{enumerate}
\end{enumerate}

\end{exercice}


\vressort{5}


\begin{exercice}{Charge d'un atome de Zinc}%\\
\begin{enumerate}
\item Combien de protons l'atome de zinc \noyau{Zn}{30}{65} contient-il ?
\item Combien d'�lectrons comporte-t-il ?
\item Calculer la charge totale des protons
sachant qu'un proton a pour charge $e = 1,6.10^{-19}~C$.
\item Calculer la charge totale des �lectrons
sachant qu'un �lectron a pour charge $-e = -1,6.10^{-19}~C$.
\item En d�duire la charge de l'atome de Zinc.
\item Ce r�sultat est-il identique pour tous les atomes ?
\item A l'issue d'une r�action dite d'oxydation, un atome de zinc $Zn$
  se transforme en un ion $Zn^{2+}$.
  \begin{enumerate}
  \item Donnez l'�quation de cette r�action
  (en faisant intervenir un ou plusieurs �lectrons not�s $e^-$).
  \item Indiquez la charge (en coulomb $C$) de cet ion.
  \end{enumerate}
\end{enumerate}
\end{exercice}

\vressort{3} % Lentilles (Images, rel de conjugaison)
\ds{Devoir Surveill�}{
%
}

\nomprenomclasse

\setcounter{numexercice}{0}

%\renewcommand{\tabularx}[1]{>{\centering}m{#1}} 

%\newcommand{\tabularxc}[1]{\tabularx{>{\centering}m{#1}}}

\vressort{3}

\begin{exercice}{Connaissance sur l'atome}%\\
\begin{enumerate}
\item De quoi est compos� un atome ?
\item Que signifie les lettres $A$, $Z$ et $X$ dans la repr�sentation \noyau{X}{Z}{A} ?
\item Comment trouve-t-on le nombre de neutrons d'un atome de l'�l�ment pr�c�dent.
\item Si un atome a $5$ protons, combien-a-t-il d'�lectrons ? Pourquoi ?
\item Qu'est-ce qui caract�rise un �l�ment chimique ?
\item Qu'est-ce qu'un isotope ?
\end{enumerate}
\end{exercice}



\vressort{3}



\begin{exercice}{Composition des atomes}\\
En vous aidant du tableau p�riodique des �l�ments,
compl�ter le tableau suivant :

\medskip

\noindent
%\begin{tabularx}{\textwidth}{|>{\centering}X|>{\centering}X|>{\centering}X|>{\centering}X|>{\centering}X|}
% \begin{tabularx}{\linewidth}{|X|X|X|X|X|}
% \hline
% \emph{nom}       & \emph{symbole}  & \emph{protons} & \emph{neutrons}
% & \emph{nucl�ons} \tbnl
% carbone   & \noyau{C}{6}{14}   &         &          & \rule[-0.5cm]{0cm}{1cm}         \tbnl
% fluor     & \noyau{F}{9}{19}   &         &          & \rule[-0.5cm]{0cm}{1cm}         \tbnl
% sodium    & \noyau{Na}{11}{23} &         &          & \rule[-0.5cm]{0cm}{1cm}         \tbnl
% oxyg�ne   & \noyau{O}{8}{16}   &         &          & \rule[-0.5cm]{0cm}{1cm}         \tbnl
% hydrog�ne &          &         & 0        & \rule[-0.5cm]{0cm}{1cm}         \tbnl
%           & \noyau{Cl}{17}{35} &         &          &  \rule[-0.5cm]{0cm}{1cm}        \tbnl
%           &          & 8       & \rule[-0.5cm]{0cm}{1cm}         & 16       \tbnl
% \end{tabularx}



\begin{tabularx}{\linewidth}{|>{\mystrut}X|X|X|X|X|}
\hline
% multicolumn pour faire dispara�tre le \mystrut
\multicolumn{1}{|X|}{\emph{nom}} & \emph{symbole}  &
\emph{protons} & \emph{neutrons} & \emph{nucl�ons} \tbnl
carbone   & \noyau{C}{6}{14}   &   &   &    \tbnl
fluor     & \noyau{F}{9}{19}   &   &   &    \tbnl
sodium    & \noyau{Na}{11}{23} &   &   &    \tbnl
oxyg�ne   & \noyau{O}{8}{16}   &   &   &    \tbnl
hydrog�ne &                    &   & 0 &    \tbnl
          & \noyau{Cl}{17}{35} &   &   &    \tbnl
          &                    & 8 &   & 16 \tbnl
\end{tabularx}


\end{exercice}


\vressort{3}


\begin{exercice}{Masse d'un atome de carbone 12}\\
Soit le carbone $12$ not� \noyau{C}{6}{12}.
\begin{enumerate}
\item L'�l�ment carbone peut-il avoir $5$ protons ? Pourquoi ?
\item Calculer la masse du noyau d'un atome de carbone $12$
sachant que la masse d'un nucl�on est $m_n = 1,67.10^{-27}~kg$
\item Calculer la masse des �lectrons de l'atome de carbone 12
sachant que la masse d'un �lectron vaut $m_e = 9,1.10^{-31}~kg$
\item Comparer la masse des �lectrons de l'atome � la masse du noyau.
Que concluez-vous ?
\item En d�duire, sans nouveau calcul, la masse de l'atome de carbone
  $12$.
\end{enumerate}
\end{exercice}

\newpage

\vressort{1}

\begin{exercice}{Couches �lectroniques}\\
Dans l'�tat le plus stable de l'atome, appel� �tat fondamental,
les �lectrons occupent successivement les couches,
en commen�ant par celles qui sont les plus proches du noyau : 
d'abord $K$ puis $L$ puis $M$.

Lorsqu'une couche est pleine, ou encore satur�e, on passe � la suivante.

La derni�re couche occup�e est appel�e couche externe.\\
Toutes les autres sont appel�es couches internes.

\medskip

\noindent
%\begin{tabularx}{\textwidth}{|>{\centering}X|>{\centering}X|>{\centering}X|>{\centering}X|}
\begin{tabularx}{\textwidth}{|>{\mystrut}X|X|X|X|}
\hline
\multicolumn{1}{|X|}{\emph{Symbole de la couche}}       & $K$ & $L$ & $M$  \tbnl
\emph{Nombre maximal d'�lectrons} & $2$ & $8$ & $18$ \tbnl
\end{tabularx}

\medskip

Ainsi, par exemple, l'atome de chlore ($Z=17$) a la configuration �lectronique :
$(K)^2(L)^8(M)^{7}$.

\begin{enumerate}
\item Indiquez le nombre d'�lectrons et donnez la configuration des atomes suivants :
  \begin{enumerate}
  \item \noyau{H}{1}{}
  \item \noyau{O}{8}{}
  \item \noyau{C}{6}{}
  \item \noyau{Ne}{10}{}
  \end{enumerate}
\item Parmi les ions ci-desssous, pr�cisez s'il s'agit d'anions ou de cations.
Indiquez le nombre d'�lectrons et donnez la configuration �lectronique
des ions suivants :
  \begin{enumerate}
  \item $Be^{2+}$ ($Z=4$)
  \item $Al^{3+}$ ($Z=13$)
  \item $O^{2-}$ ($Z=8$)
  \item $F^{-}$ ($Z=9$)
  \end{enumerate}
\end{enumerate}

\end{exercice}


\vressort{5}


\begin{exercice}{Charge d'un atome de Zinc}%\\
\begin{enumerate}
\item Combien de protons l'atome de zinc \noyau{Zn}{30}{65} contient-il ?
\item Combien d'�lectrons comporte-t-il ?
\item Calculer la charge totale des protons
sachant qu'un proton a pour charge $e = 1,6.10^{-19}~C$.
\item Calculer la charge totale des �lectrons
sachant qu'un �lectron a pour charge $-e = -1,6.10^{-19}~C$.
\item En d�duire la charge de l'atome de Zinc.
\item Ce r�sultat est-il identique pour tous les atomes ?
\item A l'issue d'une r�action dite d'oxydation, un atome de zinc $Zn$
  se transforme en un ion $Zn^{2+}$.
  \begin{enumerate}
  \item Donnez l'�quation de cette r�action
  (en faisant intervenir un ou plusieurs �lectrons not�s $e^-$).
  \item Indiquez la charge (en coulomb $C$) de cet ion.
  \end{enumerate}
\end{enumerate}
\end{exercice}

\vressort{3} % Focom�trie lentilles minces
% lunette astronomique
% autres instruments d'optique ?




\chapitre{\'Electromagn�tisme}
\ds{Devoir Surveill�}{
%
}

\nomprenomclasse

\setcounter{numexercice}{0}

%\renewcommand{\tabularx}[1]{>{\centering}m{#1}} 

%\newcommand{\tabularxc}[1]{\tabularx{>{\centering}m{#1}}}

\vressort{3}

\begin{exercice}{Connaissance sur l'atome}%\\
\begin{enumerate}
\item De quoi est compos� un atome ?
\item Que signifie les lettres $A$, $Z$ et $X$ dans la repr�sentation \noyau{X}{Z}{A} ?
\item Comment trouve-t-on le nombre de neutrons d'un atome de l'�l�ment pr�c�dent.
\item Si un atome a $5$ protons, combien-a-t-il d'�lectrons ? Pourquoi ?
\item Qu'est-ce qui caract�rise un �l�ment chimique ?
\item Qu'est-ce qu'un isotope ?
\end{enumerate}
\end{exercice}



\vressort{3}



\begin{exercice}{Composition des atomes}\\
En vous aidant du tableau p�riodique des �l�ments,
compl�ter le tableau suivant :

\medskip

\noindent
%\begin{tabularx}{\textwidth}{|>{\centering}X|>{\centering}X|>{\centering}X|>{\centering}X|>{\centering}X|}
% \begin{tabularx}{\linewidth}{|X|X|X|X|X|}
% \hline
% \emph{nom}       & \emph{symbole}  & \emph{protons} & \emph{neutrons}
% & \emph{nucl�ons} \tbnl
% carbone   & \noyau{C}{6}{14}   &         &          & \rule[-0.5cm]{0cm}{1cm}         \tbnl
% fluor     & \noyau{F}{9}{19}   &         &          & \rule[-0.5cm]{0cm}{1cm}         \tbnl
% sodium    & \noyau{Na}{11}{23} &         &          & \rule[-0.5cm]{0cm}{1cm}         \tbnl
% oxyg�ne   & \noyau{O}{8}{16}   &         &          & \rule[-0.5cm]{0cm}{1cm}         \tbnl
% hydrog�ne &          &         & 0        & \rule[-0.5cm]{0cm}{1cm}         \tbnl
%           & \noyau{Cl}{17}{35} &         &          &  \rule[-0.5cm]{0cm}{1cm}        \tbnl
%           &          & 8       & \rule[-0.5cm]{0cm}{1cm}         & 16       \tbnl
% \end{tabularx}



\begin{tabularx}{\linewidth}{|>{\mystrut}X|X|X|X|X|}
\hline
% multicolumn pour faire dispara�tre le \mystrut
\multicolumn{1}{|X|}{\emph{nom}} & \emph{symbole}  &
\emph{protons} & \emph{neutrons} & \emph{nucl�ons} \tbnl
carbone   & \noyau{C}{6}{14}   &   &   &    \tbnl
fluor     & \noyau{F}{9}{19}   &   &   &    \tbnl
sodium    & \noyau{Na}{11}{23} &   &   &    \tbnl
oxyg�ne   & \noyau{O}{8}{16}   &   &   &    \tbnl
hydrog�ne &                    &   & 0 &    \tbnl
          & \noyau{Cl}{17}{35} &   &   &    \tbnl
          &                    & 8 &   & 16 \tbnl
\end{tabularx}


\end{exercice}


\vressort{3}


\begin{exercice}{Masse d'un atome de carbone 12}\\
Soit le carbone $12$ not� \noyau{C}{6}{12}.
\begin{enumerate}
\item L'�l�ment carbone peut-il avoir $5$ protons ? Pourquoi ?
\item Calculer la masse du noyau d'un atome de carbone $12$
sachant que la masse d'un nucl�on est $m_n = 1,67.10^{-27}~kg$
\item Calculer la masse des �lectrons de l'atome de carbone 12
sachant que la masse d'un �lectron vaut $m_e = 9,1.10^{-31}~kg$
\item Comparer la masse des �lectrons de l'atome � la masse du noyau.
Que concluez-vous ?
\item En d�duire, sans nouveau calcul, la masse de l'atome de carbone
  $12$.
\end{enumerate}
\end{exercice}

\newpage

\vressort{1}

\begin{exercice}{Couches �lectroniques}\\
Dans l'�tat le plus stable de l'atome, appel� �tat fondamental,
les �lectrons occupent successivement les couches,
en commen�ant par celles qui sont les plus proches du noyau : 
d'abord $K$ puis $L$ puis $M$.

Lorsqu'une couche est pleine, ou encore satur�e, on passe � la suivante.

La derni�re couche occup�e est appel�e couche externe.\\
Toutes les autres sont appel�es couches internes.

\medskip

\noindent
%\begin{tabularx}{\textwidth}{|>{\centering}X|>{\centering}X|>{\centering}X|>{\centering}X|}
\begin{tabularx}{\textwidth}{|>{\mystrut}X|X|X|X|}
\hline
\multicolumn{1}{|X|}{\emph{Symbole de la couche}}       & $K$ & $L$ & $M$  \tbnl
\emph{Nombre maximal d'�lectrons} & $2$ & $8$ & $18$ \tbnl
\end{tabularx}

\medskip

Ainsi, par exemple, l'atome de chlore ($Z=17$) a la configuration �lectronique :
$(K)^2(L)^8(M)^{7}$.

\begin{enumerate}
\item Indiquez le nombre d'�lectrons et donnez la configuration des atomes suivants :
  \begin{enumerate}
  \item \noyau{H}{1}{}
  \item \noyau{O}{8}{}
  \item \noyau{C}{6}{}
  \item \noyau{Ne}{10}{}
  \end{enumerate}
\item Parmi les ions ci-desssous, pr�cisez s'il s'agit d'anions ou de cations.
Indiquez le nombre d'�lectrons et donnez la configuration �lectronique
des ions suivants :
  \begin{enumerate}
  \item $Be^{2+}$ ($Z=4$)
  \item $Al^{3+}$ ($Z=13$)
  \item $O^{2-}$ ($Z=8$)
  \item $F^{-}$ ($Z=9$)
  \end{enumerate}
\end{enumerate}

\end{exercice}


\vressort{5}


\begin{exercice}{Charge d'un atome de Zinc}%\\
\begin{enumerate}
\item Combien de protons l'atome de zinc \noyau{Zn}{30}{65} contient-il ?
\item Combien d'�lectrons comporte-t-il ?
\item Calculer la charge totale des protons
sachant qu'un proton a pour charge $e = 1,6.10^{-19}~C$.
\item Calculer la charge totale des �lectrons
sachant qu'un �lectron a pour charge $-e = -1,6.10^{-19}~C$.
\item En d�duire la charge de l'atome de Zinc.
\item Ce r�sultat est-il identique pour tous les atomes ?
\item A l'issue d'une r�action dite d'oxydation, un atome de zinc $Zn$
  se transforme en un ion $Zn^{2+}$.
  \begin{enumerate}
  \item Donnez l'�quation de cette r�action
  (en faisant intervenir un ou plusieurs �lectrons not�s $e^-$).
  \item Indiquez la charge (en coulomb $C$) de cet ion.
  \end{enumerate}
\end{enumerate}
\end{exercice}

\vressort{3} % Champ magn�tique
\ds{Devoir Surveill�}{
%
}

\nomprenomclasse

\setcounter{numexercice}{0}

%\renewcommand{\tabularx}[1]{>{\centering}m{#1}} 

%\newcommand{\tabularxc}[1]{\tabularx{>{\centering}m{#1}}}

\vressort{3}

\begin{exercice}{Connaissance sur l'atome}%\\
\begin{enumerate}
\item De quoi est compos� un atome ?
\item Que signifie les lettres $A$, $Z$ et $X$ dans la repr�sentation \noyau{X}{Z}{A} ?
\item Comment trouve-t-on le nombre de neutrons d'un atome de l'�l�ment pr�c�dent.
\item Si un atome a $5$ protons, combien-a-t-il d'�lectrons ? Pourquoi ?
\item Qu'est-ce qui caract�rise un �l�ment chimique ?
\item Qu'est-ce qu'un isotope ?
\end{enumerate}
\end{exercice}



\vressort{3}



\begin{exercice}{Composition des atomes}\\
En vous aidant du tableau p�riodique des �l�ments,
compl�ter le tableau suivant :

\medskip

\noindent
%\begin{tabularx}{\textwidth}{|>{\centering}X|>{\centering}X|>{\centering}X|>{\centering}X|>{\centering}X|}
% \begin{tabularx}{\linewidth}{|X|X|X|X|X|}
% \hline
% \emph{nom}       & \emph{symbole}  & \emph{protons} & \emph{neutrons}
% & \emph{nucl�ons} \tbnl
% carbone   & \noyau{C}{6}{14}   &         &          & \rule[-0.5cm]{0cm}{1cm}         \tbnl
% fluor     & \noyau{F}{9}{19}   &         &          & \rule[-0.5cm]{0cm}{1cm}         \tbnl
% sodium    & \noyau{Na}{11}{23} &         &          & \rule[-0.5cm]{0cm}{1cm}         \tbnl
% oxyg�ne   & \noyau{O}{8}{16}   &         &          & \rule[-0.5cm]{0cm}{1cm}         \tbnl
% hydrog�ne &          &         & 0        & \rule[-0.5cm]{0cm}{1cm}         \tbnl
%           & \noyau{Cl}{17}{35} &         &          &  \rule[-0.5cm]{0cm}{1cm}        \tbnl
%           &          & 8       & \rule[-0.5cm]{0cm}{1cm}         & 16       \tbnl
% \end{tabularx}



\begin{tabularx}{\linewidth}{|>{\mystrut}X|X|X|X|X|}
\hline
% multicolumn pour faire dispara�tre le \mystrut
\multicolumn{1}{|X|}{\emph{nom}} & \emph{symbole}  &
\emph{protons} & \emph{neutrons} & \emph{nucl�ons} \tbnl
carbone   & \noyau{C}{6}{14}   &   &   &    \tbnl
fluor     & \noyau{F}{9}{19}   &   &   &    \tbnl
sodium    & \noyau{Na}{11}{23} &   &   &    \tbnl
oxyg�ne   & \noyau{O}{8}{16}   &   &   &    \tbnl
hydrog�ne &                    &   & 0 &    \tbnl
          & \noyau{Cl}{17}{35} &   &   &    \tbnl
          &                    & 8 &   & 16 \tbnl
\end{tabularx}


\end{exercice}


\vressort{3}


\begin{exercice}{Masse d'un atome de carbone 12}\\
Soit le carbone $12$ not� \noyau{C}{6}{12}.
\begin{enumerate}
\item L'�l�ment carbone peut-il avoir $5$ protons ? Pourquoi ?
\item Calculer la masse du noyau d'un atome de carbone $12$
sachant que la masse d'un nucl�on est $m_n = 1,67.10^{-27}~kg$
\item Calculer la masse des �lectrons de l'atome de carbone 12
sachant que la masse d'un �lectron vaut $m_e = 9,1.10^{-31}~kg$
\item Comparer la masse des �lectrons de l'atome � la masse du noyau.
Que concluez-vous ?
\item En d�duire, sans nouveau calcul, la masse de l'atome de carbone
  $12$.
\end{enumerate}
\end{exercice}

\newpage

\vressort{1}

\begin{exercice}{Couches �lectroniques}\\
Dans l'�tat le plus stable de l'atome, appel� �tat fondamental,
les �lectrons occupent successivement les couches,
en commen�ant par celles qui sont les plus proches du noyau : 
d'abord $K$ puis $L$ puis $M$.

Lorsqu'une couche est pleine, ou encore satur�e, on passe � la suivante.

La derni�re couche occup�e est appel�e couche externe.\\
Toutes les autres sont appel�es couches internes.

\medskip

\noindent
%\begin{tabularx}{\textwidth}{|>{\centering}X|>{\centering}X|>{\centering}X|>{\centering}X|}
\begin{tabularx}{\textwidth}{|>{\mystrut}X|X|X|X|}
\hline
\multicolumn{1}{|X|}{\emph{Symbole de la couche}}       & $K$ & $L$ & $M$  \tbnl
\emph{Nombre maximal d'�lectrons} & $2$ & $8$ & $18$ \tbnl
\end{tabularx}

\medskip

Ainsi, par exemple, l'atome de chlore ($Z=17$) a la configuration �lectronique :
$(K)^2(L)^8(M)^{7}$.

\begin{enumerate}
\item Indiquez le nombre d'�lectrons et donnez la configuration des atomes suivants :
  \begin{enumerate}
  \item \noyau{H}{1}{}
  \item \noyau{O}{8}{}
  \item \noyau{C}{6}{}
  \item \noyau{Ne}{10}{}
  \end{enumerate}
\item Parmi les ions ci-desssous, pr�cisez s'il s'agit d'anions ou de cations.
Indiquez le nombre d'�lectrons et donnez la configuration �lectronique
des ions suivants :
  \begin{enumerate}
  \item $Be^{2+}$ ($Z=4$)
  \item $Al^{3+}$ ($Z=13$)
  \item $O^{2-}$ ($Z=8$)
  \item $F^{-}$ ($Z=9$)
  \end{enumerate}
\end{enumerate}

\end{exercice}


\vressort{5}


\begin{exercice}{Charge d'un atome de Zinc}%\\
\begin{enumerate}
\item Combien de protons l'atome de zinc \noyau{Zn}{30}{65} contient-il ?
\item Combien d'�lectrons comporte-t-il ?
\item Calculer la charge totale des protons
sachant qu'un proton a pour charge $e = 1,6.10^{-19}~C$.
\item Calculer la charge totale des �lectrons
sachant qu'un �lectron a pour charge $-e = -1,6.10^{-19}~C$.
\item En d�duire la charge de l'atome de Zinc.
\item Ce r�sultat est-il identique pour tous les atomes ?
\item A l'issue d'une r�action dite d'oxydation, un atome de zinc $Zn$
  se transforme en un ion $Zn^{2+}$.
  \begin{enumerate}
  \item Donnez l'�quation de cette r�action
  (en faisant intervenir un ou plusieurs �lectrons not�s $e^-$).
  \item Indiquez la charge (en coulomb $C$) de cet ion.
  \end{enumerate}
\end{enumerate}
\end{exercice}

\vressort{3} % Bobines de Helmholtz
\ds{Devoir Surveill�}{
%
}

\nomprenomclasse

\setcounter{numexercice}{0}

%\renewcommand{\tabularx}[1]{>{\centering}m{#1}} 

%\newcommand{\tabularxc}[1]{\tabularx{>{\centering}m{#1}}}

\vressort{3}

\begin{exercice}{Connaissance sur l'atome}%\\
\begin{enumerate}
\item De quoi est compos� un atome ?
\item Que signifie les lettres $A$, $Z$ et $X$ dans la repr�sentation \noyau{X}{Z}{A} ?
\item Comment trouve-t-on le nombre de neutrons d'un atome de l'�l�ment pr�c�dent.
\item Si un atome a $5$ protons, combien-a-t-il d'�lectrons ? Pourquoi ?
\item Qu'est-ce qui caract�rise un �l�ment chimique ?
\item Qu'est-ce qu'un isotope ?
\end{enumerate}
\end{exercice}



\vressort{3}



\begin{exercice}{Composition des atomes}\\
En vous aidant du tableau p�riodique des �l�ments,
compl�ter le tableau suivant :

\medskip

\noindent
%\begin{tabularx}{\textwidth}{|>{\centering}X|>{\centering}X|>{\centering}X|>{\centering}X|>{\centering}X|}
% \begin{tabularx}{\linewidth}{|X|X|X|X|X|}
% \hline
% \emph{nom}       & \emph{symbole}  & \emph{protons} & \emph{neutrons}
% & \emph{nucl�ons} \tbnl
% carbone   & \noyau{C}{6}{14}   &         &          & \rule[-0.5cm]{0cm}{1cm}         \tbnl
% fluor     & \noyau{F}{9}{19}   &         &          & \rule[-0.5cm]{0cm}{1cm}         \tbnl
% sodium    & \noyau{Na}{11}{23} &         &          & \rule[-0.5cm]{0cm}{1cm}         \tbnl
% oxyg�ne   & \noyau{O}{8}{16}   &         &          & \rule[-0.5cm]{0cm}{1cm}         \tbnl
% hydrog�ne &          &         & 0        & \rule[-0.5cm]{0cm}{1cm}         \tbnl
%           & \noyau{Cl}{17}{35} &         &          &  \rule[-0.5cm]{0cm}{1cm}        \tbnl
%           &          & 8       & \rule[-0.5cm]{0cm}{1cm}         & 16       \tbnl
% \end{tabularx}



\begin{tabularx}{\linewidth}{|>{\mystrut}X|X|X|X|X|}
\hline
% multicolumn pour faire dispara�tre le \mystrut
\multicolumn{1}{|X|}{\emph{nom}} & \emph{symbole}  &
\emph{protons} & \emph{neutrons} & \emph{nucl�ons} \tbnl
carbone   & \noyau{C}{6}{14}   &   &   &    \tbnl
fluor     & \noyau{F}{9}{19}   &   &   &    \tbnl
sodium    & \noyau{Na}{11}{23} &   &   &    \tbnl
oxyg�ne   & \noyau{O}{8}{16}   &   &   &    \tbnl
hydrog�ne &                    &   & 0 &    \tbnl
          & \noyau{Cl}{17}{35} &   &   &    \tbnl
          &                    & 8 &   & 16 \tbnl
\end{tabularx}


\end{exercice}


\vressort{3}


\begin{exercice}{Masse d'un atome de carbone 12}\\
Soit le carbone $12$ not� \noyau{C}{6}{12}.
\begin{enumerate}
\item L'�l�ment carbone peut-il avoir $5$ protons ? Pourquoi ?
\item Calculer la masse du noyau d'un atome de carbone $12$
sachant que la masse d'un nucl�on est $m_n = 1,67.10^{-27}~kg$
\item Calculer la masse des �lectrons de l'atome de carbone 12
sachant que la masse d'un �lectron vaut $m_e = 9,1.10^{-31}~kg$
\item Comparer la masse des �lectrons de l'atome � la masse du noyau.
Que concluez-vous ?
\item En d�duire, sans nouveau calcul, la masse de l'atome de carbone
  $12$.
\end{enumerate}
\end{exercice}

\newpage

\vressort{1}

\begin{exercice}{Couches �lectroniques}\\
Dans l'�tat le plus stable de l'atome, appel� �tat fondamental,
les �lectrons occupent successivement les couches,
en commen�ant par celles qui sont les plus proches du noyau : 
d'abord $K$ puis $L$ puis $M$.

Lorsqu'une couche est pleine, ou encore satur�e, on passe � la suivante.

La derni�re couche occup�e est appel�e couche externe.\\
Toutes les autres sont appel�es couches internes.

\medskip

\noindent
%\begin{tabularx}{\textwidth}{|>{\centering}X|>{\centering}X|>{\centering}X|>{\centering}X|}
\begin{tabularx}{\textwidth}{|>{\mystrut}X|X|X|X|}
\hline
\multicolumn{1}{|X|}{\emph{Symbole de la couche}}       & $K$ & $L$ & $M$  \tbnl
\emph{Nombre maximal d'�lectrons} & $2$ & $8$ & $18$ \tbnl
\end{tabularx}

\medskip

Ainsi, par exemple, l'atome de chlore ($Z=17$) a la configuration �lectronique :
$(K)^2(L)^8(M)^{7}$.

\begin{enumerate}
\item Indiquez le nombre d'�lectrons et donnez la configuration des atomes suivants :
  \begin{enumerate}
  \item \noyau{H}{1}{}
  \item \noyau{O}{8}{}
  \item \noyau{C}{6}{}
  \item \noyau{Ne}{10}{}
  \end{enumerate}
\item Parmi les ions ci-desssous, pr�cisez s'il s'agit d'anions ou de cations.
Indiquez le nombre d'�lectrons et donnez la configuration �lectronique
des ions suivants :
  \begin{enumerate}
  \item $Be^{2+}$ ($Z=4$)
  \item $Al^{3+}$ ($Z=13$)
  \item $O^{2-}$ ($Z=8$)
  \item $F^{-}$ ($Z=9$)
  \end{enumerate}
\end{enumerate}

\end{exercice}


\vressort{5}


\begin{exercice}{Charge d'un atome de Zinc}%\\
\begin{enumerate}
\item Combien de protons l'atome de zinc \noyau{Zn}{30}{65} contient-il ?
\item Combien d'�lectrons comporte-t-il ?
\item Calculer la charge totale des protons
sachant qu'un proton a pour charge $e = 1,6.10^{-19}~C$.
\item Calculer la charge totale des �lectrons
sachant qu'un �lectron a pour charge $-e = -1,6.10^{-19}~C$.
\item En d�duire la charge de l'atome de Zinc.
\item Ce r�sultat est-il identique pour tous les atomes ?
\item A l'issue d'une r�action dite d'oxydation, un atome de zinc $Zn$
  se transforme en un ion $Zn^{2+}$.
  \begin{enumerate}
  \item Donnez l'�quation de cette r�action
  (en faisant intervenir un ou plusieurs �lectrons not�s $e^-$).
  \item Indiquez la charge (en coulomb $C$) de cet ion.
  \end{enumerate}
\end{enumerate}
\end{exercice}

\vressort{3} % Force de Laplace
\ds{Devoir Surveill�}{
%
}

\nomprenomclasse

\setcounter{numexercice}{0}

%\renewcommand{\tabularx}[1]{>{\centering}m{#1}} 

%\newcommand{\tabularxc}[1]{\tabularx{>{\centering}m{#1}}}

\vressort{3}

\begin{exercice}{Connaissance sur l'atome}%\\
\begin{enumerate}
\item De quoi est compos� un atome ?
\item Que signifie les lettres $A$, $Z$ et $X$ dans la repr�sentation \noyau{X}{Z}{A} ?
\item Comment trouve-t-on le nombre de neutrons d'un atome de l'�l�ment pr�c�dent.
\item Si un atome a $5$ protons, combien-a-t-il d'�lectrons ? Pourquoi ?
\item Qu'est-ce qui caract�rise un �l�ment chimique ?
\item Qu'est-ce qu'un isotope ?
\end{enumerate}
\end{exercice}



\vressort{3}



\begin{exercice}{Composition des atomes}\\
En vous aidant du tableau p�riodique des �l�ments,
compl�ter le tableau suivant :

\medskip

\noindent
%\begin{tabularx}{\textwidth}{|>{\centering}X|>{\centering}X|>{\centering}X|>{\centering}X|>{\centering}X|}
% \begin{tabularx}{\linewidth}{|X|X|X|X|X|}
% \hline
% \emph{nom}       & \emph{symbole}  & \emph{protons} & \emph{neutrons}
% & \emph{nucl�ons} \tbnl
% carbone   & \noyau{C}{6}{14}   &         &          & \rule[-0.5cm]{0cm}{1cm}         \tbnl
% fluor     & \noyau{F}{9}{19}   &         &          & \rule[-0.5cm]{0cm}{1cm}         \tbnl
% sodium    & \noyau{Na}{11}{23} &         &          & \rule[-0.5cm]{0cm}{1cm}         \tbnl
% oxyg�ne   & \noyau{O}{8}{16}   &         &          & \rule[-0.5cm]{0cm}{1cm}         \tbnl
% hydrog�ne &          &         & 0        & \rule[-0.5cm]{0cm}{1cm}         \tbnl
%           & \noyau{Cl}{17}{35} &         &          &  \rule[-0.5cm]{0cm}{1cm}        \tbnl
%           &          & 8       & \rule[-0.5cm]{0cm}{1cm}         & 16       \tbnl
% \end{tabularx}



\begin{tabularx}{\linewidth}{|>{\mystrut}X|X|X|X|X|}
\hline
% multicolumn pour faire dispara�tre le \mystrut
\multicolumn{1}{|X|}{\emph{nom}} & \emph{symbole}  &
\emph{protons} & \emph{neutrons} & \emph{nucl�ons} \tbnl
carbone   & \noyau{C}{6}{14}   &   &   &    \tbnl
fluor     & \noyau{F}{9}{19}   &   &   &    \tbnl
sodium    & \noyau{Na}{11}{23} &   &   &    \tbnl
oxyg�ne   & \noyau{O}{8}{16}   &   &   &    \tbnl
hydrog�ne &                    &   & 0 &    \tbnl
          & \noyau{Cl}{17}{35} &   &   &    \tbnl
          &                    & 8 &   & 16 \tbnl
\end{tabularx}


\end{exercice}


\vressort{3}


\begin{exercice}{Masse d'un atome de carbone 12}\\
Soit le carbone $12$ not� \noyau{C}{6}{12}.
\begin{enumerate}
\item L'�l�ment carbone peut-il avoir $5$ protons ? Pourquoi ?
\item Calculer la masse du noyau d'un atome de carbone $12$
sachant que la masse d'un nucl�on est $m_n = 1,67.10^{-27}~kg$
\item Calculer la masse des �lectrons de l'atome de carbone 12
sachant que la masse d'un �lectron vaut $m_e = 9,1.10^{-31}~kg$
\item Comparer la masse des �lectrons de l'atome � la masse du noyau.
Que concluez-vous ?
\item En d�duire, sans nouveau calcul, la masse de l'atome de carbone
  $12$.
\end{enumerate}
\end{exercice}

\newpage

\vressort{1}

\begin{exercice}{Couches �lectroniques}\\
Dans l'�tat le plus stable de l'atome, appel� �tat fondamental,
les �lectrons occupent successivement les couches,
en commen�ant par celles qui sont les plus proches du noyau : 
d'abord $K$ puis $L$ puis $M$.

Lorsqu'une couche est pleine, ou encore satur�e, on passe � la suivante.

La derni�re couche occup�e est appel�e couche externe.\\
Toutes les autres sont appel�es couches internes.

\medskip

\noindent
%\begin{tabularx}{\textwidth}{|>{\centering}X|>{\centering}X|>{\centering}X|>{\centering}X|}
\begin{tabularx}{\textwidth}{|>{\mystrut}X|X|X|X|}
\hline
\multicolumn{1}{|X|}{\emph{Symbole de la couche}}       & $K$ & $L$ & $M$  \tbnl
\emph{Nombre maximal d'�lectrons} & $2$ & $8$ & $18$ \tbnl
\end{tabularx}

\medskip

Ainsi, par exemple, l'atome de chlore ($Z=17$) a la configuration �lectronique :
$(K)^2(L)^8(M)^{7}$.

\begin{enumerate}
\item Indiquez le nombre d'�lectrons et donnez la configuration des atomes suivants :
  \begin{enumerate}
  \item \noyau{H}{1}{}
  \item \noyau{O}{8}{}
  \item \noyau{C}{6}{}
  \item \noyau{Ne}{10}{}
  \end{enumerate}
\item Parmi les ions ci-desssous, pr�cisez s'il s'agit d'anions ou de cations.
Indiquez le nombre d'�lectrons et donnez la configuration �lectronique
des ions suivants :
  \begin{enumerate}
  \item $Be^{2+}$ ($Z=4$)
  \item $Al^{3+}$ ($Z=13$)
  \item $O^{2-}$ ($Z=8$)
  \item $F^{-}$ ($Z=9$)
  \end{enumerate}
\end{enumerate}

\end{exercice}


\vressort{5}


\begin{exercice}{Charge d'un atome de Zinc}%\\
\begin{enumerate}
\item Combien de protons l'atome de zinc \noyau{Zn}{30}{65} contient-il ?
\item Combien d'�lectrons comporte-t-il ?
\item Calculer la charge totale des protons
sachant qu'un proton a pour charge $e = 1,6.10^{-19}~C$.
\item Calculer la charge totale des �lectrons
sachant qu'un �lectron a pour charge $-e = -1,6.10^{-19}~C$.
\item En d�duire la charge de l'atome de Zinc.
\item Ce r�sultat est-il identique pour tous les atomes ?
\item A l'issue d'une r�action dite d'oxydation, un atome de zinc $Zn$
  se transforme en un ion $Zn^{2+}$.
  \begin{enumerate}
  \item Donnez l'�quation de cette r�action
  (en faisant intervenir un ou plusieurs �lectrons not�s $e^-$).
  \item Indiquez la charge (en coulomb $C$) de cet ion.
  \end{enumerate}
\end{enumerate}
\end{exercice}

\vressort{3} % Force de Lorentz





\chapitre{Devoir Surveill�} % 1S
\ds{Devoir Surveill�}{
%
}

\nomprenomclasse

\setcounter{numexercice}{0}

%\renewcommand{\tabularx}[1]{>{\centering}m{#1}} 

%\newcommand{\tabularxc}[1]{\tabularx{>{\centering}m{#1}}}

\vressort{3}

\begin{exercice}{Connaissance sur l'atome}%\\
\begin{enumerate}
\item De quoi est compos� un atome ?
\item Que signifie les lettres $A$, $Z$ et $X$ dans la repr�sentation \noyau{X}{Z}{A} ?
\item Comment trouve-t-on le nombre de neutrons d'un atome de l'�l�ment pr�c�dent.
\item Si un atome a $5$ protons, combien-a-t-il d'�lectrons ? Pourquoi ?
\item Qu'est-ce qui caract�rise un �l�ment chimique ?
\item Qu'est-ce qu'un isotope ?
\end{enumerate}
\end{exercice}



\vressort{3}



\begin{exercice}{Composition des atomes}\\
En vous aidant du tableau p�riodique des �l�ments,
compl�ter le tableau suivant :

\medskip

\noindent
%\begin{tabularx}{\textwidth}{|>{\centering}X|>{\centering}X|>{\centering}X|>{\centering}X|>{\centering}X|}
% \begin{tabularx}{\linewidth}{|X|X|X|X|X|}
% \hline
% \emph{nom}       & \emph{symbole}  & \emph{protons} & \emph{neutrons}
% & \emph{nucl�ons} \tbnl
% carbone   & \noyau{C}{6}{14}   &         &          & \rule[-0.5cm]{0cm}{1cm}         \tbnl
% fluor     & \noyau{F}{9}{19}   &         &          & \rule[-0.5cm]{0cm}{1cm}         \tbnl
% sodium    & \noyau{Na}{11}{23} &         &          & \rule[-0.5cm]{0cm}{1cm}         \tbnl
% oxyg�ne   & \noyau{O}{8}{16}   &         &          & \rule[-0.5cm]{0cm}{1cm}         \tbnl
% hydrog�ne &          &         & 0        & \rule[-0.5cm]{0cm}{1cm}         \tbnl
%           & \noyau{Cl}{17}{35} &         &          &  \rule[-0.5cm]{0cm}{1cm}        \tbnl
%           &          & 8       & \rule[-0.5cm]{0cm}{1cm}         & 16       \tbnl
% \end{tabularx}



\begin{tabularx}{\linewidth}{|>{\mystrut}X|X|X|X|X|}
\hline
% multicolumn pour faire dispara�tre le \mystrut
\multicolumn{1}{|X|}{\emph{nom}} & \emph{symbole}  &
\emph{protons} & \emph{neutrons} & \emph{nucl�ons} \tbnl
carbone   & \noyau{C}{6}{14}   &   &   &    \tbnl
fluor     & \noyau{F}{9}{19}   &   &   &    \tbnl
sodium    & \noyau{Na}{11}{23} &   &   &    \tbnl
oxyg�ne   & \noyau{O}{8}{16}   &   &   &    \tbnl
hydrog�ne &                    &   & 0 &    \tbnl
          & \noyau{Cl}{17}{35} &   &   &    \tbnl
          &                    & 8 &   & 16 \tbnl
\end{tabularx}


\end{exercice}


\vressort{3}


\begin{exercice}{Masse d'un atome de carbone 12}\\
Soit le carbone $12$ not� \noyau{C}{6}{12}.
\begin{enumerate}
\item L'�l�ment carbone peut-il avoir $5$ protons ? Pourquoi ?
\item Calculer la masse du noyau d'un atome de carbone $12$
sachant que la masse d'un nucl�on est $m_n = 1,67.10^{-27}~kg$
\item Calculer la masse des �lectrons de l'atome de carbone 12
sachant que la masse d'un �lectron vaut $m_e = 9,1.10^{-31}~kg$
\item Comparer la masse des �lectrons de l'atome � la masse du noyau.
Que concluez-vous ?
\item En d�duire, sans nouveau calcul, la masse de l'atome de carbone
  $12$.
\end{enumerate}
\end{exercice}

\newpage

\vressort{1}

\begin{exercice}{Couches �lectroniques}\\
Dans l'�tat le plus stable de l'atome, appel� �tat fondamental,
les �lectrons occupent successivement les couches,
en commen�ant par celles qui sont les plus proches du noyau : 
d'abord $K$ puis $L$ puis $M$.

Lorsqu'une couche est pleine, ou encore satur�e, on passe � la suivante.

La derni�re couche occup�e est appel�e couche externe.\\
Toutes les autres sont appel�es couches internes.

\medskip

\noindent
%\begin{tabularx}{\textwidth}{|>{\centering}X|>{\centering}X|>{\centering}X|>{\centering}X|}
\begin{tabularx}{\textwidth}{|>{\mystrut}X|X|X|X|}
\hline
\multicolumn{1}{|X|}{\emph{Symbole de la couche}}       & $K$ & $L$ & $M$  \tbnl
\emph{Nombre maximal d'�lectrons} & $2$ & $8$ & $18$ \tbnl
\end{tabularx}

\medskip

Ainsi, par exemple, l'atome de chlore ($Z=17$) a la configuration �lectronique :
$(K)^2(L)^8(M)^{7}$.

\begin{enumerate}
\item Indiquez le nombre d'�lectrons et donnez la configuration des atomes suivants :
  \begin{enumerate}
  \item \noyau{H}{1}{}
  \item \noyau{O}{8}{}
  \item \noyau{C}{6}{}
  \item \noyau{Ne}{10}{}
  \end{enumerate}
\item Parmi les ions ci-desssous, pr�cisez s'il s'agit d'anions ou de cations.
Indiquez le nombre d'�lectrons et donnez la configuration �lectronique
des ions suivants :
  \begin{enumerate}
  \item $Be^{2+}$ ($Z=4$)
  \item $Al^{3+}$ ($Z=13$)
  \item $O^{2-}$ ($Z=8$)
  \item $F^{-}$ ($Z=9$)
  \end{enumerate}
\end{enumerate}

\end{exercice}


\vressort{5}


\begin{exercice}{Charge d'un atome de Zinc}%\\
\begin{enumerate}
\item Combien de protons l'atome de zinc \noyau{Zn}{30}{65} contient-il ?
\item Combien d'�lectrons comporte-t-il ?
\item Calculer la charge totale des protons
sachant qu'un proton a pour charge $e = 1,6.10^{-19}~C$.
\item Calculer la charge totale des �lectrons
sachant qu'un �lectron a pour charge $-e = -1,6.10^{-19}~C$.
\item En d�duire la charge de l'atome de Zinc.
\item Ce r�sultat est-il identique pour tous les atomes ?
\item A l'issue d'une r�action dite d'oxydation, un atome de zinc $Zn$
  se transforme en un ion $Zn^{2+}$.
  \begin{enumerate}
  \item Donnez l'�quation de cette r�action
  (en faisant intervenir un ou plusieurs �lectrons not�s $e^-$).
  \item Indiquez la charge (en coulomb $C$) de cet ion.
  \end{enumerate}
\end{enumerate}
\end{exercice}

\vressort{3}
\ds{Devoir Surveill�}{
%
}

\nomprenomclasse

\setcounter{numexercice}{0}

%\renewcommand{\tabularx}[1]{>{\centering}m{#1}} 

%\newcommand{\tabularxc}[1]{\tabularx{>{\centering}m{#1}}}

\vressort{3}

\begin{exercice}{Connaissance sur l'atome}%\\
\begin{enumerate}
\item De quoi est compos� un atome ?
\item Que signifie les lettres $A$, $Z$ et $X$ dans la repr�sentation \noyau{X}{Z}{A} ?
\item Comment trouve-t-on le nombre de neutrons d'un atome de l'�l�ment pr�c�dent.
\item Si un atome a $5$ protons, combien-a-t-il d'�lectrons ? Pourquoi ?
\item Qu'est-ce qui caract�rise un �l�ment chimique ?
\item Qu'est-ce qu'un isotope ?
\end{enumerate}
\end{exercice}



\vressort{3}



\begin{exercice}{Composition des atomes}\\
En vous aidant du tableau p�riodique des �l�ments,
compl�ter le tableau suivant :

\medskip

\noindent
%\begin{tabularx}{\textwidth}{|>{\centering}X|>{\centering}X|>{\centering}X|>{\centering}X|>{\centering}X|}
% \begin{tabularx}{\linewidth}{|X|X|X|X|X|}
% \hline
% \emph{nom}       & \emph{symbole}  & \emph{protons} & \emph{neutrons}
% & \emph{nucl�ons} \tbnl
% carbone   & \noyau{C}{6}{14}   &         &          & \rule[-0.5cm]{0cm}{1cm}         \tbnl
% fluor     & \noyau{F}{9}{19}   &         &          & \rule[-0.5cm]{0cm}{1cm}         \tbnl
% sodium    & \noyau{Na}{11}{23} &         &          & \rule[-0.5cm]{0cm}{1cm}         \tbnl
% oxyg�ne   & \noyau{O}{8}{16}   &         &          & \rule[-0.5cm]{0cm}{1cm}         \tbnl
% hydrog�ne &          &         & 0        & \rule[-0.5cm]{0cm}{1cm}         \tbnl
%           & \noyau{Cl}{17}{35} &         &          &  \rule[-0.5cm]{0cm}{1cm}        \tbnl
%           &          & 8       & \rule[-0.5cm]{0cm}{1cm}         & 16       \tbnl
% \end{tabularx}



\begin{tabularx}{\linewidth}{|>{\mystrut}X|X|X|X|X|}
\hline
% multicolumn pour faire dispara�tre le \mystrut
\multicolumn{1}{|X|}{\emph{nom}} & \emph{symbole}  &
\emph{protons} & \emph{neutrons} & \emph{nucl�ons} \tbnl
carbone   & \noyau{C}{6}{14}   &   &   &    \tbnl
fluor     & \noyau{F}{9}{19}   &   &   &    \tbnl
sodium    & \noyau{Na}{11}{23} &   &   &    \tbnl
oxyg�ne   & \noyau{O}{8}{16}   &   &   &    \tbnl
hydrog�ne &                    &   & 0 &    \tbnl
          & \noyau{Cl}{17}{35} &   &   &    \tbnl
          &                    & 8 &   & 16 \tbnl
\end{tabularx}


\end{exercice}


\vressort{3}


\begin{exercice}{Masse d'un atome de carbone 12}\\
Soit le carbone $12$ not� \noyau{C}{6}{12}.
\begin{enumerate}
\item L'�l�ment carbone peut-il avoir $5$ protons ? Pourquoi ?
\item Calculer la masse du noyau d'un atome de carbone $12$
sachant que la masse d'un nucl�on est $m_n = 1,67.10^{-27}~kg$
\item Calculer la masse des �lectrons de l'atome de carbone 12
sachant que la masse d'un �lectron vaut $m_e = 9,1.10^{-31}~kg$
\item Comparer la masse des �lectrons de l'atome � la masse du noyau.
Que concluez-vous ?
\item En d�duire, sans nouveau calcul, la masse de l'atome de carbone
  $12$.
\end{enumerate}
\end{exercice}

\newpage

\vressort{1}

\begin{exercice}{Couches �lectroniques}\\
Dans l'�tat le plus stable de l'atome, appel� �tat fondamental,
les �lectrons occupent successivement les couches,
en commen�ant par celles qui sont les plus proches du noyau : 
d'abord $K$ puis $L$ puis $M$.

Lorsqu'une couche est pleine, ou encore satur�e, on passe � la suivante.

La derni�re couche occup�e est appel�e couche externe.\\
Toutes les autres sont appel�es couches internes.

\medskip

\noindent
%\begin{tabularx}{\textwidth}{|>{\centering}X|>{\centering}X|>{\centering}X|>{\centering}X|}
\begin{tabularx}{\textwidth}{|>{\mystrut}X|X|X|X|}
\hline
\multicolumn{1}{|X|}{\emph{Symbole de la couche}}       & $K$ & $L$ & $M$  \tbnl
\emph{Nombre maximal d'�lectrons} & $2$ & $8$ & $18$ \tbnl
\end{tabularx}

\medskip

Ainsi, par exemple, l'atome de chlore ($Z=17$) a la configuration �lectronique :
$(K)^2(L)^8(M)^{7}$.

\begin{enumerate}
\item Indiquez le nombre d'�lectrons et donnez la configuration des atomes suivants :
  \begin{enumerate}
  \item \noyau{H}{1}{}
  \item \noyau{O}{8}{}
  \item \noyau{C}{6}{}
  \item \noyau{Ne}{10}{}
  \end{enumerate}
\item Parmi les ions ci-desssous, pr�cisez s'il s'agit d'anions ou de cations.
Indiquez le nombre d'�lectrons et donnez la configuration �lectronique
des ions suivants :
  \begin{enumerate}
  \item $Be^{2+}$ ($Z=4$)
  \item $Al^{3+}$ ($Z=13$)
  \item $O^{2-}$ ($Z=8$)
  \item $F^{-}$ ($Z=9$)
  \end{enumerate}
\end{enumerate}

\end{exercice}


\vressort{5}


\begin{exercice}{Charge d'un atome de Zinc}%\\
\begin{enumerate}
\item Combien de protons l'atome de zinc \noyau{Zn}{30}{65} contient-il ?
\item Combien d'�lectrons comporte-t-il ?
\item Calculer la charge totale des protons
sachant qu'un proton a pour charge $e = 1,6.10^{-19}~C$.
\item Calculer la charge totale des �lectrons
sachant qu'un �lectron a pour charge $-e = -1,6.10^{-19}~C$.
\item En d�duire la charge de l'atome de Zinc.
\item Ce r�sultat est-il identique pour tous les atomes ?
\item A l'issue d'une r�action dite d'oxydation, un atome de zinc $Zn$
  se transforme en un ion $Zn^{2+}$.
  \begin{enumerate}
  \item Donnez l'�quation de cette r�action
  (en faisant intervenir un ou plusieurs �lectrons not�s $e^-$).
  \item Indiquez la charge (en coulomb $C$) de cet ion.
  \end{enumerate}
\end{enumerate}
\end{exercice}

\vressort{3}


