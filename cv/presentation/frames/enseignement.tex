\frame{
  \frametitle{Mon parcours au sein de l'\'Education Nationale}

  % \pause

  \begin{itemize}
  \item Professeur stagiaire (2002)
    \begin{itemize}
    \item Physique Chimie en 5$^{\mbox{i\`eme}}$,
      4$^{\mbox{i\`eme}}$\\
      (Coll�ge Jules Vernes - DEVILLE-LES-ROUEN)\\
      Stage en situation

      % \pause

    \item Physique Appliqu�e en 1$^{\mbox{i\`ere}}$ et Terminale \'Electronique\\
      (Lyc�e Marcel SEMBAT - SOTTEVILLE-LES-ROUEN)\\
      Stage de pratique accompagn�e
    
    \end{itemize}
  \end{itemize}
}

\frame{
  \frametitle{Mon parcours au sein de l'\'Education Nationale (suite)}
  

  \begin{itemize}
  \item Titulaire en Zone de Remplacement (TZR) dans le
    d�partement de la Haute-Vienne (2003)
    \begin{itemize}

    \item Classes Pr�paratoires aux Grandes \'Ecoles\\
      \small (Math. Sup. PCSI - Lyc�e Gay Lussac - LIMOGES)
      
      % \pause

      \normalsize

    \item Physique Industrielle en BTS CIRA\\
      \small (Lyc�e Raoul DAUTRY - LIMOGES)
      
      % \pause

      \normalsize
    \item Physique Appliqu�e en lyc�e technologique\\
      1$^{\mbox{i\`ere}}$ \'Electrotechnique,
      Terminale G�nie M�canique\\ \small
      (Lyc�e TURGOT - LIMOGES)
      
      % \pause

      \normalsize
      
    \item Physique Chimie en lyc�e g�n�ral\\ \small
      1$^{\mbox{i\`ere}}$ Scientifique,
      1$^{\mbox{i\`ere}}$ STL C, %Sciences et Techniques de Laboratoire
      Terminale STL Biochimie,
      \actuellement{Seconde Tronc commun et option PCL}\\
      (Lyc�e Raoul DAUTRY - LIMOGES)
      
      \normalsize   
      
    \end{itemize}
  \end{itemize}
}

\frame{
  \frametitle{Autres activit�s au sein de l'\'Education Nationale}

  \begin{itemize}
  \item Formation informatique pour les enseignants
    et les personnels de laboratoire\\
    \begin{itemize}
    \item Notions sur les r�seaux (classe de r�seau, adresse IP)
    \item Notions sur Windows (installation, s�curit�, partage...)
    \item Notions sur le web (aspirer un site, publier un site...)
    \item Quelques applications en milieu scolaire (VNC)
    \item Introduction � quelques logiciels libres
      \begin{itemize}
      \item OpenOffice.org
      \item GIMP
      \end{itemize}
    \end{itemize}
  \end{itemize}
\begin{center}
\url{http://www.celles.net/wikini/wakka.php?wiki=}\\
\url{InfoDautryFormation}
\end{center}
}

\frame{
  \frametitle{Autres activit�s au sein de l'\'Education Nationale}

  \begin{itemize}
  \item Formation informatique pour les enseignants
    et les personnels de laboratoire\\
    \begin{itemize}
    \item Notions sur les r�seaux (classe de r�seau, adresse IP)
    \item Notions sur Windows (installation, s�curit�, partage...)
    \item Notions sur le web (aspirer un site, publier un site...)
    \item Quelques applications en milieu scolaire (VNC)
    \end{itemize}
  \end{itemize}
\begin{center}
\url{http://www.celles.net/wikini/wakka.php?wiki=}\\
\url{InfoDautryFormation}
\end{center}
}

\frame{
  \frametitle{Autres activit�s au sein de l'\'Education Nationale}

  Syst�me de suivi des cours par internet pour une �l�ve hospitalis�e.

  \begin{itemize}
    \item Mise en place de SPIP
    \item Formation des �l�ves � l'utilisation en tant que r�dacteur
  \end{itemize}
  
 
\begin{center}
\url{http://lyc-dautry-87.ovh.org}\\
\end{center}
}