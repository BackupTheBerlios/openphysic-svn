% \subsection{Quelques r�alisations (informatique)}

\frame{
  % \frametitle{\insertsubsectionhead}
  \frametitle{Quelques r�alisations (informatique)}

  \begin{itemize}

  \item Site web 
    \begin{center}
      \url{http://www.celles.net}
    \end{center}
    
    \begin{itemize}
    \item Site traitant de physique, d'informatique, d'�lectronique
    \item Site bas� sur le Wiki (Wikini)
    \item Am�lioration du moteur de Wikini (PHP), ajout de fonctionnalit�s
    \end{itemize}
    
    \vressort{1}
    
  \item fieldEB, simulateur d'�lectrostatique et de magn�tostatique
    \begin{center}
      \url{http://www.celles.net/wikini/wakka.php?wiki=}\\
      \url{fieldEB}
    \end{center}

  \end{itemize}
}

\frame{
  \frametitle{Quelques r�alisations (informatique)} 
  
  \begin{itemize}
  \item Harmon, recomposition de signaux � l'aide des harmoniques
      \begin{center}
        \url{http://www.celles.net/wikini/wakka.php?wiki=}\\
        \url{Harmon}
      \end{center}

      \vressort{1} 


  \item Divers exemples de calculs num�riques en Physique et en Math�matiques avec Scilab ou avec avec un simple tableur
    
    \begin{center}
      \url{http://www.celles.net/wikini/wakka.php?wiki=}\\
      \url{CalculNumerique}
    \end{center}

  \end{itemize}
}

\frame{
  \frametitle{Quelques r�alisations (informatique)} 

  \begin{itemize}

  \item G�n�rateur en ligne de papiers sp�ciaux
    
    \begin{center}
      \url{http://www.celles.net/wikini/wakka.php?wiki=}\\
      \url{Papier}
    \end{center}

    \begin{itemize}
    \item \'Ecrit en PHP avec la biblioth�que FPDF pour g�n�rer des fichiers PDF
    \item D'autres papiers g�n�r�s avec \LaTeX\ et PSTricks sont �galement disponible
    \end{itemize}

    \vressort{1}

  \item Application en ligne de suivi de probl�mes informatiques
 
    \begin{center}
      \url{http://s.cls.free.fr/maintenance}
    \end{center}   

    \begin{itemize}
    \item \'Ecrit en PHP avec connexion � une base de donn�es MySQL
    \end{itemize}

  \end{itemize}
}