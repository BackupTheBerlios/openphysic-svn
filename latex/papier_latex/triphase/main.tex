% Fichier de style LaTeX de Sebastien CELLES
% Application : Cours, TD, TP
% \usepackage{scls}	% Definition style perso


\documentclass[a4paper]{article}
%\documentclass{report}
%\documentclass{book}
%\documentclass{slides}

\usepackage[frenchb]{babel}	% specification francaise
\usepackage[latin1]{inputenc}	% entree clavier latin1
\usepackage[T1]{fontenc}	% sortie

% aeguil ?

%\usepackage{amsmath}		% package ams-math
%\usepackage{amsfonts}		% package ams-fonts
%\usepackage{amssymb}		% package ams-symb

% pstricks ou pdftricks
\usepackage{pstricks}
%\usepackage{pst-circ} % circuit en elec
\usepackage{pst-plot} % trac� de fonctions

% gestion des images (jpg, png, eps, etc...)
%\usepackage{graphicx}

% Insertion code source
%\usepackage{verbatim}

% *----------------*
% *---- Marges ----*
% *----------------*

% R�glage manuel
%%%%%%%%%%%%%%%%

\setlength{\voffset}{-1in}              % annulation d�calage vertical (default=-1in) % fait planter \layout
\setlength{\hoffset}{-1in}              % annulation d�calage horizontal (default=-1in) % fait planter \layout
\setlength{\textwidth}{17.5cm}          % largeur de la zone de texte (default=16.5cm)
\setlength{\textheight}{26cm}           % hauteur de la zone de texte (default=24cm)
\setlength{\marginparwidth}{40pt}       % largeur marge (default=40pt)
\setlength{\topmargin}{1cm}           % hauteur de la marge sup�rieure (default=2cm)
\setlength{\evensidemargin}{2cm}      % marge de gauche pour les pages paires (default=2cm)
\setlength{\oddsidemargin}{2cm}       % marge de gauche pour les pages impaires (default=2.5cm)
%\setlength{\parskip}{\smallskipamount}  % espacement entre les paragraphes (default=\smallskipamount)
\setlength{\skip\footins}{0.6cm}        % espacement entre le corps de texte et les notes de bas de page (default=0.6cm)


% R�glage automatique via le package geometry
%%%%%%%%%%%%%%%%%%%%%%%%%%%%%%%%%%%%%%%%%%%%%
%\usepackage{geometry}
%\geometry{ hmargin=2.5cm, vmargin=1.6cm }
%\geometry{ hmargin=2.5cm, vmargin=1.5cm } % default


% Tester les marges
\usepackage[francais]{layout}
%\usepackage{fullpage}

% *-----------------*
% *---- En-t�te ----*
% *-----------------*
\usepackage{fancyhdr}
% � placer apr�s \begin{document}....
%\pagestyle{fancy}  	 
%\lhead{haut de page gauche} % haut de page gauche
%\chead{haut de page centre} % haut de page centre
%\rhead{haut de page droite} % haut de page droite
%\lfoot{pied de page gauche} % pied de page gauche
%\cfoot{pied de page centr�} % pied de page centr�
%\rfoot{pied de page droit}  % pied de page droit
%\renewcommand{\headrulewidth}{0.4pt} % Trace un trait de s�paration de largeur 0,4 point. Mettre 0pt pour supprimer le trait.
%\renewcommand{\footrulewidth}{0.4pt}



% *---------------------------*
% *---- Compteur exercice ----*
% *---------------------------*
\newcounter{num}
%\renewcommand{\thenum}{\Roman{num}}	% chiffres romains
\renewcommand{\thenum}{\arabic{num}}	% chiffres arabes
%\renewcommand{\thenum}{\Alph{num}}	% lettre majuscule
%\renewcommand{\thenum}{\alph{num}}	% lettre minuscule
\newcommand{\exo}{\addtocounter{num}{1}{\noindent \textbf{Exercice~\thenum~:~}}}
\newenvironment{exercice}{\exo}{\vskip 0.5cm}



% *************************************
% * ---- Num�rotation personnelle ----*
% *************************************
\usepackage{titlesec}
\titlelabel{\thetitle.~} % ajoute un point et une espace ins�cable apr�s un num�ro de titre
\renewcommand{\thesection}{\Roman{section}}
\renewcommand{\thesubsection}{\Alph{subsection}}
\renewcommand{\thesubsubsection}{\arabic{subsubsection}}
\renewcommand{\theparagraph}{\alph{paragraph}}
\setcounter{secnumdepth}{4}
\titleformat{\paragraph}%
    {\normalfont \normalsize \itshape}%
    {\theparagraph}{1em}{}
  \titlespacing*{\paragraph}{0pt}{1ex plus 0.2ex}{0.5ex plus .2ex}

\begin{document}


\title{Syst�mes triphas�s}
\author{S. Celles}
%\author{\null}

%\date{\today}
\date{\null}

%thanks{Merci � eux !}


% En-t�te et pied de page
% \usepackage{fancyhdr}
\pagestyle{fancy}
\lhead{S. Celles} % haut de page gauche
\chead{Physique} % haut de page centre
\rhead{Syst�mes triphas�s} % haut de page droite
%\lfoot{pied de page gauche} % pied de page gauche
\cfoot{} % pied de page centr� (pas de num�r de page)
%\rfoot{pied de page droit}  % pied de page droit
\renewcommand{\headrulewidth}{0.4pt} % Trace un trait de s�paration de largeur 0,4 point. Mettre 0pt pour supprimer le trait.
%\renewcommand{\footrulewidth}{0.4pt} % Trace un trait de s�paration de largeur 0,4 point. Mettre 0pt pour supprimer le trait.


\maketitle % affichage du titre du document, de l'auteur, de la date

%\tableofcontent % table des mati�res


%\layout % Tester les marges n�cessite \usepackage[francais]{layout}

%\newpage % nouvelle page
% \\ % saut de ligne
%\footnote{Note de bas de page}




%\newpage

%\chapter{Chapitre} % classe book ou report mais pas article

%\section{}
%Voici la premi�re section.

%\subsection{SubSection}
%Voici une subsection.

%\subsubsection{SubSubSection}
%Voici une subsubsection.

%\paragraph{Paragraph}
%Voici un paragraph.



\section{Tensions simples}

\begin{pspicture}(-1,-2.5)(13,2.5)
%\psgrid[subgriddiv=1,griddots=10]
\psset{xunit=2}
\psset{yunit=2}
\psline{->}(0,-1)(0,1.5) % Axe des y
%\rput{0}(6,-0.25){$t$}
\psline{->}(-0.25,0)(6.75,0) % Axe des x
%\rput{0}(-0.25,1.75){$u_C$}
\psplot[linecolor=red,linewidth=1pt,linestyle=dotted]{-0.5}{6.5}{x 57.296 mul cos} % sin(x)
\psplot[linecolor=green,linewidth=1pt,linestyle=dotted]{-0.5}{6.5}{x 57.296 mul 120 sub cos} % sin(x-2pi/3)
\psplot[linecolor=blue,linewidth=1pt,linestyle=dotted]{-0.5}{6.5}{x 57.296 mul 240 sub cos} % sin(x-4pi/3)=sin(x+2pi/3)
\rput{0}(-0.25,1.25){$v_1$}
\rput{0}(2,1.25){$v_2$}
\rput{0}(4,1.25){$v_3$}
\rput{0}(6,1.25){$v_1$}
\end{pspicture}

\section{Tensions simples et tensions compos�es}

\begin{pspicture}(-1,-2.5)(13,2.5)
%\psgrid[subgriddiv=1,griddots=10]
\psset{xunit=2}
\psset{yunit=1}
\psline{->}(0,-2)(0,3) % Axe des y
%\rput{0}(6,-0.25){$t$}
\psline{->}(-0.25,0)(6.75,0) % Axe des x
%\rput{0}(-0.25,1.75){$u_C$}
\psplot[linecolor=red,linewidth=1pt]{-0.5}{6.5}{x 57.296 mul cos} % sin(x)
\psplot[linecolor=green,linewidth=1pt]{-0.5}{6.5}{x 57.296 mul 120 sub cos} % sin(x-2pi/3)
\psplot[linecolor=blue,linewidth=1pt]{-0.5}{6.5}{x 57.296 mul 240 sub cos} % sin(x-4pi/3)=sin(x+2pi/3)
% 1.732 = sqrt(3) 1rad=57.296�
\psplot[linecolor=red,linewidth=1pt,linestyle=dotted]{-0.5}{6.5}{x 57.296 mul cos x 57.296 mul 120 sub cos sub} % u12
\psplot[linecolor=green,linewidth=1pt,linestyle=dotted]{-0.5}{6.5}{x 57.296 mul cos x 57.296 mul 240 sub cos sub} % u13
\psplot[linecolor=blue,linewidth=1pt,linestyle=dotted]{-0.5}{6.5}{x 57.296 mul 120 sub cos x 57.296 mul 240 sub cos sub} % u23
\psplot[linecolor=red,linewidth=1pt,linestyle=dotted]{-0.5}{6.5}{x 57.296 mul cos x 57.296 mul 120 sub cos sub neg} % -u12
\psplot[linecolor=green,linewidth=1pt,linestyle=dotted]{-0.5}{6.5}{x 57.296 mul cos x 57.296 mul 240 sub cos sub neg} % -u13
\psplot[linecolor=blue,linewidth=1pt,linestyle=dotted]{-0.5}{6.5}{x 57.296 mul 120 sub cos x 57.296 mul 240 sub cos sub neg} % -u2
%\rput{0}(-0.25,1.25){$v_1$}
%\rput{0}(2,1.25){$v_2$}
%\rput{0}(4,1.25){$v_3$}

\rput{0}(-0.5,2.5){$u_{12}$}
\rput{0}(0.5,2.5){$u_{13}$}
\rput{0}(1.5,2.5){$u_{23}$}

\rput{0}(2.5,2.5){$u_{21}$}
\rput{0}(3.5,2.5){$u_{31}$}
\rput{0}(4.5,2.5){$u_{23}$}

\rput{0}(5.5,2.5){$u_{12}$}
\end{pspicture}

\section{Tensions compos�es}

\begin{pspicture}(-1,-2.5)(13,2.5)
%\psgrid[subgriddiv=1,griddots=10]
\psset{xunit=2}
\psset{yunit=2}
\psline{->}(0,-1)(0,1.5) % Axe des y
%\rput{0}(6,-0.25){$t$}
\psline{->}(-0.25,0)(6.75,0) % Axe des x
%\rput{0}(-0.25,1.75){$u_C$}
%\psplot[linecolor=red,linewidth=1pt]{-0.5}{6}{x 57.296 mul cos} % sin(x)
%\psplot[linecolor=green,linewidth=1pt]{-0.5}{6}{x 57.296 mul 120 sub cos} % sin(x-2pi/3)
%\psplot[linecolor=blue,linewidth=1pt]{-0.5}{6}{x 57.296 mul 240 sub cos} % sin(x-4pi/3)=sin(x+2pi/3)
% 1.732 = sqrt(3) 1rad=57.296�
\psplot[linecolor=red,linewidth=1pt,linestyle=dotted]{-0.5}{6.5}{x 57.296 mul cos x 57.296 mul 120 sub cos sub 1.732 div} % u12
\psplot[linecolor=green,linewidth=1pt,linestyle=dotted]{-0.5}{6.5}{x 57.296 mul cos x 57.296 mul 240 sub cos sub 1.732 div} % u13
\psplot[linecolor=blue,linewidth=1pt,linestyle=dotted]{-0.5}{6.5}{x 57.296 mul 120 sub cos x 57.296 mul 240 sub cos sub 1.732 div} % u23
\psplot[linecolor=red,linewidth=1pt,linestyle=dotted]{-0.5}{6.5}{x 57.296 mul cos x 57.296 mul 120 sub cos sub neg 1.732 div} % -u12
\psplot[linecolor=green,linewidth=1pt,linestyle=dotted]{-0.5}{6.5}{x 57.296 mul cos x 57.296 mul 240 sub cos sub neg 1.732 div} % -u13
\psplot[linecolor=blue,linewidth=1pt,linestyle=dotted]{-0.5}{6.5}{x 57.296 mul 120 sub cos x 57.296 mul 240 sub cos sub neg 1.732 div} % -u2

\rput{0}(-0.5,1.25){$u_{12}$}
\rput{0}(0.5,1.25){$u_{13}$}
\rput{0}(1.5,1.25){$u_{23}$}

\rput{0}(2.5,1.25){$u_{21}$}
\rput{0}(3.5,1.25){$u_{31}$}
\rput{0}(4.5,1.25){$u_{23}$}

\rput{0}(5.5,1.25){$u_{12}$}

\end{pspicture}



%\section*{Exercices}

%\setcounter{num}{0}
%\begin{exercice} Titre de l'exercice \\
%Soit ce premier exercice
%\end{exercice}

%\begin{exercice} \\
%Ce deuxi�me exercice
%\end{exercice}

\end{document}
