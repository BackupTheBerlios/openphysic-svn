\documentclass[a4paper]{article}

\usepackage[frenchb]{babel}	% specification francaise
\usepackage[latin1]{inputenc}	% entree clavier latin1
\usepackage[T1]{fontenc}	% sortie

\usepackage{pstricks}
\usepackage{pst-plot} % trac� de fonctions

% R�glage marges via le package geometry
\usepackage{geometry}
%\geometry{ hmargin=2.5cm, vmargin=1.6cm }
%\geometry{ hmargin=2.5cm, vmargin=1.5cm } % default
%\geometry{ hmargin=2.5cm, vmargin=1.6cm,lmargin=.5cm }
\geometry{ hmargin=0cm, vmargin=0cm,lmargin=0cm }


\begin{document}

\title{Papier}

\author{S. \textsc{CELLES}}
%\author{\null}

%\date{\today}
\date{\null}

%thanks{Merci}

\pagestyle{empty}


\begin{center}

\vspace*{\stretch{1}}

\newcommand{\xsize}{11}
\newcommand{\ysize}{15}
\newcommand{\numx}{22}
\newcommand{\numy}{30}


\begin{pspicture}(-5,-5)(5,5)
\psgrid[subgriddiv=1,griddots=5,gridlabels=0pt]
%\psset{xunit=1}
%\psset{yunit=1}


% Axe des y
\multido{\n=-\ysize+1}{\numy}{ \psline(-\xsize,\n)(\xsize,\n) }

% Axe des x
\multido{\n=-\xsize+1}{\numx}{ \psline(\n,-\ysize)(\n,\ysize) }
\end{pspicture}

\end{center}

\vspace*{\stretch{1}}

\end{document}
