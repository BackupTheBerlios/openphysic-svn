%%%%%%%%%%%%%%%%%%%%%%%%%%%%%%%%%%%%%%%%%%%%%%%%%%%%%%%%%%%
%
% 3D extensions for MetaPost by Anthony Phan.
% file: m3Dmanual.tex
% last modification: january 11, 2006
%
%%%%%%%%%%%%%%%%%%%%%%%%%%%%%%%%%%%%%%%%%%%%%%%%%%%%%%%%%%%

\input m3Dmanmac
\input epsf
\useoptions{preprint,magstep1,english,AMSsectioning}

\def\TM{\textsuperscript{TM}}
\def\cs#1{\ifmmode\hbox{\tt#1}\else{\tt#1}\fi}%
\def\var#1{\ifmmode\hbox{\it#1\/}\else{\it#1\/}\fi}%
\newdimen\fboxsep\fboxsep=2pt
\newdimen\fboxrule\fboxrule=0.4pt
\def\mybox#1{\vcenter{\setbox0=\hbox{\vrule width\fboxrule
	\vbox{\hrule height\fboxrule\kern\fboxsep
	\hbox{\kern\fboxsep\epsfbox{#1}\kern\fboxsep}\kern\fboxsep
	\hrule height\fboxrule}\vrule width\fboxrule}
	\copy0\kern\fboxsep\hbox to\wd0{\hss\tt#1\hss}}}%
%\def\mybox#1{\hbox{\tt#1}}
\title{m3D package manual}
\author{Anthony Phan, \date}
\maketitle

\section*{Introduction}

In the second half of the nineties (of the past century),
I discovered \TeX, MetaFont and,
then, MetaPost. I was truly enthousiastic about this last tool
since it allows a very simple inclusion of images produced by
some MetaFont-like language
into \TeX\ documents. My first attempt in 3D pictures
with MetaPost was very simple: the projection system was absolutely
rigid and the picture was composed of a few lines and labels and
only 4 filled flat surfaces whose colors where static.
Then I understood that it is necessary to have a parametrizable
frame since one does not always know if a picture, viewed from
a certain angle, will be meaningfull or not. So the first and main
step was done: having a parametrizable frame, dealing with space
coordinates, drawing contours if their projection on the screen
is well-oriented.

Then I heard of Denis Roegel's package ``3D''. Its syntax didn't
fit what I could have in mind about 3D-programming in MetaPost,
but it features facilities for {\tt eps} to {\tt gif} conversion
and animation. I stolled these animation facilities in order to
have fun and also to find the best angle for my picture.
Then I began to make more complex designs
intended to illustrate my web pages\dots

What was my {\it leitmotiv}\/
was to keep the closest as possible to usual MetaFont/MetaPost
programming, but also to consider every 3-dimensional object
as enough complex to deserve a quite technical code.
So I've come to think that what could be nice should be to
have a stable and powerfull basic program and add to it libraries
of common objects. Some more complex objects could then be build up
from those more basic ones by moving them, rotating them and
rescaling them. My idea about objects is certainly not related to
object-programming, it is just the naive notion of solid bodies.

The aim one can have with such a project is boundless.
The package ``m3D'' remains still in progress. I'm only
glad to be able to use it when a friend ask me if I can
draw a better picture than he does.

Also one of my believes is that any MetaPost programmer
prefers to build his/her own macros' system than relying
on someone else's programs---especially when these ones
are claimed to be unstable. Such a programmer would
simply have an overview of the syntax, of some special
hacks, may publish his/her own programs on Internet
and in this way give some feed back to every other
people interested in such programming.
 
\section{Basic concepts}

As anyone knows, MetaPost provides standard facilities
for manipulating 3~dimensional vectors with the variable
type ``color'': type-check, addition, substraction, scalar
multiplication. What may be seen as missing is affine
transformations of 3~dimensional vectors or pictures
but no one would complain about that since MetaPost
is a 2-dimensional oriented programming language
and anyone would accept using some higher level
control sequences for such tasks.

  
\section{Coordinates system}

\section{Projection system}

\section{Rendering parameters}

\list{definitions}

\item{\tt ObsZ :=} internal numeric.

\item{\tt Resolution :=} internal numeric.

\item{\tt Phong :=} internal numeric.

\item{\tt Specularity :=} internal numeric.

\item{\tt Luminosity :=} internal numeric.

\item{\tt Contrast :=} internal numeric.

\item{\tt Fog :=} internal numeric.

\item{\tt FogZ :=} internal numeric.

\item{\tt FogHalf :=} internal numeric.

\item{\tt FinePlotFlag :=} boolean.

\item{\tt ShadedTextFlag :=} boolean.

\item{\tt LightAtInfinity :=} boolean.

\item{\tt LightSource :=} color.

\item{\tt ObjectColor :=} color.

\endlist

\section{About the light source}

There is only one light source defined in {\tt m3Dplain.mp}.
It can be located at infinity ({\tt LightAtInfinity := true})
or at some point in the scenery ({\tt LightAtInfinity := false}).
In both cases its coordinates {\tt LightSource} refer to the
global screen frame and are not modified when objects are
translated or rotated, that is that the light is fixed
for the observer.

If one wants to link the light source to a peculiar object,
one has to write within the definition of the object something like
$$
	\hbox{\tt LightSource := GDir(x,y,z)}
$$
if the light source is located at infinity, or
$$
	\hbox{\tt LightSource := GCoord(x,y,z)}
$$
if the light source is located at some point in space
({\tt (x,y,z)} are here {\it local}\/ coordinates).
One should notice that it is barely the only direct use
of the control sequences {\tt GDir} and {\tt GCoord}.

If one needs more than one light source, one has to
define anew the control sequence {\tt Light} of
{\tt m3Dplain.mp} (good luck since it is already
a heavy machinery).

\section{Ordering and hidden bodies}

Drawing a single convex body is an easy task: one just have to draw
every facet which contour is seen, or projected, as a positively
(resp. negatively) oriented path when drawing its exterior (resp.
interior) side. So, hidden facets are no problem in these cases.
Switching between inside and outside can be done with \cs{Inside}
and \cs{Outside} control sequences. They simply reverse the condition
about orientation.

When dealing with non-convex bodies, one has to decompose these bodies
into their convex parts and draw them in a proper order. Thus, we come
to ordering. A fairly tricky control sequence named \cs{QuickSort}
has been designed for this purpose. Its argument is some text that
must contain at least two terms and its output is a control sequence
named \cs{SortedList} which content is the previous (expanded) list
ordered with respect to a \cs{SortCriterion}.
\cs{SortCriterion} is a control sequence with two arguments (members of
the list which is to be sorted) whose replacement text is a boolean.
By default, these arguments are triples and the condition is about their
actual depth relatively to the current observer. Thus
\cs{QuickSort(\var{list of triples})} would output \cs{SortedList}
whose replacement text is just the expected ordered list of triples.
One can change this just by adapting \cs{SortCriterion} in order
to sort numerics, pairs, strings, \dots

It wouldn't be very elegant to sort things this way if one wants to perform
a list of actions with respect to the depth of a list of points in space.
A more natural way to do so is to use the following procedure:

\cs{OnDepth;}

\cs{Refpoint \var{triple};}

\cs{Action (\var{delimited control sequences});}

...

\cs{endOnDepth;}

\noindent
What this procedure does is the following: save and reset a few things
at the \cs{OnDepth} statement; increase the \cs{Action\_counter} and
stores the current reference point (a \var{triple}) at \cs{Refpoint}
invocation; stores the \var{delimited control sequences} into a
variable control sequence numbered (here there is a little trick that
I have been looking for a very very long time) with the current
\cs{Action\_counter}; at \cs{endOnDepth} orders the list
$1,\dots,\cs{Action\_counter}$ with respect to the depth of the
reference points and then performs actions with respect to the sorted
list. The most interesting thing with this procedure is that actions
may depend on some parameters just as loop or macro parameters.  Note
that the \cs{SortCriterion} has a special and temporary meaning when
performing \cs{endOnDepth}: its two arguments are then some indexes in
$1,\dots,\cs{Action\_counter}$ and the comparison is done between the
depths of the two corresponding reference points.

\section{Integrating text}

This is clear that for some nice and funny pictures, one has to
integrate text in the three dimensional space, for instance when
moving a text around a picture as I've done once or twice. This is
rather particular. Defining any general scheme for doing so seems to
me rather pointless: it is too complicated, it is hard to imagine what
people would like to do, etc.  Anyway, some control sequence that
allows to move flat text around would be a basic stuff. Also, basic
programming of such things may help to design special control
sequences for more complicated tasks.

\subsection{Simple text}
An object named \cs{simpletext} is defined in \cs{m3Dplain.mp}.  Its
specific parameters consist, first, in a string describing the
alignment (\cs{"left"}, \cs{"justify"}, \cs{"center"} or \cs{"right"})
and a string telling where on the text the reference point should be
(typically \cs{"right"}, \cs{"urt"}, \cs{"top"}, \cs{"ulft"},
\cs{"left"}, \cs{"llft"}, \cs{"bot"}, \cs{"lrt"} or [say!]
\cs{"center"}), then, in a list of strings which would be displayed
one above another.

For instance, at a top level ({\it see}\/ further on about the scale parameter),
\tabindent=\parindent
\verbatim
	UseObject(simpletext, Origin, (90, 0, 90), 10pt, "justify", "ulft",
		"Come let me sing into your ear;",
		"Those dancing days are gone,",
		"All that silk and satin gear;",
		"Crouch upon a stone,",
		"Wrapping that foul body up",
		"In as foul a rag:",
		"I carry the sun in a golden cup;",
		"The moon in a silver bag."));
\endverbatim
\noindent
would display, at basepoint {\tt Origin}, with frame the current one
rotated by $(90,0,90)$, with scale $10\,\rm pt$, the first part of J.B.~Yeats'
poem \og Those dancing days are gone\fg. If it is reasonable,
each line would be justified and the basepoint would correspond
to the upper left corner of the text.

The {\tt simpletext} object uses a font named by the string {\tt
mthreeDfont} (default value: {\tt"rphvb"}). Its design size is defined
by the numerical variable named {\tt mtheeDfontsize} (default value:
{\tt10pt}).  The numerical parameter \cs{baselineskip} parameter has
its usual role (default value: {\tt12pt}). These parameters are only
related to the font, not to the {\tt CurrentScale}. Thus, when using
{\tt simpletext} into an object, beware of the \var{scale} given in
statements like
$$
\displaylines{\indent
\cs{UseObject(simpletext, \var{origin}, \var{Euler angles}, \var{scale},}
\hfill\cr\hfill
\cs{\var{alignment string}, \var{location string}, \var{list of strings})}\indent}
$$
since \var{scale} must be a {\it local scale}.

Justification is obtained by stretching the normal space
width of the font up to a \cs{TextStretchFactor}
(whose default value is $2$ which is quite large). Many
successive spaces into a sentence count as as many spaces: {\it they are
not reduced into a single space.}

Each character is drawn separately just to make sure that the affine
transformation acting on the character would be approximatively
correct if the projection is not linear. {\it Since every character
would have a special scaling, one must set\/ \cs{prologues}\/ to 1 or
2 in order to not overflow MetaPost capacities (the whole font will
simply be declared at the beginning of the eps figure and not every
character with its own transformation).} Of course, one should use
PostScript\TM\ fonts for such use. This is why we have introduced {\tt
mthreeDfont}.

\subsection{Curved text}
Planned but not released.

\remark
Remember that annotating graphics can still be done with usual control
sequence like
$$
\cs{label{\it.loc}({\it label}, proj($x$, $y$, $z$))}
$$
since things like {\tt simpletext} or {\tt curvedtext}
are rather for very special effects.
\endremark

\section{Animations}

\subsection{Introduction}
Denis Roegel demonstrated that it is possible to use usual Unix tools
to merge a list of MetaPost outputs into an animated GIF image (3D package).
I've learned a lot from his ``metapost to shell'' script. The idea is the following: first, keep track of the maximum boundary limits of every eps outputs; then convert these eps outputs into PostScript\TM\ or eps files with these maximum boundary limits; convert these last files to, say, simple GIF images; then merge all these images into an animation.

\subsection{How-to with m3D}
With {\tt m3Dplain} (as for {\tt 3D}), (hidden) numerics named
\cs{xmin\_}, \cs{xmax\_}, \cs{ymin\_} and \cs{ymax\_} are updated
at every figure ends through a control sequence named {\tt compute\_bbox}
(embedded by {\tt m3Dplain.mp} into \cs{extra\_endfig}). Executing
$$
\hbox{\tt Animate({\it numeric}, {\it boolean or color})}
$$
at the end of the file would produce an external file whose name is,
by default, {\tt animate-script}. 
Once the MetaPost job finished, under Unix-like system, execute
$$
	\hbox{\tt bash animate-script}
$$
from a console ({\tt xterm} or such) in the right directory. The final
output is \cs{jobname.gif} where {\tt jobname} is the actual name of the MetaPost program.

\subsection{External programs}
This script requires the following programs: {\tt sed} and {\tt convert}.
We choose to use {\tt sed} to change the boundary parameters of every MetaPost output---the resulting temporary files are named {\tt jobname.xxx.eps} where {\tt jobname.xxx} is the name of one of the MetaPost outputs. The PostScript\TM\ to GIF conversion was performed with the Netpbm library in Roegel's macros and their merging into an animation was performed by {\tt gifmerge} (a non standard but very nice Unix program freely available on the web). These last years, ImageMagick (copyrighted first by Dupont de Nemours, then by ImageMagick Studio, but quite free in fact) has spread over almost every Linux distribution. It is a very high quality tool that converts anything into everything, even animations. One of the basic control sequence is {\tt convert} which is the one called by the former script.

\subsection{Details} The following provides more detailed explanations.

\list{definitions}

\item{\tt Animate({\it numeric}, {\it boolean or color})}
Control sequence whose first parameter is the border in {\tt bp} provided with no units, and the second one is a color or a boolean. When the second parameter is a color, this color will be made transparent in the animation. When it is a boolean, {\tt background} (color) will be made transparent if the boolean is true, and no transparency will be made if this boolean is false.


\item{\tt AnimateScript} String variable, name of the (bash) script file which is output by {\tt Animate}.

\item{\tt AnimateQuality} Numerical variable, typical values are 1, 2, 4\dots

\item{\tt AnimateDelay} Numerical variable, time in $1/100$ seconds between every image in the animation.

\item{\tt AnimateLoop} Numerical variable, parameter for the animation, its dafult value is equal to 0 (infinite loop).

\item{\tt compute\_bbox} and also \cs{xmin\_}, \cs{xmax\_}, \cs{ymin\_}, \cs{ymax\_} have been explained before.
\endlist

\section{Outputting encapsuled PostScript directly}

\cs{DirectEPS \var{filename};}

...

\cs{endDirectEPS;}

\section{Some samples}

The first picture is made with \cs{Fill} equal to \cs{TechnoFill} (I
have to change this name one day), the next one with \cs{Fill} equal
to \cs{WireFill} (more conventional). I have also added a text (W.B. Yeats)
for testing the \cs{simpletext} object. There is, in the second picture, a cylinder but the most
important is the use of an object named \cs{rope}: given a path in
space through $x(t)$, $y(t)$, $z(t)$, a radius $r$, a range for $t$,
this object is what one can expect it to be. Also computations are
quite fragile (second order) and may leads to unexpected and ugly
effects. If the object \cs{cylinder} is defined in \cs{m3Dlib01.mp},
\cs{rope} is defined in \cs{m3Dplain.mp} since I think it may be a
basic tool.
$$\mybox{m3Dmanual.1}$$
$$\mybox{m3Dmanual.2}$$ Here there are two
Sierpinski--Menger objects: the sponge (object named
\cs{sierpinskip\_sponge}) and the gasket (object named
\cs{sierpinskip\_gasket}). The sponge is drawn with \cs{Fill} equal to
\cs{SolidFill} and the gasket with \cs{Fill} equal to
\cs{SolidWireFill}. Both objects are defined in \cs{m3Dlib01.mp}. The
gasket---since it grows as $4^n$ where $n$ is the level of
recursion---is a lot easier to draw than the sponge---which grows as
$20^n$. Reaching the level 3 on my current MetaPost implementation for
the sponge was not an easy task.
$$\mybox{m3Dmanual.3}\qquad\mybox{m3Dmanual.4}$$
$$\mybox{m3Dmanual.5}\qquad\mybox{m3Dmanual.6}$$
$$
	\mybox{m3Dmanual.7}
$$
This last example shows the use of the object named \cs{cylinderlike}
which is defined in \cs{m3Dplain.mp}.
$$
	\mybox{m3Dmanual.8}
$$
Its specific parameters are an
$xOy$-cycle path and the height of the cyclinder. Thus, former picture
as be obtained with
\verbatim
	beginfig(8);
		UseObject(cylinderlike,(0,0,0),(0,0,0),1cm,
			for i = 0 upto 4:
				2dir(i/5*360)--dir((i+0.5)/5*360)--
			endfor cycle, 1);
	endfig;
\endverbatim
\noindent which is rather simple.
$$
	\mybox{m3Dmanual.9}
$$
%\verbatimfile{m3Dplain.mp}
\bye

