\section{Publier sur le web}
Le web : http://fr.wikipedia.org/wiki/Www
Le web statique / le web dynamique
La "vieille" m�thode : �diteur de texte, client ftp, serveur web http
 [[HTML]]

Les m�thode "modernes" : le web dynamique c�t� serveur
 - Wiki
 - CMS Content Management System (exemple : SPIP)
Aspect technique du web dynamique : Serveur web, Php, MySQL
Installation d'une application � base de Php (SPIP par exemple)
Le probl�me de l'h�bergement 
 http://www.google.fr/search?hl=fr&q=hebergement+web+php+mysql+gratuit
 http://developpeur.journaldunet.com/tutoriel/php/050729-php-hebergeurs-gratuits.shtml
Indexation d'un site web
 Remarque : le fichier ##robots.txt## http://www.commentcamarche.net/web/robots-txt.php3

Mettre en ligne des exercices
 voir Hot Potatoes (non libre) et Sequane
 WIMS
  http://wims.unice.fr
  http://lamia.lille.iufm.fr/~georgesk/wims-book
  http://logiciels-libres-cndp.ac-versailles.fr/article.php3?id_article=3
  http://wims.ofset.org