===== Syst�me d'exploitation : Windows  =====
Identification d'un poste
 Nom de poste
 Nom du groupe de travail ou du domaine
Authentification
 Locale
 Domaine
Utilisation des comptes �l�ves / administrateur
Avantage : s�curit�, comptes limit�
Les probl�mes que cela peut poser et comment les r�soudre ?
 - autorisations (�criture dans un r�pertoire non autoris�)
 - les profils (menu, raccourcis Bureau)
Rem : dans Windows Vista le compte par d�faut sera un compte limit� (et pas un compte administrateur)

S�curit� :
 - mise � jour Windows Update http://windowsupdate.microsoft.com/
 - antivirus http://fr.wikipedia.org/wiki/Virus_informatique
   Freeware : Bitdefender Antivir Avast
 - firewall (logiciel / ou routeur-firewall mat�riel) http://fr.wikipedia.org/wiki/Firewall
    Freeware : ZoneAlarm
 - anty-spyware http://fr.wikipedia.org/wiki/Spyware
   Freeware : SpyBot Ad-Aware
 - anti-spam http://fr.wikipedia.org/wiki/Spam
   Mozilla Thunderbird (client de messagerie libre concurrent de MS Outlook) dispose d'un filtre bay�sien afin de lutter contre le SPAM
 - utilisateur/administrateur responsable !
Partage de fichier
Partage d'imprimante

Que faire lorsqu'on doit changer un ordinateur du r�seau ?
 Regarder et noter adresse IP / Masque / DNS / Passerelle
 Regarder et noter le nom du poste, le groupe de travail (ou le domaine)
 Lister les logiciels install�s (et les besoins des enseignants)
 Pr�sence de carte d'acquisition (interface sp�cifique ?)
 ...

Administration avanc�e : Utilisation des consoles MMC ##.msc##
[[Windows]]
[[WindowsAstuces]]
http://www3.ac-nancy-metz.fr/assistance/index.php
